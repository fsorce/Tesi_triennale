\documentclass[a4paper]{beamer}
\usetheme{Boadilla}
\usecolortheme{default}
\usepackage[utf8]{inputenc}
\usepackage{amsmath,amssymb,amsfonts,amsthm,stmaryrd}
\usepackage{mathrsfs} % per mathscr
\usepackage{dsfont} % per mathbb1
\usepackage{graphicx}% ruota freccia per le azioni
\usepackage{oldgerm} % Fractur Particolare
\usepackage{marvosym}% per il \Lightning
\usepackage{array}
\usepackage{faktor} %per gli insiemi quoziente
\usepackage{hyperref}
\usepackage{xparse} % Per nuovi comandi con tanti input opzionali
\usepackage{tikz-cd}
\usepackage{multicol}
\usepackage{multirow}
\usepackage{cancel}
\usepackage[italian]{babel}
\input{style/macros.tex}
\newcommand{\Sch}[1]{{\mathrm{Sch}}/{#1}}
\newcommand{\op}{^{op}}
\newcommand{\Set}{\mathrm{Set}}
\newcommand{\Psh}[2]{\left({\mathrm{Sch}}/{#1}\right)^{op}\to #2}
\newcommand{\Fun}{\mathrm{Fun}}
\newcommand{\gr}{\mathfrak{Gr}}
\newcommand{\Gr}{\mathrm{Gr}}
\newcommand{\Pl}{\mathrm{Pl}}
\newcommand{\Can}{{\mathcal{C}\mathrm{an}}}
\newcommand{\quotf}{\mathfrak{Quot}}
\newcommand{\Quot}{\mathrm{Quot}}
\newcommand{\hilb}{\mathfrak{Hilb}}
\newcommand{\Hilb}{\mathrm{Hilb}}
\newcommand{\yo}{{h_\bullet}}
\NewDocumentCommand{\bw}{O{k}}{{\bigwedge^{#1}}}


%========= Preambolo per quiver ================
% quiver e' uno strumento che uso spesso per
% disegnare diagrammi. L'interfaccia sul loro sito
% permette di creare in modo visivo il diagramma e
% poi esportarlo come codice LaTeX da inserire nel
% documento. Il sito e' https://q.uiver.app/ 

%-----------------------------------------------
% *** quiver ***
% A package for drawing commutative diagrams exported from https://q.uiver.app.
%
% This package is currently a wrapper around the `tikz-cd` package, importing necessary TikZ
% libraries, and defining a new TikZ style for curves of a fixed height.
%
% Version: 1.2.1
% Authors:
% - varkor (https://github.com/varkor)
% - Andr\e'C (https://tex.stackexchange.com/users/138900/andr%C3%A9c)

\NeedsTeXFormat{LaTeX2e}
%\ProvidesPackage{quiver}[2021/01/11 quiver]

% `tikz-cd` is necessary to draw commutative diagrams.
\RequirePackage{tikz-cd}
% `amssymb` is necessary for `\lrcorner` and `\ulcorner`.
\RequirePackage{amssymb}
% `calc` is necessary to draw curved arrows.
\usetikzlibrary{calc}
% `pathmorphing` is necessary to draw squiggly arrows.
\usetikzlibrary{decorations.pathmorphing}

% A TikZ style for curved arrows of a fixed height, due to Andr\e'C.
\tikzset{curve/.style={settings={#1},to path={(\tikztostart)
    .. controls ($(\tikztostart)!\pv{pos}!(\tikztotarget)!\pv{height}!270:(\tikztotarget)$)
    and ($(\tikztostart)!1-\pv{pos}!(\tikztotarget)!\pv{height}!270:(\tikztotarget)$)
    .. (\tikztotarget)\tikztonodes}},
    settings/.code={\tikzset{quiver/.cd,#1}
        \def\pv##1{\pgfkeysvalueof{/tikz/quiver/##1}}},
    quiver/.cd,pos/.initial=0.35,height/.initial=0}

% TikZ arrowhead/tail styles.
\tikzset{tail reversed/.code={\pgfsetarrowsstart{tikzcd to}}}
\tikzset{2tail/.code={\pgfsetarrowsstart{Implies[reversed]}}}
\tikzset{2tail reversed/.code={\pgfsetarrowsstart{Implies}}}
% TikZ arrow styles.
\tikzset{no body/.style={/tikz/dash pattern=on 0 off 1mm}}
%=================================================


% Title page
\title{Moduli spaces and Grassmannians}
\author{Francesco Sorce}
\institute[]{Universit\`a di Pisa}
\date{12 Luglio 2024}
\logo{\includegraphics[height=1cm]{images/marchio_unipi_pant541.png}}
\definecolor{LightBlue}{HTML}{3333B2}
\definecolor{DarkBlue}{HTML}{00518B}

\begin{document}
%\frame{\titlepage}
\begin{frame}{}
\begin{center}
{\Large \textcolor{LightBlue}{Moduli spaces and Grassmannians}}
\vspace{1cm}

\begin{tabular}{lm{4.5cm}r}
\textcolor{LightBlue}{Candidato:} && \textcolor{LightBlue}{Relatore:}\\
Sorce Francesco && Talpo Mattia
\end{tabular}
\vspace{1cm}

{\footnotesize Universit\`a di Pisa

Anno accademico 2023/24}
\medskip
\vspace{0.5cm}

12 Luglio 2024
\end{center}
\end{frame}


\begin{frame}{}
\begin{center}
{\huge \textbf{Grassmanniane}}
\end{center}
\end{frame}

\begin{frame}{Grassmanniane}
\begin{block}{Grassmanniane, definizione standard}
\[\Gr'(k,n,\K)=\cpa{\text{sottospazi di $\K^n$ di dimensione $k$}}.\]
\end{block}
\bigskip

\pause
Esempi:
\begin{itemize}
\item Spazi proiettivi $\Gr'(1,n)=\Pj^{n-1}$\pause
%\item Sfera di Riemann $\Gr'(1,1,\C)=\Pj^1_\C=\C\cup \cpa{\infty}$
\item $\Gr'(2,4)$ parametrizza rette in $\Pj^3$
\end{itemize}
\end{frame}

\begin{frame}{Scrittura con quozienti}
Nota: $\Gr'(n-k,n)=\cpa{\ker A\mid A\in \Mc(k,n),\ \rnk A=k}$.
\medskip

\begin{block}{Grassmanniane con quozienti}
\[\Gr(k,n,\K)=\quot{\cpa{A\in \Mc(k,n,\K)\mid \rnk A=k}}\sim\]
con $A\sim B\coimplies \ker A=\ker B\coimplies \exists P\in\GL_k$ t.c. $A=PB$.
\end{block}
\medskip
\[\Gr'(k,n,\K)\cong\Gr(n-k,n,\K)\]
\end{frame}

\begin{frame}{}
\begin{center}
{\Large \textbf{Questo insieme \`e uno spazio?}}
\end{center}
\end{frame}

\begin{frame}{Scrittura in carte}
Scegliendo un multiindice $I=(i_1,\cdots, i_k),\ 1\leq i_r\leq n$
\[\Gr_I(k,n)=\cpa{[A]\in \Gr(k,n)\mid \det A_I\neq 0}.\]
cio\`e, i sottospazi supplementari a $\Span(\cpa{e_{i_r}\mid 1\leq r\leq k})$.\pause

\begin{center}
\begin{tikzpicture}
\draw[white, ultra thick] (-5,0) -- (5,0);
\draw[thick, ->] (-3,0) -- (3,0);
\draw[thick, ->](0,-2) -- (0,2);
\draw[red, ultra thick] (-2,0) -- (2,0);
\draw[blue, ultra thick] (-1.41,-1.41) -- (1.41,1.41);
\draw[blue, ultra thick] (-1.73,-1) -- (1.73,1);
\draw[blue, ultra thick] (-1,1.73) -- (1,-1.73);
\filldraw[black] (0,0) circle (3pt);
\draw (-4,1.5) node [anchor=north west]  [font=\large]  {$\Gr_{(1)}(1,2,\R)$};
\draw (1.8,0.5) node [anchor=north west]   {$e_1$};
\draw (0,1.8) node [anchor=north west]  {$e_2$};
%\draw[cyan, very thick,->] (-1,0) arc (180:0:1) node [anchor=south east]{$\R$};
\draw[cyan, thick, ->] (-2,0.7) -- (2,0.7) node [anchor=west]{$\R$};
\end{tikzpicture}
\end{center}
\end{frame}



\begin{frame}{Scrittura in carte}
\[A_I\ii A=\mat{
w_{I^1_{1}} & \cdots & w_{I^1_n}\\
\vdots      & \ddots & \vdots\\
w_{I^k_{1}} & \cdots & w_{I^k_n}
},\qquad\text{dove }w_J=\frac{\det A_J}{\det A_I}\]

Per esempio, con $I=(1,\cdots, k)$
\[\left(\left.
\emat{
a_{1,1} & \cdots & a_{1,k}\\
\vdots & \ddots &\vdots\\
a_{k,1} & \cdots & a_{k,k}
}\right|
\emat{
a_{1,k+1} &\cdots & a_{1,n}\\
\vdots &\ddots & \vdots\\
a_{k,k+1} &\cdots & a_{k,n}
}\right)
\leadsto
\left( \emat{1&&\\&\ddots&\\&&1}\left| \emat{
a'_{1,k+1} & \cdots & a'_{1,n}\\
\vdots & \ddots &\vdots\\
a'_{k,k+1} & \cdots & a'_{k,n}
}\right.\right)\]
\medskip
\pause

Si pu\`o mostrare che $\Gr_I(k,n,\K)\cong \A_\K^{k(n-k)}$.\bigskip
\pause

Per $\K=\R$ abbiamo variet\`a liscia.

Per $\K=\C$ abbiamo variet\`a analitica complessa.
\end{frame}

\begin{frame}{}
\begin{center}
{\huge Ma stiamo facendo geometria algebrica!} 
\pause
\bigskip

{\Large Vogliamo delle equazioni}
\end{center}
\end{frame}

\begin{frame}{Mappa di Pl\"ucker}
Siano $\omega(k,n)$ i multiindici ordinati, $e_I=e_{i_1}\wedge\cdots\wedge e_{i_k}$ per $I=(i_1,\cdots, i_k)$.
\begin{block}{Mappa di Pl\"ucker}
\[\phi:\funcDef{\Mc(k,n)}{\bw\K^n}{A}{\displaystyle\sum_{I\in \omega(k,n)}\det A_I e_I}\]
\end{block}

\pause
\begin{alertblock}{Iniettivit\`a a meno di scalare}
$\rnk A<k$ se e solo se $\phi(A)=0$. 

Se $\rnk A=k$ allora $\ker A=\ker B$ se e solo se $\phi(A)=\la \phi(B)$ per $\la\neq 0$.
\end{alertblock}
\end{frame}

\begin{frame}{Embedding di Pl\"ucker}
\begin{block}{Embedding di Pl\"ucker}
\[\Pl:\funcDef{\Gr(k,n)}{\Pj^{\binom nk-1}}{[A]}{[\det A_I\mid I\in\omega(k,n)]}\]
\end{block}
\pause
\begin{alertblock}{Grassmanniane sono una variet\`a}
L'immagine di $\phi$ \`e un cono algebrico e un chiuso di Zariski di $\bw \K^n$.
\end{alertblock}
\pause
In particolare $\Gr(k,n)$ \`e anche uno schema proiettivo.
\end{frame}





%\begin{frame}{Grassmanniane come schema}
%$\phi:\Mc(k,n)\to \bw \K^n$ corrisponde a
%\[\phi^\#:\funcDef{\K[z_I\mid I\in\omega(k,n)]}{\K[x_{1,1},\cdots, x_{k,n}]}{z_I}{\det X_I}\]
%da cui $\cpa{\phi^\#=0}=\ol{\imm\phi}=\imm\phi$\pause 
%~e quindi
%\[V_+(\ker\phi^\#)\cong t(\imm \Pl)\cong t(\Gr(k,n)).\]
%\pause

%Ritroviamo il ricoprimento $\Gr_I(k,n)=\Gr(k,n)_{z_I}\cong \A^{k(n-k)}_\K$.
%\end{frame}

\begin{frame}{}
\begin{center}
{\huge \textbf{Spazi di moduli}}
\end{center}
\end{frame}

\begin{frame}{Problema di moduli}
\begin{block}{Problema di classificazione geometrico}
Dati degli \textbf{oggetti} e una \textbf{equivalenza} tra questi cerchiamo uno \textbf{spazio} che parametrizza le classi ``ragionevolmente".
\end{block}
\pause
Sia $F(T)$ l'insieme delle \textbf{famiglie} di oggetti parametrizzate da $T$ a meno di isomorfismo
\begin{center}
\begin{tikzpicture}
%\draw[gray, thick, ->] (-3,0) -- (3,0);
%\draw[gray, thick, ->](0,-2) -- (0,2);
\draw[white] (-4.5,0) -- (5,0);
\draw (-3,-0.3) node {$T$};
\draw (4.2,-0.8) node {$a\in F(T)$};
\draw[thick] (-1.2,1) .. controls (-4,0.6) and (-0.5,-0.4) .. (-3.5,-0.8);
\draw[thick, ->] (-0.5,0) -- (0.7,0);
\draw[thick] (-0.5,0.2) -- (-0.5,-0.2);
\draw (0.1,0.6) node {$F$};
\draw[thick] (3.8,1) .. controls (1,0.6) and (4.5,-0.4) .. (1.5,-0.8);
\filldraw[black] (1.7,-0.77) circle (1pt);
\filldraw[black] (2.35,-0.63) circle (1pt);
\filldraw[black] (2.81,-0.2) circle (1pt);
\filldraw[black] (2.63,0.4) circle (1pt);
\filldraw[black] (2.92,0.77) circle (1pt);
\filldraw[black] (3.53,0.95) circle (1pt);
\draw[thick] (1.70,-1.17) -- (1.70,-0.37);
\draw[thick] (2.60,-1.10) -- (2.16,-0.27);
\draw[thick] (3.21,-0.28) -- (2.42,-0.12);
\draw[thick] (2.92,0.51) -- (2.34,0.29);
\draw[thick] (3.17,1.08) -- (2.78,0.60);
\draw[thick] (3.66,1.32) -- (3.44,0.69);
\end{tikzpicture}
\end{center}
\pause
Un \textbf{problema di moduli} \`e un funtore $F:\Sch S\op\to \Set$
\end{frame}

%\begin{frame}{Approccio funtoriale}
%\begin{block}{Funtore hom}
%\[h_X:\functorDef{\Cc\op}{\Set}{T}{\Hom_\Cc(T,X)}{f:S\to T}{(g:T\to X)\mapsto (g\circ f:S\to X)}\]
%\end{block}
%\pause
%\begin{alertblock}{Lemma di Yoneda}
%\[\text{mappe classificanti}=\Hom(h_T,F)\cong F(T)=\text{famiglie}\]
%\end{alertblock}
%\pause
%\begin{itemize}
%\item Valutare il funtore in un punto restituisce gli oggetti da classificare.
%\item La funtorialit\`a permette di definite un pullback di famiglie
%\[f:S\to T,\; a\in F(T)\implies f^\ast a=F(f)(a)\in F(S).\]
%\end{itemize}
%\end{frame}

\begin{frame}{Esempio di problema di moduli}
Una \textbf{famiglia di curve lisce di genere $g$} su uno schema $S$ \`e un morfismo liscio, proprio e finitamente presentato $C\to S$ tale che ogni fibra $C_s$ \`e una curva liscia connessa e propria di genere $g$.
\[F_{M_g}:\functorDef{\Sch\C\op}{\Set}{S}{\quot{\cpa{\text{famiglia di curve lisce di genere $g$ su $S$}}}\sim}{T\to S}{(C\to S)\mapsto (C_T\doteqdot C\times_S T\to T)}\]
dove due famiglie $C\to S$ e $C'\to S$ sono equivalenti se esiste un isomorfismo tra $C$ e $C'$ compatibile con le mappe verso $S$.
\begin{center}
\begin{tikzpicture}
\draw[white, thick, ->] (-5,0) -- (5,0);
%\draw[gray, thick, ->](0,-1) -- (0,2);

\draw[thick] (-3.5,-0.5) .. controls (-1,-0.2) and (0.6,-1) .. (3.5,-0.8);
\draw (-2.5,-0.8) node {$S$};
\draw[thick, ->] (0,0.15) -- (0,-0.35);

\draw[gray, thick] (-3.5,1.4) .. controls (-1,1.8) and (0.6,1) .. (3.5,1.7);
\draw[gray, thick] (-3.5,0.5) .. controls (-1,0.6) and (0.6,0) .. (3.5,0.6);
\draw (-0.2,1.2) node {$C$};

\filldraw[black] (-2,-0.45) circle (2pt);
\draw (-2,-0.15) node {$s_1$};
\draw[cyan, thick] (-2,1) ellipse (0.8 and 0.5);
\draw[cyan, thick] (-1.6,1) arc (309:231:0.64);
\draw[cyan, thick] (-1.7,0.93) arc (55.1:124.9:0.524);
\draw (-2.4,1.8) node {$C_{s_1}$};

\filldraw[black] (1.8,-0.81) circle (2pt);
\draw (1.8,-0.51) node {$s_2$};
\draw[cyan, thick] (1.8,0.9) ellipse (1 and 0.5);
\draw[cyan, thick] (2.3,1) arc (315:225:0.71);
\draw[cyan, thick] (2.113,0.88) arc (55.1:124.9:0.55);
\draw (1.4,1.7) node {$C_{s_2}$};
\end{tikzpicture}
\end{center}
\end{frame}

\begin{frame}{Spazi di moduli}
$M$ \`e uno spazio di moduli\pause
\begin{itemize}
\item \textbf{fine} se $h_M\cong F$. La famiglia $u\in F(M)$ che corrisponde all'isomorfismo \`e detta \textbf{famiglia universale}.\pause
\item \textbf{grezzo} se 
\begin{itemize}
\item ogni famiglia induce un morfismo verso $M$ (fissiamo $F\to h_M$ naturale)
\item $M(\K)\leftrightarrow F(\Spec\K)$ per ogni campo algebricamente chiuso
\item $M$ \`e universale ($F\to h_N$ si fattorizza in $F\to h_M\to h_N$).
\end{itemize}
\end{itemize}
\medskip\pause

Esempio: $F_{M_g}$ ammette spazio di moduli grezzo ma non fine.
\end{frame}








\begin{frame}{Problema di moduli delle Grassmanniane}
\`E naturale aspettarsi che $\Gr'(k,n)$ sia uno spazio di moduli per
\[\gr'(k,n):\functorDef{(\Sch\K)\op}{\Set}{T}{\cpa{\emat{\Fc\text{ sottofibrato vettoriale di $\Oc_T^n$ di}\\ \text{rango $k$ t.c. $\Oc^n_T/\Fc$ loc. libero}}}}{f:S\to T}{\Fc\mapsto f^\ast\Fc}\]
\begin{center}
\begin{tikzpicture}
%\draw[gray, thick, ->] (-3,0) -- (3,0);
%\draw[gray, thick, ->](0,-2) -- (0,2);
\draw[white] (-4.5,0) -- (5,0);
\draw (-3,-0.3) node {$T$};
%\draw (4.2,-0.8) node {$a\in F(T)$};
\draw[thick] (-1.2,1) .. controls (-4,0.6) and (-0.5,-0.4) .. (-3.5,-0.8);
\draw[thick, ->] (-0.5,0) -- (0.7,0);
\draw[thick] (-0.5,0.2) -- (-0.5,-0.2);
\draw (0.1,0.6) node {$\gr'(1,3)$};
\draw[thick] (3.8,1) .. controls (1,0.6) and (4.5,-0.4) .. (1.5,-0.8);
\filldraw[black] (1.7,-0.77) circle (1pt);
\filldraw[black] (2.35,-0.63) circle (1pt);
\filldraw[black] (2.81,-0.2) circle (1pt);
\filldraw[black] (2.63,0.4) circle (1pt);
\filldraw[black] (2.92,0.77) circle (1pt);
\filldraw[black] (3.53,0.95) circle (1pt);
\draw[thick] (1.70,-1.17) -- (1.70,-0.37);
\draw[thick] (2.60,-1.10) -- (2.16,-0.27);
\draw[thick] (3.21,-0.28) -- (2.42,-0.12);
\draw[thick] (2.92,0.51) -- (2.34,0.29);
\draw[thick] (3.17,1.08) -- (2.78,0.60);
\draw[thick] (3.66,1.32) -- (3.44,0.69);
\end{tikzpicture}
\end{center}
\end{frame}

\begin{frame}{Problema di moduli delle Grassmanniane con quozienti}
\[\begin{tikzcd}[ampersand replacement=\&, column sep=small]
	{\Gr'(n-k,n)} \& {\Gr(k,n)} \\
	{\gr'(n-k,n)} \& {\gr(k,n)}
	\arrow["\cong"{marking, allow upside down}, draw=none, from=1-1, to=1-2]
	\arrow[tail reversed, from=1-1, to=2-1]
	\arrow[tail reversed, from=1-2, to=2-2]
	\arrow["\cong"{marking, allow upside down}, draw=none, from=2-1, to=2-2]
\end{tikzcd}\]
\pause
\[\gr(k,n):\functorDef{(\Sch\K)\op}{\Set}{T}{\quot{\cpa{\al:\Oc_T^n\onto Q}}\sim}{f:S\to T}{(\al:\Oc_T^n\to Q)\mapsto (f^\ast\al:\Oc_S^n\to f^\ast Q)}\]
dove $Q$ fibrato vettoriale su $T$ di rango $k$ e $\al\sim \beta\coimplies \ker\al=\ker\beta$.
\pause
\[\gr(k,n)(\Spec\K)\cong{\cpa{\vp:\K^n\onto \K^k}}/_\sim=\Gr(k,n)(\K).\]
\end{frame}






%\begin{frame}{}
%\begin{center}
%{\Large Vogliamo:}
%\pause
%\bigskip

%dare a $\Gr(k,n)$ la struttura di uno schema
%\pause
%\bigskip

%mostrare che $h_{\Gr(k,n)}\cong \gr(k,n)$
%\end{center}
%\end{frame}







%\begin{frame}{}
%\begin{center}
%{\Large \textbf{Per studiare la rappresentabilit\`a costruiamo un ricoprimento}}
%\end{center}
%\end{frame}

\begin{frame}{Sottofuntori e ricoprimenti aperti di funtori}
\[\begin{tikzcd}[ampersand replacement=\&, column sep=small]
	U \& {h_{U}} \& G \\
	T \& {h_T} \& F
	\arrow["\yo", from=1-1, to=1-2]
	\arrow["\subseteq"{marking, allow upside down}, draw=none, from=1-1, to=2-1]
	\arrow[dashed, from=1-2, to=1-3]
	\arrow[dashed, from=1-2, to=2-2]
	\arrow["\ulcorner"{anchor=center, pos=0.125}, draw=none, from=1-2, to=2-3]
	\arrow[from=1-3, to=2-3]
	\arrow["\yo", from=2-1, to=2-2]
	\arrow[from=2-2, to=2-3]
\end{tikzcd}\]
$\cpa{G_i\to F}$ ricoprimento quando gli $U_i$ coprono $T$.
\pause
\begin{block}{Sottofuntori aperti principali di $\gr(k,n)$}
\[\gr_I(k,n):\funcDef{(\Sch\K)\op}{\Set}{T}{\quot{\cpa{\Oc_T^n\overset{\al}\onto Q\mid \Oc_T^k\overset{\al_I}{\longrightarrow} Q\text{ surgettiva}}}\sim}\]
\end{block}
\pause
Dato $T$ e $h_T\to \gr(k,n)\leftrightarrow [\al]\in \gr(k,n)(T)$, abbiamo \[U_I=T\bs (\Supp (\coker(\al_I))).\]
Sono anche un ricoprimento.
\end{frame}

\begin{frame}{Fasci di Zariski}
\begin{block}{Fascio di Zariski}
Dato $\cpa{U_i\to X}$, $\qquad F(X)\to \displaystyle \prod_k F(U_k) \rightrightarrows \displaystyle \prod_{i,j}F(U_i\cap U_j)$
\end{block}
\pause
\begin{alertblock}{Incollamento di morfismi su ricoprimenti di fasci di Zariski}
Se $F$ e $G$ fasci di Zariski, $\cpa{F_i}_{i\in J}$ e $\cpa{G_i}_{i\in J}$ ricoprimenti aperti e 
\[f_i:F_i\to G_i,\qquad f_i\res{F_i\cap F_j}=f_j\res{F_i\cap F_j},\]
esiste $f:F\to G$. 
Se $f_i$ isomorfismo per ogni $i$ allora $f$ isomorfismo.
\end{alertblock}
\pause
\begin{itemize}
\item $h_X$ \`e un fascio di Zariski.\pause
\item $\gr(k,n)$ \`e un fascio di Zariski: 
%\begin{itemize}
%\item incolla i fasci codominio (unicit\`a del cambio di base $\implies$ cociclo)
%\item incolla i morfismi
%\end{itemize}
\end{itemize}
\end{frame}

%\begin{frame}{Incollamento di trasformazioni naturali}
%\pause
%\begin{proof}
%\[\begin{tikzcd}[ampersand replacement=\&, column sep=small]
%	{h_{U_i}} \& {F_i} \& {G_i} \\
%	{h_T} \& F \& G
%	\arrow["{\eta_i}", from=1-1, to=1-2]
%	\arrow["{h_{\iota_i}}"', from=1-1, to=2-1]
%	\arrow["\ulcorner"{anchor=center, pos=0.125}, draw=none, from=1-1, to=2-2]
%	\arrow["{f_i}", from=1-2, to=1-3]
%	\arrow["\subseteq"{marking, allow upside down}, draw=none, from=1-2, to=2-2]
%	\arrow["\subseteq"{marking, allow upside down}, draw=none, from=1-3, to=2-3]
%	\arrow["\zeta"', from=2-1, to=2-2]
%	\arrow["f"', dashed, from=2-2, to=2-3]
%\end{tikzcd}\]
%Siano $g_i=f_i\circ \eta_i$. $G$ \`e un fascio, quindi si incollano a $\zeta':h_T\to G$. Abbiamo $\Hom(h_T,F)\to \Hom(h_T,G)$ naturale, cio\`e $f:F\to G$.
%\end{proof}
%\pause
%Se gli $f_i$ sono isomorfismi anche $f$ lo \`e.\\
%~
%\end{frame}

\begin{frame}{Rappresentabilit\`a del funtore delle Grassmanniane}
\begin{alertblock}{Grassmanniana \`e uno spazio di moduli fine}
\[h_{\Gr(k,n)}\cong \gr(k,n).\]
\end{alertblock}
\begin{proof}
Applichiamo il risultato di prima verificando che $h_{\Gr_I(k,n)}\cong \gr_I(k,n)$ con trasformazioni compatibili con l'intersezione.
%\medskip
%\[\correspDef{\Hom_{\K\text{-alg}}\pa{\K\spa{\frac{\det X_J}{\det X_I}},\Oc_T(T)}}{\cpa{\Oc_T^n\overset\al\to \Oc_T^k\mid \al\circ s_I=id_{\Oc_T^k}}}{\vp}{\eta(\vp)}{\rho(\al):\frac{\det X_J}{\det X_I}\mapsto \frac{\det (\Mc^{\Can_n}_{\Can_k}(\al_T))_J}{\det (\Mc^{\Can_n}_{\Can_k}(\al_T))_I}}{\al}\]
%dove $\eta(\vp)$ \`e data dalla matrice $(\mathrm{res}^T_V\circ \vp)\pa{X_I\ii X}$.
\end{proof}
%~\\~\\~
\end{frame}

\begin{frame}{}
\begin{center}
{\huge \textbf{Generalizziamo il funtore} }
\end{center}
\end{frame}

\begin{frame}{Funtore dei quozienti}
\begin{center}
Fibrati vettoriali su $(\Spec\K)_T$ 

$\downarrow$ 

fasci LFP su $X_T$, piatti e con supporto proprio su $T$ tali che il polinomio di Hilbert sulle fibre di $X_T\to T$ calcolato con $\Lc$ sia costante.
\end{center}
\pause

Se $X\in\Sch S$, $\Ec$ LFP su $X$ e $\Phi\in \Q[\la]$, definiamo $\quotf^{\Phi,\Lc}_{\Ec/X/S}$ come
\[\functorDef{\Sch S\op}{\Set}{T}{\cpa{q:\Ec_T\onto Q\left| \emat{Q\text{ come sopra dove $\forall t\in T$,}\\\chi(Q\res{X_t}(r))=\Phi(r)}\right.}/_\sim}{f:T'\to T}{[q:\Ec_T\to Q]\mapsto [f^\ast q:\Ec_{T'}\to f^\ast Q]}\]
dove $q\sim q'$ se $\ker q=\ker q'$.
\end{frame}

\begin{frame}{Casi particolari: Grassmanniane e funtore di Hilbert}
Generalizza Grassmanniane: $\gr(k,n)=\quotf^{k,\Oc_{\K}}_{\Oc_{\K}^n/\K/\K}$.\pause
\bigskip

Se $\hilb^{\Phi,\Lc}_X$ problema dei moduli di sottoschemi chiusi di $X$ loc. noeth. con polinomio di Hilbert $\Phi$ allora, poich\'e
\begin{center}
sottoschemi chiusi di $X$ 

$\updownarrow$ 

fasci quasi-coerenti di ideali di $\Oc_X$ 

$\updownarrow$ 

classi di quozienti di $\Oc_X$
\end{center}
si ha
\[\hilb^{\Phi,\Lc}_X=\quotf^{\Phi,\Lc}_{\Oc_X/X/\K}\]

\end{frame}

\begin{frame}{Esistenza di Quot}
\begin{alertblock}{Esistenza degli schemi $\Quot$}
Sia $X$ un sottoschema chiuso di $\Pj^n_\K$, $\Lc=\Oc_{\Pj^n_\K}(1)\res X$, $\Ec$ un quoziente coerente di $\Oc_{X}(\nu)^p$ e $\Phi\in\Q[\la]$. Allora il funtore $\quotf^{\Phi,\Lc}_{\Ec/X/\K}$ \`e rappresentabile. 
\end{alertblock}
\pause
\begin{proof}
Ci riconduciamo a $X=\Pj^n_\K$ e $\Ec=\Oc^p_{\Pj^n_\K}$ e poi mostriamo che il seguente morfismo \`e una immersione localmente chiusa per $r\gg0$
\[\funcDef{\quotf^{\Phi,\Lc}_{\Oc^p_{\Pj^n_\K}/\Pj^n_\K/\K}(T)}{\gr(\Phi(r),\dim_\K\pi_\ast\Oc^p_{\Pj^n_\K}(r))(T)}{[\Oc^p_{\Pj^n_T}\onto Q]}{[{\pi_T}_\ast\Oc_{\Pj^n_T}(r)^p\to {\pi_T}_\ast Q(r)]}\]
\end{proof}
~\\~\\~
\end{frame}

%\begin{frame}{Schemi di Hilbert}
%Problema di classificazione $\hilb^{\Phi,\Lc}_{X}$ per $X$ sottoschema chiuso di $\Pj^n_\K$: 

%sottoschemi chiusi di $X$ con polinomio di Hilbert $\Phi$.
%\[\funcDef{(\Sch{\K})\op}{\Set}{T}{\cpa{Y\subseteq X\times T\mid \emat{Y\text{ piatto e finitamente presentato}\\\text{su $T$ tale che $Y_t$ ha polinomio}\\\text{di Hilbert $\Phi$ per ogni $t\in T$}}}}\]
%\pause

%Idea: sottoschema chiuso $\leftrightarrow$ fascio quasicoerente di ideali
%\[\hilb^{\Phi,\Lc}_X=\quotf^{\Phi,\Lc}_{\Oc_X/X/\K}\]
%\end{frame}

\begin{frame}
\begin{center}
{\huge \textbf{Grazie per l'attenzione!}}
\end{center}
\end{frame}

\end{document}
