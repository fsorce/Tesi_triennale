\chapter{Grassmanians}

\section{Set-theoretic definition}
\begin{definition}[Grassmannian]
Let $k\leq n$ be a pair of positive integers. We define the \textbf{$(n,k)$-Grassmannian}, denoted\footnote{we shall often omit the field when clear from context} $\Gr(k,n,\K)$, as the set of $(n-k)$-dimensional $\K$-vector subspaces of $\K^n$.
\end{definition}
\begin{remark}[Definition via quotients]
We may equivalently define $\Gr(k,n)$ to be the following set:
\[\cpa{\ker \vp\mid \vp\in \Hom_\K(\K^n,\K^k),\ \rnk \vp=k}.\]
\end{remark}

\begin{lemma}\label{kerAkerBVSActionOfGLk}
Let $\vp,\psi\in \Hom_\K(\K^n,\K^k)$ be linear maps of full rank. The following conditions are equivalent:
\begin{enumerate}
    \item $\ker\vp=\ker\psi$,
    \item there exists $\theta\in \GL(\K^k)$ such that $\vp=\theta\circ \psi$. 
\end{enumerate}
\end{lemma}
\begin{proof}
We shall prove the two implications:
\setlength{\leftmargini}{0cm}
\begin{itemize}
\item[$\boxed{2.\implies 1.}$] $\ker(\theta\circ \psi)=\psi\ii(\ker \theta)=\psi\ii(\cpa{0})=\ker \psi$. 
\item[$\boxed{1.\implies 2.}$] Let $z_1,\cdots, z_{n-k}$ be a basis of $\ker \vp=\ker \psi$ and let $z_1,\cdots, z_{n-k}, v_1,\cdots, v_k$ be a completion of it to a basis of $\K^n$. By construction $\vp(v_1),\cdots, \vp(v_k)$ and $\psi(v_1),\cdots, \psi(v_k)$ are bases of $\K^k$. Let $\theta$ be the linear automorphism of $\K^k$ determined by $\theta(\psi(v_i))=\vp(v_i)$ for all $i$. By construction $\theta$ is nonsingular and $\vp$ agrees with $\theta\circ \psi$ on a basis of $\K^n$.
\end{itemize}
\setlength{\leftmargini}{0.5cm}
\end{proof}

\begin{corollary}\label{LinearQuotientDefinition}
We may identify Grassmannians in terms of linear maps as follows:
\[\Gr(k,n)=\quot{\cpa{\vp\in \Hom_\K(\K^n,\K^k)\mid \vp\ \text{surjective.}}}\sim\]
where $\vp\sim \psi$ if and only if $\exists \theta\in \GL(\K^k)$ such that $\vp=\theta\circ \psi$.
\end{corollary}

\subsection{The Pl\"ucker embedding}
To make the study of Grassmanians easier, we want to identify $\Gr(k,n)$ with a subset of some projective space.

\begin{definition}[Pl\"ucker map]
Let $k\leq n$ be a pair of positive integers. We define the \textbf{Pl\"ucker map} as follows:
\[\phi:\funcDef{\Hom_\K(\K^n,\K^k)}{\bigwedge^k\Hom_\K(\K^n,\K^k)}{\vp}{\wedge^k\vp},\]
where $(\wedge^k\vp)(v_1,\cdots, v_k)=\vp(v_1)\wedge\cdots\wedge\vp(v_k)$.
\end{definition}
\begin{remark}
$\rnk \vp<k$ if and only if $\phi(\vp)=0$.
\end{remark}
\begin{proof}
$\phi(\vp)$ is the zero map if an only if $\vp(v_1),\cdots, \vp(v_k)$ are always linearly dependent, i.e. if and only if $\vp$ is not of full rank.
\end{proof}

\begin{proposition}\label{PluckerMapInjectiveOnGrassmaniansUpToScalar}
Let $\sim$ be the equivalence relation defined in corollary (\ref{LinearQuotientDefinition}), then for linear full rank maps $\vp,\psi:\K^n\to \K^k$
\[\vp\sim \psi\coimplies \exists \la\in\K^\ast\ s.t.\ \phi(\vp)=\la\phi(\psi).\]
\end{proposition}
\begin{proof}

\end{proof}

\section{Definition as a projective scheme}
In order to write $\K$-algebra morphisms that correspond to what we've done geometrically, we shall switch to the language of matricies.
\begin{definition}[Multiindicies]
We define a \textbf{$(k,n)$-multiindex} as an element of $\cpa{1,\cdots, n}^k$. Our notation for a multiindex $I$ will usually be $I=(i_1,\cdots, i_k)$.\\
If $A$ is a $k\times n$ matrix and $I$ is a $(k,n)$-multiindex, we denote the $I$-minor by $A_I$, i.e.
\[A_I=\mat{a_{1,i_1} &\cdots &a_{1,i_{k}}\\\vdots &\ddots &\vdots\\a_{k,i_1}&\cdots&a_{k,i_k}}.\]
\end{definition}



\section{Moduli functor}

\subsection{Affine cover}

\subsection{Representability of the Grassmannian functor}