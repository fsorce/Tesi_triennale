\chapter{Grassmannians}

\section{Set-theoretic definition}
\begin{definition}[Grassmannian]
Let $k\leq n$ be a pair of positive integers. We define the \textbf{$(n,k)$-Grassmannian}, denoted\footnote{we shall often omit the field when clear from context} $\Gr(k,n,\K)$, as the set of $(n-k)$-dimensional $\K$-vector subspaces of $\K^n$.
\end{definition}
\begin{remark}[Definition via quotients]
We may equivalently define $\Gr(k,n)$ to be the following set:
\[\cpa{\ker \vp\mid \vp\in \Hom_\K(\K^n,\K^k),\ \rnk \vp=k}.\]
\end{remark}

\begin{lemma}\label{kerAkerBVSActionOfGLk}
Let $\vp,\psi\in \Hom_\K(\K^n,\K^k)$ be linear maps of full rank. The following conditions are equivalent:
\begin{enumerate}
    \item $\ker\vp=\ker\psi$,
    \item there exists $\theta\in \GL(\K^k)$ such that $\vp=\theta\circ \psi$. 
\end{enumerate}
\end{lemma}
\begin{proof}
We shall prove the two implications:
\setlength{\leftmargini}{0cm}
\begin{itemize}
\item[$\boxed{2.\implies 1.}$] $\ker(\theta\circ \psi)=\psi\ii(\ker \theta)=\psi\ii(\cpa{0})=\ker \psi$. 
\item[$\boxed{1.\implies 2.}$] Let $z_1,\cdots, z_{n-k}$ be a basis of $\ker \vp=\ker \psi$ and let $z_1,\cdots, z_{n-k}, v_1,\cdots, v_k$ be a completion of it to a basis of $\K^n$. By construction $\vp(v_1),\cdots, \vp(v_k)$ and $\psi(v_1),\cdots, \psi(v_k)$ are bases of $\K^k$. Let $\theta$ be the linear automorphism of $\K^k$ determined by $\theta(\psi(v_i))=\vp(v_i)$ for all $i$. By construction $\theta$ is nonsingular and $\vp$ agrees with $\theta\circ \psi$ on a basis of $\K^n$.
\end{itemize}
\setlength{\leftmargini}{0.5cm}
\end{proof}

\begin{corollary}\label{LinearQuotientDefinition}
We may identify Grassmannians in terms of linear maps as follows:
\[\Gr(k,n)=\quot{\cpa{\vp\in \Hom_\K(\K^n,\K^k)\mid \vp\ \text{surjective.}}}\sim\]
where $\vp\sim \psi$ if and only if $\exists \theta\in \GL(\K^k)$ such that $\vp=\theta\circ \psi$.
\end{corollary}

\subsection{The Pl\"ucker embedding}
To make the study of Grassmannians easier, we want to identify $\Gr(k,n)$ with a subset of some projective space.

\begin{notation}
We shall use the following notation for brevity:
\[\cpa{\al:(\K^n)^k\to \bigwedge^k\K^k\mid \al\text{ multilinear, alternating}}=\bigwedge^k\Hom_{\K}(\K^n,\K^k)\]
\end{notation}

\begin{definition}[Pl\"ucker map]
Let $k\leq n$ be a pair of positive integers. We define the \textbf{Pl\"ucker map} as:
\[\phi:\funcDef{\Hom_\K(\K^n,\K^k)}{\bigwedge^k\Hom_\K(\K^n,\K^k)}{\vp}{\wedge^k\vp},\]
where $(\wedge^k\vp)(v_1,\cdots, v_k)=\vp(v_1)\wedge\cdots\wedge\vp(v_k)=\det\mat{\vp(v_1)|\cdots|\vp(v_k)} e_1\wedge\cdots\wedge e_k$.
\end{definition}

\begin{remark}\label{CodomainOfPluckerMap}
The codomain of the Pl\"ucker map is isomorphic to $\bigwedge^k\K^n$, indeed
\[\bigwedge^k\Hom_\K\pa{\K^n,\K^k}\pasgnl\cong{Univ.Prop.}\Hom_\K\pa{\bigwedge^k\K^n,\bigwedge^k\K^k}\cong \pa{\bigwedge^k\K^n}^{\vee}\cong \bigwedge^k\K^n.\]
Under this isomorphism the map takes on the following form
\[\funcDef{\Hom_\K(\K^n,\K^k)}{\bigwedge^k\K^n}{\vp}{\displaystyle\sum_{1\leq i_1<\cdots< i_k\leq n} \det(\vp(e_{i_1})\mid \cdots\mid \vp(e_{i_k}))e_{i_1}\wedge\cdots\wedge e_{i_k}}.\]
\end{remark}


\begin{remark}
The image of the Pl\"ucker map is a cone.
\end{remark}
\begin{proof}
For any $\la\in \K^\ast$ and any map $\vp\in \Hom_\K(\K^n,\K^k)$ we see that
\[\la\phi(\vp)=\phi(\al\circ \vp),\]
for any automorphism $\al$ of $\K^k$ with determinant\footnote{For example $\al(e_i)=\begin{cases}
\la e_1 &\text{if }i=1\\
e_i &\text{otherwise}
\end{cases}$} $\la$.
\end{proof}


\begin{remark}\label{WhenPluckerMapIsZero}
$\rnk \vp<k$ if and only if $\phi(\vp)=0$.
\end{remark}
\begin{proof}
$\phi(\vp)$ is the zero map if an only if $\vp(v_1),\cdots, \vp(v_k)$ are always linearly dependent, i.e. if and only if $\vp$ is not of full rank.
\end{proof}

\begin{lemma}\label{CharacterizationOfKernels}
Let $\vp:\K^n\to \K^k$ be a full rank linear map, then
\[\ker\vp=\cpa{z\in\K^n\mid \forall w_2,\cdots, w_k\in \K^n,\ \phi(\vp)(z,w_2,\cdots,w_k)=0}.\]
\end{lemma}
\begin{proof}
If $\vp(z)=0$ then for any $w_2,\cdots, w_k\in \K^k$ we see that 
\[\phi(\vp)(z, w_2,\cdots, w_k)=0\wedge \vp(w_2)\wedge\cdots\wedge \vp(w_k)=0.\]
Suppose now that $\vp(z)\neq 0$ and let $v_2,\cdots, v_k$ be such that $\cpa{\vp(z), v_2,\cdots,v_k}$ form a basis for $\K^k$. Since $\vp$ is surjective there exist $w_2,\cdots, w_k$ such that $\vp(w_i)=v_i$ for all $2\leq i\leq k$.
By construction 
\[\phi(\vp)(z,w_2,\cdots, w_k)=\vp(z)\wedge v_2\wedge\cdots\wedge v_k\neq0.\]
\end{proof}

\begin{proposition}\label{PluckerMapInjectiveOnGrassmanniansUpToScalar}
Let $\sim$ be the equivalence relation defined in corollary (\ref{LinearQuotientDefinition}), then for any two full rank linear maps $\vp,\psi:\K^n\to \K^k$
\[\vp\sim \psi\coimplies \exists \la\in\K^\ast\ s.t.\ \phi(\vp)=\la\phi(\psi).\]
\end{proposition}
\begin{proof}
Let us prove both implications:
\setlength{\leftmargini}{0cm}
\begin{itemize}
\item[$\boxed{\implies}$] If $\vp=\theta\circ \psi$ for $\theta\in GL(\K^k)$ then it follows easily from known properties of the determinant that \[\phi(\vp)=\phi(\theta\circ \psi)=(\det\theta) \phi(\psi).\]
\item[$\boxed{\impliedby}$] From lemma (\ref{kerAkerBVSActionOfGLk}) we see that it is enough to prove that $\ker \vp=\ker \psi$. We conclude by applying lemma (\ref{CharacterizationOfKernels}) as follows:
\begin{align*}
\ker\vp=&\cpa{z\in\K^n\mid \forall w_2,\cdots, w_k\in \K^n,\ \phi(\vp)(z,w_2,\cdots,w_k)=0}\\
=&\cpa{z\in\K^n\mid \forall w_2,\cdots, w_k\in \K^n,\ \la\phi(\psi)(z,w_2,\cdots,w_k)=0}=\\
=&\cpa{z\in\K^n\mid \forall w_2,\cdots, w_k\in \K^n,\ \phi(\psi)(z,w_2,\cdots,w_k)=0}=\ker \psi.
\end{align*}
\end{itemize}
\setlength{\leftmargini}{0.5cm}
\end{proof}

\begin{remark}
Because of proposition (\ref{PluckerMapInjectiveOnGrassmanniansUpToScalar}) and remark (\ref{WhenPluckerMapIsZero}) there exists a unique $h$ such that the diagram commutes
% https://q.uiver.app/#q=WzAsMyxbMCwwLCJcXGNwYXtcXHZwXFxpbiBcXEhvbV9cXEsoXFxLXm4sXFxLXmspXFxtaWQgXFxybmsgXFx2cD1rfSJdLFswLDEsIlxcR3IoayxuKSJdLFsxLDAsIlxcUGooXFxiaWd3ZWRnZV5rXFxIb21fXFxLKFxcS15uLFxcS15rKSkiXSxbMCwxLCJcXHBpIl0sWzAsMiwiW1xccGhpXSJdLFsxLDIsImgiLDIseyJzdHlsZSI6eyJib2R5Ijp7Im5hbWUiOiJkYXNoZWQifX19XV0=
\[\begin{tikzcd}
	{\cpa{\vp\in \Hom_\K(\K^n,\K^k)\mid \rnk \vp=k}} & {\Pj(\bigwedge^k\Hom_\K(\K^n,\K^k))} \\
	{\Gr(k,n)}
	\arrow["\pi", from=1-1, to=2-1]
	\arrow["{[\phi]}", from=1-1, to=1-2]
	\arrow["h"', dashed, from=2-1, to=1-2]
\end{tikzcd}\]
Moreover, such an $h$ must be injective by proposition (\ref{PluckerMapInjectiveOnGrassmanniansUpToScalar}).
\end{remark}

\begin{definition}[Pl\"ucker embedding]
We define the \textbf{Pl\"ucker embedding} as follows
\[\Pl:\funcDef{\Gr(k,n)}{\Pj(\bigwedge^k\K^n)}{[\vp]_\sim}{[(\det(\vp(e_{i_1})\mid\cdots\mid\vp(e_{i_k})))_{1\leq i_1<\cdots<i_k\leq n}]_{\K^\ast}}.\]
\end{definition}

\begin{remark}
If $\zeta$ is the isomorphism $\bigwedge^k\Hom_\K(\K^n,\K^k)\to \bigwedge^k\K^n$ discussed during remark (\ref{CodomainOfPluckerMap}), we see that the following diagram commutes
% https://q.uiver.app/#q=WzAsMyxbMCwxLCJcXEdyKGssbikiXSxbMCwwLCJcXFBqKFxcYmlnd2VkZ2Vea1xcSG9tX1xcSyhcXEtebixcXEteaykpIl0sWzEsMCwiXFxQaihcXGJpZ3dlZGdlXmtcXEtebikiXSxbMCwxLCJoIiwyXSxbMSwyLCJcXFBqKFxcemV0YSkiXSxbMCwyLCJcXG1hdGhybXtQbH0iLDJdXQ==
\[\begin{tikzcd}
	{\Pj(\bigwedge^k\Hom_\K(\K^n,\K^k))} & {\Pj(\bigwedge^k\K^n)} \\
	{\Gr(k,n)}
	\arrow["h"', from=2-1, to=1-1]
	\arrow["{\Pj(\zeta)}", from=1-1, to=1-2]
	\arrow["{\Pl}"', from=2-1, to=1-2]
\end{tikzcd}\]
This proves that the Pl\"ucker embedding is well defined and injective.
\end{remark}

\noindent We have thus identified $\Gr(k,n)$ with a subset of some projective space. We now need to show that it is a closed subset in the Zariski topology.***********************

*******************************
\bigskip

\noindent Let us consider the following map: let $\psi:\bigwedge^k\Hom_\K(\K^n,\K^k)$ be any alternating multilinear map, we define $\Phi(\psi)$ as
\[\Phi(\psi):\funcDef{\K^n}{\bigwedge^{k+1}\K^n}{v}{\displaystyle\sum_{I\in \omega(k,n)}\psi(e_{i_1},\cdots,e_{i_k})e_I\wedge v}.\]

\begin{proposition}
An alternating multilinear map $\psi\in \bigwedge^k\Hom_\K(\K^n,\K^k)$ is in the image of the Pl\"ucker map $\phi$ if and only if $\Phi(\psi)$ has rank at most $n-k$.
\end{proposition}
\begin{proof}
Suppose that $\psi=\phi(\vp)$ and let $\cpa{z_1,\cdots, z_k,z_{k+1},\cdots, z_{n}}$ be a basis of $\K^n$ such that the first $k$ vectors are a basis of $\ker \vp$. Then
\[\Phi(\psi)(v)=\sum_{I\in \omega(k,n)}\det(\vp(z_{i_1}\mid \cdots\mid \vp(z_{i_k})))z_I\wedge v.\]
Notice that if $v\in\ker \vp$ then for any choice of $I$ either the determinant vanishes or $k<i_1$*********************************
\end{proof}




















\section{Definition as a projective scheme}
In order to write $\K$-algebra morphisms that correspond to what we've done geometrically, we shall switch to the language of matricies.
\begin{definition}[Multiindicies]
We define a \textbf{$(k,n)$-multiindex} as an element of $\cpa{1,\cdots, n}^k$. Our notation for a multiindex $I$ will usually be $I=(i_1,\cdots, i_k)$.\\
If $A$ is a $k\times n$ matrix and $I$ is a $(k,n)$-multiindex, we denote the $I$-minor by $A_I$, i.e.
\[A_I=\mat{a_{1,i_1} &\cdots &a_{1,i_{k}}\\\vdots &\ddots &\vdots\\a_{k,i_1}&\cdots&a_{k,i_k}}.\]
We denote the set of \textbf{ordered $(k,n)$-multiindicies} with
\[\omega(k,n)=\cpa{(i_1,\cdots, i_k)\in \cpa{1,\cdots, n}^k\mid i_1<\cdots<i_k}.\]
\end{definition}

\begin{remark}
The set
\[\cpa{e_{i_1}\wedge\cdots\wedge e_{i_k}\mid 1\leq i_1<\cdots<i_k\leq n}\]
forms a basis for $\bigwedge^k\K^n$. For brevity, for all multiindicies $I=(i_1,\cdots, i_k)$ we shall define
\[e_I=e_{i_1}\wedge\cdots\wedge e_{i_k}.\]
\end{remark}
\medskip

\noindent
Under the isomorphism $\Hom_\K(\K^n,\K^k)\cong \Mc(k,n)$ given by choosing a basis, we may redefine
\[\Gr(k,n)=\quot{\cpa{A\in \Mc(k,n)\mid \rnk A=k}}{\sim},\] 
where $A\sim B\coimplies \exists P\in \GL_k\ s.t.\ A=PB$, 
\[\phi:\funcDef{\Mc(k,n)}{\bigwedge^k\K^n}{A}{\sum_{I\in \omega(k,n)}\det A_I e_I}\]
and
\[\Pl:\funcDef{Gr(k,n)}{\Pj(\bigwedge^k\K^n)}{[A]_\sim}{[\sum_{I\in \omega(k,n)}\det A_I e_I]_{\K^\ast}}.\]


To connect Grassmannians to the world of representable functors we shall redefine them scheme-theoretically by mimicing the Pl\"ucker embedding using rings

\begin{definition}[Braket ring]
We define the \textbf{braket ring} as the ring of polynomial functions on $\bigwedge^k\K^n$, i.e.
\[\Bc_{k,n}\doteqdot\frac{\K[z_I\mid I\in \cpa{1,\cdots, n}^k]}{(\cpa{z_I-\sgn(\sigma)z_{\sigma(I)}}_{\sigma\in S_k})}\cong \K[z_I\mid I\in \omega(k,n)]\]
\end{definition}

\begin{notation}
Let $\K[x_{1,1},\cdots,x_{k,n}]$ denote the polynomial ring with $k\cdot n$ variables. We will interpret this as the coordinate ring of $\Mc(k,n)$. Following this description we denote the \textbf{generic matrix} by
\[X=\mat{
    x_{1,1} & \cdots & x_{1,n}\\
    \vdots & \ddots & \vdots\\
    x_{k,1} &\cdots & x_{k,n}}\]
and by the same token we denote by $X_I$ the generic $k\times k$ minor determined by the multiindex $I$ and by $\det X_I$ the formal determinant of this minor.\\
We shall also denote $\K[x_{1,1},\cdots, x_{k,n}]$ by the compacter notation $\Oc(\Mc(k,n))$.
\end{notation}

\begin{remark}
The familiar $\Mc(k,n)$ and $\bigwedge^k\K^n$ can be identified with the $\K$-points of the affine schemes $\Spec \Oc(\Mc(k,n))$ and $\Spec \Bc_{k,n}$ respectively\footnote{Example 2.3.32 from Qing Liu}.
\end{remark}

\begin{definition}[Pl\"ucker ring homomorphism]
We define the \textbf{Pl\"ucker ring homomorphism} as
\[\phi^\#:\funcDef{\Bc_{k,n}}{\Oc(\Mc(k,n))}{z_I}{\det X_I}\]
\end{definition}

\begin{proposition}\label{PluckerRingHomomorphismWorksForKPoints}
The induced map $\Spec \phi^\#:\A^{kn}(\K)\to \A^{\binom nk}(\K)$ is equal to the Pl\"ucker map $\phi:\Mc(k,n)\to \bigwedge^k\K^n$ under the afformentioned identification, i.e. for any matrix $A\in\Mc(k,n)$ with entries $a_{i,j}$ we have
\[(\phi^\#)\ii((x_{i,j}-a_{i,j}))=(z_I-\det A_I).\]
\end{proposition}
\begin{proof}
First we observe that for any multiindex $I$
\[\det X_I-\det A_I\in (x_{i,j}-a_{i,j}),\]
thus $(z_I-\det A_I)\subseteq (\phi^\#)\ii((x_{i,j}-a_{i,j}))$.\\
Since $(z_I-\det A_I)$ is a $\K$-point, it is in particular a maximal ideal of the Braket ring, thus we have the desired equality if $1\notin (\phi^\#)\ii((x_{i,j}-a_{i,j}))$, which is the case because otherwise $(x_{i,j}-a_{i,j})$ would not be proper.
\end{proof}

\begin{lemma}
The kernel of the Pl\"ucker homomorphism is homogeneous.
\end{lemma}
\begin{proof}
By definition of homogeneous ideal, we want to show that if $f=\sum f_d$ for $d$ homogeneous and $f\in \ker\phi^\#$ then $f_d\in \ker \phi^\#$ for all $d$.\\
Looking at the definition of $\phi^\#$ we see that $\phi^\#(f_d)$ is a homogeneous polynomial of degree $kd$, in particular if $d\neq h$ then $\deg \phi^\#(f_d)\neq \deg \phi^\#(f_h)$. Since
\[0=\phi^\#(f)=\sum \phi^\#(f_d)\]
this proves that $\phi^\#(f_d)=0$ for all $d$.
\end{proof}

Since $\imm \phi$ is closed and\footnote{Known result of algebraic geometry} $V(\ker\phi^\#)=\ol{\imm \phi}$, we can identify $\Gr(k,n)$ with $V_+(\ker \phi^\#)$. This identification corresponds to the equality \[\Pj(\imm \phi)=V_+(\ker \phi^\#)(\K).\]


\section{Moduli functor}

Let us consider the following functor
\[\G(k,n):\funcDef{(\Sch\K)\op}{Set}{T}{\quot{\cpa{\al:\Oc_T^n\onto Q}}\sim}\]
where $Q$ is a locally free sheaf of rank $k$ on $T$ and two surjections $\al:\Oc_T^n\onto Q$, $\beta:\Oc_T^n\onto V$ are equivalent if and only if there exist an isomorphism of sheaves $\theta:Q\to V$ such that the diagram commutes
\[\begin{tikzcd}
	{\Oc_T^n} & Q \\
	& V
	\arrow["\al", two heads, from=1-1, to=1-2]
	\arrow["\beta"', two heads, from=1-1, to=2-2]
	\arrow["\theta", from=1-2, to=2-2]
\end{tikzcd}\]

\noindent In this this section we will prove that the Grassmannian scheme $V_+(\ker\phi^\#)$ defined above represents this functor.

\subsection{Affine cover}

\subsection{Representability of the Grassmannian functor}