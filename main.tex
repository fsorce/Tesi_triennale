\documentclass[a4paper]{report}
\usepackage[utf8]{inputenc}
\usepackage{amsmath,amssymb,amsfonts,amsthm,stmaryrd}
\usepackage{mathrsfs} % per mathscr
\usepackage{dsfont} % per mathbb1
\usepackage{graphicx}% ruota freccia per le azioni
\usepackage{oldgerm} % Fractur Particolare
\usepackage{marvosym}% per il \Lightning
\usepackage{array}
\usepackage{faktor} %per gli insiemi quoziente
\usepackage{hyperref}
\usepackage{xparse} % Per nuovi comandi con tanti input opzionali
\usepackage{tikz-cd}
\usepackage{multicol}
\usepackage{multirow}
\usepackage{cancel}
%\usepackage[italian]{babel}


% Ambienti per teoremi =================================
% <name> 
% <space above> 
% <space below> 
% <body font> 
% <indent amount> 
% <Theorem head font> 
% <punctuation after theorem head> 
% <space after theorem head> (default .5em) 
% <Theorem head spec>

% I nuovi ambienti sono costruiti in modo da andare alla riga successiva

%\newtheoremstyle{customth}
%{\topsep}{\topsep}{\itshape}{}{\bfseries}{.}{\newline}{}
%\newtheoremstyle{customdef}
%{\topsep}{\topsep}{\normalfont}{}{\bfseries}{.}{\newline}{}
%\newtheoremstyle{customrem}
%{\topsep}{\topsep}{\normalfont}{}{\itshape}{.}{\newline}{}


\newtheorem{theorem}{Theorem}[chapter]
\newtheorem{lemma}[theorem]{Lemma}
\newtheorem{corollary}[theorem]{Corollary}
\newtheorem{proposition}[theorem]{Proposition}
\newtheorem{fact}[theorem]{Fact}
\newtheorem{application}[theorem]{Application}
\theoremstyle{remark}
\newtheorem{remark}[theorem]{Remark}
\theoremstyle{definition}
\newtheorem{definition}[theorem]{Definition}
\newtheorem{notation}[theorem]{Notation}
\newtheorem{example}[theorem]{Example}

%\makeatletter
%\renewenvironment{proof}[1][\proofname]
%{
%    \par
%    \pushQED{\qed}
%    \normalfont \topsep6\p@\@plus6\p@\relax
%    \trivlist
%    \item[\hskip\labelsep\itshape#1\@addpunct{.}]\mbox{}\\*
%}
%{
%    \popQED\endtrivlist\@endpefalse
%}
%\makeatother



%============ Simboli standard =================
%----------------- Lettere ---------------------
\newcommand{\A}{\mathbb{A}}
\newcommand{\B}{\mathbb{B}}
\newcommand{\C}{\mathbb{C}}
\newcommand{\D}{\mathbb{D}}
\newcommand{\E}{\mathbb{E}}
\newcommand{\F}{\mathbb{F}}
\newcommand{\G}{\mathbb{G}}
\newcommand{\Hb}{\mathbb{H}}
\newcommand{\I}{\mathbb{I}}
\newcommand{\J}{\mathbb{J}}
\newcommand{\K}{\mathbb{K}}
\newcommand{\Lb}{\mathbb{L}}
\newcommand{\M}{\mathbb{M}}
\newcommand{\N}{\mathbb{N}}
\newcommand{\Ob}{\mathbb{O}}
\newcommand{\Pj}{\mathbb{P}}
\newcommand{\Q}{\mathbb{Q}}
\newcommand{\R}{\mathbb{R}}
\newcommand{\Sb}{\mathbb{S}}
\newcommand{\T}{\mathbb{T}}
\newcommand{\U}{\mathbb{U}}
\newcommand{\V}{\mathbb{V}}
\newcommand{\W}{\mathbb{W}}
\newcommand{\X}{\mathbb{X}}
\newcommand{\Y}{\mathbb{Y}}
\newcommand{\Z}{\mathbb{Z}}

\newcommand{\Ac}{\mathcal{A}}
\newcommand{\Bc}{\mathcal{B}}
\newcommand{\Cc}{\mathcal{C}}
\newcommand{\Dc}{\mathcal{D}}
\newcommand{\Ec}{\mathcal{E}}
\newcommand{\Fc}{\mathcal{F}}
\newcommand{\Gc}{\mathcal{G}}
\newcommand{\Hc}{\mathcal{H}}
\newcommand{\Ic}{\mathcal{I}}
\newcommand{\Jc}{\mathcal{J}}
\newcommand{\Kc}{\mathcal{K}}
\newcommand{\Lc}{\mathcal{L}}
\newcommand{\Mc}{\mathcal{M}}
\newcommand{\Nc}{\mathcal{N}}
\newcommand{\Oc}{\mathcal{O}}
\newcommand{\Pc}{\mathcal{P}}
\newcommand{\Qc}{\mathcal{Q}}
\newcommand{\Rc}{\mathcal{R}}
\newcommand{\Sc}{\mathcal{S}}
\newcommand{\Tc}{\mathcal{T}}
\newcommand{\Uc}{\mathcal{U}}
\newcommand{\Vc}{\mathcal{V}}
\newcommand{\Wc}{\mathcal{W}}
\newcommand{\Xc}{\mathcal{X}}
\newcommand{\Yc}{\mathcal{Y}}
\newcommand{\Zc}{\mathcal{Z}}

\newcommand{\Af}{\mathfrak{A}}
\newcommand{\Bf}{\mathfrak{B}}
\newcommand{\Cf}{\mathfrak{C}}
\newcommand{\Df}{\mathfrak{D}}
\newcommand{\Ef}{\mathfrak{E}}
\newcommand{\Ff}{\mathfrak{F}}
\newcommand{\Gf}{\mathfrak{G}}
\newcommand{\Hf}{\mathfrak{H}}
\newcommand{\If}{\mathfrak{I}}
\newcommand{\Jf}{\mathfrak{J}}
\newcommand{\Kf}{\mathfrak{K}}
\newcommand{\Lf}{\mathfrak{L}}
\newcommand{\Mf}{\mathfrak{M}}
\newcommand{\Nf}{\mathfrak{N}}
\newcommand{\Of}{\mathfrak{O}}
\newcommand{\Pf}{\mathfrak{P}}
\newcommand{\Qf}{\mathfrak{Q}}
\newcommand{\Rf}{\mathfrak{R}}
\newcommand{\Sf}{\mathfrak{S}}
\newcommand{\Tf}{\mathfrak{T}}
\newcommand{\Uf}{\mathfrak{U}}
\newcommand{\Vf}{\mathfrak{V}}
\newcommand{\Wf}{\mathfrak{W}}
\newcommand{\Xf}{\mathfrak{X}}
\newcommand{\Yf}{\mathfrak{Y}}
\newcommand{\Zf}{\mathfrak{Z}}

\newcommand{\af}{\mathfrak{a}}

\newcommand{\cf}{\mathfrak{c}}
\newcommand{\df}{\mathfrak{d}}
\newcommand{\ef}{\mathfrak{e}}
\newcommand{\ff}{\mathfrak{f}}
\newcommand{\gf}{\mathfrak{g}}
\newcommand{\hf}{\mathfrak{h}}

\newcommand{\jf}{\mathfrak{j}}
\newcommand{\kf}{\mathfrak{k}}
\newcommand{\lf}{\mathfrak{l}}
\newcommand{\mf}{\mathfrak{m}}
\newcommand{\nf}{\mathfrak{n}}
\newcommand{\of}{\mathfrak{o}}
\newcommand{\pf}{\mathfrak{p}}
\newcommand{\qf}{\mathfrak{q}}
\newcommand{\rf}{\mathfrak{r}}

\newcommand{\tf}{\mathfrak{t}}
\newcommand{\uf}{\mathfrak{u}}
\newcommand{\vf}{\mathfrak{v}}
\newcommand{\wf}{\mathfrak{w}}
\newcommand{\xf}{\mathfrak{x}}
\newcommand{\yf}{\mathfrak{y}}
\newcommand{\zf}{\mathfrak{z}}

\newcommand{\As}{\mathscr{A}}
\newcommand{\Bs}{\mathscr{B}}
\newcommand{\Cs}{\mathscr{C}}
\newcommand{\Ds}{\mathscr{D}}
\newcommand{\Es}{\mathscr{E}}
\newcommand{\Fs}{\mathscr{F}}
\newcommand{\Gs}{\mathscr{G}}
\newcommand{\Hs}{\mathscr{H}}
\newcommand{\Is}{\mathscr{I}}
\newcommand{\Js}{\mathscr{J}}
\newcommand{\Ks}{\mathscr{K}}
\newcommand{\Ls}{\mathscr{L}}
\newcommand{\Ms}{\mathscr{M}}
\newcommand{\Ns}{\mathscr{N}}
\newcommand{\Os}{\mathscr{O}}
\newcommand{\Ps}{\mathscr{P}}
\newcommand{\Qs}{\mathscr{Q}}
\newcommand{\Rs}{\mathscr{R}}
\newcommand{\Ss}{\mathscr{S}}
\newcommand{\Ts}{\mathscr{T}}
\newcommand{\Us}{\mathscr{U}}
\newcommand{\Vs}{\mathscr{V}}
\newcommand{\Ws}{\mathscr{W}}
\newcommand{\Xs}{\mathscr{X}}
\newcommand{\Ys}{\mathscr{Y}}
\newcommand{\Zs}{\mathscr{Z}}

\newcommand{\ula}{{\underline{a}}}
\newcommand{\ulb}{{\underline{b}}}
\newcommand{\ulc}{{\underline{c}}}
\newcommand{\uld}{{\underline{d}}}
\newcommand{\ule}{{\underline{e}}}
\newcommand{\ulf}{{\underline{f}}}
\newcommand{\ulg}{{\underline{g}}}
\newcommand{\ulh}{{\underline{h}}}
\newcommand{\uli}{{\underline{i}}}
\newcommand{\ulj}{{\underline{j}}}
\newcommand{\ulk}{{\underline{k}}}
\newcommand{\ull}{{\underline{l}}}
\newcommand{\ulm}{{\underline{m}}}
\newcommand{\uln}{{\underline{n}}}
\newcommand{\ulo}{{\underline{o}}}
\newcommand{\ulp}{{\underline{p}}}
\newcommand{\ulq}{{\underline{q}}}
\newcommand{\ulr}{{\underline{r}}}
\newcommand{\uls}{{\underline{s}}}
\newcommand{\ult}{{\underline{t}}}
\newcommand{\ulu}{{\underline{u}}}
\newcommand{\ulv}{{\underline{v}}}
\newcommand{\ulw}{{\underline{w}}}
\newcommand{\ulx}{{\underline{x}}}
\newcommand{\uly}{{\underline{y}}}
\newcommand{\ulz}{{\underline{z}}}

%---------- Funzioni standard ------------------
\newcommand{\Adj}{\mathrm{Adj}\,}
\newcommand{\adj}{\mathrm{adj}\,}
\newcommand{\Ann}{\mathrm{Ann}\,}
\newcommand{\Arg}{\mathrm{Arg}\,}
\newcommand{\Ass}{\mathrm{Ass}\,}
\newcommand{\cha}{\mathrm{char}\,}
\newcommand{\cod}{\mathrm{cod}}
\newcommand{\coker}{\mathrm{coker}\,}
\newcommand{\comb}{\mathrm{Comb}\,}
\newcommand{\dom}{\mathrm{dom}}
\newcommand{\End}{\mathrm{End}\,}
\newcommand{\Fix}{\mathrm{Fix}\;}
\newcommand{\Hom}{\mathrm{Hom}\,}
\newcommand{\imm}{\mathrm{Imm}\,}
\newcommand{\Ind}{\mathrm{Ind}}
\newcommand{\mcd}{\mathrm{mcd}\,}
\newcommand{\mcm}{\mathrm{mcm}\,}
\newcommand{\Min}{\mathrm{Min}\,}
\newcommand{\Mor}{\mathrm{Mor}}
\newcommand{\obj}{\mathrm{obj}}
\newcommand{\orb}{\mathrm{orb}\,}
\newcommand{\ord}{\mathrm{ord}\;}
\newcommand{\Proj}{\mathrm{Proj}\,}
\newcommand{\Res}{\mathrm{Res}}
\newcommand{\rnk}{\mathrm{rnk}\,}
\newcommand{\sgn}{\mathrm{sgn}\,}
\newcommand{\Span}{\mathrm{Span}\,}
\newcommand{\Spec}{\mathrm{Spec}\,}
\newcommand{\stab}{\mathrm{stab}\,}
\newcommand{\Supp}{\mathrm{Supp}\,}
\newcommand{\supp}{\mathrm{supp}\,}
\newcommand{\tr}{\mathrm{tr}\,}

\newcommand{\Real}{\,\Re\mathfrak{e}}
\newcommand{\Imag}{\,\Im\mathfrak{m}}

%-------------- Frecce -------------------------
\newcommand{\coimplies}{\Longleftrightarrow}
\newcommand{\inj}{\hookrightarrow}
\newcommand{\onto}{\twoheadrightarrow}
\newcommand{\ot}{\leftarrow}
\newcommand{\acts}{\curvearrowright}

%----------- Lettere greche -------------------
\newcommand{\al}{\alpha}
\newcommand{\de}{\delta}
\newcommand{\e}{\varepsilon}
%\newcommand{\th}{\theta}
\newcommand{\la}{\lambda}
\newcommand{\vp}{\varphi}

%-------------- Derivate ----------------------
\newcommand{\raiseargument}[1]{\raisebox{.8ex}{$#1$}}
\newcommand{\centersmallmath}[1]{\vcenter{\hbox{\scalebox{.8}{$#1$}}}}
\newcommand{\raiseargumentsmall}[1]{\raisebox{.4ex}{\scalebox{.8}{$#1$}}}
\newcommand*{\emptyfrac}[2]{\genfrac{}{}{0pt}{}{#1}{#2}}

\NewDocumentCommand{\ddxi}{O{x}mm}{
    {\frac{d^{}{#3}}{d{#1}_{#2}}}
}

\NewDocumentCommand{\dd}{O{}mm}{
    {\frac{d^{#1}{#3}}{d{#2}^{#1}}}
}

\NewDocumentCommand{\ppxi}{O{x}mm}{
    {{\frac{\partial^{}{#3}}{\partial{#1}_{#2}}}}
}

\NewDocumentCommand{\pp}{O{}mm}{
    {{\frac{\partial^{#1}{#3}}{\partial{#2}}}}
}





%========== Comandi dattilografici ============
%--------- Passaggi in derivazioni ------------
\newcommand{\pasg}[3]{\overset{\hyperref[#3]{\text{#2}}}{#1}}
\newcommand{\pasgnl}[2]{\overset{\text{#2}}{#1}}
\newcommand{\pasgnlmath}[2]{\overset{#2}{#1}}
\newcommand{\pasgmath}[3]{\overset{\hyperref[#3]{{#2}}}{#1}}

%----------- Modifica testo -------------------
\newcommand{\ul}[1]{\underline{#1}}
\newcommand{\ol}[1]{\overline{#1}}
\newcommand{\wt}[1]{\widetilde{#1}}
\newcommand{\wh}[1]{\widehat{#1}}
\newcommand{\td}[1]{\Tilde{#1}}
\newcommand{\rg}[1]{{\mathring {#1}}}
\newcommand{\under}[2]{\underset{#1}{\underbrace{#2}}}

%-------------- Parentesi ---------------------
\newcommand{\pa}[1]{\left({#1}\right)}
\newcommand{\spa}[1]{\left[{#1}\right]}
\newcommand{\cpa}[1]{\left\{{#1}\right\}}
\newcommand{\abs}[1]{\left|{#1}\right|}
\newcommand{\norm}[1]{\left\Vert{#1}\right\Vert}
\newcommand{\ps}[1]{\left\langle {#1}\right\rangle}
\newcommand{\floor}[1]{\left\lfloor {#1}\right\rfloor}
\newcommand{\ceil}[1]{\left\lceil {#1}\right\rceil}
\newcommand{\rbar}[1]{\left.{#1}\right|}

%--------------- Matrici ----------------------
\newcommand{\mat}[1]{\begin{pmatrix}#1\end{pmatrix}}
\newcommand{\emat}[1]{\begin{matrix}#1\end{matrix}}
\newcommand{\dmat}[1]{\begin{vmatrix}#1\end{vmatrix}}
\newcommand{\smat}[1]{\begin{smallmatrix}#1\end{smallmatrix}}
\newcommand{\BIG}[1]{\mathlarger{\mathlarger{\mathlarger{\mathlarger{#1}}}}}

%--------------- Funzioni ---------------------
\newcommand{\funcDef}[4]{
\begin{array}{ccc}
{#1} & \longrightarrow & {#2}\\
{#3} & \longmapsto & {#4}
\end{array}}
\newcommand{\functorDef}[6]{
\begin{array}{ccc}
{#1} & \longrightarrow & {#2}\\
{#3} & \longmapsto & {#4}\\
{#5} & \longmapsto & {#6}
\end{array}}
\newcommand{\correspDef}[6]{
\begin{array}{ccc}
{#1} & \longleftrightarrow & {#2}\\
{#3} & \longmapsto & {#4}\\
{#5} & \longmapsfrom & {#6}
\end{array}}

%---------------- Altro -----------------------
\newcommand{\bs}{\setminus}
\newcommand{\res}[1]{\raisebox{-.5ex}{$|$}_{#1}}
\newcommand{\quot}[2]{\faktor{#1}{#2}}
\newcommand{\sep}{\,\middle|\,}

\newcommand{\ii}{^{-1}}
\newcommand{\nz}{\bs\{0\}}

\newcommand{\powerset}{\mathscr{P}}
\newcommand{\del}{\partial}
\newcommand{\0}{{\underline{0}}}
\newcommand{\1}{{\vcenter{\hbox{\scalebox{1.2}{$\mathds{1}$}}}}}


\newcommand{\GL}{\mathrm{GL}}
\newcommand{\PGL}{\mathrm{PGL}}
%\NewDocumentCommand{\PGL}{o m}{
%    \IfNoValueTF{#1}
%        {{\mathbb{P}GL({#2})}}
%    {{\mathbb{P}GL_{#1}({#2})}}
%}
%\NewDocumentCommand{\GL}{o m}{
%    \IfNoValueTF{#1}
%        {{GL({#2})}}
%    {{GL_{#1}({#2})}}
%}
\newcommand{\znz}[1]{{\Z/{#1}\Z}}






%========= Preambolo per quiver ================
% quiver e' uno strumento che uso spesso per
% disegnare diagrammi. L'interfaccia sul loro sito
% permette di creare in modo visivo il diagramma e
% poi esportarlo come codice LaTeX da inserire nel
% documento. Il sito e' https://q.uiver.app/ 

%-----------------------------------------------
% *** quiver ***
% A package for drawing commutative diagrams exported from https://q.uiver.app.
%
% This package is currently a wrapper around the `tikz-cd` package, importing necessary TikZ
% libraries, and defining a new TikZ style for curves of a fixed height.
%
% Version: 1.2.1
% Authors:
% - varkor (https://github.com/varkor)
% - Andr\e'C (https://tex.stackexchange.com/users/138900/andr%C3%A9c)

\NeedsTeXFormat{LaTeX2e}
%\ProvidesPackage{quiver}[2021/01/11 quiver]

% `tikz-cd` is necessary to draw commutative diagrams.
\RequirePackage{tikz-cd}
% `amssymb` is necessary for `\lrcorner` and `\ulcorner`.
\RequirePackage{amssymb}
% `calc` is necessary to draw curved arrows.
\usetikzlibrary{calc}
% `pathmorphing` is necessary to draw squiggly arrows.
\usetikzlibrary{decorations.pathmorphing}

% A TikZ style for curved arrows of a fixed height, due to Andr\e'C.
\tikzset{curve/.style={settings={#1},to path={(\tikztostart)
    .. controls ($(\tikztostart)!\pv{pos}!(\tikztotarget)!\pv{height}!270:(\tikztotarget)$)
    and ($(\tikztostart)!1-\pv{pos}!(\tikztotarget)!\pv{height}!270:(\tikztotarget)$)
    .. (\tikztotarget)\tikztonodes}},
    settings/.code={\tikzset{quiver/.cd,#1}
        \def\pv##1{\pgfkeysvalueof{/tikz/quiver/##1}}},
    quiver/.cd,pos/.initial=0.35,height/.initial=0}

% TikZ arrowhead/tail styles.
\tikzset{tail reversed/.code={\pgfsetarrowsstart{tikzcd to}}}
\tikzset{2tail/.code={\pgfsetarrowsstart{Implies[reversed]}}}
\tikzset{2tail reversed/.code={\pgfsetarrowsstart{Implies}}}
% TikZ arrow styles.
\tikzset{no body/.style={/tikz/dash pattern=on 0 off 1mm}}
%=================================================

%\usepackage{biblatex}
%PER CAMBIARE I MARGINI
\usepackage[margin=4cm]{geometry}

%----------- Setup stilistico ----------------
\definecolor{DarkRed}{HTML}{B6321C}
\hypersetup{
    colorlinks=true,
    linkcolor=DarkRed,
    filecolor=blue,
    citecolor = black,
    urlcolor=cyan,
}
\renewcommand\thefootnote{\textcolor{blue}{\arabic{footnote}}}
% ============================================


%---------- Comandi specifici ----------------
\newcommand{\Sch}[1]{{\mathrm{Sch}}/{#1}}
\newcommand{\op}{^{op}}
\newcommand{\Set}{\mathrm{Set}}
\newcommand{\Psh}[2]{\left({\mathrm{Sch}}/{#1}\right)^{op}\to #2}
\newcommand{\Fun}{\mathrm{Fun}}
\newcommand{\Gr}{\mathrm{Gr}}
\newcommand{\Pl}{\mathrm{Pl}}
\newcommand{\Can}{{\mathcal{C}\mathrm{an}}}


%\usepackage[utf8]{inputenc}
%\DeclareFontFamily{U}{min}{}
%\DeclareFontShape{U}{min}{m}{n}{<-> udmj30}{}
%\newcommand\yo{\!\text{\usefont{U}{min}{m}{n}\symbol{'207}}\!}
\newcommand{\yo}{{h_\bullet}}


%--------- Comandi dattilografici ------------
\newcommand{\comm}[1]{}
\NewDocumentCommand{\bw}{O{k}}{{\bigwedge^{#1}}}

% ============================================
\title{Moduli Spaces and Grassmannians}

\author{Francesco Sorce}
\date{Università di Pisa\\
Dipartimento di Matematica}


\begin{document}
\maketitle

\begin{abstract}
In this document we introduce the concept of moduli spaces in algebraic geometry through the example of the Grassmannian scheme.

The first chapter introduces the basics of the functorial approach to algebraic geometry and its relation to moduli problems.

The second chapter is a quick overview of Grassmannians as defined set theoretically. We focus our attention on the Pl\"ucker embedding and prove that it identifies the Grassmannian with a projective variety.

In the third chapter we describe the reduced scheme structure on the Grassmannian and prove that it is a fine moduli space for the functor of quotients from $\Oc_T^n$ to a rank $k$ vector bundle on $T$.

%The last chapter is a brief overview of the construction of the Hilbert and Quot schemes and how they generalize Grassmannians.
\end{abstract}

%\newpage
\tableofcontents
\newpage

\chapter{Introduction}
The following type of \textit{classification problem} occurs often in math:
\begin{center}
Consider some type of object and a notion of isomorphism which can be defined between them. We are interested in understanding the behavior of isomorphism classes and how they relate to each other.
\end{center}
The set or class of isomorphism classes is a tautological answer to the set-theoretic question, but for an answer to a classification problem to be satisfactory we usually require it to encode some information on \textit{families} of isomorphism classes.\medskip

Miraculously, many such classification problems turn out to have a natural answer in the form of some geometric object. 
In general the object can only be defined as the category of families together with some geometric structure (this is the realm of the theory of stacks), but in more special circumstances one can find a more concrete space, usually a scheme, whose points represent isomorphism classes for our problem and whose geometric structure encodes information on the families. 
Such objects are called \textit{moduli spaces} for the classification problem.\medskip

The best result we can hope for is finding a space which completely encodes how families behave\footnote{what will be formalized as a fine moduli space}, but this requirement is usually too strict.
In this document we mostly deal with problems for which such a nice space exists: the Grassmannian, Quot and Hilbert schemes.


\section*{Historical background}
The history of moduli theory aligns remarkably well with that of the moduli space of smooth curves of fixed genus. Indeed the word moduli was introduced by Riemann in the article \cite{riemann54theorie} to denote what we would now call the dimension of $M_g$, the moduli space of smooth projective algebraic curves of genus $g$, which he computed to be $3g-3$. 

Although the argument given by Riemann can be made rigorous in modern language, he did not prove the existence of the space $M_g$ itself.
The first general construction of $M_g$ as a space of some kind can be attributed to Teichm\"uller, which realized $M_g$ as the quotient of the Teichm\"uller space $T_g$ parametrizing complex structures up to isomorphism on a surface of genus $g$ by the action of the group $\Gamma_g$ of diffeomorphisms of the surface up to isotopy. The paper which establishes these ideas is \cite{teichmuller1939extremale}.

The basis for the modern theory were laid by Alexander Grothendieck and his functorial approach.
He first introduced his methods to analytic moduli theory and later on to algebraic geometry in general.
Grothendieck was very interested in algebraic moduli theory and contributed to it greatly by introducing the Hilbert, Quot and Picard functors and showing their representability by schemes. However, Grothendieck did not end up publishing on $M_g$.

Among the first to study moduli spaces systematically was David Mumford.
Inspired by invariant theory, Grothendieck's functorial approach and the existing constructions of moduli spaces like the one of principally polarized abelian varieties or the Chow varieties, Mumford developed Geometric Invariant Theory (commonly referred to as GIT), which can be described as a method to study and construct moduli spaces as quotients of algebraic groups.
In the book \cite{GIT} Mumford gives two constructions of $M_g$ as a coarse moduli space.

For a more detailed history and more references see section 0.1 in \cite{Alper}.

\section*{Why category theory?}
As we briefly mentioned, the modern approach to moduli problems if formalized via functors. It might not be clear why this is the most appropriate tool, and indeed it can seem more complicated than more concrete treatments in simple cases like the classification of lines through a point via projective space.

Nevertheless, the functorial approach has proven itself to be effective in many aspects, chief among them the formalization of the nebulous concept of ``family" described above.\medskip

Following Grothendieck's ideas, a moduli problem is expressed as a contravariant functor
\[F:T\mapsto \quot{\cpa{\text{families of objects over $T$}}}\sim\]
where $\sim$ is the isomorphism relation imposed on families of objects.
Since we are mostly concerned about problems in algebraic geometry, and thus families over schemes, the functor is usually taken to be a presheaf on $\Sch S$ for some base scheme\footnote{usually $\Spec \K$ for an algebraically closed field $\K$ or $\Spec \Z$.} $S$.
To find the set of objects we want to classify up to isomorphism we can simply evaluate $F$ on a point.

The functorial language allows for families to be \textit{pulled back} via morphisms: if $f:S\to T$ is a morphism and $a\in F(T)$ is a family over $T$, then $F(f):F(T)\to F(S)$ by contravariance and thus $F(f)(a)\doteqdot f^\ast a\in F(S)$ is a family over $S$.
\medskip

There are several ways in which we can define a moduli space. The two most relevant are \textit{fine} and \textit{coarse} moduli spaces. A scheme $M$ is a fine moduli space if we can recover the whole moduli functor from it\footnote{formally, when $h_M$ and $F$ are naturally isomorphic functors.}. $M$ is a coarse moduli space if its $\K$-points are in bijection with $F(\Spec \K)$, all families over $T$ induce a morphism $T\to M$ which behaves well with pullbacks and $M$ is universal for these properties.

In both cases we can interpret a family of objects over a scheme $T$ as a morphism from $T$ to $M$ called \textit{classifying map}.  Intuitively this is the function that to each point of $T$ assigns the corresponding isomorphism class. The added structure of a scheme morphism serves to define a ``niceness" on families. If $M$ is a fine moduli space, then every family over $T$ can be viewed as the pullback under a morphism $T\to M$ of a specific family $u\in F(M)$, called the \textit{universal family}.

\section*{Why Grassmannians?}
Grassmannians are among the first nontrivial examples of spaces whose points represent some type of object one can encounter in their mathematical career. Given two positive integers $k$ and $n$, the first definition of a Grassmannian $\Gr(k,n)$ one encounters is
\[\Gr'(k,n)=\cpa{H\subseteq\K^n\mid \text{$H$ vector subspace, }\dim_\K H=k}.\]
This definition invites us to think about the classification problem of $k$-dimensional vector subspaces of $n$-dimensional space. This classification problem is best formalized in terms vector bundle quotients as
\[\gr(k,n):\functorDef{(\Sch\K)\op}{\Set}{T}{\quot{\cpa{\al:\Oc_T^n\onto Q}}\sim}{f:S\to T}{(\al:\Oc_T^n\to Q)\mapsto (f^\ast\al:\Oc_S^n\to f^\ast Q)}\]
where $q\sim q'\coimplies \ker q=\ker q'$, so the definition we will use for $\Gr(k,n)$ is actually
\[\Gr(k,n)=\quot{\cpa{\vp:\K^n\to \K^k\mid \rnk \vp=k}}\sim\qquad \text{where }\vp\sim \psi\coimplies \ker\vp=\ker\psi,\]
but the two are related, up to canonical identifications, by $\Gr(k,n)=\Gr'(n-k,n)$.
Showing that Grassmannians are schemes and that they are fine moduli spaces for this classification problem is a good introduction to the elementary tools of the theory of fine moduli spaces. Grassmannians also serve as a warm up and necessary stepping stone in the construction of the Quot schemes, which generalize Grassmannians and yield important results like the existence of Hilbert schemes.
\chapter{Moduli Spaces}
The following type of \textit{classification problem} occurs often in math:
\begin{center}
Consider some type of object and a notion of isomorphism which can be defined between them. We are interested in understanding the behaviour of isomorphism classes and how they relate to each other.
\end{center}
Finding a bijection between isomorphism classes and known objects is usually trivial\footnote{for example, if the classes form a set they can be identified with a canonical set of the same cardinality}, but for an answer to a classification problem to be satisfactory we usually require some information on \textit{families} of isomorphism classes.\medskip

Miraculously, many such classification problems turn out to have a natural answer in the form of a space whose points are isomorphism classes and whose geometry encodes how families behave. Such an object is usually called a \textit{moduli space}.\\
The best result we can hope for is finding a space which completely encodes how families behave\footnote{what will be formalized as a fine moduli space}, but this requirement is usually too strict. 
In this document we shall mostly deal with problems for which such a nice space exists.\medskip

The most fruitful formalization of family as intended above has been the functorial approach, the basics of which will be introduced in this chapter.
\medskip

\noindent Most definitions given in this chapter follow \cite{Alper} and \cite{Bejleri1}.

\section{Representable functors}
In this section we introduce the basic concepts of the functorial approach and some of the required category theory.

\subsection{Yoneda lemma and embedding}
All categories considered in this document will be small. If $\Cc$ is a category, we shall write $X\in \Cc$ to mean ``$X$ is an object in $\Cc$".\\
If $A$ and $B$ are objects in a category (resp. categories / functors) then $\Hom(A,B)$ denotes the set of morphisms (resp. functors / natural transformations) from $A$ to $B$.\\
The notation $\Fun(\Cc,\Dc)$ denotes the category of functors from $\Cc$ to $\Dc$, with morphisms being natural transformations.

\begin{definition}[Presheaf]
A contravariant functor $F:\Cc\op\to \Set$ is called a \textbf{presheaf} on $\Cc$. For any fixed category $\Cc$, the presheaves on $\Cc$ form a category $\Fun(\Cc\op,\Set)$ with morphisms given by natural transformations.\\
If $T\in \Cc$ then we call the elements of $F(T)$ \textbf{families} over $T$.
\end{definition}

\begin{definition}[Hom-functor]
Let $\Cc$ be a category and $X\in \Cc$. We define the \textbf{Hom-functor} of $X$ to be
\[h_X:\functorDef{\Cc\op}{\Set}{T}{\Hom(T,X)}{f:T\to S}{\circ f:\Hom(S,X)\to \Hom(T,X)}\]
\end{definition}
\begin{remark}
The Hom-functor is a presheaf.
\end{remark}

\begin{definition}[Moduli problem]
Let $S$ be a scheme. A presheaf on $\Sch S$ is called a \textbf{moduli problem}.
\end{definition}

\begin{remark}
Usually we study moduli problems of the following form
\[\functorDef{\Sch S}{\Set}{T}{\quot{\cpa{\text{families over $T$}}}\sim}{f:T\to R}{\text{pullback of families along }f}\]
where $\sim$ refers to the equivalence relation we are studying the families of classes of.
\end{remark}

\noindent To proceed we will need the following

\begin{lemma}[Yoneda Lemma]\label{YonedaLemma}
Let $\Cc$ be a category and $X\in \Cc$. For all contravariant functors $F:\Cc\op\to\Set$ the following sets are in a natural bijection
\[\Hom(h_X,F)\longleftrightarrow F(X).\]
\end{lemma}
\begin{proof}
Given a natural trasformation $\zeta$, we can take its image in $F(X)$ to be $\zeta_X(id_X)$.\\
On the other hand, for any given element $u\in F(X)$ we can define an arrow $h_X(T)\to F(T)$ for any $T\in \Cc$ by taking $f\mapsto F(f)(u)$. This collection of maps defines a natural transformation from $h_X$ to $F$ by commutativity of the following diagram for all $g:S\to T$
\[\begin{tikzcd}
	{\Hom(X,X)} \\
	& {\Hom(T,X)} & {\Hom(S,X)} \\
	& {F(T)} & {F(S)} \\
	{F(X)}
	\arrow["{\circ f}"', from=1-1, to=2-2]
	\arrow["{\circ (f\circ g)}", from=1-1, to=2-3]
	\arrow["{F(\cdot)(u)}"', from=1-1, to=4-1]
	\arrow["{\circ g}"', from=2-2, to=2-3]
	\arrow["{F(\cdot)(u)}"', from=2-2, to=3-2]
	\arrow["{F(\cdot)(u)}", from=2-3, to=3-3]
	\arrow["{F(g)}", from=3-2, to=3-3]
	\arrow["{F(f)}", from=4-1, to=3-2]
	\arrow["{F(f\circ g)}"', from=4-1, to=3-3]
\end{tikzcd}\]
\end{proof}

\begin{definition}[Yoneda embedding]
We define the \textbf{Yoneda embedding} of a category $\Cc$ to be the following functor
\[\yo:\functorDef{\Cc}{\Fun(\Cc\op,\Set)}{X}{h_X}{f:X\to Y}{f\circ:h_X\to h_Y}\]
\end{definition}

\begin{proposition}
The functor $\yo$ is fully faithful.
\end{proposition}
\begin{proof}
Recall that a functor $F:\Cc\to \Dc$ is fully faithful if for any two objects $A,B\in\Cc$ we have $\Hom_\Cc(A,B)\cong \Hom_\Dc(F(A),F(B))$. In our case we want to check if
\[\Hom(X,Y)\cong\Hom(h_X,h_Y),\]
which is exactly the statement of the Yoneda lemma (\ref{YonedaLemma}).
\end{proof}
\begin{remark}\label{YonedaEmbeddingInjectiveOnIsoClasses}
The Yoneda embedding is injective up to isomorphism on isomorphism classes of objects in $\Cc$.
\end{remark}
\begin{proof}
Suppose $\zeta:h_X\to h_Y$ is an isomorphism of functors, then $id_X\in h_X(X)$ corresponds to some $f=\zeta_X(id_X):X\to Y$. Similarly $id_Y\in h_Y(Y)$ corresponds to some $g=\zeta_Y\ii(id_Y):Y\to X$. By naturality of $\zeta$ we see can pull back $id_X$ and $f$ by $g$ to see that 
% https://q.uiver.app/#q=WzAsOCxbMywwLCJcXEhvbShYLFkpIl0sWzIsMSwiZiJdLFswLDAsIlxcSG9tKFgsWCkiXSxbMSwxLCJpZF9YIl0sWzAsMywiXFxIb20oWSxYKSJdLFsxLDIsImciXSxbMywzLCJcXEhvbShZLFkpIl0sWzIsMiwiaWRfWT1mXFxjaXJjIGciXSxbMywxLCIiLDAseyJzdHlsZSI6eyJ0YWlsIjp7Im5hbWUiOiJtYXBzIHRvIn19fV0sWzIsMCwiXFx6ZXRhX1giXSxbMiw0LCJcXGNpcmMgZyIsMl0sWzMsNSwiIiwyLHsic3R5bGUiOnsidGFpbCI6eyJuYW1lIjoibWFwcyB0byJ9fX1dLFs0LDYsIlxcemV0YV9ZIiwyXSxbMCw2LCJcXGNpcmMgZyJdLFsxLDcsIiIsMCx7InN0eWxlIjp7InRhaWwiOnsibmFtZSI6Im1hcHMgdG8ifX19XSxbNSw3LCIiLDIseyJzdHlsZSI6eyJ0YWlsIjp7Im5hbWUiOiJtYXBzIHRvIn19fV1d
\[\begin{tikzcd}
	{\Hom(X,X)} &&& {\Hom(X,Y)} \\
	& {id_X} & f \\
	& g & {id_Y=f\circ g} \\
	{\Hom(Y,X)} &&& {\Hom(Y,Y)}
	\arrow["{\zeta_X}", from=1-1, to=1-4]
	\arrow["{\circ g}"', from=1-1, to=4-1]
	\arrow["{\circ g}", from=1-4, to=4-4]
	\arrow[maps to, from=2-2, to=2-3]
	\arrow[maps to, from=2-2, to=3-2]
	\arrow[maps to, from=2-3, to=3-3]
	\arrow[maps to, from=3-2, to=3-3]
	\arrow["{\zeta_Y}"', from=4-1, to=4-4]
\end{tikzcd}\]
and similarly we can show that $g\circ f=id_X$.
\end{proof}

\begin{lemma}[Yoneda embedding is continuous]\label{YonedaEmbeddingPreservesLimits}
The Yoneda embedding preserves limits.
\end{lemma}
\begin{proof}
Suppose $X$ is the limit of the diagram $\cpa{f_{ij}:X_j\to X_i}$. If we apply the Yoneda embedding to the diagram we obtain
\[\cpa{\circ f_{ij}:h_{X_j}\to h_{X_i}}\]
Let $F$ be any presheaf on $\Cc$ and suppose that we have morphisms $F\to h_{X_i}$ which make the diagram commute, then for all $T\in \Cc$ we have compatible and natural $F(T)\to \Hom(T,X_i)$. 
If $f\in F(T)$ then these arrows define several $f_i\in \Hom(T,X_i)$ which compose respecting the diagram. 
By the universal property of limits this defines uniquely a morphism $f_\ell\in \Hom(T,X)$ and we see that the assignment $f\mapsto f_\ell$ is the unique map from $F(T)$ to $\Hom(T,X)$ which makes the diagram in $\Set$ commute. 
Since all that we have done is natural in $T$ we have effectively constructed a morphism $F\to h_X$ as we desired.
\end{proof}

\subsection{Moduli spaces}
\begin{definition}[Representable functor]
A presheaf $F$ on $\Cc$ is \textbf{representable} if there exists a natural isomorphism $\zeta:F\to h_X$ for some $X\in \Cc$. In this case we say that the pair\footnote{usually we just say that $X$ represents $F$} $(X,\zeta)$ \textbf{represents} $F$.\\
If $a\in F(T)$ we call the map $\zeta_T(a):T\to X$ the \textbf{classifying map} of the family $a$.
\end{definition}
\begin{definition}[Universal family]
Instead of the pair $(X,\zeta)$ for $\zeta:F\to h_X$, we may consider the pair $(X,\xi)$ where $\xi\in F(X)$ is the image of $\zeta\ii$ under the bijection predicated by the Yoneda lemma.\\
The object $\xi$ is called the \textbf{universal family} of $X$.
\end{definition}
\begin{remark}
The universal family is given by
\[\zeta\ii_X(id_X).\]
\end{remark}


\begin{definition}[Fine moduli space]
Let $F$ be a moduli functor. A scheme $X\in \Sch S$ is a \textbf{fine moduli space} for $F$ if $X$ represents $F$.
\end{definition}

\begin{notation}
If $U$ is a subscheme of $T$ and $i:U\to T$ is the inclusion morphism, then if $\xi\in F(T)$ we will define its restriction to $U$ to be
\[\xi\res U=F(i)(\xi).\]
\end{notation}

\begin{remark}
Because the Yoneda embedding is injective on isomorphism classes up to isomorphism (\ref{YonedaEmbeddingInjectiveOnIsoClasses}), fine moduli spaces are unique up to isomorphism.
\end{remark}

\begin{example}[Projective space]
Consider the functor
\[\Pj_n:\functorDef{\mathrm{Sch}\op}{\Set}{S}{\quot{\cpa{(\Lc,s_0,\cdots,s_n)\mid \emat{\Lc\text{ line bundle on $S$, }s_0,\cdots,s_n\in\Lc(S),\\\forall x\in S,\ \ps{(s_0)_x,\cdots,(s_n)_x}_{\Oc_{S,x}}=\Lc_x}}}\sim}{f}{\text{pullback of sheaves and sections via $f$}}\]
where $(\Lc,(s_i))\sim (\Lc',(s_i'))$ is there exists a sheaf isomorphism $\al:\Lc\to\Lc'$ such that $s_i=\al^\ast s_i'$ for all $i\in\cpa{0,\cdots, n}$.\medskip

\noindent It is a well know fact (Proposition 5.1.31 in \cite{QingLiu}) that $\Pj_n(S)\cong \Hom(S,\Pj^n_\Z)$ and that pullbacks behave as expected, thus $\Pj^n_\Z$ is a fine moduli space for $\Pj_n$.\\
From the statement of Proposition 5.1.31 in \cite{QingLiu} it is also clear that $\Oc_{\Pj^n_\Z}(1)$ is a universal family.
\end{example}

\noindent Fine moduli spaces do not always exist. The simplest obstructions to having a fine moduli spaces are
\begin{itemize}
\item the functor is not a Zariski sheaf (see proposition (\ref{RepresentableModuliFunctorsAreZariskiSheaves}))
\item existence of non trivial automorphisms.
\end{itemize}

\noindent To get an idea of why the second condition is an obstruction consider the following
\begin{proposition}
Let $F\in \Psh{\C}{\Set}$ be a moduli functor. If there exists an \textbf{isotrivial family} $\Ec\in F(S)$ for $S\in \Sch\C$ variety, i.e.
\begin{itemize}
\item for all $s,t\in S(\C)$, the fiber $F(s)(\Ec)=\Ec_s=\Ec_t$ and
\item the family $\Ec$ is not the pullback of an object $E\in F(\Spec \C)$ along the structural morphism $S\to \Spec \C$,
\end{itemize}
then there exists no fine moduli space for $F$.
\end{proposition}
\begin{proof}
This is Proposition 0.3.21 in \cite{Alper}. [ISERT LATER MAYBE]
\end{proof}

\noindent A weaker notion of moduli space is that of coarse moduli space:
\begin{definition}[Coarse moduli space]
Let $F$ be a moduli problem. A pair $(X,\zeta)$ for $X\in \Sch S$ and $\zeta:F\to h_X$ natural transformation is a \textbf{coarse moduli space} for $F$ if
\begin{itemize}
\item $\zeta_{\Spec \K}:F(\Spec\K)\to \Hom(\Spec\K,X)$ is a bijection for all algebraically closed fields $\K$
\item for any scheme $Y$ and $\eta:F\to h_{Y}$ natural transformation there exists a unique morphism $\al:X\to Y$ such that $\eta=h_\al\circ \zeta$.
\end{itemize}
\end{definition}
\begin{remark}
A fine moduli space is also a coarse moduli space.
\end{remark}
\begin{proof}
The first condition is trivially verified. For the second condition, if $(Y,\eta)$ is defined as above and $(X,u)$ is the fine moduli space with universal family $u$ then we can take $\al=\eta_X(u)$.
\end{proof}


\section{Zariski sheaves and gluing of fine moduli spaces}
One approach to show representability of a moduli problem is emulating the gluing properties of sheaves.
Indeed it is possible to show that representable functors are sheaves of some kind and this realization will eventually lead us to the representability criterion we will use later in this document. 

\subsection{Zariski sheaves}
First, let us formalize a way in which a functor can be a sheaf. First we recall the definition of equalizer:

\begin{definition}[Equalizer]
Let $\Cc$ be a category, $A,B,C\in \Cc$ and $f,g:B\to C$. We say that the pair $(A,h)$ is an \textbf{equalizer} of the diagram \[B\underset{g}{\overset{f}{\rightrightarrows}} C\]
if $h:A\to B$ is such that $f\circ h=g\circ h$ and if $(Q,q)$ is another such pair then there exists a unique morphism $Q\to A$ which makes the diagram commute
\[\begin{tikzcd}
	A & B & C \\
	Q
	\arrow["h", from=1-1, to=1-2]
	\arrow["f", shift left, from=1-2, to=1-3]
	\arrow["g"', shift right, from=1-2, to=1-3]
	\arrow["q"', from=2-1, to=1-2]
	\arrow[dashed, from=2-1, to=1-1]
\end{tikzcd}\]
\end{definition}

\begin{definition}[Zariski sheaf]
A moduli problem $F\in \Psh S\Set$ is a \textbf{Zariski sheaf} if for any $S$-scheme $X$ and any Zariski open cover $\cpa{U_i\to X}$ the following diagram is an equalizer
% https://q.uiver.app/#q=WzAsMyxbMCwwLCJGKFgpIl0sWzIsMCwiXFxkaXNwbGF5c3R5bGUgXFxwcm9kX2sgRihVX2spIl0sWzQsMCwiXFxkaXNwbGF5c3R5bGUgXFxwcm9kX3tpLGp9RihVX2lcXGNhcCBVX2opIl0sWzAsMV0sWzEsMiwiIiwwLHsib2Zmc2V0IjotMX1dLFsxLDIsIiIsMix7Im9mZnNldCI6MX1dXQ==
\[\begin{tikzcd}
	{F(X)} && {\displaystyle \prod_k F(U_k)} && {\displaystyle \prod_{i,j}F(U_i\cap U_j)}
	\arrow[from=1-1, to=1-3]
	\arrow[shift left, from=1-3, to=1-5]
	\arrow[shift right, from=1-3, to=1-5]
\end{tikzcd}\]
where the arrows are induced by the inclusions.
\end{definition}


\begin{proposition}[Representable moduli functors are Zariski sheaves]\label{RepresentableModuliFunctorsAreZariskiSheaves}
Let $F:\Psh S\Set$ be a moduli problem, then if there exists a fine moduli space $M$ for $F$ it must be the case that $F$ is a Zariski sheaf.
\end{proposition}
\begin{proof}
Up to composing with the natural isomorphism, we may assume $F=h_M$. 
Let $X$ be an $S$-scheme and $\cpa{U_i\to X}$ a Zariski open cover for it. 
We want to show that the following diagram is an equalizer
% https://q.uiver.app/#q=WzAsMyxbMCwwLCJIb20oVSxNKSJdLFsyLDAsIlxcZGlzcGxheXN0eWxlIFxccHJvZF9pIEhvbShVX2ksTSkiXSxbNCwwLCJcXGRpc3BsYXlzdHlsZSBcXHByb2Rfe2ksan1Ib20oVV9pXFx0aW1lc19VVV9qLE0pIl0sWzAsMSwiUmVzIl0sWzEsMiwicHJfMV5cXGFzdCIsMCx7Im9mZnNldCI6LTF9XSxbMSwyLCJwcl8yXlxcYXN0IiwyLHsib2Zmc2V0IjoxfV1d
\[\begin{tikzcd}
	{\Hom(X,M)} && {\displaystyle \prod_k \Hom(U_k,M)} && {\displaystyle \prod_{i,j}\Hom(U_i\cap U_j,M)}
	\arrow[from=1-1, to=1-3]
	\arrow[shift left, from=1-3, to=1-5]
	\arrow[shift right, from=1-3, to=1-5]
\end{tikzcd}\]
The arrows in this case correspond to restriction of morphisms, so what we need to verify is
\begin{itemize}
\item $\mathrm{res}^{U_i}_{U_i\cap U_j}\circ \mathrm{res}^X_{U_i}=\mathrm{res}^{U_j}_{U_i\cap U_j}\circ \mathrm{res}^X_{U_j}$
\item a collection of maps $\cpa{f_i:U_i\to M}$ such that $f_i\res{U_i\cap U_j}=f_j\res{U_i\cap U_j}$ glues uniquely to a map $f:X\to M$
\end{itemize}
both of which are true.
\end{proof}

\subsection{Open cover of a moduli problem}
Let us now define the analogue of an open cover for functors
\begin{definition}[Subfunctor]
A functor $F:\Cc\to \Set$ is a \textbf{subfunctor} of $G:\Cc\to \Set$ if for all $X\in \Cc$ and for all $f\in \Hom_\Cc(A,B)$
\[F(X)\subseteq G(X),\qquad\text{and}\qquad F(f)=G(f)\res{F(A)}.\]
In this case we write $F\subseteq G$.
\end{definition}

\begin{definition}[Fibered product of presheaves]
Let $F,G,H:\Cc\op\to\Set$ be presheaves together with two natural transformations $\eta:F\to H$ and $\zeta:G\to H$. We define their fibered product as the following functor
\[F\times_H G:\functorDef{\Cc\op}{\Set}{X}{F(X)\times_{H(X)}G(X)}{f:A\to B}{(b_1,b_2)\mapsto(F(f)(b_1),G(f)(b_2))}\]
where the fibered product $F(X)\times_{H(X)}G(X)$ in defined through the maps $\eta_X$ and $\zeta_X$. The map $(F\times_H G)(f)$ is well defined because if $(b_1,b_2)\in F(B)\times_{H(B)}G(B)$ then $\eta_B(b_1)=\zeta_B(b_2)$, thus
\[\eta_A(F(f)(b_1))=H(f)(\eta_B(b_1))=H(f)(\zeta_B(b_2))=\zeta_A(G(f)(b_2)).\]
\end{definition}

\begin{definition}[Open subfunctor]
Let $F:\Psh S\Set$ be a moduli problem. We say that a subfunctor $G\subseteq F$ is \textbf{open} if for any $S$-scheme $T$ and any natural transformation $h_T\to F$, there exists an open subscheme $U$ of $T$ such that $U$ represents $h_T\times_FG$, i.e.
% https://q.uiver.app/#q=WzAsNixbMSwwLCJoX1UiXSxbMSwxLCJoX1QiXSxbMiwxLCJGIl0sWzIsMCwiRyJdLFswLDAsIlUiXSxbMCwxLCJUIl0sWzMsMl0sWzEsMl0sWzAsMywiIiwwLHsic3R5bGUiOnsiYm9keSI6eyJuYW1lIjoiZGFzaGVkIn19fV0sWzAsMSwiIiwyLHsic3R5bGUiOnsiYm9keSI6eyJuYW1lIjoiZGFzaGVkIn19fV0sWzAsMiwiIiwxLHsic3R5bGUiOnsibmFtZSI6ImNvcm5lci1pbnZlcnNlIn19XSxbNCw1LCJcXHN1YnNldGVxIiwzLHsic3R5bGUiOnsiYm9keSI6eyJuYW1lIjoibm9uZSJ9LCJoZWFkIjp7Im5hbWUiOiJub25lIn19fV0sWzQsMCwiXFx5byJdLFs1LDEsIlxceW8iXV0=
\[\begin{tikzcd}
	U & {h_U} & G \\
	T & {h_T} & F
	\arrow["\yo", from=1-1, to=1-2]
	\arrow["\subseteq"{marking, allow upside down}, draw=none, from=1-1, to=2-1]
	\arrow[dashed, from=1-2, to=1-3]
	\arrow[dashed, from=1-2, to=2-2]
	\arrow["\ulcorner"{anchor=center, pos=0.125}, draw=none, from=1-2, to=2-3]
	\arrow[from=1-3, to=2-3]
	\arrow["\yo", from=2-1, to=2-2]
	\arrow[from=2-2, to=2-3]
\end{tikzcd}\]
\end{definition}

\begin{remark}
By the Yoneda lemma, giving a natural transformation like in the above definition is equivalent to choosing a family $\xi\in F(T)$. We can thus rephrase the definition as follows:\medskip

\noindent
A subfunctor $G\subseteq F$ is open if for any $S$-scheme $T$ and any family $\xi\in F(T)$ there exists an open subscheme $U\subseteq T$ such that the following diagram is natural in $R$, commutes and for any map $f:R\to U$ there exists a $g:R\to U$ such that $f=\subseteq \circ g$ if and only if $F(f)(\xi)\in G(R)$\footnote{the ``only if" is trivially true by commutativity but for the ``if" we are using the fact that $h_U\cong h_T\times_FG$.}
% https://q.uiver.app/#q=WzAsNCxbMCwwLCJcXE1vcihSLFUpIl0sWzAsMSwiXFxNb3IoUixUKSJdLFsyLDAsIkcoUikiXSxbMiwxLCJGKFIpIl0sWzAsMSwiXFxzdWJzZXRlcVxcY2lyYyIsMix7InN0eWxlIjp7InRhaWwiOnsibmFtZSI6Imhvb2siLCJzaWRlIjoidG9wIn19fV0sWzAsMiwiRyhcXHN1YnNldGVxIFxcY2lyYyB+flxcY2RvdH4pKFxceGkpIl0sWzEsMywiRihcXGNkb3QpKFxceGkpIiwyXSxbMiwzLCJcXHN1YnNldGVxIiwwLHsic3R5bGUiOnsidGFpbCI6eyJuYW1lIjoiaG9vayIsInNpZGUiOiJ0b3AifX19XV0=
\[\begin{tikzcd}
	{\Hom(R,U)} && {G(R)} \\
	{\Hom(R,T)} && {F(R)}
	\arrow["\subseteq\circ"', hook, from=1-1, to=2-1]
	\arrow["{G(\subseteq \circ ~~\cdot~)(\xi)}", from=1-1, to=1-3]
	\arrow["{F(\cdot)(\xi)}"', from=2-1, to=2-3]
	\arrow["\subseteq", hook, from=1-3, to=2-3]
\end{tikzcd}\]
\end{remark}

\begin{definition}[Open cover of a functor]
Let $F:\Psh S\Set$ be a moduli problem. A collection of open subfunctors $\cpa{F_i\to F}$ is an \textbf{open cover} of $F$ if for any $S$-scheme $T$ and any natural transformation $h_T\to F$, there exists an open cover $\cpa{U_i\to T}$ of $T$ such that $U_i$ represents $h_T\times_F F_i$ for all $i$.
\end{definition}
\begin{remark}
Like above, we can rephrase the definition as follows:\\
A collections of open subfunctors $F_i\subseteq F$ form an open cover of $F$ if for any $S$-scheme $T$ and any family $\xi\in F(T)$, there exists an open cover $\cpa{U_i\to T}$ of $T$ such that $\xi\res{U_i}\in F_i(U_i)$ for all $i$.
\end{remark}

\subsection{Representability criterion}
Finally, we come to the main result of this chapter

\begin{theorem}[Representability by open cover]\label{RepresentabilityByOpenSubfunctorCover}
Let $F:\Psh{S}{\Set}$ be a Zariski sheaf and let $\cpa{F_i\to F}$ be an open cover of it by representable subfunctors, then $F$ is representable.
\end{theorem}
\begin{proof}
For this proof we will mainly follow the version presented in \cite{GortzAlgGeoISchemes} (Theorem 8.9, page 212).\\
Let us fix $X_i$ schemes and $\xi_i\in F_i(X_i)$ such that $(X_i,\xi_i)$ is a fine moduli space for $F_i$.\\
For all $S$-schemes $T$ we have
\[(F_i\times_F F_j)(T)=F_i(T)\times_{F(T)}F_j(T)=F_i(T)\cap F_j(T)\subseteq F(T),\]
thus $F_i\times_F F_j=F_j\times_F F_i\doteqdot F_{i,j}$. We define $F_{i,j,k}$ analogously.
\medskip

\noindent Since $F_j$ is an open subfunctor of $F$, there exists an open subscheme $U_{ij}\subseteq X_i$ which represents $h_{X_i}\times_F F_j\cong F_i\times_F F_j=F_{i,j}$. We can define $U_{ji}\subseteq X_j$ similarly and since they are both moduli spaces for $F_{i,j}$ they are isomorphic. Let $\vp_{ji}:U_{ij}\to U_{ji}$ be the isomorphism given by $\vp_{ji}=\al_{U_{ij}}(id_{U_{ij}})$ for $\al$ natural isomorphism which makes the following diagram commute
% https://q.uiver.app/#q=WzAsNixbMCwwLCJoX3tVX3tpan19Il0sWzAsMSwiaF97VV97aml9fSJdLFsxLDAsImhfe1hfaX1cXHRpbWVzX0YgRl9qIl0sWzIsMCwiRl97aSxqfSJdLFsyLDEsIkZfe2ksan0iXSxbMSwxLCJGX2lcXHRpbWVzX0YgaF97WF9qfSJdLFsyLDMsIlxcY29uZyIsMyx7InN0eWxlIjp7ImJvZHkiOnsibmFtZSI6Im5vbmUifSwiaGVhZCI6eyJuYW1lIjoibm9uZSJ9fX1dLFs1LDQsIlxcY29uZyIsMyx7InN0eWxlIjp7ImJvZHkiOnsibmFtZSI6Im5vbmUifSwiaGVhZCI6eyJuYW1lIjoibm9uZSJ9fX1dLFszLDQsIj0iLDMseyJzdHlsZSI6eyJib2R5Ijp7Im5hbWUiOiJub25lIn0sImhlYWQiOnsibmFtZSI6Im5vbmUifX19XSxbMCwxLCJcXGFscGhhIiwyXSxbMCwyLCJcXGNvbmciLDMseyJzdHlsZSI6eyJib2R5Ijp7Im5hbWUiOiJub25lIn0sImhlYWQiOnsibmFtZSI6Im5vbmUifX19XSxbMSw1LCJcXGNvbmciLDMseyJzdHlsZSI6eyJib2R5Ijp7Im5hbWUiOiJub25lIn0sImhlYWQiOnsibmFtZSI6Im5vbmUifX19XV0=
\[\begin{tikzcd}
	{h_{U_{ij}}} & {h_{X_i}\times_F F_j} & {F_{i,j}} \\
	{h_{U_{ji}}} & {F_i\times_F h_{X_j}} & {F_{i,j}}
	\arrow["\cong"{marking, allow upside down}, draw=none, from=1-2, to=1-3]
	\arrow["\cong"{marking, allow upside down}, draw=none, from=2-2, to=2-3]
	\arrow["{=}"{marking, allow upside down}, draw=none, from=1-3, to=2-3]
	\arrow["\alpha"', from=1-1, to=2-1]
	\arrow["\cong"{marking, allow upside down}, draw=none, from=1-1, to=1-2]
	\arrow["\cong"{marking, allow upside down}, draw=none, from=2-1, to=2-2]
\end{tikzcd}\]
Note that if $T$ is an $S$-scheme and $f\in h_{U_{ij}}(T)$ then
\[h_{\vp_{ji}}(f)=\al_{U_{ij}}(id_{U_{ij}})\circ f =\al_T(id_{U_{ij}}\circ f)=\al_T(f),\]
so $\al$ is the image of $\vp_{ji}$ under the Yoneda embedding.
\medskip

\noindent We want to show that the $X_i$ can be glued along the $U_{ij}$ using the isomorphisms $\vp_{ji}$. First we need to show that $\vp_{ji}(U_{ij}\cap U_{ik})=U_{ji}\cap U_{jk}$ and then we have to verify the cocycle condition $\vp_{ki}=\vp_{kj}\circ \vp_{ji}$.

The first condition follows immediately from the fact that $F_k$ is an open subfunctor and our construction of the $\vp_{ji}$.

Since the Yoneda embedding preserves limits (\ref{YonedaEmbeddingPreservesLimits}) it preserves fibered products, so we see that the following diagram commutes
% https://q.uiver.app/#q=WzAsOCxbMSwwLCJoX3tVX3tpan19XFx0aW1lc197aF97WF9pfX1oX3tVX3tpa319Il0sWzEsMSwiaF97VV97aml9fVxcdGltZXNfe2hfe1hfan19aF97VV97amt9fSJdLFsyLDAsIkZfe2ksan1cXHRpbWVzX3tGX2l9IEZfe2ksa30iXSxbMywwLCJGX3tpLGosa30iXSxbMywxLCJGX3tpLGosa30iXSxbMiwxLCJGX3tpLGp9XFx0aW1lc197Rl9qfSBGX3tqLGt9Il0sWzAsMCwiaF97VV97aWp9XFxjYXAgVV97aWt9fSJdLFswLDEsImhfe1Vfe2ppfVxcY2FwIFVfe2prfX0iXSxbMiwzLCJcXGNvbmciLDMseyJzdHlsZSI6eyJib2R5Ijp7Im5hbWUiOiJub25lIn0sImhlYWQiOnsibmFtZSI6Im5vbmUifX19XSxbNSw0LCJcXGNvbmciLDMseyJzdHlsZSI6eyJib2R5Ijp7Im5hbWUiOiJub25lIn0sImhlYWQiOnsibmFtZSI6Im5vbmUifX19XSxbMyw0LCI9IiwzLHsic3R5bGUiOnsiYm9keSI6eyJuYW1lIjoibm9uZSJ9LCJoZWFkIjp7Im5hbWUiOiJub25lIn19fV0sWzAsMiwiXFxjb25nIiwzLHsic3R5bGUiOnsiYm9keSI6eyJuYW1lIjoibm9uZSJ9LCJoZWFkIjp7Im5hbWUiOiJub25lIn19fV0sWzEsNSwiXFxjb25nIiwzLHsic3R5bGUiOnsiYm9keSI6eyJuYW1lIjoibm9uZSJ9LCJoZWFkIjp7Im5hbWUiOiJub25lIn19fV0sWzYsMCwi44KIIl0sWzcsMSwi44KIIl0sWzYsNywiXFx2cF97aml9IiwyXV0=
\[\begin{tikzcd}
	{h_{U_{ij}\cap U_{ik}}} & {h_{U_{ij}}\times_{h_{X_i}}h_{U_{ik}}} & {F_{i,j}\times_{F_i} F_{i,k}} & {F_{i,j,k}} \\
	{h_{U_{ji}\cap U_{jk}}} & {h_{U_{ji}}\times_{h_{X_j}}h_{U_{jk}}} & {F_{i,j}\times_{F_j} F_{j,k}} & {F_{i,j,k}}
	\arrow["\cong"{marking, allow upside down}, draw=none, from=1-3, to=1-4]
	\arrow["\cong"{marking, allow upside down}, draw=none, from=2-3, to=2-4]
	\arrow["{=}"{marking, allow upside down}, draw=none, from=1-4, to=2-4]
	\arrow["\cong"{marking, allow upside down}, draw=none, from=1-2, to=1-3]
	\arrow["\cong"{marking, allow upside down}, draw=none, from=2-2, to=2-3]
	\arrow["{\yo}", from=1-1, to=1-2]
	\arrow["{\yo}", from=2-1, to=2-2]
	\arrow["{\vp_{ji}}"', from=1-1, to=2-1]
\end{tikzcd}\]
therefore, to prove that $\vp_{kj}\circ \vp_{ji}=\vp_{ki}$ it is enough to verify the trivial equality \[id_{F_{i,j,k}}\circ id_{F_{i,j,k}}=id_{F_{i,j,k}}.\]

\noindent
Let $X$ to be the scheme obtained by gluing the $X_i$ as indicated above.\\
Note that $\xi_i=\vp_{ji}^\ast\xi_j$, so if we look at these families as elements of $F(X)$ we see that $\xi_i\res{U_ij}=\xi_j\res{U_{ij}}$. Since $F$ is a Zariski sheaf, the $\xi_i$ can be glued to a family $\xi\in F(X)$.\medskip

\noindent To finish the proof we need to verify that $(X,\xi)$ is a fine moduli space for $F$:\\
Let $T$ be an $S$-scheme and and let us consider a family $\zeta\in F(T)$. Since $\cpa{F_i\to F}$ is an open cover of $F$, there exists an open cover $\cpa{V_i\to T}$ of $T$ such that $\zeta\res{V_i}\in F_i(V_i)\cong \Hom(V_i,X_i)$. Since $F$ is a sheaf and $\zeta_i\res{V_i\cap V_j}=\zeta_j\res{V_i\cap V_j}$, the morphisms $V_i\to X_i$ corresponding to the $\zeta_i$ glue to a morphism $f:T\to X$ such that $f^\ast\xi=\zeta$ (by construction).
\end{proof}
\begin{remark}
Adopting the notation of the proof we see that if we have a candidate fine moduli space $X$, an open cover $\cpa{X_i\to X}$ such that $X_i$ represents $F_i$ and we can verify that $\vp_{ji}:X_i\cap X_j\to X_i\cap X_j$ is the identity for all $i,j$, then $X$ represents $F$.
\end{remark}
\chapter{Grassmannians as projective varieties}
In this chapter we introduce Grassmannians from the point of view of classical algebraic geometry. We are interested in Grassmannians in the context of classification problems because their definition leads us to suspect that they are a moduli space for certain families of vector spaces. In the next chapter we will indeed find that they are fine moduli spaces for a functor that formalizes \textit{families of $k$-vector subspaces of $\K^n$}.\medskip

We first define Grassmannians set-theoretically, then we will find a bijection between this set and Zariski-closed subset of some projective space. This bijection will allow us to endow the Grassmannian with the structure of a projective algebraic variety.

\section{First definitions and conventions}
\begin{definition}[Grassmannian]
Let $k\leq n$ be a pair of positive integers. We define the \textbf{$(n,k)$-Grassmannian}, denoted\footnote{the field will be omitted when clear from context} $\Gr(k,n,\K)$, as the set of $(n-k)$-dimensional $\K$-vector subspaces of $\K^n$.
\end{definition}

\begin{remark}
We may equivalently define $\Gr(k,n)$ to be the following set:
\[\cpa{\ker \vp\mid \vp\in \Hom_\K(\K^n,\K^k),\ \rnk \vp=k}.\]
\end{remark}
%\begin{proof}
%If $H\in \Gr(k,n)$, let $z_1,\cdots, z_n$ be a basis of $\K^n$ such that $z_1,\cdots, z_{n-k}$ is a basis of $H$ and let $e_1,\cdots, e_k$ be any basis of $\K^k$. We can view $H$ as the kernel of the (rank $k$) linear map given by
%\[\vp(z_i)=\begin{cases}
%0 &\text{if }i\leq n-k\\
%e_{i-n+k} &\text{otherwise}
%\end{cases}\]
%On the other hand, if $\vp$ is a rank $k$ linear map then, by the rank-nullity theorem, its kernel is an $n-k$ dimensional subspace of $\K^n$.
%\end{proof}

\begin{lemma}\label{kerAkerBVSActionOfGLk}
Let $\vp,\psi\in \Hom_\K(\K^n,\K^k)$ be linear maps of full rank. The following conditions are equivalent:
\begin{enumerate}
    \item $\ker\vp=\ker\psi$,
    \item there exists $\theta\in \GL(\K^k)$ such that $\vp=\theta\circ \psi$. 
\end{enumerate}
\end{lemma}
\begin{proof}
The implication $2.\implies 1.$ is a straight forward computation, the other can be derived by completing a basis of $H$ to a basis $\Bc$ of $\K^n$ and defining $\theta$ to be the change of basis between the images of $\Bc$ under $\vp$ and $\psi$.
%Let us prove both implications:
%\setlength{\leftmargini}{0cm}
%\begin{itemize}
%\item[$\boxed{2.\implies 1.}$] $\ker \vp=\ker(\theta\circ \psi)=\psi\ii(\ker \theta)=\psi\ii(\cpa{0})=\ker \psi$. 
%\item[$\boxed{1.\implies 2.}$] Let $z_1,\cdots, z_n$ be a basis of $\K^n$ such that $\ker\vp=\ker\psi=\Span(z_1,\cdots, z_{n-k})$. By construction $\vp(z_{n-k+1}),\cdots, \vp(z_n)$ and $\psi(z_{n-k+1}),\cdots, \psi(z_n)$ are bases of $\K^k$. 
%Let $\theta$ be the change of basis on $\K^k$ determined by $\theta(\psi(z_i))=\vp(z_i)$ for all $n-k<i\leq n$. By construction $\theta$ is nonsingular and $\vp$ agrees with $\theta\circ \psi$ on a basis of $\K^n$.
%\end{itemize}
%\setlength{\leftmargini}{0.5cm}
\end{proof}

\begin{corollary}\label{LinearQuotientDefinition}
We may redefine Grassmannians in terms of linear maps as follows:
\[\Gr(k,n)=\quot{\cpa{\vp\in \Hom_\K(\K^n,\K^k)\mid \vp\ \text{surjective.}}}\sim\]
where $\vp\sim \psi$ if and only if $\exists \theta\in \GL(\K^k)$ such that $\vp=\theta\circ \psi$.
\end{corollary}
\bigskip

\noindent We conclude this introductory section with some notation and conventions.
\begin{definition}[Multiindicies]
We define a \textbf{$(k,n)$-multiindex} as an element of $\cpa{1,\cdots, n}^k$. Our notation for a multiindex $I$ will usually be $I=(i_1,\cdots, i_k)$.\\
We denote the set of \textbf{ordered $(k,n)$-multiindicies} with
\[\omega(k,n)=\cpa{(i_1,\cdots, i_k)\in \cpa{1,\cdots, n}^k\mid i_1<\cdots<i_k}.\]
If $I\in \omega(k,n)$, we write\footnote{$\cup$ and $\ast$ denote the union of the underlying sets and concatenation respectively.}
\begin{itemize}
\item $\wh I$ for the element of $\omega(n-k,n)$ such that $I\cup \wh I=\cpa{1,\cdots, n}$ and 
\item $\sigma_I$ for the permutation that sends $\wh I\ast I$ to $\pa{1,\cdots, n}$.
\end{itemize}
If $A$ is a $k\times n$ matrix and $I$ is a $(k,n)$-multiindex, we denote the \textbf{$I$-minor of $A$} by $A_I$, i.e.
\[A_I=\mat{a_{1,i_1} &\cdots &a_{1,i_{k}}\\\vdots &\ddots &\vdots\\a_{k,i_1}&\cdots&a_{k,i_k}}.\]
If $B$ is an $\al\times \beta$ matrix, $i\in\cpa{1,\cdots, \al}$ and $j\in \cpa{1,\cdots, \beta}$ we denote the $(\al-1)\times (\beta-1)$ matrix obtained from $B$ by deleting the $i$-th row and the $j$-th column with $B_{\times i,\times j}$.
\end{definition}

\begin{remark}
If $I=(i_1,\cdots, i_k)$ is a $(k,n)$-multiindex and $\Bc=\cpa{e_1,\cdots,e_n}$ is a basis of $\K^n$ we define
\[e_I=e_{i_1}\wedge\cdots\wedge e_{i_k}.\]
Note that
\[\cpa{e_{i_1}\wedge\cdots\wedge e_{i_k}\mid 1\leq i_1<\cdots<i_k\leq n}=\cpa{e_I\mid I\in \omega(k,n)}\]
forms a basis for $\bigwedge^k\K^n$, which we call the \textbf{basis induced by $\Bc$} and denote with $\wedge^k\Bc$ or simply $\Bc$ with a slight abuse of notation.
\end{remark}

\begin{notation}
Whenever a basis $\Bc$ of $\K^\ell$ is fixed, we will identify $\bigwedge^\ell\K^\ell$ with $\K$ by sending only element of $\wedge^\ell\Bc$ to $1\in\K$. This isomorphism is denoted $\eta_\Bc:\bigwedge^\ell\K^\ell\to \K$. If $\Dc$ is a basis of $\K^m$ then we define $\eta^\Bc_\Dc=\eta_\Dc\ii\circ \eta_\Bc$.
\end{notation}

\begin{remark}[Matrix form for the Grassmannian]
If we fix bases $e_1,\cdots, e_n$ of $\K^n$ and $e_1,\cdots, e_k$ of $\K^k$, then we can identify $\Hom_\K(\K^n,\K^k)$ with the set of $k\times n$ matrices with coefficients in $\K$. As a consequence of this we find yet another form for $\Gr(k,n)$:
\[\Gr(k,n)=\quot{\cpa{A\in \Mc(k,n)\mid \rnk A=k}}{\sim},\] 
where $A\sim B\coimplies \exists P\in \GL(k)\ s.t.\ A=PB$.
\end{remark}

\section{The Pl\"ucker embedding}
In this section we define an injection from the Grassmannian to a projective space.
The idea behind this map is to take appropriate wedge products in such a way as to trasform the several vectors defining a vector subspace into a single vector and then to projectivize. 
Our approach differs slightly from the usual one\footnote{briefly illustrated in \cite{matroids}, pages 79 and 80} because we consider equivalence classes of maps rather than equivalence classes of bases.

\begin{definition}[Pl\"ucker map]
Let $k\leq n$ be a pair of positive integers. We define the \textbf{Pl\"ucker map} as\footnote{the map $\wedge^k\vp$ is well defined because if we view it as a map $\wedge^k\vp:(\K^n)^k\to \bigwedge^k\K^k$ then it is multilinear and alternating.}
\[\wedge^k:\funcDef{\Hom_\K(\K^n,\K^k)}{\Hom_\K(\bigwedge^k\K^n,\bigwedge^k\K^k)}{\vp}{\wedge^k\vp},\]
where $(\wedge^k\vp)(v_1\wedge\cdots\wedge v_k)=\vp(v_1)\wedge\cdots\wedge\vp(v_k).$
\end{definition}

\begin{remark}
If $\Bc=\cpa{v_1,\cdots, v_k}$ is a basis of $\K^k$, $\Can=\cpa{e_1,\cdots,e_k}$ is the canonical basis and $[\cdot]_\Bc:\K^k\to\K^k$ is the isomorphism which sends $v_i$ to $e_i$ then
\[\wedge^k(\vp)(v_1\wedge\cdots\wedge v_k)=\det\mat{[\vp(v_1)]_\Bc|\cdots|[\vp(v_k)]_\Bc} e_1\wedge\cdots\wedge e_k.\]
\end{remark}

\begin{remark}\label{CodomainOfPluckerMap}
The codomain of the Pl\"ucker map is isomorphic to $\bigwedge^k\K^n$, indeed
\[\Hom_\K\pa{\bigwedge^k\K^n,\bigwedge^k\K^k}\cong \pa{\bigwedge^k\K^n}^{\vee}\cong \bigwedge^k\K^n.\]
If $\Bc=\cpa{e_1,\cdots, e_n}$ is a basis of $\K^n$ and $\Dc=\cpa{e_1,\cdots, e_k}$ is a basis of $\K^k$ then we can write one such isomorphism concretely as
\[\zeta_{\Bc,\Dc}:\funcDef{\Hom_\K(\bigwedge^k\K^n,\bigwedge^k\K^k)}{\bigwedge^k\K^n}{\psi}{\displaystyle\sum_{I\in\omega(k,n)} \eta_\Dc(\psi(e_I))e_I}.\]
When the bases are fixed we simply write $\zeta$.
\end{remark}

\begin{notation}
If bases are fixed we define $\phi\doteqdot \zeta\circ \wedge^k$.
\end{notation}

\begin{remark}[Matrix form of the Pl\"ucker map]
If we fix bases $e_1,\cdots, e_n$ of $\K^n$ and $e_1,\cdots, e_k$ of $\K^k$ then, up to identifying $\Hom_\K(\K^n,\K^k)$ with $\Mc(k,n)$, we have
\[\phi:\funcDef{\Mc(k,n)}{\bigwedge^k\K^n}{A}{\sum_{I\in \omega(k,n)}\det A_I e_I}\]
\end{remark}


\begin{proposition}\label{ImagePluckerMapIsCone}
The image of the Pl\"ucker map is a cone.
\end{proposition}
\begin{proof}
We have $\la\wedge^k\vp=\wedge^k(\al\circ \vp)$ for any $\al\in \GL(\K^k)$ with determinant $\la$. 
%For any $\la\in \K^\ast$ and any map $\vp\in \Hom_\K(\K^n,\K^k)$ note that
%\[\la\wedge^k(\vp)=\wedge^k(\al\circ \vp)\]
%for any automorphism $\al$ of $\K^k$ with determinant $\la$. We can construct one such $\al$ by fixing a basis of $\K^k$ and defining $\al$ to be the map corresponding to the matrix
%\[\mat{
%\la &   &        &\\
%    & 1 &        &\\
%	&   & \ddots &\\
%	&   &        &  1
%}\]
\end{proof}


\begin{lemma}\label{WhenPluckerMapIsZero}
If $\vp\in\Hom_\K(\K^n,\K^k)$ then $\rnk \vp<k$ if and only if $\wedge^k(\vp)=0$.
\end{lemma}
\begin{proof}
$\wedge^k(\vp)$ is the zero map if an only if the set $\cpa{\vp(v_1),\cdots, \vp(v_k)}$ is linearly dependent for any choice of $v_1,\cdots, v_k$, i.e. $\vp$ is not of full rank.
\end{proof}

\begin{lemma}\label{CharacterizationOfKernels}
Let $\vp:\K^n\to \K^k$ be a full rank linear map, then
\[\ker\vp=\cpa{z\in\K^n\mid \forall w_2,\cdots, w_k\in \K^n,\ \wedge^k(\vp)(z\wedge w_2\wedge\cdots\wedge w_k)=0}.\]
\end{lemma}
\begin{proof}
The inclusion $\subseteq$ is trivial. If $\vp(z)\neq 0$ we can find $k-1$ vectors of the desired form by completing $\vp(z)$ to a basis $\vp(z),v_2,\cdots,v_k$ of $\K^k$ and then taking $w_i$ to be any element of $\vp\ii(v_i)$. The set is not empty by surjectivity of $\vp$.
%If $\vp(z)=0$ then for any $w_2,\cdots, w_k\in \K^k$ we see that 
%\[\wedge^k(\vp)(z\wedge w_2\wedge\cdots\wedge w_k)=0\wedge \vp(w_2)\wedge\cdots\wedge \vp(w_k)=0.\]
%Suppose now that $\vp(z)\neq 0$ and let $v_2,\cdots, v_k$ be such that $\cpa{\vp(z), v_2,\cdots,v_k}$ forms a basis for $\K^k$. Since $\vp$ is surjective, there exist $w_2,\cdots, w_k$ such that $\vp(w_i)=v_i$ for all $2\leq i\leq k$.
%By construction 
%\[\wedge^k(\vp)(z\wedge w_2\wedge\cdots\wedge w_k)=\vp(z)\wedge v_2\wedge\cdots\wedge v_k\neq0.\]
\end{proof}

\begin{proposition}[Injectivity of the Pl\"ucker map up to scalars]\label{PluckerMapInjectiveOnGrassmanniansUpToScalar}
Let $\sim$ be the equivalence relation defined in corollary (\ref{LinearQuotientDefinition}), then for any two full rank linear maps $\vp,\psi:\K^n\to \K^k$
\[\vp\sim \psi\coimplies \exists \la\in\K^\ast\ s.t.\ \wedge^k(\vp)=\la\wedge^k(\psi).\]
\end{proposition}
\begin{proof}
We prove both implications:
\setlength{\leftmargini}{0cm}
\begin{itemize}
\item[$\boxed{\implies}$] If $\vp=\theta\circ \psi$ for $\theta\in \GL(\K^k)$ then
\[\wedge^k(\vp)=\wedge^k(\theta\circ \psi)=(\det\theta) \wedge^k(\psi).\]
\item[$\boxed{\impliedby}$] From lemma (\ref{kerAkerBVSActionOfGLk}) we see that it is enough to prove that $\ker \vp=\ker \psi$. We conclude by applying lemma (\ref{CharacterizationOfKernels}) as follows:
\begin{align*}
\ker\vp=&\cpa{z\in\K^n\mid \forall w_2,\cdots, w_k\in \K^n,\ \wedge^k(\vp)(z\wedge w_2\wedge \cdots\wedge w_k)=0}=\\
=&\cpa{z\in\K^n\mid \forall w_2,\cdots, w_k\in \K^n,\ \la\wedge^k(\psi)(z\wedge w_2\wedge\cdots\wedge w_k)=0}=\\
=&\cpa{z\in\K^n\mid \forall w_2,\cdots, w_k\in \K^n,\ \wedge^k(\psi)(z\wedge w_2\wedge \cdots\wedge w_k)=0}=\ker \psi.
\end{align*}
\end{itemize}
\setlength{\leftmargini}{0.5cm}
\end{proof}

\begin{remark}
Because of proposition (\ref{PluckerMapInjectiveOnGrassmanniansUpToScalar}) and lemma (\ref{WhenPluckerMapIsZero}), there exists a unique $h$ such that the diagram commutes
% https://q.uiver.app/#q=WzAsMyxbMCwwLCJcXGNwYXtcXHZwXFxpbiBcXEhvbV9cXEsoXFxLXm4sXFxLXmspXFxtaWQgXFxybmsgXFx2cD1rfSJdLFswLDEsIlxcR3IoayxuKSJdLFsxLDAsIlxcUGooXFxiaWd3ZWRnZV5rXFxIb21fXFxLKFxcS15uLFxcS15rKSkiXSxbMCwxLCJcXHBpIl0sWzAsMiwiW1xccGhpXSJdLFsxLDIsImgiLDIseyJzdHlsZSI6eyJib2R5Ijp7Im5hbWUiOiJkYXNoZWQifX19XV0=
\[\begin{tikzcd}
	{\cpa{\vp\in \Hom_\K(\K^n,\K^k)\mid \rnk \vp=k}} & {\Pj(\Hom_\K(\bigwedge^k\K^n,\bigwedge^k\K^k))} \\
	{\Gr(k,n)}
	\arrow["\pi_\sim", from=1-1, to=2-1]
	\arrow["{[\wedge^k]}", from=1-1, to=1-2]
	\arrow["h"', dashed, from=2-1, to=1-2]
\end{tikzcd}\]
Moreover, such an $h$ must be injective by proposition (\ref{PluckerMapInjectiveOnGrassmanniansUpToScalar}).
\end{remark}


\begin{definition}[Pl\"ucker embedding]
Let us fix a basis $e_1,\cdots, e_n$ of $\K^n$ and a basis $e_1,\cdots, e_k$ of $\K^k$. We define the \textbf{Pl\"ucker embedding} as follows\footnote{we can omit the basis in which we calulated the determinats because we will soon see that the resulting point in $\Pj(\bigwedge^k\K^n)$ does not depend on this choice.}
\[\Pl:\funcDef{\Gr(k,n)}{\Pj(\bigwedge^k\K^n)}{[\vp]}{\displaystyle\spa{\sum_{1\leq i_1<\cdots<i_k\leq n}\det(\vp(e_{i_1})\mid\cdots\mid\vp(e_{i_k}))e_{i_1}\wedge\cdots\wedge e_{i_k}}}\]
The entries of the homogeneous $\binom nk$-tuple associated to $[\vp]\in \Gr(k,n)$ are called the \textbf{Pl\"ucker coordinates} of $[\vp]$. 
\end{definition}

\begin{remark}[Well defined and injective]
If we fix bases for $\K^n$ and $\K^k$ and $\zeta$ is the isomorphism $\Hom_\K(\bigwedge^k\K^n,\bigwedge^k\K^k)\to \bigwedge^k\K^n$ discussed during remark (\ref{CodomainOfPluckerMap}), we see that the following diagram commutes
% https://q.uiver.app/#q=WzAsMyxbMCwxLCJcXEdyKGssbikiXSxbMCwwLCJcXFBqKFxcYmlnd2VkZ2Vea1xcSG9tX1xcSyhcXEtebixcXEteaykpIl0sWzEsMCwiXFxQaihcXGJpZ3dlZGdlXmtcXEtebikiXSxbMCwxLCJoIiwyXSxbMSwyLCJcXFBqKFxcemV0YSkiXSxbMCwyLCJcXG1hdGhybXtQbH0iLDJdXQ==
\[\begin{tikzcd}
	{\Pj(\Hom_\K(\bigwedge^k\K^n,\bigwedge^k\K^k))} & {\Pj(\bigwedge^k\K^n)} \\
	{\Gr(k,n)}
	\arrow["h"', from=2-1, to=1-1]
	\arrow["{\Pj(\zeta)}", from=1-1, to=1-2]
	\arrow["{\Pl}"', from=2-1, to=1-2]
\end{tikzcd}\]
This proves that the Pl\"ucker embedding is well defined and injective.
\end{remark}

\begin{remark}
$\Pl\circ \pi_\sim=\Pj(\zeta\circ\wedge^k)=\Pj(\phi)$.
\end{remark}


\begin{remark}
The Pl\"ucker embedding depends on the choice of basis for $\K^n$ but not on the one for $\K^k$. 

Changing the basis of $\K^k$ simply multiplies all Pl\"ucker coordinates by the same nonzero scalar (the determinant of the change of basis), which does not change the point they describe in $\Pj(\bigwedge^k\K^k)$.

The dependence on the basis of $\K^n$ is inevitable because $\GL(\K^n)$ acts transitively on $\Gr(k,n)$ viewed as the set of $(n-k)$-dimentional subspaces of $\K^n$.
\end{remark}

\begin{remark}[Matrix form of the Pl\"ucker embedding]
If we fix a basis $e_1,\cdots, e_n$ of $\K^n$ and identify $\Hom_\K(\K^n,\K^k)$ with $\Mc(k,n)$  then
\[\Pl:\funcDef{\Gr(k,n)}{\Pj(\bigwedge^k\K^n)}{[A]_\sim}{\spa{\sum_{I\in \omega(k,n)}\det A_I e_I}_{\K^\ast}}\]
\end{remark}



\section{The image of the Pl\"ucker embedding is closed}\label{ImagePluckerEmbeddingIsClosed}
Thus far we have identified $\Gr(k,n)$ with a subset of some projective space. We seek to show that this subset is closed in the Zariski topology.

\subsection{Some linear algebra results}
\begin{definition}[Divisibility]
We say that $\omega\in \bigwedge^k\K^n$ is \textbf{divisible} by $v\in\K^n$ if there exists $\e\in \bigwedge^{k-1}\K^n$ such that $\omega=\e\wedge v$.
\end{definition}
\begin{lemma}\label{Divisibility}
Let $\omega\in \bigwedge^k\K^n$. For any given nonzero vector $v$, $\omega$ is divisible by $v$ if and only if $\omega\wedge v=0$.
\end{lemma}
\begin{proof}
If $\omega=\e\wedge v$ then $\omega\wedge v=\e\wedge v\wedge v=0$. If $\omega\wedge v=0$ then by writing $\omega$ in a base containing $v$ we can see that the simple multivectors with nonzero coefficients must contain $v$ as a factor, so can factor out $v$ by multilinearity and get a decomposition of the form $\omega=\e\wedge v$.
%If $\omega=\e\wedge v$ then $\omega\wedge v=\e\wedge v\wedge v=0$.\\
%Suppose now that $\omega\wedge v=0$. Let $v_1,\cdots, v_n$ be a basis of $\K^n$ such that $v_1=v$. 
%If we write
%\[\omega=\sum_{I\in\omega(k,n)}p_I v_I\]
%then we see that for any given multiindex $I$, either $p_I=0$ or $v_I\wedge v=0$. 
%Since $v_1,\cdots, v_n$ is a basis, $v_I\wedge v_1=0$ if and only if $1\in I$, i.e. $v_I=v\wedge v_{\pa{i_2,\cdots, i_k}}$, therefore
%\[\omega=v\wedge \under{\doteqdot (-1)^{k-1}\e}{\pa{\sum_{2\leq i_2<\cdots i_k\leq n} p_{\pa{1,i_2,\cdots, i_k}} e_{\pa{i_2,\cdots, i_k}}}}=\e\wedge v.\]
\end{proof}

\begin{corollary}[Total decomposibility criterion]\label{TotalDecomposibilityCriterion}
Let $\omega\in \bigwedge^k\K^n$ and define 
\[D_\omega=\cpa{v\in\K^n\mid \omega\wedge v=0}.\]
If $\dim D_\omega\geq k$ then $\omega=\la v_1\wedge \cdots\wedge v_k$ for any set of linearly independent vectors $\cpa{v_1,\cdots, v_k}$ in $D_\omega$ and some scalar $\la$. 
Moreover $\la\neq 0$ if and only if $\dim D_\omega= k$.
\end{corollary}
\begin{proof}
For the first part of the result we may just iterate the above lemma. If $\la=0$ then $D_\omega=\K^n$, so its dimention is not $k$. If the dimention is greater than $k$ then we may subtract two total decompositions differing only by one vector and use linear independence to check that the coefficients must have been zero.
%The set $D_\omega$ is clearly a subspace of $\K^n$. Let $\cpa{v_1,\cdots, v_k}$ be linearly independent vectors of this space. By iterating the above lemma we see that
%\[\omega=\la\wedge v_1\wedge\cdots\wedge v_k\]
%for some $\la\in \bigwedge^0\K^n=\K$.\\
%If $\la=0$ then clearly $D_\omega=\K^n$. If $v_{k+1}$ is such that $\omega\wedge v_{k+1}=0$ and $\cpa{v_1,\cdots, v_{k+1}}$ is linearly independent then, proceding as above,
%\[\la v_1\wedge\cdots\wedge v_{k}=\omega=\mu v_1\wedge\cdots\wedge v_{k-1}\wedge v_{k+1}.\]
%By multilinearity
%\[0=v_1\wedge\cdots\wedge v_{k-1}\wedge(\la v_k-\mu v_{k+1}),\]
%i.e. $\la v_k-\mu v_{k+1}\in \Span(v_1,\cdots, v_{k-1})$. By linear independence $\la v_k-\mu v_{k+1}=0$ and thus, again by linear independence, $\la=\mu=0$.
\end{proof}


\begin{proposition}\label{CanonicalIso}
There is a canonical isomorphism between $\Hom_\K(\bigwedge^k\K^n,\bigwedge^n\K^n)$ and $\bigwedge^{n-k}\K^n$ given by
\[\Xi:\funcDef{\bigwedge^{n-k}\K^n}{\Hom_\K(\bigwedge^k\K^n, \bigwedge^n\K^n)}{\omega}{\omega\wedge \cdot}.\]
For any basis $\Bc=\cpa{e_1,\cdots, e_n}$ of $\K^n$ the map 
\[\Gamma_\Bc:\funcDef{\Hom_\K(\bigwedge^k\K^n,\bigwedge^n\K^n)}{\bigwedge^{n-k}\K^n}{\psi}{\sum_{I\in\omega(n-k,n)}\sgn\sigma_I\eta_\Bc(\psi(e_{\wh I}))e_I}\]
is the inverse of $\Xi$.
\end{proposition}
\begin{proof}
The map is clearly base independent and linear. Let $\wedge^k\Bc$ be the basis induced on $\bigwedge^k\K^n$ by $\Bc$. Concluding from here is simply a matter of computing $\Gamma_\Bc(\Xi(\omega))$ by writing $\omega$ in terms of its coordinates in $\wedge^k\Bc$ and verifying that $\Xi(\Gamma_\Bc(\psi))$ and $\psi$ agree on $\wedge^k\Bc$. 
%The map is clearly base independent and linear.\\
%For all $e_J$
%\begin{align*}
%\Xi(\Gamma_\Bc(\psi))(e_J)=&\sum_{I\in\omega(n-k,n)}\sgn\sigma_I\eta_\Bc(\psi(e_{\wh I}))e_I\wedge e_J=\\
%=&\sgn\sigma_{\wh J}\eta_\Bc(\psi(e_{J}))e_{\wh J}\wedge e_J=\\
%=&\eta_\Bc(\psi(e_{J}))e_{(1,\cdots,n)}=\\
%=&\psi(e_{J}),
%\end{align*}
%so $\Xi(\Gamma_\Bc(\psi))$ agrees with $\psi$ on a basis.\medskip

%\noindent
%If $\omega=\sum_{J\in\omega(n-k,n)} p_J e_J$ then
%\begin{align*}
%\sgn\sigma_I\eta_\Bc(\omega\wedge e_{\wh I})=&\sum_{J\in\omega(n-k,n)}p_J\sgn\sigma_I\eta_\Bc(e_J\wedge e_{\wh I})=\\
%=&p_I\eta_\Bc(\sgn \sigma_I e_I\wedge e_{\wh I})=\\
%=&p_I\eta_\Bc(e_{(1,\cdots, n)})=\\
%=&p_I,
%\end{align*}
%thus \[\Gamma_\Bc(\Xi(\omega))=\sum_{I\in\omega(n-k,n)}\sgn\sigma_I\eta_\Bc(\omega\wedge e_{\wh I})e_I=\sum_{I\in\omega(n-k,n)}p_Ie_I=\omega.\]
\end{proof}
\begin{corollary}\label{UpToScalarCanonicalIso}
Let $\psi\in \Hom_\K(\bigwedge^k\K^n,\bigwedge^k\K^k)$. If $\Bc=\cpa{e_1,\cdots, e_n}$ and $\Bc'=\cpa{e_1',\cdots, e_n'}$ are two bases for $\K^n$ and $\Dc=\cpa{e_1,\cdots,e_k}$ and $\Dc'=\cpa{e_1',\cdots,e_k'}$ are bases for $\K^k$, there exists $\mu\in\K\nz$ such that
\[\sum_{I\in\omega(n-k,n)}\sgn\sigma_I\eta_\Dc(\psi(e_{\wh I}))e_I=\mu \sum_{I\in\omega(n-k,n)}\sgn\sigma_I\eta_{\Dc'}(\psi(e'_{\wh I}))e'_I.\]
\end{corollary}
\begin{proof}
Note that
\[\sum_{I\in\omega(n-k,n)}\sgn\sigma_I\eta_\Dc(\psi(e_{\wh I}))e_I=\Xi\ii(\eta^\Dc_\Bc\circ \psi)\]
and similarly the other expression is $\Xi\ii(\eta^{\Dc'}_{\Bc'}\circ \psi)$. 
It is therefore enough to show that $\eta^\Bc_\Dc=\mu\eta^{\Bc'}_{\Dc'}$ for some $\mu\in \K\nz$, which is true because $\dim_\K\Hom_\K(\bigwedge^n\K^n,\bigwedge^k\K^k)=1$ and both $\eta^{\Bc}_{\Dc}$ and $\eta^{\Bc'}_{\Dc'}$ are not the zero map.
\end{proof}

\subsection{Rank condition for the image}
\begin{lemma}\label{DecomposabilityOfMultilinearForm}
Fix bases $\Bc=\cpa{e_1,\cdots, e_n}$ of $\K^n$ and $\Dc=\cpa{e_1,\cdots, e_k}$ of $\K^k$. A multilinear alternating form $\psi\in \Hom_\K(\bigwedge^k\K^n,\bigwedge^k\K^k)$ is in the image of the Pl\"ucker map $\wedge^k$ if and only if there exists $\la\in\K$ and linearly independent vectors $z_1,\cdots,z_{n-k}$ such that
\[\sum_{I\in \omega(n-k,n)}\sgn\sigma_I\eta_\Dc(\psi(e_{\wh I}))e_I=\la z_{(1,\cdots, n-k)}.\]
\end{lemma}
\begin{proof}
We show both implications
\setlength{\leftmargini}{0cm}
\begin{itemize}
\item[$\boxed{\implies}$] If $\psi=\wedge^k\vp$, the equality follows by choosing $z_1,\cdots, z_{n-k}$ to be a basis of $\ker\vp$. Completing this set to a basis of $\K^n$ and using corollary (\ref{UpToScalarCanonicalIso}) gives the result after a simple calculation.
\item[$\boxed{\impliedby}$] Let $\Zc=\cpa{z_1,\cdots, z_n}$ be a basis of $\K^n$ which extends the given $z_1,\cdots, z_{n-k}$. We can take $\vp$ to be
\[\vp(z_i)=\begin{cases}
0 & \text{if }1\leq i\leq n-k\\
\pa{\mu\la\sgn\sigma_{(1,\cdots,n-k)}} e_1 & \text{if }i=n-k+1\\
e_{i-n+k} & \text{if }i>n-k+1
\end{cases}\]
where $\mu\in\K\nz$ is such that $\eta^\Bc_\Dc=\mu \eta^{\Zc}_{\Dc}$.
\end{itemize}
\setlength{\leftmargini}{0.5cm}
%For simplicity we omit the $\eta_\Dc$. 
%\setlength{\leftmargini}{0cm}
%\begin{itemize}
%\item[$\boxed{\implies}$] Suppose that $\psi=\wedge^k(\vp)$ and let $\cpa{z_1,\cdots, z_n}$ be a basis of $\K^n$ such that $z_1,\cdots, z_{n-k}$ are linearly independent vectors in $\ker \vp$, then
%\begin{align*}
%&\sum_{I\in \omega(n-k,n)}\sgn\sigma_I\wedge^k\vp(e_{\wh I})e_I\pasgnl={(\ref{UpToScalarCanonicalIso})}\\
%&\qquad\qquad\qquad=\mu\sum_{I\in \omega(n-k,n)}\sgn\sigma_I\wedge^k\vp(z_{\wh I})z_I=\\
%&\qquad\qquad\qquad=\pa{\mu\ \sgn\sigma_{(1,\cdots,n-k)}\wedge^k\vp(z_{(n-k+1,\cdots,n)}) }z_{\pa{1,\cdots, n-k}}.
%\end{align*}
%\item[$\boxed{\impliedby}$] Let $z_1,\cdots, z_n$ be a basis of $\K^n$ which extends $\cpa{z_1,\cdots, z_{n-k}}$ and define $\wt\vp$ by
%\[\wt\vp(z_i)=\begin{cases}
%0 & \text{if }1\leq i\leq n-k\\
%e_1 & \text{if }i=n-k+1\\
%e_{i-n+k} & \text{if }i>n-k+1
%\end{cases}\]
%Let $\al=\wedge^k\wt\vp(z_{(n-k+1,\cdots, n)})$ and consider the following chain of equalities
%\begin{align*}
%\sum_{I\in \omega(n-k,n)}\sgn\sigma_I\psi(e_{\wh I})e_I=&\la z_{(1,\cdots, n-k)}=\\
%=&\frac\la\al\wedge^k\wt\vp(z_{(n-k+1,\cdots, n)}) z_{(1,\cdots, n-k)}=\\
%=&\pa{\frac\la\al\sgn\sigma_{(1,\cdots,n-k)}}\sum_{I\in \omega(n-k,n)}\sgn\sigma_I\wedge^k\wt\vp(z_{\wh I})z_I=\\
%=&\pa{\frac\la\al\sgn\sigma_{(1,\cdots,n-k)}}\mu\sum_{I\in \omega(n-k,n)}\sgn\sigma_I\wedge^k\wt\vp(e_{\wh I})e_I
%\end{align*}
%where the third equality follows from the construction of $\wt\vp$ and the last is (\ref{UpToScalarCanonicalIso}). If we now define $\vp$ by
%\[\vp(z_i)=\begin{cases}
%0 & \text{if }1\leq i\leq n-k\\
%\mu\pa{\frac\la\al\sgn\sigma_{(1,\cdots,n-k)}} e_1 & \text{if }i=n-k+1\\
%e_{i-n+k} & \text{if }i>n-k+1
%\end{cases}\]
%with the same reasoning we see that
%\[\sum_{I\in \omega(n-k,n)}\sgn\sigma_I\psi(e_{\wh I})e_I=\sum_{I\in \omega(n-k,n)}\sgn\sigma_I\wedge^k\vp(e_{\wh I})e_I.\]
%By linear independence, this shows that for all $J\in\omega(k,n)$ we have
%\[\psi(e_J)=\wedge^k \vp(e_J),\]
%so $\psi$ and $\wedge^k(\vp)$ agree on a basis of $\bigwedge^k\K^n$ and are thus the same map. 
%\end{itemize}
%\setlength{\leftmargini}{0.5cm}
\end{proof}

\begin{definition}
Let $\Bc=\cpa{e_1,\cdots, e_n}$ and $\Dc=\cpa{e_1,\cdots, e_k}$ be bases of $\K^n$ and $\K^k$ respectively. For any $\psi\in \Hom_\K(\bigwedge^k\K^n,\bigwedge^k\K^k)$ we define $\Phi_{\Bc,\Dc}(\psi)$ to be
\[\Phi_{\Bc,\Dc}(\psi):\funcDef{\K^n}{\bigwedge^{n-k+1}\K^n}{v}{\displaystyle\sum_{I\in \omega(n-k,n)}\sgn\sigma_I\eta_\Dc(\psi(e_{\wh I}))e_I\wedge v}.\]
\end{definition}

\begin{remark}\label{RankPhiIsBaseIndependent}
The rank of $\Phi_{\Bc,\Dc}(\psi)$ does not depend on the choice of basis. Indeed if we change basis, by corollary (\ref{UpToScalarCanonicalIso}) we see that
\[\sum_{I\in \omega(n-k,n)}\sgn\sigma_I\eta_\Dc(\psi(e_{\wh I}))e_I\wedge v=\mu\sum_{I\in \omega(n-k,n)}\sgn\sigma_I\eta_{\Dc'}(\psi(e'_{\wh I}))e'_I\wedge v,\]
so $\ker \Phi_{\Bc,\Dc}(\psi)$ does not depend on the basis and thus neither do nullity or rank.\bigskip

For this reason we will write propositions which only concern the rank of $\Phi_{\Bc,\Dc}(\psi)$ omitting the bases.
\end{remark}

\begin{remark}
$\Phi_{\Bc,\Dc}(\psi)$ is linear in $\psi$.
\end{remark}

\begin{proposition}\label{RankCriterionForImageOfPlucker}
An alternating multilinear map $\psi\in \Hom_\K(\bigwedge^k\K^n,\bigwedge^k\K^k)$ is in the image of the Pl\"ucker map $\wedge^k$ if and only if $\Phi(\psi)$ has rank at most $k$.
\end{proposition}
\begin{proof}
For the $\implies$ arrow, choose a basis $\Zc=\cpa{z_1,\cdots, z_n}$ for $\K^n$ which extends a basis for $\ker\vp$. Because of how we proved lemma (\ref{DecomposabilityOfMultilinearForm}), we see that if $v\in\ker\vp$ then $\Phi_{\Zc,\Dc}(\wedge^k\vp)(v)=\la z_{(1,\cdots, n-k)}\wedge v$, which is zero by linear dependence. Thus the nullity of $\Phi(\wedge^k\vp)$ is at least $\dim\ker\vp=n-k$.\bigskip

Given $n-k$ linearly independent vectors in $\ker\Phi(\psi)$, by the total decomposibility criterion (\ref{TotalDecomposibilityCriterion}) there exists $\la\in\K$ such that
\[\sum_{I\in \omega(n-k,n)}\sgn\sigma_I\eta_\Dc(\psi(z_{\wh I}))z_I=\la z_1\wedge\cdots \wedge z_{n-k}.\]
This concludes by lemma (\ref{DecomposabilityOfMultilinearForm}).
%Let $\Dc$ be any basis of $\K^k$. We prove both implications
%\setlength{\leftmargini}{0cm}
%\begin{itemize}
%\item[$\boxed{\implies}$] Suppose that $\psi=\wedge^k(\vp)$ and let $\Zc=\cpa{z_1,\cdots, z_{n}}$ be a basis of $\K^n$ such that the first $n-k$ vectors are a basis of $\ker \vp$. Note that
%\[\sum_{I\in \omega(n-k,n)}\sgn\sigma_I\eta_\Dc(\psi(z_{\wh I}))z_I=\sgn\sigma_{(1,\cdots,n-k)}\eta_\Dc(\wedge^k\vp(z_{(n-k+1,\cdots,n)}))z_{(1,\cdots, n-k)}.\]
%If $v\in \ker \vp$ then $z_{(1,\cdots, n-k)}\wedge v=0$ and therefore by the above equality $v\in \ker \Phi_{\Zc,\Dc}(\psi)$. This means that $\Phi(\psi)$ has a nullity of at least $n-k$ (i.e. rank at most $k$).
%\item[$\boxed{\impliedby}$] Suppose that $\cpa{z_1\cdots, z_{n-k}}$ are linearly independent elements of $\ker \Phi(\psi)$. By the total decomposability criterion (\ref{TotalDecomposibilityCriterion}) there exists $\la\in\K$ such that
%\[\sum_{I\in \omega(n-k,n)}\sgn\sigma_I\eta_\Dc(\psi(z_{\wh I}))z_I=\la z_1\wedge\cdots \wedge z_{n-k}.\]
%This concludes by lemma (\ref{DecomposabilityOfMultilinearForm}).
%\end{itemize}
%\setlength{\leftmargini}{0.5cm}
\end{proof}

\begin{definition}
Let $\Bc=\cpa{e_1,\cdots, e_n}$ and $\Dc=\cpa{e_1,\cdots, e_k}$ be bases of $\K^n$ and $\K^k$ respectively. Let $\zeta_{\Bc,\Dc}:\Hom_\K(\bigwedge^k\K^n,\bigwedge^k\K^k)\to\bigwedge^k\K^n$ be the isomorphism discussed during remark (\ref{CodomainOfPluckerMap}). We define
\[\wt \Phi_{\Bc,\Dc}:\funcDef{\bigwedge^k\K^n}{ \Hom_\K\pa{\K^n,\bigwedge^{n-k+1}\K^n}}{\omega}{\Phi_{\Bc,\Dc}(\zeta_{\Bc,\Dc}\ii(\omega))}\]
\end{definition}

\begin{remark}
The rank of $\wt \Phi_{\Bc,\Dc}(\omega)$ does not depend on the choice of basis.
\end{remark}
\begin{proof}
If $\omega=\sum_{I\in\omega(k,n)}p_Ie_I$ then an easy calculation shows that $v$ is an element of $\ker \Phi_{\Bc,\Dc}(\zeta_{\Bc,\Dc}\ii(\omega))$ if and only if for all $I\in \omega(k,n)$ either $p_I=0$ or $e_{\wh I}\wedge v=0$, which is the same as saying $\omega\wedge v= 0$. The last condition is base independent so $\ker \Phi_{\Bc,\Dc}(\zeta_{\Bc,\Dc}\ii(\omega))$ must be also.
%Let $\omega=\sum_{I\in\omega(k,n)}p_Ie_I$.
%\begin{align*}
%\Phi_{\Bc,\Dc}(\zeta_{\Bc,\Dc}\ii(\omega))(v)=&\sum_{I\in\omega(n-k,n)}\sgn\sigma_I\eta_\Dc(\zeta_{\Bc,\Dc}\ii(\omega)(e_{\wh I}))e_I\wedge v=\\
%=&\sum_{I\in\omega(n-k,n)}\sgn\sigma_Ip_{\wh I}e_I\wedge v,
%\end{align*}
%so $v\in \ker \Phi_{\Bc,\Dc}(\zeta_{\Bc,\Dc}\ii(\omega))$ if and only if for all $I\in \omega(k,n)$ either $p_I=0$ or $e_{\wh I}\wedge v=0$, which is the same as saying $\omega\wedge v= 0$ if and only if $\omega=0$ or $v=0$. This is a base independent condition, so the rank of $\wt\Phi_{\Bc,\Dc}(\omega)$ does not depend on the choice of basis.
\end{proof}

\begin{remark}
$\wt\Phi_{\Bc,\Dc}$ is linear.
\end{remark}



\noindent
Let us define a matrix with coefficients in\footnote{what we will later call the Bracket ring} $\K[z_I\mid I\in\omega(k,n)]$ which represents $\wt\Phi_{\Bc,\Dc}$:\medskip

Let $B^I\in \Mc\pa{\binom{n}{n-k+1},n,\K}$ be the matrix which represents $\Phi_{\Bc,\Dc}(\zeta_{\Bc,\Dc}\ii(e_I))$ in the bases induced by $\Bc$ and $\Dc$. By linearity
\[\wt \Phi_{\Bc,\Dc}\pa{\sum_{I\in \omega(k,n)}a_I e_I}(v)=\sum_{I\in \omega(k,n)}a_I \Phi_{\Bc,\Dc}(\zeta_{\Bc,\Dc}\ii(e_I))(v)=\sum_{I\in \omega(k,n)}a_I B^I v.\]
We define the matrix which represents $\wt\Phi_{\Bc,\Dc}$ to be
\[M_{\Bc,\Dc}=\sum_{I,\omega(k,n)}B^I z_I=\pa{\sum_{I\in\omega(k,n)}(B^I)_{i,j}z_I}_{i,j}.\]


\begin{remark}
The rank of $\Phi_{\Bc,\Dc}(\sum_{I\in\omega(k,n)}p_Ie_I)$ is exactly the rank of $\rbar{M}_{z_I=p_I}$.
\end{remark}

\noindent The previous remark together with proposition (\ref{RankCriterionForImageOfPlucker}) tells us that
\begin{align*}
\imm (\zeta_{\Bc,\Dc}\circ \wedge^k)=&\cpa{\sum_{I\in \omega(k,n)}p_Ie_I\mid \rnk \rbar{M}_{z_I=p_I}<k+1}=\\
=&V(\cpa{\det m\mid m\text{ is a $(k+1)\times(k+1)$ minor of $M$}}),
\end{align*}
which is evidently a Zariski-closed subset of $\bigwedge^k\K^n$.\bigskip

It follows trivially that the projectivization\footnote{recall (\ref{ImagePluckerMapIsCone}) that $\imm \wedge^k$ is a cone.} of this set (i.e. the image of $\Pl$) is closed in $\Pj(\bigwedge^k\K^n)$, so we found a bijection between $\Gr(k,n)$ and a projective variety, which we can use to endow $\Gr(k,n)$ with the structure of one.


\begin{remark}
The determinants we used to show that the image of the Pl\"ucker embedding is closed do not generate the ideal of that variety. The most well known set of generators for that ideal are the \textbf{Pl\"ucker relations} (Theorem 2.4.3 in \cite{matroids}, page 80).
\end{remark}










\chapter{Representability of the Grassmannian functor}
In this chapter, unless otherwise specified, we have fixed a basis $e_1,\cdots, e_n$ of $\K^n$ and a basis $e_1,\cdots, e_k$ of $\K^k$. In case of ambiguity we will refer to these bases as \textit{canonical}.
\medskip

\noindent Having fixed a base, we will identify $\Hom_\K(\K^n,\K^k)$ with the set of $k\times n$ matrices with coefficients in $\K$, which we will denote $\Mc(k,n)$. As a consequence of this we find yet another form for $\Gr(k,n)$:
\[\Gr(k,n)=\quot{\cpa{A\in \Mc(k,n)\mid \rnk A=k}}{\sim},\] 
where $A\sim B\coimplies \exists P\in \GL(k)\ s.t.\ A=PB$.
\bigskip

\noindent We may rewrite the maps from the previous chapter as follows:
\[\phi^s:\funcDef{\Mc(k,n)}{\bigwedge^k\K^n}{A}{\sum_{I\in \omega(k,n)}\det A_I e_I}\]
\[\Pl^s:\funcDef{\Gr(k,n)}{\Pj(\bigwedge^k\K^n)}{[A]_\sim}{\spa{\sum_{I\in \omega(k,n)}\det A_I e_I}_{\K^\ast}}\]
where we use the superscript $s$ to distinguish these maps with the ones we will define for schemes.

\section{Grassmannians as projective schemes}
To connect Grassmannians to the world of representable functors we shall redefine them scheme-theoretically by emulating the construction from the prievious chapter using rings and ring homomorphisms.

\begin{definition}[Braket ring]
We define the \textbf{braket ring}(see page 79 of \cite{matroids}) as the ring of polynomial functions on $\bigwedge^k\K^n$, i.e.
\[\Bc_{k,n}\doteqdot\frac{\K[z_I\mid I\in \cpa{1,\cdots, n}^k]}{(\cpa{z_I-\sgn(\sigma)z_{\sigma(I)}}_{\sigma\in S_k})}\cong \K[z_I\mid I\in \omega(k,n)].\]
\end{definition}

\begin{definition}[Ring of generic matrices]
Let $\K[X_{k,n}]\doteqdot\K[x_{1,1},\cdots,x_{k,n}]$ denote the polynomial ring with $k\cdot n$ variables. We define the \textbf{generic matrix} by
\[X=\mat{
    x_{1,1} & \cdots & x_{1,n}\\
    \vdots & \ddots & \vdots\\
    x_{k,1} &\cdots & x_{k,n}}\]
and by the same token we denote by $X_I$ the generic $k\times k$ minor determined by the multiindex $I$ and by $\det X_I$ the formal determinant of this minor.
\end{definition}
\begin{remark}
The ring $\K[X_{k,n}]$ is the coordinate ring of $\Mc(k,n)$. 
\end{remark}

\begin{remark}
The familiar $\Mc(k,n)$ and $\bigwedge^k\K^n$ can be identified with the $\K$-points of the affine schemes $\Spec \K[X_{k,n}]$ and $\Spec \Bc_{k,n}$ respectively (Example 2.3.32 of \cite{QingLiu}).
\end{remark}

\begin{definition}[Pl\"ucker ring homomorphism]
We define the \textbf{Pl\"ucker ring homomorphism} or simply \textbf{Pl\"ucker homomorphism} as
\[\phi^\#:\funcDef{\Bc_{k,n}}{\K[X_{k,n}]}{z_I}{\det X_I}\]
For brevity we will denote $\Spec \phi^\#$ by $\phi$.\\
This definition is inspired by that of $\phi$ at page 79 of \cite{matroids}.
\end{definition}

\begin{proposition}
The kernel of the Pl\"ucker homomorphism is an homogeneous prime ideal which does not contain $(\cpa{z_I}_{I\in\omega(k,n)})$.
\end{proposition}
\begin{proof}
Since $\K[X_{k,n}]$ is an integral domain, $\ker\phi^\#$ is prime.\\
By definition of homogeneous ideal, we want to show that if $f=\sum f_d$ for $d$ homogeneous and $f\in \ker\phi^\#$ then $f_d\in \ker \phi^\#$ for all $d$.\\
Looking at the definition of $\phi^\#$ we see that $\phi^\#(f_d)$ is a homogeneous polynomial of degree $kd$, in particular if $d\neq h$ then $\deg \phi^\#(f_d)\neq \deg \phi^\#(f_h)$. Since
\[0=\phi^\#(f)=\sum \phi^\#(f_d)\]
this proves that $\phi^\#(f_d)=0$ for all $d$.\\
Finally, observe that $\deg\phi^\#(z_I)=\deg(\det X_I)=k>0$, so $z_I\notin \ker\phi^\#$.
\end{proof}

\begin{proposition}\label{PluckerRingHomomorphismWorksForKPoints}
The induced map $\phi\res{\Spec(\K[X_{k,n}])(\K)}:\Spec(\K[X_{k,n}])(\K)\to \Spec(\Bc_{k,n})(\K)$ is equal to $\phi^s:\Mc(k,n)\to \bigwedge^k\K^n$ under the identification mentioned above, i.e. for all matrices $A\in\Mc(k,n)$ with entries $a_{i,j}$ we have
\[(\phi^\#)\ii((x_{i,j}-a_{i,j}))=(z_I-\det A_I).\]
\end{proposition}
\begin{proof}
First we observe that for any multiindex $I$
\[\det X_I-\det A_I\in (x_{i,j}-a_{i,j}),\]
thus $(z_I-\det A_I)\subseteq (\phi^\#)\ii((x_{i,j}-a_{i,j}))$.\\
Since $(z_I-\det A_I)$ is a $\K$-point, it is in particular a maximal ideal of the Braket ring, thus we have the desired equality if $1\notin (\phi^\#)\ii((x_{i,j}-a_{i,j}))$, which is the case because otherwise $(x_{i,j}-a_{i,j})$ would not be proper.
\end{proof}

\begin{proposition}
The $\K$-points of $V_+(\ker(\phi^\#))$ correspond to $\imm \Pl^s$.
\end{proposition}
\begin{proof}
First we note that
\[V_+(\ker\phi^\#)=\Proj\frac{\Bc_{k,n}}{\ker\phi^\#}\subseteq \Proj\Bc_{k,n}.\]
Since $\phi$ becomes $\phi^s$ on $\K$-points we see that
\[Z(\ker\phi^\#)=\ol{\imm \pa{\Spec\phi^\#}\res{\Spec(\K[X_{k,n}])(\K)}}=\ol{\imm \phi^s}\pasgnl={$\emat{\text{previous}\\\text{chapter}}$}\imm \phi^s.\] 
It follows from Corollary 2.3.44 in \cite{QingLiu} that the $\K$-points of $V_+(\ker\phi^\#)$ correspond to 
\[Z_+(\ker\phi^\#)=\Pj(Z(\ker\phi^\#))=\Pj(\imm \phi^s)=\imm \Pl^s.\]
\end{proof}

\noindent This result allows us to redefine the Grassmannian as a projective scheme. We can obtain the Pl\"ucker embedding of the classical Grassmannian back by looking at $\K$-points.
\medskip

\noindent From now on $\Gr(k,n)$ will denote $V_+(\ker \phi^\#)$, while $\Gr(k,n)(\K)$ will denote what we used to write as $\Gr(k,n)$.

\subsection{Standard affine cover of the Grassmannian scheme}






\section{Grassmannian moduli functor}

Let us consider the following functor
\[\G(k,n):\funcDef{(\Sch\K)\op}{Set}{T}{\quot{\cpa{\al:\Oc_T^n\onto Q}}\sim}\]
where $Q$ is a locally free sheaf of rank $k$ on $T$ and two surjections $\al:\Oc_T^n\onto Q$, $\beta:\Oc_T^n\onto V$ are equivalent if and only if there exist an isomorphism of sheaves $\theta:Q\to V$ such that the diagram commutes
\[\begin{tikzcd}
	{\Oc_T^n} & Q \\
	& V
	\arrow["\al", two heads, from=1-1, to=1-2]
	\arrow["\beta"', two heads, from=1-1, to=2-2]
	\arrow["\theta", from=1-2, to=2-2]
\end{tikzcd}\]

\noindent In this this section we will prove that the Grassmannian scheme $V_+(\ker\phi^\#)$ defined above represents this functor.

\subsection{Open subfunctor cover of the Grassmannian}
\begin{notation}
For any multiindex $I$ and any scheme $T$ we define the following morphism of sheaves
\[s_I:\funcDef{\Oc_T^k}{\Oc_T^n}{e_j}{e_{i_j}}.\]
\end{notation}

\begin{definition}[Principal subfunctor of the Grassmannian]
Fixed a multiindex $I\in \omega(k,n)$ we define the following functor
\[\G_I(k,n):\funcDef{(\Sch\K)\op}{Set}{T}{\quot{\cpa{0\to \Oc^k_T\overset{s_I}\to\Oc_T^n\overset{\al}\to Q\to 0}}\sim}\]
where the elements are short exact sequences and the equivalence relation is the same as the one defined for $\G(k,n)$.
\end{definition}


\begin{proposition}\label{GrIAreOpenSubfunctors}
The $\G_I(k,n)$ are open subfunctors of $\G(k,n)$.
\end{proposition}
\begin{proof}
We will follow the approach showcased in \cite{Bejleri2}.\\
Let us prove first that it is a subfunctor, then we shall prove representability of the appropriate fibered products.
\setlength{\leftmargini}{0cm}
\begin{itemize}
\item[$\boxed{Subfunctor}$] It is a functor because if $\psi=\theta\circ \al$ with $\theta$ isomorphism of sheaves, then $\ker \psi=\ker\al=\imm s_I$. The inclusion is obvious.
\item[$\boxed{Open}$] Let $T$ be any $\K$-scheme and let us fix a quotient $\al:\Oc_T^n\onto Q$ in $\G(k,n)(T)$. We can define $\al\circ s_I:\Oc_T^k\to Q$. The locus where this map is surjective is the complement of the support of its cokernel sheaf $\Kc$, i.e. $\al\circ s_I$ is surjective on $\Oc_{T,x}^k$ if and only if $x\notin \supp\Kc$. Since the support of a locally free sheaf is closed, the set $U_I$ where $\al\circ s_I$ is surjective is an open subset of $T$.\\
We now want to show that $U_I$ represents the functor $h_T\times_{\G(k,n)}\G_I(k,n)$, that is we want to show that if $f:S\to T$ is a morphism of $\K$-schemes then $f$ factors through $U_I$ if and only if $f^\ast\al:\Oc_S^n\to f^\ast Q \in \Gr_I(S)$. Let us fix $x\in S$ and $y\in T$ so that $f(x)=y$. By definition $y\in U_i$ if and only if $\al\circ s_I$ is surjective on $\Oc_{T,y}^k$, which by Nakayama's lemma is the same as asking that the following map is surjective
\[(\al\circ s_I)\res{y}:k(y)^k\to \frac{Q_y}{\mf_y Q_y}.\]
If we now pull back via $f$ we obtain a map
\[(f^\ast \al\circ f^\ast s_I)\res{x}:k(x)^k\to\frac{f^\ast Q_x}{\mf_x f^\ast Q_x} \]
and *******************************************
\end{itemize}
\setlength{\leftmargini}{0.5cm}

\end{proof}


\subsection{Representability of the Grassmannian functor}
\begin{proposition}
The grassmannian functor $\G(k,n)$ is a Zariski sheaf.
\end{proposition}
\begin{proof}
Consider a $\K$-scheme $T$ and an open cover $\cpa{U_i}$. Consider now quotiens $\al_i:\Oc_{U_i}^n\onto Q_i$ such that 
\[\al_i\res{U_i\cap U_j}\sim \al_j\res{U_i\cap U_j}.\]	
By definition of $\sim$ there exist isomorphisms of sheaves $\vp_{ji}:Q_i\res{U_{i}\cap U_j}\to Q_j\res{U_i\cap U_j}$. If we define $\vp_{ii}=id_{Q_i}$ and fix the isomorphisms in such a way that $\vp_{ki}=\vp_{kj}\circ \vp_{ji}$ we have the data to glue the $Q_i$ to a locally free sheaf of rank $k$ over $T$, which we denote by $Q$. Now, up to isomorphism let us consider $\al_i:\Oc_{U_i}^n\onto Q\res{U_i}$ for all $i$. If we fix any open set $V\subseteq T$ we see that if $s\in \Oc_T^n(V)$ is a section, we can define $\al_V(s)$ by gluing the $\al_i(s\res{U_i})$, which we can do by construction of $Q$ and the choice of representative for the $\al_i$. By construction $\al_{U_i}=\al_i$ and it is in fact the only such morphism, so we have verified the gluing property of sheaves for $\G(k,n)$.
\end{proof}



\bibliographystyle{plain}
\bibliography{refs}


%\appendix
%\include{Riconoscimenti.tex}

\end{document}
