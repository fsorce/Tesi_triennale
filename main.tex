\documentclass[a4paper]{report}
\usepackage[utf8]{inputenc}
\usepackage{amsmath,amssymb,amsfonts,amsthm,stmaryrd}
\usepackage{mathrsfs} % per mathscr
\usepackage{dsfont} % per mathbb1
\usepackage{graphicx}% ruota freccia per le azioni
\usepackage{oldgerm} % Fractur Particolare
\usepackage{marvosym}% per il \Lightning
\usepackage{array}
\usepackage{faktor} %per gli insiemi quoziente
\usepackage{hyperref}
\usepackage{xparse} % Per nuovi comandi con tanti input opzionali
\usepackage{tikz-cd}
\usepackage{multicol}
\usepackage{multirow}
\usepackage{cancel}
%\usepackage[italian]{babel}


% Ambienti per teoremi =================================
% <name> 
% <space above> 
% <space below> 
% <body font> 
% <indent amount> 
% <Theorem head font> 
% <punctuation after theorem head> 
% <space after theorem head> (default .5em) 
% <Theorem head spec>

% I nuovi ambienti sono costruiti in modo da andare alla riga successiva
\newtheoremstyle{customth}
{\topsep}{\topsep}{\itshape}{}{\bfseries}{.}{\newline}{}
\newtheoremstyle{customdef}
{\topsep}{\topsep}{\normalfont}{}{\bfseries}{.}{\newline}{}
\newtheoremstyle{customrem}
{\topsep}{\topsep}{\normalfont}{}{\itshape}{.}{\newline}{}

\theoremstyle{customth}
\newtheorem{theorem}{Theorem}[chapter]
\newtheorem{lemma}[theorem]{Lemma}
\newtheorem{corollary}[theorem]{Corollary}
\newtheorem{proposition}[theorem]{Proposition}
\newtheorem{fact}[theorem]{Fact}
\newtheorem{application}[theorem]{Application}
\theoremstyle{customrem}
\newtheorem{remark}[theorem]{Remark}
\theoremstyle{customdef}
\newtheorem{definition}[theorem]{Definition}
\newtheorem{notation}[theorem]{Notation}
\newtheorem{example}[theorem]{Example}

\makeatletter
\renewenvironment{proof}[1][\proofname]
{
    \par
    \pushQED{\qed}
    \normalfont \topsep6\p@\@plus6\p@\relax
    \trivlist
    \item[\hskip\labelsep\itshape#1\@addpunct{.}]\mbox{}\\*
}
{
    \popQED\endtrivlist\@endpefalse
}
\makeatother



%============ Simboli standard =================
%----------------- Lettere ---------------------
\newcommand{\A}{\mathbb{A}}
\newcommand{\B}{\mathbb{B}}
\newcommand{\C}{\mathbb{C}}
\newcommand{\D}{\mathbb{D}}
\newcommand{\E}{\mathbb{E}}
\newcommand{\F}{\mathbb{F}}
\newcommand{\G}{\mathbb{G}}
\newcommand{\Hb}{\mathbb{H}}
\newcommand{\I}{\mathbb{I}}
\newcommand{\J}{\mathbb{J}}
\newcommand{\K}{\mathbb{K}}
\newcommand{\Lb}{\mathbb{L}}
\newcommand{\M}{\mathbb{M}}
\newcommand{\N}{\mathbb{N}}
\newcommand{\Ob}{\mathbb{O}}
\newcommand{\Pj}{\mathbb{P}}
\newcommand{\Q}{\mathbb{Q}}
\newcommand{\R}{\mathbb{R}}
\newcommand{\Sb}{\mathbb{S}}
\newcommand{\T}{\mathbb{T}}
\newcommand{\U}{\mathbb{U}}
\newcommand{\V}{\mathbb{V}}
\newcommand{\W}{\mathbb{W}}
\newcommand{\X}{\mathbb{X}}
\newcommand{\Y}{\mathbb{Y}}
\newcommand{\Z}{\mathbb{Z}}

\newcommand{\Ac}{\mathcal{A}}
\newcommand{\Bc}{\mathcal{B}}
\newcommand{\Cc}{\mathcal{C}}
\newcommand{\Dc}{\mathcal{D}}
\newcommand{\Ec}{\mathcal{E}}
\newcommand{\Fc}{\mathcal{F}}
\newcommand{\Gc}{\mathcal{G}}
\newcommand{\Hc}{\mathcal{H}}
\newcommand{\Ic}{\mathcal{I}}
\newcommand{\Jc}{\mathcal{J}}
\newcommand{\Kc}{\mathcal{K}}
\newcommand{\Lc}{\mathcal{L}}
\newcommand{\Mc}{\mathcal{M}}
\newcommand{\Nc}{\mathcal{N}}
\newcommand{\Oc}{\mathcal{O}}
\newcommand{\Pc}{\mathcal{P}}
\newcommand{\Qc}{\mathcal{Q}}
\newcommand{\Rc}{\mathcal{R}}
\newcommand{\Sc}{\mathcal{S}}
\newcommand{\Tc}{\mathcal{T}}
\newcommand{\Uc}{\mathcal{U}}
\newcommand{\Vc}{\mathcal{V}}
\newcommand{\Wc}{\mathcal{W}}
\newcommand{\Xc}{\mathcal{X}}
\newcommand{\Yc}{\mathcal{Y}}
\newcommand{\Zc}{\mathcal{Z}}

\newcommand{\Af}{\mathfrak{A}}
\newcommand{\Bf}{\mathfrak{B}}
\newcommand{\Cf}{\mathfrak{C}}
\newcommand{\Df}{\mathfrak{D}}
\newcommand{\Ef}{\mathfrak{E}}
\newcommand{\Ff}{\mathfrak{F}}
\newcommand{\Gf}{\mathfrak{G}}
\newcommand{\Hf}{\mathfrak{H}}
\newcommand{\If}{\mathfrak{I}}
\newcommand{\Jf}{\mathfrak{J}}
\newcommand{\Kf}{\mathfrak{K}}
\newcommand{\Lf}{\mathfrak{L}}
\newcommand{\Mf}{\mathfrak{M}}
\newcommand{\Nf}{\mathfrak{N}}
\newcommand{\Of}{\mathfrak{O}}
\newcommand{\Pf}{\mathfrak{P}}
\newcommand{\Qf}{\mathfrak{Q}}
\newcommand{\Rf}{\mathfrak{R}}
\newcommand{\Sf}{\mathfrak{S}}
\newcommand{\Tf}{\mathfrak{T}}
\newcommand{\Uf}{\mathfrak{U}}
\newcommand{\Vf}{\mathfrak{V}}
\newcommand{\Wf}{\mathfrak{W}}
\newcommand{\Xf}{\mathfrak{X}}
\newcommand{\Yf}{\mathfrak{Y}}
\newcommand{\Zf}{\mathfrak{Z}}

\newcommand{\af}{\mathfrak{a}}

\newcommand{\cf}{\mathfrak{c}}
\newcommand{\df}{\mathfrak{d}}
\newcommand{\ef}{\mathfrak{e}}
\newcommand{\ff}{\mathfrak{f}}
\newcommand{\gf}{\mathfrak{g}}
\newcommand{\hf}{\mathfrak{h}}

\newcommand{\jf}{\mathfrak{j}}
\newcommand{\kf}{\mathfrak{k}}
\newcommand{\lf}{\mathfrak{l}}
\newcommand{\mf}{\mathfrak{m}}
\newcommand{\nf}{\mathfrak{n}}
\newcommand{\of}{\mathfrak{o}}
\newcommand{\pf}{\mathfrak{p}}
\newcommand{\qf}{\mathfrak{q}}
\newcommand{\rf}{\mathfrak{r}}

\newcommand{\tf}{\mathfrak{t}}
\newcommand{\uf}{\mathfrak{u}}
\newcommand{\vf}{\mathfrak{v}}
\newcommand{\wf}{\mathfrak{w}}
\newcommand{\xf}{\mathfrak{x}}
\newcommand{\yf}{\mathfrak{y}}
\newcommand{\zf}{\mathfrak{z}}

\newcommand{\As}{\mathscr{A}}
\newcommand{\Bs}{\mathscr{B}}
\newcommand{\Cs}{\mathscr{C}}
\newcommand{\Ds}{\mathscr{D}}
\newcommand{\Es}{\mathscr{E}}
\newcommand{\Fs}{\mathscr{F}}
\newcommand{\Gs}{\mathscr{G}}
\newcommand{\Hs}{\mathscr{H}}
\newcommand{\Is}{\mathscr{I}}
\newcommand{\Js}{\mathscr{J}}
\newcommand{\Ks}{\mathscr{K}}
\newcommand{\Ls}{\mathscr{L}}
\newcommand{\Ms}{\mathscr{M}}
\newcommand{\Ns}{\mathscr{N}}
\newcommand{\Os}{\mathscr{O}}
\newcommand{\Ps}{\mathscr{P}}
\newcommand{\Qs}{\mathscr{Q}}
\newcommand{\Rs}{\mathscr{R}}
\newcommand{\Ss}{\mathscr{S}}
\newcommand{\Ts}{\mathscr{T}}
\newcommand{\Us}{\mathscr{U}}
\newcommand{\Vs}{\mathscr{V}}
\newcommand{\Ws}{\mathscr{W}}
\newcommand{\Xs}{\mathscr{X}}
\newcommand{\Ys}{\mathscr{Y}}
\newcommand{\Zs}{\mathscr{Z}}

\newcommand{\ula}{{\underline{a}}}
\newcommand{\ulb}{{\underline{b}}}
\newcommand{\ulc}{{\underline{c}}}
\newcommand{\uld}{{\underline{d}}}
\newcommand{\ule}{{\underline{e}}}
\newcommand{\ulf}{{\underline{f}}}
\newcommand{\ulg}{{\underline{g}}}
\newcommand{\ulh}{{\underline{h}}}
\newcommand{\uli}{{\underline{i}}}
\newcommand{\ulj}{{\underline{j}}}
\newcommand{\ulk}{{\underline{k}}}
\newcommand{\ull}{{\underline{l}}}
\newcommand{\ulm}{{\underline{m}}}
\newcommand{\uln}{{\underline{n}}}
\newcommand{\ulo}{{\underline{o}}}
\newcommand{\ulp}{{\underline{p}}}
\newcommand{\ulq}{{\underline{q}}}
\newcommand{\ulr}{{\underline{r}}}
\newcommand{\uls}{{\underline{s}}}
\newcommand{\ult}{{\underline{t}}}
\newcommand{\ulu}{{\underline{u}}}
\newcommand{\ulv}{{\underline{v}}}
\newcommand{\ulw}{{\underline{w}}}
\newcommand{\ulx}{{\underline{x}}}
\newcommand{\uly}{{\underline{y}}}
\newcommand{\ulz}{{\underline{z}}}

%---------- Funzioni standard ------------------
\newcommand{\Adj}{\mathrm{Adj}\,}
\newcommand{\adj}{\mathrm{adj}\,}
\newcommand{\Ann}{\mathrm{Ann}\,}
\newcommand{\Arg}{\mathrm{Arg}\,}
\newcommand{\Ass}{\mathrm{Ass}\,}
\newcommand{\cha}{\mathrm{char}\,}
\newcommand{\cod}{\mathrm{cod}}
\newcommand{\coker}{\mathrm{coker}\,}
\newcommand{\comb}{\mathrm{Comb}\,}
\newcommand{\dom}{\mathrm{dom}}
\newcommand{\End}{\mathrm{End}\,}
\newcommand{\Fix}{\mathrm{Fix}\;}
\newcommand{\Hom}{\mathrm{Hom}\,}
\newcommand{\imm}{\mathrm{Imm}\,}
\newcommand{\Ind}{\mathrm{Ind}}
\newcommand{\mcd}{\mathrm{mcd}\,}
\newcommand{\mcm}{\mathrm{mcm}\,}
\newcommand{\Min}{\mathrm{Min}\,}
\newcommand{\Mor}{\mathrm{Mor}}
\newcommand{\obj}{\mathrm{obj}}
\newcommand{\orb}{\mathrm{orb}\,}
\newcommand{\ord}{\mathrm{ord}\;}
\newcommand{\Proj}{\mathrm{Proj}\,}
\newcommand{\Res}{\mathrm{Res}}
\newcommand{\rnk}{\mathrm{rnk}\,}
\newcommand{\sgn}{\mathrm{sgn}\,}
\newcommand{\Span}{\mathrm{Span}\,}
\newcommand{\Spec}{\mathrm{Spec}\,}
\newcommand{\stab}{\mathrm{stab}\,}
\newcommand{\Supp}{\mathrm{Supp}\,}
\newcommand{\supp}{\mathrm{supp}\,}
\newcommand{\tr}{\mathrm{tr}\,}

\newcommand{\Real}{\,\Re\mathfrak{e}}
\newcommand{\Imag}{\,\Im\mathfrak{m}}

%-------------- Frecce -------------------------
\newcommand{\coimplies}{\Longleftrightarrow}
\newcommand{\inj}{\hookrightarrow}
\newcommand{\onto}{\twoheadrightarrow}
\newcommand{\ot}{\leftarrow}
\newcommand{\acts}{\curvearrowright}

%----------- Lettere greche -------------------
\newcommand{\al}{\alpha}
\newcommand{\de}{\delta}
\newcommand{\e}{\varepsilon}
%\newcommand{\th}{\theta}
\newcommand{\la}{\lambda}
\newcommand{\vp}{\varphi}

%-------------- Derivate ----------------------
\newcommand{\raiseargument}[1]{\raisebox{.8ex}{$#1$}}
\newcommand{\centersmallmath}[1]{\vcenter{\hbox{\scalebox{.8}{$#1$}}}}
\newcommand{\raiseargumentsmall}[1]{\raisebox{.4ex}{\scalebox{.8}{$#1$}}}
\newcommand*{\emptyfrac}[2]{\genfrac{}{}{0pt}{}{#1}{#2}}

\NewDocumentCommand{\ddxi}{O{x}mm}{
    {\frac{d^{}{#3}}{d{#1}_{#2}}}
}

\NewDocumentCommand{\dd}{O{}mm}{
    {\frac{d^{#1}{#3}}{d{#2}^{#1}}}
}

\NewDocumentCommand{\ppxi}{O{x}mm}{
    {{\frac{\partial^{}{#3}}{\partial{#1}_{#2}}}}
}

\NewDocumentCommand{\pp}{O{}mm}{
    {{\frac{\partial^{#1}{#3}}{\partial{#2}}}}
}





%========== Comandi dattilografici ============
%--------- Passaggi in derivazioni ------------
\newcommand{\pasg}[3]{\overset{\hyperref[#3]{\text{#2}}}{#1}}
\newcommand{\pasgnl}[2]{\overset{\text{#2}}{#1}}
\newcommand{\pasgnlmath}[2]{\overset{#2}{#1}}
\newcommand{\pasgmath}[3]{\overset{\hyperref[#3]{{#2}}}{#1}}

%----------- Modifica testo -------------------
\newcommand{\ul}[1]{\underline{#1}}
\newcommand{\ol}[1]{\overline{#1}}
\newcommand{\wt}[1]{\widetilde{#1}}
\newcommand{\wh}[1]{\widehat{#1}}
\newcommand{\td}[1]{\Tilde{#1}}
\newcommand{\rg}[1]{{\mathring {#1}}}
\newcommand{\under}[2]{\underset{#1}{\underbrace{#2}}}

%-------------- Parentesi ---------------------
\newcommand{\pa}[1]{\left({#1}\right)}
\newcommand{\spa}[1]{\left[{#1}\right]}
\newcommand{\cpa}[1]{\left\{{#1}\right\}}
\newcommand{\abs}[1]{\left|{#1}\right|}
\newcommand{\norm}[1]{\left\Vert{#1}\right\Vert}
\newcommand{\ps}[1]{\left\langle {#1}\right\rangle}
\newcommand{\floor}[1]{\left\lfloor {#1}\right\rfloor}
\newcommand{\ceil}[1]{\left\lceil {#1}\right\rceil}
\newcommand{\rbar}[1]{\left.{#1}\right|}

%--------------- Matrici ----------------------
\newcommand{\mat}[1]{\begin{pmatrix}#1\end{pmatrix}}
\newcommand{\emat}[1]{\begin{matrix}#1\end{matrix}}
\newcommand{\dmat}[1]{\begin{vmatrix}#1\end{vmatrix}}
\newcommand{\smat}[1]{\begin{smallmatrix}#1\end{smallmatrix}}
\newcommand{\BIG}[1]{\mathlarger{\mathlarger{\mathlarger{\mathlarger{#1}}}}}

%--------------- Funzioni ---------------------
\newcommand{\funcDef}[4]{
\begin{array}{ccc}
{#1} & \longrightarrow & {#2}\\
{#3} & \longmapsto & {#4}
\end{array}}
\newcommand{\functorDef}[6]{
\begin{array}{ccc}
{#1} & \longrightarrow & {#2}\\
{#3} & \longmapsto & {#4}\\
{#5} & \longmapsto & {#6}
\end{array}}

%---------------- Altro -----------------------
\newcommand{\bs}{\setminus}
\newcommand{\res}[1]{\raisebox{-.5ex}{$|$}_{#1}}
\newcommand{\quot}[2]{\faktor{#1}{#2}}
\newcommand{\sep}{\,\middle|\,}

\newcommand{\ii}{^{-1}}
\newcommand{\nz}{\bs\{0\}}

\newcommand{\powerset}{\mathscr{P}}
\newcommand{\del}{\partial}
\newcommand{\0}{{\underline{0}}}
\newcommand{\1}{{\vcenter{\hbox{\scalebox{1.2}{$\mathds{1}$}}}}}


\newcommand{\GL}{\mathrm{GL}}
\newcommand{\PGL}{\mathrm{PGL}}
%\NewDocumentCommand{\PGL}{o m}{
%    \IfNoValueTF{#1}
%        {{\mathbb{P}GL({#2})}}
%    {{\mathbb{P}GL_{#1}({#2})}}
%}
%\NewDocumentCommand{\GL}{o m}{
%    \IfNoValueTF{#1}
%        {{GL({#2})}}
%    {{GL_{#1}({#2})}}
%}
\newcommand{\znz}[1]{{\Z/{#1}\Z}}






%========= Preambolo per quiver ================
% quiver e' uno strumento che uso spesso per
% disegnare diagrammi. L'interfaccia sul loro sito
% permette di creare in modo visivo il diagramma e
% poi esportarlo come codice LaTeX da inserire nel
% documento. Il sito e' https://q.uiver.app/ 

%-----------------------------------------------
% *** quiver ***
% A package for drawing commutative diagrams exported from https://q.uiver.app.
%
% This package is currently a wrapper around the `tikz-cd` package, importing necessary TikZ
% libraries, and defining a new TikZ style for curves of a fixed height.
%
% Version: 1.2.1
% Authors:
% - varkor (https://github.com/varkor)
% - Andr\e'C (https://tex.stackexchange.com/users/138900/andr%C3%A9c)

\NeedsTeXFormat{LaTeX2e}
%\ProvidesPackage{quiver}[2021/01/11 quiver]

% `tikz-cd` is necessary to draw commutative diagrams.
\RequirePackage{tikz-cd}
% `amssymb` is necessary for `\lrcorner` and `\ulcorner`.
\RequirePackage{amssymb}
% `calc` is necessary to draw curved arrows.
\usetikzlibrary{calc}
% `pathmorphing` is necessary to draw squiggly arrows.
\usetikzlibrary{decorations.pathmorphing}

% A TikZ style for curved arrows of a fixed height, due to Andr\e'C.
\tikzset{curve/.style={settings={#1},to path={(\tikztostart)
    .. controls ($(\tikztostart)!\pv{pos}!(\tikztotarget)!\pv{height}!270:(\tikztotarget)$)
    and ($(\tikztostart)!1-\pv{pos}!(\tikztotarget)!\pv{height}!270:(\tikztotarget)$)
    .. (\tikztotarget)\tikztonodes}},
    settings/.code={\tikzset{quiver/.cd,#1}
        \def\pv##1{\pgfkeysvalueof{/tikz/quiver/##1}}},
    quiver/.cd,pos/.initial=0.35,height/.initial=0}

% TikZ arrowhead/tail styles.
\tikzset{tail reversed/.code={\pgfsetarrowsstart{tikzcd to}}}
\tikzset{2tail/.code={\pgfsetarrowsstart{Implies[reversed]}}}
\tikzset{2tail reversed/.code={\pgfsetarrowsstart{Implies}}}
% TikZ arrow styles.
\tikzset{no body/.style={/tikz/dash pattern=on 0 off 1mm}}
%=================================================

%\usepackage{biblatex}
%PER CAMBIARE I MARGINI
\usepackage[margin=4cm]{geometry}

%----------- Setup stilistico ----------------
\definecolor{DarkRed}{HTML}{B6321C}
\hypersetup{
    colorlinks=true,
    linkcolor=DarkRed,
    filecolor=blue,
    citecolor = black,
    urlcolor=cyan,
}
\renewcommand\thefootnote{\textcolor{blue}{\arabic{footnote}}}
% ============================================


%---------- Comandi specifici ----------------
\newcommand{\Sch}[1]{{\mathrm{Sch}}/{#1}}
\newcommand{\op}{^{op}}
\newcommand{\Set}{\mathrm{Set}}
\newcommand{\Psh}[2]{\left({\mathrm{Sch}}/{#1}\right)^{op}\to #2}
\newcommand{\Fun}{\mathrm{Fun}}
\newcommand{\Gr}{\mathrm{Gr}}
\newcommand{\Pl}{\mathrm{Pl}}
\newcommand{\Can}{{\mathcal{C}\mathrm{an}}}


%\usepackage[utf8]{inputenc}
%\DeclareFontFamily{U}{min}{}
%\DeclareFontShape{U}{min}{m}{n}{<-> udmj30}{}
%\newcommand\yo{\!\text{\usefont{U}{min}{m}{n}\symbol{'207}}\!}
\newcommand{\yo}{{h_\bullet}}


%--------- Comandi dattilografici ------------
\newcommand{\comm}[1]{}
\NewDocumentCommand{\bw}{O{k}}{{\bigwedge^{#1}}}

% ============================================
\title{Moduli Spaces and Grassmannians}

\author{Francesco Sorce}
\date{Università di Pisa\\
Dipartimento di Matematica}


\begin{document}
\maketitle

\begin{abstract}
In this document we introduce the concept of moduli spaces in algebraic geometry through the example of the Grassmannian scheme.

The first chapter introduces the basics of the functorial approach to algebraic geometry and its relation to moduli problems.

The second chapter is a quick overview of Grassmannians as defined set theoretically. We focus our attention on the Pl\"ucker embedding and prove that it identifies the Grassmannian with a projective variety.

In the third chapter we describe the reduced scheme structure on the Grassmannian and prove that it is a fine moduli space for the functor of quotients from $\Oc_T^n$ to a rank $k$ vector bundle on $T$.

%The last chapter is a brief overview of the construction of the Hilbert and Quot schemes and how they generalize Grassmannians.
\end{abstract}

%\newpage
\tableofcontents
\newpage

\chapter{Introduction}
The following type of \textit{classification problem} occurs often in math:
\begin{center}
Consider some type of object and a notion of isomorphism which can be defined between them. We are interested in understanding the behavior of isomorphism classes and how they relate to each other.
\end{center}
The set or class of isomorphism classes is a tautological answer to the set-theoretic question, but for an answer to a classification problem to be satisfactory we usually require it to encode some information on \textit{families} of isomorphism classes.\medskip

Miraculously, many such classification problems turn out to have a natural answer in the form of some geometric object. 
In general the object can only be defined as the category of families together with some geometric structure (this is the realm of the theory of stacks), but in more special circumstances one can find a more concrete space, usually a scheme, whose points represent isomorphism classes for our problem and whose geometric structure encodes information on the families. 
Such objects are called \textit{moduli spaces} for the classification problem.\medskip

The best result we can hope for is finding a space which completely encodes how families behave\footnote{what will be formalized as a fine moduli space}, but this requirement is usually too strict.
In this document we mostly deal with problems for which such a nice space exists: the Grassmannian, Quot and Hilbert schemes.


\section*{Historical background}
The history of moduli theory aligns remarkably well with that of the moduli space of smooth curves of fixed genus. Indeed the word moduli was introduced by Riemann in the article \cite{riemann54theorie} to denote what we would now call the dimension of $M_g$, the moduli space of smooth projective algebraic curves of genus $g$, which he computed to be $3g-3$. 

Although the argument given by Riemann can be made rigorous in modern language, he did not prove the existence of the space $M_g$ itself.
The first general construction of $M_g$ as a space of some kind can be attributed to Teichm\"uller, which realized $M_g$ as the quotient of the Teichm\"uller space $T_g$ parametrizing complex structures up to isomorphism on a surface of genus $g$ by the action of the group $\Gamma_g$ of diffeomorphisms of the surface up to isotopy. The paper which establishes these ideas is \cite{teichmuller1939extremale}.

The basis for the modern theory were laid by Alexander Grothendieck and his functorial approach.
He first introduced his methods to analytic moduli theory and later on to algebraic geometry in general.
Grothendieck was very interested in algebraic moduli theory and contributed to it greatly by introducing the Hilbert, Quot and Picard functors and showing their representability by schemes. However, Grothendieck did not end up publishing on $M_g$.

Among the first to study moduli spaces systematically was David Mumford.
Inspired by invariant theory, Grothendieck's functorial approach and the existing constructions of moduli spaces like the one of principally polarized abelian varieties or the Chow varieties, Mumford developed Geometric Invariant Theory (commonly referred to as GIT), which can be described as a method to study and construct moduli spaces as quotients of algebraic groups.
In the book \cite{GIT} Mumford gives two constructions of $M_g$ as a coarse moduli space.

For a more detailed history and more references see section 0.1 in \cite{Alper}.

\section*{Why category theory?}
As we briefly mentioned, the modern approach to moduli problems if formalized via functors. It might not be clear why this is the most appropriate tool, and indeed it can seem more complicated than more concrete treatments in simple cases like the classification of lines through a point via projective space.

Nevertheless, the functorial approach has proven itself to be effective in many aspects, chief among them the formalization of the nebulous concept of ``family" described above.\medskip

Following Grothendieck's ideas, a moduli problem is expressed as a contravariant functor
\[F:T\mapsto \quot{\cpa{\text{families of objects over $T$}}}\sim\]
where $\sim$ is the isomorphism relation imposed on families of objects.
Since we are mostly concerned about problems in algebraic geometry, and thus families over schemes, the functor is usually taken to be a presheaf on $\Sch S$ for some base scheme\footnote{usually $\Spec \K$ for an algebraically closed field $\K$ or $\Spec \Z$.} $S$.
To find the set of objects we want to classify up to isomorphism we can simply evaluate $F$ on a point.

The functorial language allows for families to be \textit{pulled back} via morphisms: if $f:S\to T$ is a morphism and $a\in F(T)$ is a family over $T$, then $F(f):F(T)\to F(S)$ by contravariance and thus $F(f)(a)\doteqdot f^\ast a\in F(S)$ is a family over $S$.
\medskip

There are several ways in which we can define a moduli space. The two most relevant are \textit{fine} and \textit{coarse} moduli spaces. A scheme $M$ is a fine moduli space if we can recover the whole moduli functor from it\footnote{formally, when $h_M$ and $F$ are naturally isomorphic functors.}. $M$ is a coarse moduli space if its $\K$-points are in bijection with $F(\Spec \K)$, all families over $T$ induce a morphism $T\to M$ which behaves well with pullbacks and $M$ is universal for these properties.

In both cases we can interpret a family of objects over a scheme $T$ as a morphism from $T$ to $M$ called \textit{classifying map}.  Intuitively this is the function that to each point of $T$ assigns the corresponding isomorphism class. The added structure of a scheme morphism serves to define a ``niceness" on families. If $M$ is a fine moduli space, then every family over $T$ can be viewed as the pullback under a morphism $T\to M$ of a specific family $u\in F(M)$, called the \textit{universal family}.

\section*{Why Grassmannians?}
Grassmannians are among the first nontrivial examples of spaces whose points represent some type of object one can encounter in their mathematical career. Given two positive integers $k$ and $n$, the first definition of a Grassmannian $\Gr(k,n)$ one encounters is
\[\Gr'(k,n)=\cpa{H\subseteq\K^n\mid \text{$H$ vector subspace, }\dim_\K H=k}.\]
This definition invites us to think about the classification problem of $k$-dimensional vector subspaces of $n$-dimensional space. This classification problem is best formalized in terms vector bundle quotients as
\[\gr(k,n):\functorDef{(\Sch\K)\op}{\Set}{T}{\quot{\cpa{\al:\Oc_T^n\onto Q}}\sim}{f:S\to T}{(\al:\Oc_T^n\to Q)\mapsto (f^\ast\al:\Oc_S^n\to f^\ast Q)}\]
where $q\sim q'\coimplies \ker q=\ker q'$, so the definition we will use for $\Gr(k,n)$ is actually
\[\Gr(k,n)=\quot{\cpa{\vp:\K^n\to \K^k\mid \rnk \vp=k}}\sim\qquad \text{where }\vp\sim \psi\coimplies \ker\vp=\ker\psi,\]
but the two are related, up to canonical identifications, by $\Gr(k,n)=\Gr'(n-k,n)$.
Showing that Grassmannians are schemes and that they are fine moduli spaces for this classification problem is a good introduction to the elementary tools of the theory of fine moduli spaces. Grassmannians also serve as a warm up and necessary stepping stone in the construction of the Quot schemes, which generalize Grassmannians and yield important results like the existence of Hilbert schemes.
\chapter{Moduli Spaces}

In this chapter we introduce the basic category theory used in the study of moduli spaces. After a quick review of the Yoneda embedding, we define representability of a functor and give the definition of fine and coarse moduli space. After that we give a quick overview of Zariski sheaves and prove representability results that we will need in the third chapter.\medskip

We adopt the following conventions:
\begin{itemize}
\item All categories considered in this document will be small.
\item If $\Cc$ is a category, we shall write $X\in \Cc$ to mean ``$X$ is an object in $\Cc$".
\item If $A,B\in \Cc$, we denote the set of morphisms from $A$ to $B$ with $\Hom(A,B)$ or $\Hom_\Cc(A,B)$ for specificity.
\item If $A$ and $B$ are $R$-modules we write $\Hom_R(A,B)$ instead of $\Hom_{R\text{-Mod}}(A,B)$.
\end{itemize}
Most definitions given in this chapter follow section 0.3 of \cite{Alper}.


\section{Yoneda lemma}
\begin{definition}[Presheaf]
A contravariant functor $F:\Cc\op\to \Set$ is called a \textbf{presheaf} on $\Cc$. If $T\in \Cc$ then we call the elements of $F(T)$ \textbf{families} over $T$.
\end{definition}
\begin{definition}[Presheaf category]
For any fixed category $\Cc$, the presheaves on $\Cc$ form a category $\Fun(\Cc\op,\Set)$ with morphisms given by natural transformations.
\end{definition}

\begin{definition}[Hom-functor]
Let $\Cc$ be a category and $X\in \Cc$. We define the \textbf{Hom-functor} of $X$ to be
\[h_X:\functorDef{\Cc\op}{\Set}{T}{\Hom(T,X)}{f:T\to S}{h_X(f):\funcDef{\Hom(S,X)}{\Hom(T,X)}{g}{g\circ f}}\]
\end{definition}
\begin{remark}
The Hom-functor is a presheaf.
\end{remark}

\begin{lemma}[Yoneda Lemma]\label{YonedaLemma}
Let $\Cc$ be a category and $X\in \Cc$. If $F$ is a presheaf on $\Cc$ then the following sets are in a natural bijection
\[\Hom(h_X,F)\longleftrightarrow F(X).\]
\end{lemma}
\begin{proof}
Given a natural transformation $\zeta$, we can take its image in $F(X)$ to be $\zeta_X(id_X)$.
On the other hand, for any given element $u\in F(X)$ we can define an arrow $h_X(T)\to F(T)$ for any $T\in \Cc$ by taking $f\mapsto F(f)(u)$. This collection of maps defines a natural transformation from $h_X$ to $F$ because for all $g:S\to T$ and for all $f\in h_X(T)$
\[F(g)(F(f)(u))=(F(g)\circ F(f))(u)=F(f\circ g)(u)=F(h_X(g)(f))(u).\]
To conclude it is enough to check that the two assignments are inverses:
\[F(f)(\zeta_X(id_X))=\zeta_T(h_X(f)(id_X))=\zeta_T(f),\quad F(id_X)(u)=u.\]
\end{proof}

\begin{definition}[Yoneda embedding]
We define the \textbf{Yoneda embedding} of a category $\Cc$ to be the following functor
\[\yo:\functorDef{\Cc}{\Fun(\Cc\op,\Set)}{X}{h_X}{f:X\to Y}{h_f:h_X\to h_Y}\]
where if $g:T\to X$ then $h_f(g)=f\circ g:T\to Y$.
\end{definition}


\begin{proposition}
The functor $\yo$ is fully faithful.
\end{proposition}
\begin{proof}
Recall that a functor $F:\Cc\to \Dc$ is fully faithful if for any two objects $A,B\in\Cc$ we have $\Hom_\Cc(A,B)\cong \Hom_\Dc(F(A),F(B))$. In our case we want to verify that
\[\Hom(X,Y)\cong\Hom(h_X,h_Y),\]
which is exactly the statement of the Yoneda lemma (\ref{YonedaLemma}) for $F=h_Y$.
\end{proof}
\begin{proposition}\label{YonedaEmbeddingInjectiveOnIsoClasses}
The Yoneda embedding is injective on isomorphism classes of objects in $\Cc$.
\end{proposition}
\begin{proof}
A natural isomorphism $\zeta:h_A\to h_B$ and its inverse $\zeta'$ correspond to maps $f:A\to B$ and $f':B\to A$ via the Yoneda lemma. Note that
\[\yo(f\circ f')=h_{f\circ f'}=h_f\circ h_{f'}=h_B(\cdot)(f)\circ h_A(\cdot)(f')\overset{\text{Yoneda}}=\zeta\circ \zeta'=id_{h_B},\]
thus, because $\yo$ if fully faithful, we see that $f\circ f'=id_B$. An analogous argument works for $f'\circ f$.
\end{proof}

\begin{lemma}\label{YonedaEmbeddingPreservesLimits}
The Yoneda embedding preserves limits.
\end{lemma}
\begin{proof}
Suppose $X$ is the limit of the diagram $\cpa{f_{ij}:X_j\to X_i}$. If we apply the Yoneda embedding to the diagram we obtain
\[\cpa{h_{f_{ij}}:h_{X_j}\to h_{X_i}}\]
Let $F$ be any presheaf on $\Cc$ and suppose that we have morphisms $F\to h_{X_i}$ which make the diagrams commute, then for all $T\in \Cc$ we have compatible and natural $F(T)\to \Hom(T,X_i)$. 
If $f\in F(T)$ then these arrows define several $f_i\in \Hom(T,X_i)$ which compose with the $f_{ij}$ respecting the diagram. 
By the universal property of limits this defines uniquely a morphism $f_\ell\in \Hom(T,X)$ and we see that the assignment $f\mapsto f_\ell$ is the unique map from $F(T)$ to $\Hom(T,X)$ which makes the diagram in $\Set$ commute. 
Since all that we have done is natural in $T$, we have effectively constructed a morphism $F\to h_X$ as we desired.
\end{proof}



\section{Moduli problems}

\begin{definition}[Representable functor]
A presheaf $F$ on $\Cc$ is \textbf{representable} if there exists a natural isomorphism $\zeta:F\to h_X$ for some $X\in \Cc$. 
In this case we say that the pair\footnote{usually we just say that $X$ represents $F$} $(X,\zeta)$ \textbf{represents} $F$.
If $a\in F(T)$ we call $\zeta_T(a):T\to X$ the \textbf{classifying map} of the family $a$.
\end{definition}
\begin{definition}[Universal family]
Given a functor $F$ and an object $X\in \Cc$ that represents it via the isomorphism $\zeta:F\to h_X$, the \textbf{universal family} of $X$ is
\[\zeta\ii_X(id_X)\in F(X).\]
\end{definition}
\begin{remark}
The universal family is the element of $F(X)$ which corresponds to $\zeta\ii$ under the Yoneda lemma (\ref{YonedaLemma}).
\end{remark}

We now specify our study to the category of schemes:

\begin{definition}[Moduli problem]
Let $S$ be a scheme. A presheaf on $\Sch S$ is called a \textbf{moduli problem} or \textbf{moduli functor}.
\end{definition}

A classical example of moduli problem is
\begin{example}[Moduli problem of smooth curves of fixed genus]
A \textit{family of smooth curves of genus $g$} over a scheme $S$ is a smooth, proper and finitely presented scheme morphism $C\to S$ such that for all $s\in S$ the fiber $C_s$ is a connected, smooth and proper curve of genus $g$. The moduli problem of smooth curves of genus $g$ is the functor
\[F_{M_g}:\functorDef{\Sch\C\op}{\Set}{S}{\quot{\cpa{\text{families of smooth curves of genus $g$ over $S$}}}\sim}{T\to S}{(C\to S)\mapsto (C\times_S T\to T)}\]
where two families $C\to S$ and $C'\to S$ are equivalent if there exists an isomorphism between $C$ and $C'$ which is compatible with the structure maps over $S$.
\end{example}

\begin{definition}[Fine moduli space]
Let $F$ be a moduli functor. A scheme $X\in \Sch S$ is a \textbf{fine moduli space} for $F$ if $X$ represents $F$.
\end{definition}

\begin{remark}
Because of proposition (\ref{YonedaEmbeddingInjectiveOnIsoClasses}), fine moduli spaces are unique up to isomorphism.
\end{remark}

\begin{example}[Projective space]
Consider the functor $\Pj_n$ given by
\[\functorDef{\mathrm{Sch}\op}{\Set}{S}{\quot{\cpa{(\Lc,s_0,\cdots,s_n)\mid \emat{\Lc\text{ line bundle on $S$,}s_0,\cdots,s_n\in\Lc(S),\\\forall x\in S,\ \ps{(s_0)_x,\cdots,(s_n)_x}_{\Oc_{S,x}}=\Lc_x}}}\sim}{f}{\text{pullback of sheaves and sections via $f$}}\]
where $(\Lc,(s_i))\sim (\Lc',(s_i'))$ is there exists a sheaf isomorphism $\al:\Lc\to\Lc'$ such that $s_i=\al^\ast s_i'$ for all $i\in\cpa{0,\cdots, n}$.\medskip

It is a well known property of projective spaces (see Proposition 5.1.31 in \cite{QingLiu}) that $\Pj_n(S)\cong \Hom(S,\Pj^n_\Z)$ and that pullbacks behave as expected, thus $\Pj^n_\Z$ is a fine moduli space for $\Pj_n$.
From the statement of Proposition 5.1.31 in \cite{QingLiu} it is also clear that $\Oc_{\Pj^n_\Z}(1)$ is a universal family.
\end{example}

Fine moduli spaces do not always exist. The simplest obstructions to having a fine moduli spaces are
\begin{itemize}
\item the functor is not a Zariski sheaf (see proposition (\ref{RepresentableModuliFunctorsAreZariskiSheaves}))
\item existence of non trivial automorphisms.
\end{itemize}

To get an idea for why the second condition is an obstruction we cite the following
\begin{proposition}
Let $F\in \Psh{\C}{\Set}$ be a moduli functor. If there exists a variety $S\in \Sch \C$ such that $\Ec\in F(S)$ is an \textbf{isotrivial family}, i.e.
\begin{itemize}
\item for all $s,t\in S(\C)$, the fiber $F(s)(\Ec)=\Ec_s=\Ec_t$ and
\item the family $\Ec$ is not the pullback of an object $E\in F(\Spec \C)$ along the structural morphism $S\to \Spec \C$,
\end{itemize}
then there exists no fine moduli space for $F$.
\end{proposition}
\begin{proof}
This is Proposition 0.3.28 in \cite{Alper}.
\end{proof}
\begin{remark}
This proposition can be used to show that $F_{M_g}$ is not representable.
\end{remark}

A weaker notion of moduli space is that of coarse moduli space:
\begin{definition}[Coarse moduli space]
Let $F$ be a moduli problem. A pair $(X,\zeta)$ for $X\in \Sch S$ and $\zeta:F\to h_X$ natural transformation is a \textbf{coarse moduli space} for $F$ if
\begin{itemize}
\item $\zeta_{\Spec \K}:F(\Spec\K)\to \Hom(\Spec\K,X)$ is a bijection for all algebraically closed fields $\K$
\item for any scheme $Y$ and $\eta:F\to h_{Y}$ natural transformation there exists a unique morphism $\al:X\to Y$ such that $\eta=h_\al\circ \zeta$.
\end{itemize}
\end{definition}
\begin{proposition}
A fine moduli space is also a coarse moduli space.
\end{proposition}
\begin{proof}
The first condition is trivially verified. For the second condition, if $(Y,\eta)$ is defined as above and $(X,u)$ is the fine moduli space with universal family $u$ then we can take $\al=\eta_X(u)$.
\end{proof}

\begin{remark}
There exists a coarse moduli space $M_g$ for the moduli problem $F_{M_g}$. This is a classic result in geometric invariant theory, see \cite{GIT}.
\end{remark}


\section{Zariski sheaves and gluing of fine moduli spaces}
One approach to show representability of a moduli problem is emulating the gluing properties of sheaves.
Indeed it is possible to show that representable functors are sheaves of some kind. This realization will lead to some results that aid in showing representability. 

\subsection{Zariski sheaves}
First, let us formalize a way in which a functor can be a sheaf. First we recall the definition of equalizer:

\begin{definition}[Equalizer]
Let $\Cc$ be a category, $A,B,C\in \Cc$ and $f,g:B\to C$. We say that the the diagram 
% https://q.uiver.app/#q=WzAsMyxbMCwwLCJBIl0sWzEsMCwiQiJdLFsyLDAsIkMiXSxbMCwxLCJoIl0sWzEsMiwiZiIsMCx7Im9mZnNldCI6LTF9XSxbMSwyLCJnIiwyLHsib2Zmc2V0IjoxfV1d
\[\begin{tikzcd}
	A & B & C
	\arrow["h", from=1-1, to=1-2]
	\arrow["f", shift left, from=1-2, to=1-3]
	\arrow["g"', shift right, from=1-2, to=1-3]
\end{tikzcd}\]
is an \textbf{equalizer} if $h:A\to B$ is such that $f\circ h=g\circ h$ and if $(Q,q)$ is another such pair then there exists a unique morphism $Q\to A$ which makes the diagram commute
\[\begin{tikzcd}
	A & B & C \\
	Q
	\arrow["h", from=1-1, to=1-2]
	\arrow["f", shift left, from=1-2, to=1-3]
	\arrow["g"', shift right, from=1-2, to=1-3]
	\arrow["q"', from=2-1, to=1-2]
	\arrow[dashed, from=2-1, to=1-1]
\end{tikzcd}\]
\end{definition}

\begin{definition}[Zariski sheaf]
A moduli problem $F\in \Psh S\Set$ is a \textbf{Zariski sheaf} if for any $S$-scheme $X$ and any Zariski open cover $\cpa{U_i\to X}$ the following diagram is an equalizer
% https://q.uiver.app/#q=WzAsMyxbMCwwLCJGKFgpIl0sWzEsMCwiXFxkaXNwbGF5c3R5bGUgXFxwcm9kX2sgRihVX2spIl0sWzIsMCwiXFxkaXNwbGF5c3R5bGUgXFxwcm9kX3tpLGp9RihVX2lcXGNhcCBVX2opIl0sWzAsMV0sWzEsMiwiIiwwLHsib2Zmc2V0IjotMX1dLFsxLDIsIiIsMix7Im9mZnNldCI6MX1dXQ==
\[\begin{tikzcd}
	{F(X)} & {\displaystyle \prod_k F(U_k)} & {\displaystyle \prod_{i,j}F(U_i\cap U_j)}
	\arrow[from=1-1, to=1-2]
	\arrow[shift left, from=1-2, to=1-3]
	\arrow[shift right, from=1-2, to=1-3]
\end{tikzcd}\]
where the arrows are induced by the inclusions.
\end{definition}

\begin{remark}
Using the Yoneda lemma (\ref{YonedaLemma}), we may equivalently consider
% https://q.uiver.app/#q=WzAsMyxbMCwwLCJcXEhvbShoX1gsRikiXSxbMSwwLCJcXGRpc3BsYXlzdHlsZSBcXHByb2RfayBcXEhvbShoX3tVX2t9LEYpIl0sWzIsMCwiXFxkaXNwbGF5c3R5bGUgXFxwcm9kX3tpLGp9XFxIb20oaF97VV9pXFxjYXAgVV9qfSxGKSJdLFswLDFdLFsxLDIsIiIsMCx7Im9mZnNldCI6LTF9XSxbMSwyLCIiLDIseyJvZmZzZXQiOjF9XV0=
\[\begin{tikzcd}
	{\Hom(h_X,F)} & {\displaystyle \prod_k \Hom(h_{U_k},F)} & {\displaystyle \prod_{i,j}\Hom(h_{U_i\cap U_j},F)}
	\arrow[from=1-1, to=1-2]
	\arrow[shift left, from=1-2, to=1-3]
	\arrow[shift right, from=1-2, to=1-3]
\end{tikzcd}\]
\end{remark}


\begin{proposition}[Representable moduli functors are Zariski sheaves]\label{RepresentableModuliFunctorsAreZariskiSheaves}
Let $F$ be a moduli problem, then if there exists a fine moduli space $M$ for $F$ it must be the case that $F$ is a Zariski sheaf.
\end{proposition}
\begin{proof}
Up to composing with the natural isomorphism, we may assume $F=h_M$. 
Let $X$ be an $S$-scheme and $\cpa{U_i\to X}$ a Zariski open cover for it. 
We want to show that the following diagram is an equalizer
% https://q.uiver.app/#q=WzAsMyxbMCwwLCJcXEhvbShVLE0pIl0sWzEsMCwiXFxkaXNwbGF5c3R5bGUgXFxwcm9kX2kgXFxIb20oVV9pLE0pIl0sWzIsMCwiXFxkaXNwbGF5c3R5bGUgXFxwcm9kX3tpLGp9XFxIb20oVV9pXFxjYXAgVV9qLE0pIl0sWzAsMSwiUmVzIl0sWzEsMiwicHJfMV5cXGFzdCIsMCx7Im9mZnNldCI6LTF9XSxbMSwyLCJwcl8yXlxcYXN0IiwyLHsib2Zmc2V0IjoxfV1d
\[\begin{tikzcd}
	{\Hom(U,M)} & {\displaystyle \prod_i \Hom(U_i,M)} & {\displaystyle \prod_{i,j}\Hom(U_i\cap U_j,M)}
	\arrow["Res", from=1-1, to=1-2]
	\arrow["{pr_1^\ast}", shift left, from=1-2, to=1-3]
	\arrow["{pr_2^\ast}"', shift right, from=1-2, to=1-3]
\end{tikzcd}\]
The arrows correspond to restriction of morphisms, so what we need to verify is that
\begin{itemize}
\item $\mathrm{res}^{U_i}_{U_i\cap U_j}\circ \mathrm{res}^X_{U_i}=\mathrm{res}^{U_j}_{U_i\cap U_j}\circ \mathrm{res}^X_{U_j}$ and that
\item a collection of maps $\cpa{f_i:U_i\to M}$ such that $f_i\res{U_i\cap U_j}=f_j\res{U_i\cap U_j}$ glues uniquely to a map $f:X\to M$.
\end{itemize}
Both propositions are well known properties of scheme morphisms.
\end{proof}

\subsection{Open cover of a moduli problem}
\begin{definition}[Subfunctor]
A functor $G:\Cc\to \Set$ is a \textbf{subfunctor} of $F:\Cc\to \Set$ if for all $X,A,B\in \Cc$ and for all $f\in \Hom(A,B)$
\[G(X)\subseteq F(X),\qquad\text{and}\qquad G(f)=F(f)\res{G(A)}.\]
In this case we write $G\subseteq F$.
\end{definition}
\begin{remark}
If $F$ and $G$ are presheaves and $f:A\to B$ then $G(f)=F(f)\res{G(B)}$.
\end{remark}

\begin{definition}[Fibered product of presheaves]
Let $F,G,H:\Cc\op\to\Set$ be presheaves together with two natural transformations $\eta:F\to H$ and $\zeta:G\to H$. We define their fibered product as the following functor
\[F\times_H G:\functorDef{\Cc\op}{\Set}{X}{F(X)\times_{H(X)}G(X)}{f:A\to B}{(b_1,b_2)\mapsto(F(f)(b_1),G(f)(b_2))}\]
where the fibered product $F(X)\times_{H(X)}G(X)$ in defined through the maps $\eta_X$ and $\zeta_X$. The map $(F\times_H G)(f)$ is well defined because if $(b_1,b_2)\in F(B)\times_{H(B)}G(B)$ then 
\[\eta_A(F(f)(b_1))=H(f)(\eta_B(b_1))\pasgnlmath={\eta_B(b_1)=\zeta_B(b_2)}H(f)(\zeta_B(b_2))=\zeta_A(G(f)(b_2)).\]
\end{definition}

\begin{definition}[Functor immersions]
Let $\zeta:G\to F$ be a natural transformation of moduli problems. $\zeta$ is an \textbf{open immersion} if $\zeta_T$ is injective for every scheme $T\in\Sch S$ and for every natural transformation $h_T\to F$ there is an open subscheme $U\subseteq T$ such that
% https://q.uiver.app/#q=WzAsNixbMSwwLCJoX3tVfSJdLFsxLDEsImhfVCJdLFsyLDEsIkYiXSxbMiwwLCJHIl0sWzAsMCwiVSJdLFswLDEsIlQiXSxbMywyXSxbMSwyXSxbMCwzLCIiLDAseyJzdHlsZSI6eyJib2R5Ijp7Im5hbWUiOiJkYXNoZWQifX19XSxbMCwxLCIiLDIseyJzdHlsZSI6eyJib2R5Ijp7Im5hbWUiOiJkYXNoZWQifX19XSxbMCwyLCIiLDEseyJzdHlsZSI6eyJuYW1lIjoiY29ybmVyLWludmVyc2UifX1dLFs0LDUsIlxcc3Vic2V0ZXEiLDMseyJzdHlsZSI6eyJib2R5Ijp7Im5hbWUiOiJub25lIn0sImhlYWQiOnsibmFtZSI6Im5vbmUifX19XSxbNCwwLCJcXHlvIl0sWzUsMSwiXFx5byJdXQ==
\[\begin{tikzcd}
	U & {h_{U}} & G \\
	T & {h_T} & F
	\arrow["\yo", from=1-1, to=1-2]
	\arrow["\subseteq"{marking, allow upside down}, draw=none, from=1-1, to=2-1]
	\arrow[dashed, from=1-2, to=1-3]
	\arrow[dashed, from=1-2, to=2-2]
	\arrow["\ulcorner"{anchor=center, pos=0.125}, draw=none, from=1-2, to=2-3]
	\arrow[from=1-3, to=2-3]
	\arrow["\yo", from=2-1, to=2-2]
	\arrow[from=2-2, to=2-3]
\end{tikzcd}\]
We define \textbf{closed immersions} and \textbf{locally closed immersions} analogously.
\end{definition}

Because of the Yoneda lemma, giving a natural transformation like in the above definition is equivalent to choosing a family $\xi\in F(T)$. We can thus rephrase the definition as follows

\begin{definition}[Functor immersions v.2]
Let $\zeta:G\to F$ be a natural transformation of moduli problems. $\zeta$ is an \textbf{open immersion} if $\zeta_T$ is injective for every scheme $T\in\Sch S$ and for every $\xi\in F(T)$ there exists an open subscheme $\iota:U\inj T$ such that the following diagram is natural in $R$ for all $R\in \Sch S$, commutes and is cartesian\footnote{for any map $f:R\to U$ there exists a $g:R\to U$ such that $f=\iota\circ g$ if and only if $F(f)(\xi)\in G(R)$.}
% https://q.uiver.app/#q=WzAsNCxbMCwwLCJcXEhvbShSLFUpIl0sWzAsMSwiXFxIb20oUixUKSJdLFsyLDAsIkcoUikiXSxbMiwxLCJGKFIpIl0sWzAsMSwiaF9cXGlvdGEiLDIseyJzdHlsZSI6eyJ0YWlsIjp7Im5hbWUiOiJob29rIiwic2lkZSI6InRvcCJ9fX1dLFswLDIsIkdcXGNpcmMgaF9cXGlvdGEoXFxjZG90KShcXHhpKSJdLFsxLDMsIkYoXFxjZG90KShcXHhpKSIsMl0sWzIsMywiXFx6ZXRhX1IiLDAseyJzdHlsZSI6eyJ0YWlsIjp7Im5hbWUiOiJob29rIiwic2lkZSI6InRvcCJ9fX1dLFswLDYsIiIsMSx7ImxldmVsIjoxLCJzdHlsZSI6eyJuYW1lIjoiY29ybmVyLWludmVyc2UifX1dXQ==
\[\begin{tikzcd}
	{\Hom(R,U)} && {G(R)} \\
	{\Hom(R,T)} && {F(R)}
	\arrow["{G\circ h_\iota(\cdot)(\xi)}", from=1-1, to=1-3]
	\arrow["{h_\iota}"', hook, from=1-1, to=2-1]
	\arrow["{\zeta_R}", hook, from=1-3, to=2-3]
	\arrow[""{name=0, anchor=center, inner sep=0}, "{F(\cdot)(\xi)}"', from=2-1, to=2-3]
	\arrow["\ulcorner"{anchor=center, pos=0.125}, draw=none, from=1-1, to=0]
\end{tikzcd}\]
\textbf{Closed immersions} and \textbf{locally closed immersions} of moduli problems are defined in the same way.
\end{definition}


\begin{definition}[Open subfunctor]
Let $F:\Psh S\Set$ be a moduli problem. We say that a subfunctor $G\subseteq F$ is \textbf{open} if the natural transformation given by the inclusion is an open immersion.
\end{definition}

\begin{definition}[Open cover of a functor]
Let $F:\Psh S\Set$ be a moduli problem. A collection of open subfunctors $\cpa{F_i\to F}$ is an \textbf{open cover} of $F$ if for any $S$-scheme $T$ and any natural transformation $h_T\to F$, the open subschemes $U_i$ of $T$ determined by the $F_i$ form an open cover of $T$.
\end{definition}



\begin{definition}[Restriction of a family]
If $U$ is a subscheme of $T$ and $\iota:U\to T$ is the inclusion morphism, then if $\xi\in F(T)$ we define its \textbf{restriction} to $U$ to be
\[\xi\res U=F(\iota)(\xi).\]
\end{definition}

\begin{remark}
If $\cpa{F_i\to F}$ is an open cover of the functor $F$ then for any $S$-scheme $T$ and any family $\xi\in F(T)$ there exists an open cover $\cpa{U_i\to T}$ of $T$ such that $\xi\res{U_i}\in F_i(U_i)$ for all $i$.
\end{remark}

\subsection{Representability criterion}
Finally, we come to the main results of this chapter

\begin{proposition}\label{MapGluingForZariskiSheaves}
Let $F$ and $G$ be Zariski sheaves, $\cpa{F_i\to F}$ and $\cpa{G_i\to G}$ be open covers with the same indexation and $f_i:F_i\to G_i$ be natural transformations such that\footnote{for a natural transformation $\zeta:F\to H$ and a subfunctor $G\subseteq F$, we define $\zeta\res G$ as the natural transformation $G\to H$ given by $(\zeta\res G)_T=\zeta_T\res{G(T)}$. Naturality follows from the naturality of $f$ and the definition of subfunctor.} $f_i\res{F_i\cap F_j}=f_j\res{F_i\cap F_j}$. 
Then there exists a natural transformation $f:F\to G$ which restricts to $f_i$ on $F_i$.
\end{proposition}
\begin{proof}
Let $T$ be a scheme and $\zeta:h_T\to F$ a natural transformation. Let $\cpa{\iota_i:U_i\to T}$ be the open cover induced by $\cpa{F_i\to F}$ through $\zeta$ by the definition of open subfunctor cover.
% https://q.uiver.app/#q=WzAsNixbMSwxLCJGIl0sWzEsMCwiRl9pIl0sWzIsMCwiR19pIl0sWzIsMSwiRyJdLFswLDEsImhfVCJdLFswLDAsImhfe1VfaX0iXSxbMSwyLCJmX2kiXSxbMSwwLCJcXHN1YnNldGVxIiwzLHsic3R5bGUiOnsiYm9keSI6eyJuYW1lIjoibm9uZSJ9LCJoZWFkIjp7Im5hbWUiOiJub25lIn19fV0sWzIsMywiXFxzdWJzZXRlcSIsMyx7InN0eWxlIjp7ImJvZHkiOnsibmFtZSI6Im5vbmUifSwiaGVhZCI6eyJuYW1lIjoibm9uZSJ9fX1dLFs1LDQsImhfe1xcaW90YV9pfSIsMl0sWzQsMCwiXFx6ZXRhIiwyXSxbNSwxLCJcXGV0YV9pIl0sWzUsMCwiIiwyLHsic3R5bGUiOnsibmFtZSI6ImNvcm5lci1pbnZlcnNlIn19XV0=
\[\begin{tikzcd}
	{h_{U_i}} & {F_i} & {G_i} \\
	{h_T} & F & G
	\arrow["{\eta_i}", from=1-1, to=1-2]
	\arrow["{h_{\iota_i}}"', from=1-1, to=2-1]
	\arrow["\ulcorner"{anchor=center, pos=0.125}, draw=none, from=1-1, to=2-2]
	\arrow["{f_i}", from=1-2, to=1-3]
	\arrow["\subseteq"{marking, allow upside down}, draw=none, from=1-2, to=2-2]
	\arrow["\subseteq"{marking, allow upside down}, draw=none, from=1-3, to=2-3]
	\arrow["\zeta"', from=2-1, to=2-2]
\end{tikzcd}\]
where $\eta_i$ is the map $\zeta\circ h_{\iota_i}$ with its codomain restricted. This map is well defined because the square is cartesian.
Let $g_i=f_i\circ \eta_i$ and note that
\[g_i\res{h_{U_i\cap U_j}}=f_i\res{F_i\cap F_j}\circ \eta_i \res{h_{U_i\cap U_j}}=f_j\res{F_i\cap F_j}\circ \eta_j\res{h_{U_i\cap U_j}}=g_j\res{h_{U_i\cap U_j}}.\]
Because $G$ is a Zariski sheaf, there exists $\zeta':h_T\to G$ such that $\zeta'\circ h_{\iota_i}=g_i$. We have thus constructed a map $\Hom(h_T,F)\to \Hom(h_T,G)$ which is functorial in $T$ by naturality of the maps involved. Applying the Yoneda lemma (\ref{YonedaLemma}) gives a map $F(T)\to G(T)$ which is functorial in $T$, i.e. $f:F\to G$. By construction it is also clear that $f\res{F_i}=f_i$.
\end{proof}
\begin{corollary}\label{IsomorphismGluingForZariskiSheaves}
With the same setup as above, if each $f_i$ is an isomorphism then $f$ too is an isomorphism.
\end{corollary}
\begin{proof}
Let $f$ be the morphism $F\to G$ obtained as above and let $g:G\to F$ be the morphism obtained the same way but by gluing the $f_i\ii:G_i\to F_i$. It is easy to see that $f$ and $g$ are inverses with a local argument.
\end{proof}


\begin{theorem}[Representability by open cover]\label{RepresentabilityByOpenSubfunctorCover}
Let $F:\Psh{S}{\Set}$ be a Zariski sheaf and let $\cpa{F_i\to F}$ be an open cover of it by representable subfunctors, then $F$ is representable.
\end{theorem}
\begin{proof}[Sketch.]
We fix schemes $X_i$ and families $\xi_i\in F_i(X_i)$ such that $(X_i,\xi_i)$ is a fine moduli space for $F_i$. For all $S$-schemes $T$ we have
\[(F_i\times_F F_j)(T)=F_i(T)\times_{F(T)}F_j(T)=F_i(T)\cap F_j(T)\subseteq F(T),\]
thus $F_i\times_F F_j=F_j\times_F F_i\doteqdot F_{i,j}$. 

Using the openness of $F_j$ we find $U_{ij}\subseteq X_i$ which represents $h_{X_i}\times_FF_j\cong F_{i,j}$. By uniqueness of moduli spaces we see that there exists an isomorphism $\vp_{ji}:U_{ij}\to U_{ji}$, which we can choose to correspond to the identity $F_{i,j}= F_{j,i}$.

Our choice for the maps $\vp_{ji}$ makes the cocycle condition $\vp_{ki}=\vp_{kj}\circ \vp_{ji}$ hold trivially. We can thus glue the $X_i$ to a scheme $X$. Since $\xi_i\res{U_ij}=\xi_j\res{U_{ij}}$ by construction of $\vp_{ji}$, we find a family $\xi\in F(X)$ by the sheaf property of $F$. It follows easily that $(X,\xi)$ represents $F$.
\end{proof}


\chapter{Classical Grassmannians}

\section{First definitions and conventions}
\begin{definition}[Grassmannian]
Let $k\leq n$ be a pair of positive integers. We define the \textbf{$(n,k)$-Grassmannian}, denoted\footnote{we shall often omit the field when clear from context} $\Gr(k,n,\K)$, as the set of $(n-k)$-dimensional $\K$-vector subspaces of $\K^n$.
\end{definition}

\begin{remark}
We may equivalently define $\Gr(k,n)$ to be the following set:
\[\cpa{\ker \vp\mid \vp\in \Hom_\K(\K^n,\K^k),\ \rnk \vp=k}.\]
\end{remark}
\begin{proof}
If $H\in \Gr(k,n)$, let $z_1,\cdots, z_n$ be a basis of $\K^n$ such that $H=\Span(z_1,\cdots, z_{n-k})$ and let $e_1,\cdots, e_k$ be any basis of $\K^k$, then we can view $H$ as the kernel of the (rank $k$) linear map defined by
\[\vp(z_i)=\begin{cases}
0 &\text{if }i\leq n-k\\
e_{i-n+k} &\text{otherwise}
\end{cases}\]
On the other hand, if $\vp$ is a rank $k$ linear map then by the rank-nullity theorem its kernel is an $n-k$ dimensional subspace of $\K^n$.
\end{proof}

\begin{lemma}\label{kerAkerBVSActionOfGLk}
Let $\vp,\psi\in \Hom_\K(\K^n,\K^k)$ be linear maps of full rank. The following conditions are equivalent:
\begin{enumerate}
    \item $\ker\vp=\ker\psi$,
    \item there exists $\theta\in \GL(\K^k)$ such that $\vp=\theta\circ \psi$. 
\end{enumerate}
\end{lemma}
\begin{proof}
Let us prove both implications:
\setlength{\leftmargini}{0cm}
\begin{itemize}
\item[$\boxed{2.\implies 1.}$] $\ker(\theta\circ \psi)=\psi\ii(\ker \theta)=\psi\ii(\cpa{0})=\ker \psi$. 
\item[$\boxed{1.\implies 2.}$] Let $z_1,\cdots, z_n$ be a basis of $\K^n$ such that $\ker\vp=\ker\psi=\Span(z_1,\cdots, z_{n-k})$. By construction $\vp(z_{n-k+1}),\cdots, \vp(z_n)$ and $\psi(z_{n-k+1}),\cdots, \psi(z_n)$ are bases of $\K^k$. Let $\theta$ be the linear automorphism of $\K^k$ determined by $\theta(\psi(z_i))=\vp(z_i)$ for all $n-k<i\leq n$. By construction $\theta$ is nonsingular and $\vp$ agrees with $\theta\circ \psi$ on a basis of $\K^n$.
\end{itemize}
\setlength{\leftmargini}{0.5cm}
\end{proof}

\begin{corollary}\label{LinearQuotientDefinition}
We may redefine Grassmannians in terms of linear maps as follows:
\[\Gr(k,n)=\quot{\cpa{\vp\in \Hom_\K(\K^n,\K^k)\mid \vp\ \text{surjective.}}}\sim\]
where $\vp\sim \psi$ if and only if $\exists \theta\in \GL(\K^k)$ such that $\vp=\theta\circ \psi$.
\end{corollary}

\noindent We end the prelude of this chapter by introducing some notation and conventions
\begin{definition}[Multiindicies]
We define a \textbf{$(k,n)$-multiindex} as an element of $\cpa{1,\cdots, n}^k$. Our notation for a multiindex $I$ will usually be $I=(i_1,\cdots, i_k)$.\\
We denote the set of \textbf{ordered $(k,n)$-multiindicies} with
\[\omega(k,n)=\cpa{(i_1,\cdots, i_k)\in \cpa{1,\cdots, n}^k\mid i_1<\cdots<i_k}.\]
If $A$ is a $k\times n$ matrix and $I$ is a $(k,n)$-multiindex, we denote the $I$-minor by $A_I$, i.e.
\[A_I=\mat{a_{1,i_1} &\cdots &a_{1,i_{k}}\\\vdots &\ddots &\vdots\\a_{k,i_1}&\cdots&a_{k,i_k}}.\]
If $B$ is an $\al\times \beta$ matrix, $i\in\cpa{1,\cdots, \al}$ and $j\in \cpa{1,\cdots, \beta}$ we use $B_{\times i,\times j}$ to denote the $(\al-1)\times (\beta-1)$ matrix obtained from $B$ by deleting the $i$-th row and the $j$-th column.
\end{definition}

\begin{remark}
The set
\[\cpa{e_{i_1}\wedge\cdots\wedge e_{i_k}\mid 1\leq i_1<\cdots<i_k\leq n}\]
forms a basis for $\bigwedge^k\K^n$. For brevity, for all multiindicies $I=(i_1,\cdots, i_k)$ we shall define
\[e_I=e_{i_1}\wedge\cdots\wedge e_{i_k}.\]
\end{remark}

\begin{notation}
Whenever a base of $\K^\ell$ is fixed, we will identify $\bigwedge^\ell\K^\ell$ with $\K$ by sending the wedge of the ordered basis to $1\in\K$.
\end{notation}

\section{The Pl\"ucker embedding}
To make the study of Grassmannians easier, we want to identify $\Gr(k,n)$ with a projective variety.\\
Intuitively we seek to transform objects defined by several vectors into objects given by a single vector and then take the projective space construction. This conversion can be made by taking wedge products appropriately.

\begin{definition}[Pl\"ucker map]
Let $k\leq n$ be a pair of positive integers. We define the \textbf{Pl\"ucker map} as\footnote{the map $\wedge^k\vp$ is well defined because if we view it as a map $\wedge^k\vp:(\K^n)^k\to \bigwedge^k\K^k$ then it is multilinear and alternating.}:
\[\wedge^k:\funcDef{\Hom_\K(\K^n,\K^k)}{\Hom_\K(\bigwedge^k\K^n,\bigwedge^k\K^k)}{\vp}{\wedge^k\vp},\]
where $(\wedge^k\vp)(v_1\wedge\cdots\wedge v_k)=\vp(v_1)\wedge\cdots\wedge\vp(v_k).$\\
Note that if $e_1,\cdots, e_k$ is a basis of $\K^k$ then\footnote{the columns of the determinant are the coordinates of the vectors given in the base $e_1,\cdots, e_k$.}
\[\wedge^k(\vp)(v_1\wedge\cdots\wedge v_k)=\det\mat{\vp(v_1)|\cdots|\vp(v_k)} e_1\wedge\cdots\wedge e_k,\]
which we will often identify with $\det\mat{\vp(v_1)|\cdots|\vp(v_k)}$ when the choice of basis is clear.
\end{definition}

\begin{remark}\label{CodomainOfPluckerMap}
The codomain of the Pl\"ucker map is isomorphic to $\bigwedge^k\K^n$, indeed
\[\Hom_\K\pa{\bigwedge^k\K^n,\bigwedge^k\K^k}\cong \pa{\bigwedge^k\K^n}^{\vee}\cong \bigwedge^k\K^n.\]
The isomorphism depends on the choice of basis for $\K^n$ and $\K^k$. If $e_1,\cdots, e_n$ is a basis of $\K^n$ and $e_1,\cdots, e_k$ is a basis of $\K^k$ then the isomorphism takes on the following form
\[\funcDef{\Hom_\K(\K^n,\K^k)}{\bigwedge^k\K^n}{\vp}{\displaystyle\sum_{1\leq i_1<\cdots< i_k\leq n} \det(\vp(e_{i_1})\mid \cdots\mid \vp(e_{i_k}))e_{i_1}\wedge\cdots\wedge e_{i_k}},\]
where the determinant is defined assuming that the $\vp(e_j)$ are viewed as their coordinates in the base $e_1,\cdots, e_k$.
\end{remark}


\begin{remark}
The image of the Pl\"ucker map is a cone.
\end{remark}
\begin{proof}
For any $\la\in \K^\ast$ and any map $\vp\in \Hom_\K(\K^n,\K^k)$ we see that
\[\la\wedge^k(\vp)=\wedge^k(\al\circ \vp),\]
for any automorphism $\al$ of $\K^k$ with determinant\footnote{For example we may fix a base $e_1,\cdots, e_k$ of $\K^k$ and define $\al(e_i)=\begin{cases}
\la e_1 &\text{if }i=1\\
e_i &\text{otherwise}
\end{cases}$} $\la$.
\end{proof}


\begin{remark}\label{WhenPluckerMapIsZero}
$\rnk \vp<k$ if and only if $\wedge^k(\vp)=0$.
\end{remark}
\begin{proof}
$\wedge^k(\vp)$ is the zero map if an only if the set $\cpa{\vp(v_1),\cdots, \vp(v_k)}$ is linearly dependent for any choice of $v_1,\cdots, v_k$, i.e. $\vp$ is not of full rank.
\end{proof}

\begin{lemma}\label{CharacterizationOfKernels}
Let $\vp:\K^n\to \K^k$ be a full rank linear map, then
\[\ker\vp=\cpa{z\in\K^n\mid \forall w_2,\cdots, w_k\in \K^n,\ \wedge^k(\vp)(z\wedge w_2\wedge\cdots\wedge w_k)=0}.\]
\end{lemma}
\begin{proof}
If $\vp(z)=0$ then for any $w_2,\cdots, w_k\in \K^k$ we see that 
\[\wedge^k(\vp)(z\wedge w_2\wedge\cdots\wedge w_k)=0\wedge \vp(w_2)\wedge\cdots\wedge \vp(w_k)=0.\]
Suppose now that $\vp(z)\neq 0$ and let $v_2,\cdots, v_k$ be such that $\cpa{\vp(z), v_2,\cdots,v_k}$ form a basis for $\K^k$. Since $\vp$ is surjective there exist $w_2,\cdots, w_k$ such that $\vp(w_i)=v_i$ for all $2\leq i\leq k$.
By construction 
\[\wedge^k(\vp)(z\wedge w_2\wedge\cdots\wedge w_k)=\vp(z)\wedge v_2\wedge\cdots\wedge v_k\neq0.\]
\end{proof}

\begin{proposition}\label{PluckerMapInjectiveOnGrassmanniansUpToScalar}
Let $\sim$ be the equivalence relation defined in corollary (\ref{LinearQuotientDefinition}), then for any two full rank linear maps $\vp,\psi:\K^n\to \K^k$
\[\vp\sim \psi\coimplies \exists \la\in\K^\ast\ s.t.\ \wedge^k(\vp)=\la\wedge^k(\psi).\]
\end{proposition}
\begin{proof}
Let us prove both implications:
\setlength{\leftmargini}{0cm}
\begin{itemize}
\item[$\boxed{\implies}$] If $\vp=\theta\circ \psi$ for $\theta\in GL(\K^k)$ then it follows easily from known properties of the determinant that \[\wedge^k(\vp)=\wedge^k(\theta\circ \psi)=(\det\theta) \wedge^k(\psi).\]
\item[$\boxed{\impliedby}$] From lemma (\ref{kerAkerBVSActionOfGLk}) we see that it is enough to prove that $\ker \vp=\ker \psi$. We conclude by applying lemma (\ref{CharacterizationOfKernels}) as follows:
\begin{align*}
\ker\vp=&\cpa{z\in\K^n\mid \forall w_2,\cdots, w_k\in \K^n,\ \wedge^k(\vp)(z\wedge w_2\wedge \cdots\wedge w_k)=0}\\
=&\cpa{z\in\K^n\mid \forall w_2,\cdots, w_k\in \K^n,\ \la\wedge^k(\psi)(z\wedge w_2\wedge\cdots\wedge w_k)=0}=\\
=&\cpa{z\in\K^n\mid \forall w_2,\cdots, w_k\in \K^n,\ \wedge^k(\psi)(z\wedge w_2\wedge \cdots\wedge w_k)=0}=\ker \psi.
\end{align*}
\end{itemize}
\setlength{\leftmargini}{0.5cm}
\end{proof}

\begin{remark}
Because of proposition (\ref{PluckerMapInjectiveOnGrassmanniansUpToScalar}) and remark (\ref{WhenPluckerMapIsZero}) there exists a unique $h$ such that the diagram commutes
% https://q.uiver.app/#q=WzAsMyxbMCwwLCJcXGNwYXtcXHZwXFxpbiBcXEhvbV9cXEsoXFxLXm4sXFxLXmspXFxtaWQgXFxybmsgXFx2cD1rfSJdLFswLDEsIlxcR3IoayxuKSJdLFsxLDAsIlxcUGooXFxiaWd3ZWRnZV5rXFxIb21fXFxLKFxcS15uLFxcS15rKSkiXSxbMCwxLCJcXHBpIl0sWzAsMiwiW1xccGhpXSJdLFsxLDIsImgiLDIseyJzdHlsZSI6eyJib2R5Ijp7Im5hbWUiOiJkYXNoZWQifX19XV0=
\[\begin{tikzcd}
	{\cpa{\vp\in \Hom_\K(\K^n,\K^k)\mid \rnk \vp=k}} & {\Pj(\Hom_\K(\bigwedge^k\K^n,\bigwedge^k\K^k))} \\
	{\Gr(k,n)}
	\arrow["\pi", from=1-1, to=2-1]
	\arrow["{[\wedge^k]}", from=1-1, to=1-2]
	\arrow["h"', dashed, from=2-1, to=1-2]
\end{tikzcd}\]
Moreover, such an $h$ must be injective by proposition (\ref{PluckerMapInjectiveOnGrassmanniansUpToScalar}).
\end{remark}


\begin{definition}[Pl\"ucker embedding]
Let us fix a basis $e_1,\cdots, e_n$ of $\K^n$ and a basis $e_1,\cdots, e_k$ of $\K^k$. We define the \textbf{Pl\"ucker embedding} as follows
\[\Pl:\funcDef{\Gr(k,n)}{\Pj(\bigwedge^k\K^n)}{[\vp]_\sim}{[(\det(\vp(e_{i_1})\mid\cdots\mid\vp(e_{i_k})))_{1\leq i_1<\cdots<i_k\leq n}]_{\K^\ast}}.\]
\end{definition}

\begin{remark}
If we fix bases for $\K^n$ and $\K^k$ and $\zeta$ is the isomorphism $\Hom_\K(\bigwedge^k\K^n,\bigwedge^k\K^k)\to \bigwedge^k\K^n$ discussed during remark (\ref{CodomainOfPluckerMap}), we see that the following diagram commutes
% https://q.uiver.app/#q=WzAsMyxbMCwxLCJcXEdyKGssbikiXSxbMCwwLCJcXFBqKFxcYmlnd2VkZ2Vea1xcSG9tX1xcSyhcXEtebixcXEteaykpIl0sWzEsMCwiXFxQaihcXGJpZ3dlZGdlXmtcXEtebikiXSxbMCwxLCJoIiwyXSxbMSwyLCJcXFBqKFxcemV0YSkiXSxbMCwyLCJcXG1hdGhybXtQbH0iLDJdXQ==
\[\begin{tikzcd}
	{\Pj(\bigwedge^k\Hom_\K(\K^n,\K^k))} & {\Pj(\bigwedge^k\K^n)} \\
	{\Gr(k,n)}
	\arrow["h"', from=2-1, to=1-1]
	\arrow["{\Pj(\zeta)}", from=1-1, to=1-2]
	\arrow["{\Pl}"', from=2-1, to=1-2]
\end{tikzcd}\]
This proves that the Pl\"ucker embedding is well defined and injective.
\end{remark}

\begin{remark}
If we define $\phi=\zeta\circ \wedge^k$ then we see that \[\Pl\circ \pi=\Pj(\phi).\]
This form will be instrumental in the next chapter.
\end{remark}

\noindent
The entries of the homogeneous $\binom nk$-tuple associated to $[\vp]\in \Gr(k,n)$ are called the \textbf{Pl\"ucker coordinates} of $[\vp]$. The Pl\"ucker coordinates are unique up to multiplying each by the same nonzero scalar.

\begin{remark}
The Pl\"ucker embedding depends on the choice of basis for $\K^n$ but not on the one for $\K^k$, since the effect of changing the basis of $\K^k$ is that of multiplying all Pl\"ucker coordinates by the same scalar (the determinant of the change of basis), which doesn't change the point they describe in $\Pj(\bigwedge^k\K^k)$.\\
The dependence on the base of $\K^n$ corresponds to the fact that $\GL(\K^n)$ acts transitively on $\Gr(k,n)$ viewed as the set of $(n-k)$-dimentional subspaces of $\K^n$.
\end{remark}



\section{The image of the Pl\"ucker embedding is closed}
Thus far we have identified $\Gr(k,n)$ with a subset of some projective space. We now seek to show that it is a closed subset in the Zariski topology. Our approach mostly readapts parts of \cite{McKernan}.
\medskip

\noindent First we need some linear algebra results

\begin{lemma}\label{Divisibility}
Let $\omega\in \bigwedge^k\K^n$. For any given nonzero vector $v$ there exists $\e\in \bigwedge^{k-1}\K^n$ such that $\omega=\e\wedge v$ if and only if $\omega\wedge v=0$.
\end{lemma}
\begin{proof}
If $\omega=\e\wedge v$ then $\omega\wedge v=\e\wedge v\wedge v=0$.\\
Suppose now that $\omega\wedge v$. Let $v_1,\cdots, v_n$ be a basis of $\K^n$ such that $v_1=v$. If we write
\[\omega=\sum_{I\in\omega(k,n)}p_I v_I\]
then we see that for any given multiindex $I$ either $p_I=0$ or $v_I\wedge v=0$. Since $v_1,\cdots, v_n$ is a basis, $v_I\wedge v_1=0$ if and only if $1\in I$, i.e. $v_I=v\wedge v_{\pa{i_2,\cdots, i_k}}$, therefore
\[\omega=v\wedge \pa{\sum_{2\leq i_2<\cdots i_k\leq n} p_{\pa{1,i_2,\cdots, i_k}} e_{\pa{i_2,\cdots, i_k}}}\]
\end{proof}

\begin{corollary}[Total decomposibility criterion]\label{TotalDecomposibilityCriterion}
Let $\omega\in \bigwedge^k\K^n$. If $\dim\cpa{v\in\K^n\mid \omega\wedge v=0}\geq k$ then $\omega=\la v_1\wedge \cdots\wedge v_k$ for any set of linearly independent vectors $\cpa{v_1,\cdots, v_k}$ in $\cpa{v\in\K^n\mid \omega\wedge v=0}$ and some scalar $\la$.
\end{corollary}
\begin{proof}
The set $\cpa{v\in\K^n\mid \omega\wedge v=0}$ is clearly a subspace of $\K^n$. Let $\cpa{v_1,\cdots, v_k}$ be linearly independent vectors of this space. By iterating the above lemma we see that
\[\omega=\la\wedge v_1\wedge\cdots\wedge v_k\]
for some $\la\in \bigwedge^0\K^n=\K$.
\end{proof}

\begin{lemma}\label{DecomposabilityOfMultilinearForm}
A multilinear alternating form $\psi\in \Hom_\K(\bigwedge^k\K^n,\bigwedge^k\K^k)$ is in the image of the Pl\"ucker map $\wedge^k$ if and only if there exists a basis $e_1,\cdots, e_n$ of $\K^n$ and an element $\la$ of $\K$ such that
\[\sum_{I\in \omega(n-k,n)}\psi(e_{\wh I})e_I=\la e_{(1,\cdots, n-k)}.\]
\end{lemma}
\begin{proof}
Let us fix a basis $e_1,\cdots, e_k$ of $\K^k$. Using this basis we identify $\bigwedge^k\K^k$ with $\K$.\\
Suppose that $\psi=\wedge^k(\vp)$ and let $e_1,\cdots, e_n$ be a basis of $\K^n$ such that $e_1,\cdots, e_{n-k}$ is a basis of $\ker \vp$, then
\[\sum_{I\in \omega(n-k,n)}\wedge^k\vp(e_{\wh I})e_I=\pa{\vp(e_{n-k+1})\wedge \cdots\wedge \vp(e_{n})} e_{\pa{1,\cdots, n-k}}.\]
Suppose now that we have the decomposability above and let us define $\vp$ by
\[\vp(e_i)=\begin{cases}
0 & \text{if }1\leq i\leq n-k\\
\la e_1 & \text{if }i=n-k+1\\
e_{i-n+k} & \text{if }i>n-k+1
\end{cases}\]
We now compute a second form for $\sum_{I\in \omega(n-k,n)}\psi(e_{\wh I})e_I$
\begin{align*}
\sum_{I\in \omega(n-k,n)}\psi(e_{\wh I})e_I=&\la e_{(1,\cdots, n-k)}=\\
=&\vp(e_{n-k+1})\wedge\cdots\wedge\vp(e_n) e_{(1,\cdots, n-k)}=\\
=&\sum_{I\in \omega(n-k,n)}\wedge^k\vp(e_{\wh I})e_I
\end{align*}
where the second equality follows from the construction of $\vp$.\\
We have shown that for all $J\in\omega(k,n)$ we have
\[\psi(e_J)=\wedge^k \vp(e_J),\]
so $\psi$ and $\wedge^k(\vp)$ agree on a basis of $\bigwedge^k\K^n$ and are thus the same map. 
\end{proof}

\noindent Let us consider the following map: let $\psi:\Hom_\K(\bigwedge^k\K^n,\bigwedge^k\K^k)$ be any alternating multilinear map, we define $\Phi(\psi)$ as
\[\Phi(\psi):\funcDef{\K^n}{\bigwedge^{n-k+1}\K^n}{v}{\displaystyle\sum_{I\in \omega(n-k,n)}\psi(e_{\wh I})e_I\wedge v}\]
where $\wh I$ is the ordered $k$-tuple of the indices in $\cpa{1,\cdots, n}$ missing from $I$.



\begin{proposition}\label{RankCriterionForImageOfPlucker}
An alternating multilinear map $\psi\in \Hom_\K(\bigwedge^k\K^n,\bigwedge^k\K^k)$ is in the image of the Pl\"ucker map $\wedge^k$ if and only if $\Phi(\psi)$ has rank at most $k$.
\end{proposition}
\begin{proof}
Suppose that $\psi=\wedge^k(\vp)$ and let $\cpa{z_1,\cdots, z_{n-k},z_{n-k+1},\cdots, z_{n}}$ be a basis of $\K^n$ such that the first $n-k$ vectors are a basis of $\ker \vp$. From the proof of lemma (\ref{DecomposabilityOfMultilinearForm}) we see that if $v\in \ker \vp$ then $v\in \ker \Phi(\psi)$, i.e. $\Phi(\psi)$ has a nullity of at least $n-k$ (or rank at most $k$).
\medskip

\noindent
Suppose now that $\cpa{z_1\cdots, z_{n-k}}$ are linearly independent elements of $\ker \Phi(\psi)$. By the Total decomposability criterion (\ref{TotalDecomposibilityCriterion}) we have that there exists $\la\in\K$ such that
\[\sum_{I\in \omega(n-k,n)}\psi(e_{\wh I})e_I=\la z_1\wedge\cdots \wedge z_{n-k}.\]This concludes by lemma (\ref{DecomposabilityOfMultilinearForm}).
\end{proof}

\noindent Let us now consider the map
\[\Phi:\funcDef{\bigwedge^k\K^n}{ \Hom_\K\pa{\K^n,\bigwedge^{n-k+1}\K^n}}{\psi^\ast}{\Phi(\zeta\ii(\psi^\ast))}\]
Note that $\Phi$ is linear, so we can represent $\Phi$ as a matrix with coefficients in $\bigwedge^k\K^n$:\\
we fix a basis $\cpa{e_I}$ of $\bigwedge^k\K^n$ and write $\Phi(\sum a_I e_I)=\sum a_I \Phi(\zeta\ii e_I)$. Since each $\Phi(\zeta\ii e_I)$ is linear, it can be viewed as a matrix $\pa{b_{i,j}^I}_{i,j}$ and so the matrix associated to $\Phi$ is $\pa{\sum b_{i,j}^I e_I}_{i,j}$.
\bigskip

\noindent Proposition (\ref{RankCriterionForImageOfPlucker}) tells us that the image of $\zeta\circ\wedge^k$ can by identified by imposing that the rank of the matrix representing $\Phi$ defined above is at most $n-k$, which is equivalent to the vanishing of its $(n-k+1)\times (n-k+1)$ minors, which is a closed condition.\smallskip

\noindent
It follows trivially that the projectivization\footnote{recall that $\imm \wedge^k$ is a cone.} of this set (i.e. the image of $\Pl$) is also closed in $\Pj(\bigwedge^k\K^n)$, so we found a bijection between $\Gr(k,n)$ and a projective variety, which we can use to endow $\Gr(k,n)$ with the structure of one.


\begin{remark}
The determinants we used to show that the image of the Pl\"ucker embedding is closed do not generate the ideal of that variety. The most well known set of generators for that ideal are the \textbf{Pl\"ucker relations}. For their construction [BUT NOT THE PROOF THAT THEY ARE GENERATORS, STILL HAVE TO FIND THAT REFERENCE] see \cite{McKernan}.
\end{remark}










\chapter{Representability of the Grassmannian functor}
In this chapter we will work with $\K^n$ and $\K^k$ instead of abstract vector spaces. This means that we have canonical bases $\Can_n=\cpa{e_1,\cdots, e_n}$ and $\Can_k=\cpa{e_1,\cdots, e_k}$ and that we identify $\Hom_\K(\K^n,\K^k)$ with $\Mc(k,n)$.
\medskip

To differentiate the scheme morphisms we define in this chapter from the morphisms of varieties defined previously we use a superscript $s$ for the latter, i.e.
\[\phi^s:\funcDef{\Mc(k,n)}{\bigwedge^k\K^n}{A}{\displaystyle\sum_{I\in \omega(k,n)}\det A_I e_I},\quad \Pl^s:\funcDef{\Gr(k,n)}{\Pj(\bigwedge^k\K^n)}{[A]_\sim}{\spa{\phi^s(A)}_{\K^\ast}}\]

\begin{notation}
Let $I$ be an ideal of the ring $A$ and $J$ be a homogeneous ideal of the graded ring $B$. We adopt the following notation
\[V(I)=\cpa{\pf\in\Spec A\mid I\subseteq \pf},\qquad V_+(J)=\cpa{\pf\in\Proj B\mid I\subseteq \pf}.\]
If the sets above are considered as closed subschemes we take the reduced structure.
\end{notation}


\section{Grassmannians as projective schemes}

\begin{definition}[Bracket ring]
We define the \textbf{bracket ring} (see page 79 of \cite{matroids}) as the ring of polynomial functions on $\bigwedge^k\K^n$, i.e.
\[\Bc_{k,n}\doteqdot\frac{\K[z_I\mid I\in \cpa{1,\cdots, n}^k]}{(\cpa{z_I-\sgn(\sigma)z_{\sigma(I)}}_{\sigma\in S_k})}\cong \K[z_I\mid I\in \omega(k,n)].\]
We define $\Bc_{k,n}^+$ to be the ideal generated by the indeterminates $z_I$.
\end{definition}

\begin{definition}[Ring of generic matrices]
Let $\K[X_{k,n}]\doteqdot\K[x_{1,1},\cdots,x_{k,n}]$ denote the polynomial ring with $k\cdot n$ variables. We define the \textbf{generic matrix} as
\[X=\mat{
    x_{1,1} & \cdots & x_{1,n}\\
    \vdots & \ddots & \vdots\\
    x_{k,1} &\cdots & x_{k,n}}.\]
By the same token we use $X_I$ to denote the generic $k\times k$ minor determined by the multiindex $I$ and $\det X_I$ to write the formal determinant of this minor.
\end{definition}
\begin{remark}
The ring $\K[X_{k,n}]$ is the coordinate ring of $\Mc(k,n)$. 
\end{remark}

\begin{remark}
The familiar $\Mc(k,n)$ and $\bigwedge^k\K^n$ can be identified with the $\K$-points of the affine schemes $\Spec \K[X_{k,n}]$ and $\Spec \Bc_{k,n}$ respectively (Example 2.3.32 of \cite{QingLiu}). We will impose this identification for the rest of the chapter.
\end{remark}

\begin{definition}[Pl\"ucker ring homomorphism]
We define the \textbf{Pl\"ucker ring homomorphism} or simply \textbf{Pl\"ucker homomorphism} as
\[\phi^\#:\funcDef{\Bc_{k,n}}{\K[X_{k,n}]}{z_I}{\det X_I}\]
For brevity we will denote $\Spec \phi^\#$ by $\phi$.
\end{definition}

\begin{remark}\label{PluckerRingHomomorphismWorksForKPoints}
It is clear by construction that
\[\phi\res{\Mc(k,n)}(A)=\sum_{I\in\omega(k,n)}\det A_I e_I=\phi^s(A).\]
\end{remark}

\begin{proposition} $\ker\phi^\#$ is a homogeneous prime ideal and $\Bc_{k,n}^+\not\subseteq \ker\phi^\#$.
\end{proposition}
\begin{proof}
$\ker\phi^\#$ is prime because $\K[X_{k,n}]$ is an integral domain and $z_I\notin \ker \phi^\#$ because $\deg \phi^\#(z_I)=\deg(\det X_I)=k>0$. 
To show homogeneity let us note that if $g$ is homogeneous of degree $d$ then $\phi^\#(g)$ is homogeneous of degree $kd$. If follows that if $f_d$ is the homogeneous component of $f$ of degree $d$ and $0=\phi^\#(f)=\sum_{d\in\N}\phi^\#(f_d)$ then $\phi^\#(f_d)=0$ for all $d\in\N$.
\end{proof}

\begin{proposition}
Let $t:\mathrm{Var}/\K\to \Sch \K$ be the fully faithful functor defined as in Proposition 2.6 of \cite{Hartshorne}. Then $V_+(\ker\phi^\#)\cong t(\imm\Pl^s)$.
\end{proposition}
\begin{proof}
Because $t$ is fully faithful, we only need to show that $V_+(\ker(\phi^\#))(\K)\cong \imm \Pl^s$.
Passing to the corresponding cones, this is equivalent to 
\[\imm \phi^s\cong V(\ker\phi^\#)(\K)=\ol{\imm \phi\res{\Mc(k,n)}}=\ol{\imm \phi^s},\]
which is true because $\imm \phi^s\pasgnl={(\ref{ImageOfPhiClosed})}\ol{\imm \phi^s}$.
\end{proof}

From now on $\Gr(k,n)$ will also have the scheme structure of $V_+(\ker \phi^\#)$. What we used to write $\Gr(k,n)$ corresponds to $\Gr(k,n)(\K)$.

\subsection{Standard affine cover of the Grassmannian scheme}
Recall that projective space admits a standard affine cover given by the loci where one indeterminate does not vanish. In our case we see that
\[\Proj \Bc_{k,n}=\bigcup_{I\in\omega(k,n)}\Spec \pa{\pa{\Bc_{k,n}}_{z_I}^0}=\bigcup_{I\in\omega(k,n)}\Spec \pa{\K\spa{\frac{z_J}{z_I}\mid J\in \omega(k,n)}},\]
where the subscript denotes localization with multiplicative part $\cpa{1,z_I,z_I^2,\cdots}$ and the superscript $0$ denotes the fact that we are only considering terms of degree $0$ in this ring (this is the notation used in \cite{QingLiu}).
\medskip

This open affine cover of $\Proj\Bc_{k,n}$ induces an open cover on $\Gr(k,n)$ as follows:
\[\Gr(k,n)=V_+(\ker\phi^\#)=\bigcup_{I\in \omega(k,n)}\Spec\pa{\pa{\frac{\Bc_{k,n}}{\ker\phi^\#}}_{z_I}^0}.\]

\begin{notation}
Let us fix $I\in\omega(k,n)$, then we denote the restriction of $\phi^\#$ as
\[\phi^\#_I:\funcDef{\K\spa{\frac{z_J}{z_I}\mid J\in \omega(k,n)}}{\K[X_{k,n}]_{\det X_I}^0}{\frac{z_J}{z_I}}{\frac{\det X_J}{\det X_I}}\]
\end{notation}

\begin{remark}
By the first isomorphism theorem we have
\[\frac{\pa{\Bc_{k,n}}_{z_I}^0}{\ker \phi^\#_I}\cong \imm \phi^\#_I=\K\spa{\frac{\det X_J}{\det X_I}\mid J\in \omega(k,n)}\doteqdot \K\spa{\frac{\det X_J}{\det X_I}}.\]
\end{remark}

\begin{remark}
Applying a property of localization we have
\[\pa{\frac{\Bc_{k,n}}{\ker\phi^\#}}_{z_I}=\frac{\pa{\Bc_{k,n}}_{z_I}}{(\ker\phi^\#)_{z_I}},\]
thus
\[\pa{\frac{\Bc_{k,n}}{\ker\phi^\#}}_{z_I}^0=\pa{\frac{\pa{\Bc_{k,n}}_{z_I}}{(\ker\phi^\#)_{z_I}}}^0=\frac{\pa{\Bc_{k,n}}_{z_I}^0}{\ker \phi^\#_I}\cong \K\spa{\frac{\det X_J}{\det X_I}}\]
\end{remark}

In summary we have shown that, up to some canonical identifications,
\[\Gr(k,n)=\bigcup_{I\in \omega(k,n)}\Spec\pa{\K\spa{\frac{\det X_J}{\det X_I}}}\doteqdot \bigcup_{I\in \omega(k,n)}\Gr_I(k,n).\]

\begin{notation}
Let $I$ be a $(k,n)$-multiindex, $i\in\cpa{1,\cdots,k}$ and $j\in\cpa{1,\cdots, n}$. We define $I^i_j$ to be the multiindex which is the same as $I$ but with the $i$-th entry replaced with $j$.
\end{notation}

\begin{lemma}\label{FormalBasisChange}
If $I\in\omega(k,n)$ then the following equality holds in $\K[X_{k,n}]_{\det X_I}$
\[X_I\ii X=\mat{
w_{I^1_{1}} & \cdots & w_{I^1_n}\\
\vdots      & \ddots & \vdots\\
w_{I^k_{1}} & \cdots & w_{I^k_n}
},\qquad\text{where }w_J=\frac{\det X_J}{\det X_I}\]
\end{lemma}
\begin{proof}
Recall that if $\Adj(X_I)$ is the adjugate matrix of $X_I$ then
\[\pa{X_I}\ii=\frac1{\det X_I}\Adj(X_I)=\frac1{\det X_I}\mat{(-1)^{i+j}\det (X_I)_{\times j\times i}}_{\smat{1\leq i\leq k\\1\leq j\leq k}}.\]
We can verify the identity for each element:
\[\frac{(\Adj(X_I) X)_{i,j}}{\det X_I}=\frac1{\det X_I}{\sum_{\ell=1}^k\pa{(-1)^{i+\ell} \det \pa{X_I}_{\times \ell,\times i}}x_{\ell,j}}=\frac{\det X_{I^i_j}}{\det X_I}=w_{I^i_j}.\]
\end{proof}

\begin{remark}
$(X_I\ii X)_J=X_I\ii X_J$, in particular
$(X_I\ii X)_I$ is the identity matrix.
\end{remark}

\begin{proposition}\label{GrassmannianIsCoveredByAffineSpaces}
$\Gr_I(k,n)$ is isomorphic to $\A^{k(n-k)}_\K$ as a scheme.
\end{proposition}
\begin{proof}
Since both schemes are affine, it is enough to show that their coordinate rings are isomorphic. Without loss of generality we may assume that $I=\pa{1,\cdots, k}$. 
For brevity we set $w_J=\frac{\det X_J}{\det X_I}$. 

Let $M$ be the formal matrix whose $(i,j)$-entry is $w_{I^i_j}$. Lemma (\ref{FormalBasisChange}) shows that $M=X_I\ii X$, so $\det M_J=\det X_I\ii \det X_J=w_J$. This shows that
\[\K\spa{\frac{\det X_J}{\det X_I}\mid J\in \omega(k,n)}=\K\spa{\frac{\det X_J}{\det X_I}\mid J=I^j_{\ell_j},\ j\in \cpa{1,\cdots, k},\ \ell_j\notin I}.\]
Let $R$ denote this ring. To conclude we want to show that it is isomorphic to $\K[Y_{k,n-k}]=\K[y_{1,1},\cdots, y_{k,n-k}]$.
\medskip

Let us consider the following ring homomorphism
\[\chi:\funcDef{\K[Y_{k,n-k}]}{R}{y_{i,j}}{w_{I^i_{j+k}}}.\]
It is surjective by construction, so we just need to show that it is injective to find the desired isomorphism.

Suppose by contradiction that there exists a nonzero polynomial $p\in \K[Y_{k,n-k}]$ which maps to $0$. If $\ol \K$ is an algebraic closure\footnote{we can take any field extension $\K\subseteq \F$ where $\F$ is an infinite field.} of $\K$ we can consider the lift 
\[\wt\chi:\funcDef{\ol\K[Y_{k,n-k}]}{\wt R=\ol \K[w_{I^i_j}]}{y_{i,j}}{w_{I^i_{j+k}}}\]
Note that if $\chi(p)=0$ then $\wt\chi(p)=0$ because $R\subseteq \wt R$ and $\wt \chi\res{\K[Y_{k,n-k}]}=\chi$. Consider now any matrix of the form
\[A=\mat{I_k\mid \wt A}=\pa{a_{i,j}}_{i,j}\]
where $I_k$ is the $k\times k$ identity matrix and $\wt A\in \Mc(k,n-k,\ol \K)$. 
From what we have said above it follows that $\det A_{I^i_j}=a_{i,j}$, so 
\[p(\wt A)=p\pa{\pa{\det A_{I^i_j}}_{\smat{i\in\cpa{1,\cdots, k},~~~\\ j\in \cpa{k+1,\cdots, n}}}}=\wt \chi(p)(A)=0.\] 
This shows that $p$ has infinitely many roots in $\ol K$, so if we fix the value of $k(n-k)-1$ indeterminates the resulting polynomial is the $0$ polynomial. 
If we reiterate this reasoning we eventually prove that $p=0$ in $\ol \K[Y_{k,n-k}]$, but $0\in \K[Y_{k,n-k}]\subseteq \ol \K[Y_{k,n-k}]$, so $p$ is the zero polynomial in the original ring, contradicting our hypothesis.
\end{proof}

\begin{remark}\label{IntersectionIsAffine}
Since $\Gr_I(k,n)$ and $\Gr_J(k,n)$ are affine and $\Gr(k,n)$ is projective and thus separated, $\Gr_I(k,n)\cap \Gr_J(k,n)$ is affine for any choice of multiindices.
\end{remark}




\section{Grassmannian moduli problem}

Let us consider the following moduli problem
\[\gr(k,n):\functorDef{(\Sch\K)\op}{\Set}{T}{\quot{\cpa{\al:\Oc_T^n\onto Q}}\sim}{f:S\to T}{(\al:\Oc_T^n\to Q)\mapsto (f^\ast\al:\Oc_S^n\to f^\ast Q)}\]
where $Q$ is a locally free sheaf of rank $k$ on $T$ and two surjections $\al:\Oc_T^n\onto Q$, $\beta:\Oc_T^n\onto V$ are equivalent if and only if there exist an isomorphism of sheaves $\theta:Q\to V$ such that the diagram commutes
\[\begin{tikzcd}
	{\Oc_T^n} & Q \\
	& V
	\arrow["\al", two heads, from=1-1, to=1-2]
	\arrow["\beta"', two heads, from=1-1, to=2-2]
	\arrow["\theta", from=1-2, to=2-2]
\end{tikzcd}\]
We have functoriality because of the composition properties of pullbacks.

\begin{remark}
This functor formalizes the classification problem of $(n-k)$-dimensional subspaces of an $n$-dimensional space. Indeed
\[\gr(k,n)(\Spec\K)=\quot{\cpa{\al:\Oc_{\Spec \K}^n\onto Q}}\sim\cong\quot{\cpa{\vp:\K^n\onto \K^k}}\sim=\Gr(k,n)(\K).\]
For the middle isomorphism we used the fact that sheaves over a point are skyscrapers and that $\Oc_{\Spec\K,\Spec \K}=\K$. The last equality is our first definition for the Grassmannian up to the choice of a basis.
\end{remark}

\noindent In this this section we prove that the Grassmann scheme represents this functor.

\subsection{Open subfunctor cover of the Grassmannian}
\begin{notation}
For any multiindex $I\in\omega(k,n)$ and any scheme $T$ we define the following morphism of sheaves
\[s_I^T:\funcDef{\Oc_T^k}{\Oc_T^n}{e_j}{e_{i_j}}.\]
If there is no ambiguity we write $s_I$.
\end{notation}

\begin{definition}[Principal subfunctors of the Grassmannian]
Fixed a multiindex $I\in \omega(k,n)$ we define the following functor
\[\gr_I(k,n):\functorDef{(\Sch\K)\op}{\Set}{T}{\quot{\cpa{\Oc_T^n\overset{\al}\onto Q\mid \al\circ s_I\text{ surjective}}}\sim}{f}{\al\mapsto f^\ast \al}\]
where the equivalence relation is the same as the one defined for $\gr(k,n)$.
\end{definition}

\begin{proposition}
The functor $\gr_I(k,n)$ is well defined.
\end{proposition}
\begin{proof}
First we observe that $\gr_I(k,n)(T)$ is well defined because if $\psi=\theta\circ \al$ with $\theta$ isomorphism of sheaves then on each stalk we have
\[\psi_x\circ (s_I)_x=\theta_x\circ \vp_x\circ (s_I)_x,\]
which is surjective if and only if $\vp_x\circ (s_I)_x$ is surjective.

Consider now a morphism $f:S\to T$, then
\[f^\ast\al \circ s_I^S=f^\ast\al\circ f^\ast s_I^T=f^\ast(\al\circ s_I^T)\]
is surjective if and only if it is surjective on all stalks, i.e. if and only if for all $s\in S$ we have that the following map is surjective
\[f^\ast(\al\circ s_I^T)_s=(\al\circ s_I^T)_{f(s)}\otimes_{\Oc_{T,f(s)}}id_{\Oc_{S,s}},\]
which is true because the tensor product is right-exact.
\end{proof}


\begin{proposition}\label{GrIAreOpenSubfunctors}
The $\gr_I(k,n)$ are open subfunctors of $\gr(k,n)$.
\end{proposition}
\begin{proof}
The inclusion $\gr_I(k,n)(T)\subseteq \gr(k,n)(T)$ is apparent, so we just need to show that if we fix a quotient $[\al:\Oc_T^n\onto Q]$ in $\gr(k,n)(T)$ then we can find an open subscheme of $T$ which represents $h_T\times_{\gr(k,n)}\gr_I(k,n)$.\medskip

Let us fix a representative $\al$ for the given quotient. The locus where $\al\circ s_I:\Oc_T^k\to Q$ is surjective is the complement of the support of its cokernel sheaf $\Kc$, i.e. 
\[(\al\circ s_I)_x\text{ surjective} \coimplies x\notin \Supp\Kc.\]
Note that by the definition of $\sim$ and properties of isomorphisms of sheaves, the first condition does not depend of the choice of representative for $[\al]$, so $\Supp\Kc$ only depends on $[\al]$. Note that $\Kc$ is of finite type because the codomains are locally free of finite rank, so $\Supp\Kc$ is closed\footnote{For more detail see \href{https://stacks.math.columbia.edu/tag/01B4}{Section 01B4} in \cite{stacks}} and hence $U_I=T\bs \Supp \Kc$ is open.\medskip

We now want to show that $U_I$ represents the functor $h_T\times_{\gr(k,n)}\gr_I(k,n)$, that is we want to show that if $f:S\to T$ is a morphism of $\K$-schemes then $f$ factors through $U_I$ if and only if $[f^\ast\al:\Oc_S^n\to f^\ast Q] \in \Gr_I(S)$.\medskip

Note that $f(s)\in U_I$ if and only if $(\al\circ s_I^T)_{f(s)}$ is surjective which, by Nakayama's lemma applied to the cokernels, is equivalent to the surjectivity of
\[(\al\circ s_I^T)\res{f(s)}:k(f(s))^n\to Q_{f(s)}\otimes_{\Oc_{T,f(s)}}k(f(s)).\]
Observe that, up to standard identifications,
\begin{align*}
f^\ast(\al\circ s^T_I)\res s=&f^\ast(\al\circ s^T_I)_s\otimes_{\Oc_{S,s}} id_{k(s)}=\\
=&((\al\circ s^T_I)_{f(s)}\otimes_{\Oc_{T,f(s)}}id_{\Oc_{S,s}})\otimes_{\Oc_{S,s}}id_{k(s)}=\\
=&(\al\circ s^T_I)_{f(s)}\otimes_{\Oc_{T,f(s)}}id_{k(s)}=\\
=&((\al\circ s^T_I)_{f(s)}\otimes_{\Oc_{T,f(s)}}id_{k(f(s))})\otimes_{k(f(s))}id_{k(s)}=\\
=&(\al\circ s^T_I)\res{f(s)}\otimes_{k(f(s))} id_{k(s)}.
\end{align*}
Note that we used the fact that $\Oc_{T,f(s)}\to \Oc_{S,s}\to k(s)=\Oc_{T,f(s)}\to k(f(s))\to k(s)$. Since field extensions do not change the rank of linear maps, this shows that
\[f^\ast(\al\circ s^T_I)\res s\text{ is surjective }\quad\coimplies\quad (\al\circ s^T_I)\res{f(s)}\text{ is surjective}.\]
By Nakayama's lemma we can again consider equivalently $f^\ast(\al\circ s^T_I)_s=(f^\ast\al)_{s}\circ (s^S_I)_s$.


We have thus shown that $f(s)\in U_I$ if and only if $(f^\ast\al)_{s}\circ (s^S_I)_s$ is surjective, i.e. $f$ factors through $U_I$ if and only if $(f^\ast\al)\circ s^S_I$ is surjective, i.e. $f^\ast\al\in \gr_I(k,n)(S)$.
\end{proof}


\begin{proposition}\label{GrIAreOpenCover}
The collection $\cpa{\gr_I(k,n)}$ is a Zariski open cover of $\gr(k,n)$.
\end{proposition}
\begin{proof}
For any $\K$-scheme $S$ and any quotient $[\al]\in \Gr(k,n)(S)$ (without loss of generality we choose a representative $\al$) we need to show that for any $s\in S$ there exists a multiindex $I$ such that $s\in U_I$ defined as in the previous proposition.

We are therefore looking for a multiindex $I$ such that $(\al\circ s_I)_s$ is surjective. By Nakayama's lemma this is equivalent to showing that there exists an $I$ such that
\[k(s)^k\overset{s_I}\to k(s)^n\overset{\al_s}\to Q_s\otimes_{\Oc_{S,s}}k(s)\]
is surjective, which is trivially true since $\rnk \al_s=k$.
\end{proof}


\subsection{Representability of the Grassmannian functor}
\begin{lemma}\label{UniqueIsomorphismInGrassmannFunctor}
Let $T$ be a scheme and $[\al:\Oc_T^n\onto Q],[\beta:\Oc_T^n\onto Q']\in\gr(k,n)$. If $[\al]=[\beta]$ then the isomorphism $\theta:Q\to Q'$ such that $\beta=\theta\circ \al$ is unique.
\end{lemma}
\begin{proof}
First, observe that if $\al=\beta$ then by surjectivity and commutativity $\theta=id_Q$.
Let $\theta,\eta:Q\to Q'$ be isomorphisms such that $\beta=\theta\circ \al$ and $\beta=\eta\circ \al$. Then $\theta\ii \circ \eta:Q\to Q$ is an isomorphism such that $\theta\ii\circ \eta\circ \al=\theta\ii\circ \beta=\al$, so $\theta\ii\circ \eta=id_Q$ and thus $\theta=\eta$.
\end{proof}

\begin{proposition}\label{GrassmannianIsSheaf}
The Grassmannian functor $\gr(k,n)$ is a Zariski sheaf.
\end{proposition}
\begin{proof}
Consider a $\K$-scheme $T$ and an open cover $\cpa{U_i\to T}$. Let $\al_i:\Oc_{U_i}^n\onto Q_i$ be representatives of quotients such that 
\[\al_i\res{U_i\cap U_j}\sim \al_j\res{U_i\cap U_j}.\]
Because of lemma (\ref{UniqueIsomorphismInGrassmannFunctor}), the isomorphism giving the equivalence above is unique. Let $\vp_{ji}:Q_i\res{U_{i}\cap U_j}\to Q_j\res{U_i\cap U_j}$ be this isomorphism. Because of the uniqueness $\vp_{ii}=id_{Q_i}$ and $\vp_{ki}=\vp_{kj}\circ \vp_{ji}$, so we have the data to glue the $Q_i$ to a locally free sheaf of rank $k$ over $T$, which we denote by $Q$. 

By construction $\al_i:\Oc_{U_i}^n\onto Q\res{U_i}$ for all $i$. 
Let $V\subseteq T$ be an open subset. For any section $s\in \Oc_T^n(V)$ we can define $\al_V(s)$ by gluing the $(\al_i)_V(s\res{U_i})$, which we can do by construction\footnote{More precicely, the $\vp_{ji}$ are the gluing functions on $Q$ and \[\al_j(s\res{U_j})\res{U_i\cap U_j}=\al_j(s\res{U_i\cap U_j})=\vp_{ji}\circ \al_i(s\res{U_i\cap U_j})= \vp_{ji}(\al_j(s\res{U_j})\res{U_i\cap U_j}).\]} of $Q$. It is well known that a sheaf morphism is determined by its restrictions to open sets.
\end{proof}

\begin{proposition}\label{GrIRepresentGrIFunctors}
The affine scheme $\Gr_I(k,n)$ represents the functor $\gr_I(k,n)$.
\end{proposition}
\begin{proof}
First we prove that for any $\K$-scheme $T$, $\Hom_{\Sch\K}(T,\Gr_I(k,n))\cong \gr_I(T)$, then we need to check naturality.

By definition $\Gr_I(k,n)=\Spec\pa{\K \spa{\frac{\det X_J}{\det X_I}}}$, so
\[\Hom_{\Sch\K}(T,\Gr_I(k,n))\cong \Hom_{\K\text{-alg}}\pa{\K\spa{\frac{\det X_J}{\det X_I}},\Oc_T(T)}.\]
For a map $\al:\Oc^n_T\to \Oc_T^k$, we define $M(U)$ as the matrix which represents $\al_U:\Oc_T^n(U)\to\Oc_T^k(U)$ in the canonical bases.
We define the following maps
\[\correspDef{\Hom_{\K\text{-alg}}\pa{\K\spa{\frac{\det X_J}{\det X_I}},\Oc_T(T)}}{\cpa{\al:\Oc_T^n\to \Oc_T^k\mid \al\circ s_I=id_{\Oc_T^k}}}{\vp}{\eta(\vp)}{\rho(\al):\frac{\det X_J}{\det X_I}\mapsto \frac{\det (M(T))_J}{\det (M(T))_I}}{\al}\]
where $\eta(\vp)$ is defined on an open subset $V$ of $T$ by
\[\eta(\vp)_V(e_j)=
\sum_{i=1}^k (\mathrm{res}^T_V\circ\vp)\pa{\frac{\det X_{I^{i}_j}}{\det X_I}}e_r\pasgnl={(\ref{FormalBasisChange})}(\mathrm{res}^T_V\circ \vp)\pa{X_I\ii X} e_j.
\]
The maps are well defined because $\al\circ s_I=id_{\Oc^k_T}\coimplies M(T)_I=I_k$ and 
\[\dfrac{\det X_{I^{r}_{i_s}}}{\det X_I}=\delta_{r,s}\implies \eta(\vp)\circ s_I=id_{\Oc^k_T}.\]
We can see that $\eta$ and $\rho$ are inverses via the following computations:
\[\mathrm{res}^T_V\circ \rho(\al)(X_I\ii X)=\mathrm{res}^T_V({M(T)_I}\ii M(T))=\mathrm{res}^T_V(I_k\ii M(T))=M(V),\]
\[\rho(\eta(\vp))\pa{\frac{\det X_J}{\det X_I}}=\frac{\det((\mathrm{res}^T_T\circ \vp)\pa{X_I\ii X}_J)}1=\vp(\det((X_I\ii X)_J))=\vp\pa{\frac{\det X_J}{\det X_I}}.\]
Observe now that
\[\correspDef{\cpa{\al:\Oc_T^n\to \Oc_T^k\mid \al\circ s_I=id_{\Oc_T^k}}}{\quot{\cpa{\al:\Oc_T^n\onto Q\mid \al\circ s_I\text{ isomorphism}}}\sim}{\al}{[\al]}{(\beta\circ s_I)\ii\circ \beta}{[\beta]}\]
is a bijection. The second map is well defined because if $\beta=\theta\circ \beta'$ then \[(\beta\circ s_I)\ii\circ \beta=(\beta'\circ s_I)\ii\circ \theta\ii\circ \theta\circ \beta'=(\beta'\circ s_I)\ii\circ \beta'\]
and they are inverses because $\beta\sim (\beta\circ s_I)\ii\circ \beta$ by definition of $\sim$ and if $\al\circ s_I=id_{\Oc_T^k}$ then $(\al\circ s_I)\ii\circ \al=\al$.
We conclude by noticing that
\[\quot{\cpa{\al:\Oc_T^n\onto Q\mid \al\circ s_I\text{ isomorphism}}}\sim=\quot{\cpa{\al:\Oc_T^n\onto Q\mid \al\circ s_I\text{ surjective}}}\sim\]
because on all stalks $\al\circ s_I$ is an endomorphism of finitely generated modules.
\bigskip

To prove naturality we consider a morphism  $f:S\to T$ of $\K$-schemes. Recall that
\[\funcDef{\gr_I(k,n)(T)}{\gr_I(k,n)(S)}{[\al]}{[f^\ast \al]}.\]
Under the bijection above, imposing naturality gives
\[\funcDef{\cpa{\al:\Oc_T^n\to \Oc_T^k\mid \al\circ s_I=id_{\Oc_T^k}}}{\cpa{\beta:\Oc_S^n\to \Oc_S^k\mid \beta\circ s_I=id_{\Oc_S^k}}}{\al}{f^\ast\al}\]
since $f^\ast\al\circ s_I^S=f^\ast(\al\circ s_I^T)=f^\ast(id_{\Oc^k_T})=id_{\Oc^k_S}$.
If we impose naturality again we get
\[\funcDef{\Hom_{\K\text{-alg}}\pa{\K\spa{\frac{\det X_J}{\det X_I}},\Oc_T(T)}}{\Hom_{\K\text{-alg}}\pa{\K\spa{\frac{\det X_J}{\det X_I}},\Oc_S(S)}}{\vp}{\rho(f^\ast\eta(\vp))}\]
We claim that $\rho(f^\ast(\eta(\vp)))=f^\#(T)\circ \vp$. Since $\eta$ is the inverse of $\rho$, it is enough to prove that $f^\ast(\eta(\vp))=\eta(f^\#(T)\circ \vp)$. 
Equality holds because for all $s\in S$ both of the maps induced on stalks are represented by the matrix
\[f^\#_s\pa{\pa{\vp(X_I\ii X)}_{f(s)}}.\]
We conclude by recalling that the following diagram commutes
% https://q.uiver.app/#q=WzAsNCxbMSwwLCJcXEhvbV97XFxLXFx0ZXh0ey1hbGd9fVxccGF7XFxLXFxzcGF7XFxmcmFje1xcZGV0IFhfSn17XFxkZXQgWF9JfX0sXFxPY19UKFQpfSJdLFsxLDEsIlxcSG9tX3tcXEtcXHRleHR7LWFsZ319XFxwYXtcXEtcXHNwYXtcXGZyYWN7XFxkZXQgWF9KfXtcXGRldCBYX0l9fSxcXE9jX1MoUyl9Il0sWzAsMCwiXFxIb21fe1xcU2NoXFxLfShULFxcR3JfSShrLG4pKSJdLFswLDEsIlxcSG9tX3tcXFNjaFxcS30oUyxcXEdyX0koayxuKSkiXSxbMiwwLCJcXFNwZWMiXSxbMywxLCJcXFNwZWMiXSxbMiwzLCJoX3tcXEdyX0koayxuKX0oZikiLDJdLFswLDEsIlxcSG9tXFxwYXtcXEtcXHNwYXtcXGZyYWN7XFxkZXQgWF9KfXtcXGRldCBYX0l9fSxmXlxcIyhUKX0iXV0=
\[\begin{tikzcd}
	{\Hom_{\Sch\K}(T,\Gr_I(k,n))} & {\Hom_{\K\text{-alg}}\pa{\K\spa{\frac{\det X_J}{\det X_I}},\Oc_T(T)}} \\
	{\Hom_{\Sch\K}(S,\Gr_I(k,n))} & {\Hom_{\K\text{-alg}}\pa{\K\spa{\frac{\det X_J}{\det X_I}},\Oc_S(S)}}
	\arrow["\Spec", from=1-1, to=1-2]
	\arrow["{h_{\Gr_I(k,n)}(f)}"', from=1-1, to=2-1]
	\arrow["{\Hom\pa{\K\spa{\frac{\det X_J}{\det X_I}},f^\#(T)}}", from=1-2, to=2-2]
	\arrow["\Spec", from=2-1, to=2-2]
\end{tikzcd}\]
\end{proof}

\begin{theorem}\label{GrassmannianIsModuliSpace}
The Grassmann scheme $\Gr(k,n)$ is a fine moduli space for the Grassmann functor $\gr(k,n)$.
\end{theorem}
\begin{proof}
We know that $\cpa{\gr_I(k,n)\to\gr(k,n)}$ is an open cover (\ref{GrIAreOpenCover}), that $\gr(k,n)$ is a Zariski sheaf (\ref{GrassmannianIsSheaf}) and that $h_{\Gr_I(k,n)}\cong \gr_I(k,n)$ (\ref{GrIRepresentGrIFunctors}). If we can show that these isomorphisms restrict well to double intersections we have the desired result by proposition (\ref{MapGluingForZariskiSheaves}).
Let $T$ be a scheme and let us consider a morphism
\[f\in\Hom_{\Sch\K}\pa{T,\Gr_I(k,n)\cap \Gr_J(k,n)}=\Hom_{\Sch\K}\pa{T,\Gr(k,n)\cap D_+(z_Iz_J)}.\]
Applying a standard result for morphisms towards an affine scheme\footnote{see remark (\ref{IntersectionIsAffine}).} we get 
\[f^\#(T)\in\Hom_{\K\text{-alg}}\pa{\pa{\K\spa{\det X_L}_{\det X_I\det X_J}}^0,\Oc_T(T)}.\]
By the universal property of localization, we may identify this set with
\[\cpa{\beta\in \Hom_{\K\text{-alg}}\pa{\K\spa{\frac{\det X_L}{\det X_I}},\Oc_T(T)}\mid \beta\pa{\frac{\det X_J}{\det X_I}}\in \Oc_T(T)^\ast}.\]
Applying the functor $\eta$  defined during the proof of proposition (\ref{GrIRepresentGrIFunctors}), which we will denote $\eta^I$ to emphasize which determinant we consider at the denominator, we obtain\footnote{the condition on the image of $\frac{\det X_J}{\det X_I}$ corresponds to $\det(\al\circ s_J)$ being invertible, and thus to $\al\circ s_J$ being an isomorphism.} 
\[\eta^I(f^\#(T))\in\cpa{\al:\Oc_T^n\to \Oc^k_T\mid \al\circ s_I=id_{\Oc^k_T},\ \al\circ s_J\text{ isomorphism}}.\]
Observe that we can identify this set with
\[\quot{\cpa{\al:\Oc^n_T\to Q\mid \al\circ s_I \text{ and }\al\circ s_J\text{ surjective}}}{\sim}=(\gr_I(k,n)\times_{\gr(k,n)}\gr_J(k,n))(T),\]
so to conclude the proof we just need to verify that $\eta^I(f^\#(T))\sim \eta^J(f^\#(T))$ in $\gr(k,n)$. By lemma (\ref{FormalBasisChange}), the matrix $X_J\ii X_I$ can be described only using elements in the ring $\K\spa{\det X_L}_{\det X_I\det X_J}^0$. We can thus define $\theta$ by setting $\theta_V(e_j)=f^\#(V)(X_J\ii X_I)e_j$. It is clear by construction that $\theta\circ \eta^I(f^\#(T))=\eta^J(f^\#(T))$. Defining $\delta$ from $X_I\ii X_J$ analogously yields an inverse of $\theta$, realizing the sought out equivalence. 
\end{proof}

\bibliographystyle{plain}
\bibliography{refs}


%\appendix
%\include{Riconoscimenti.tex}

\end{document}
