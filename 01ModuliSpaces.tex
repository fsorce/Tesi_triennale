\chapter{Moduli Spaces}
Often in math we want to answer ``solve" classification problems, that is, if we fix some type of object and impose some equivalence, usually an isomorphism relation, we want to identify and study the equivalence classes.\\
Simply giving the isomorphism classes is trivial\footnote{if the classes form a set they can be identified with a canonical set of the same cardinality} but for an answer to a classification problem to be satisfactory we usually require some information on families of isomorphism classes. The best result we can hope for is finding a space which completely encodes how these families behave (what will be called a fine moduli space), but this can not always be done. Sometimes we can still find a space that encodes the isomorphism classes and some information on the families (coarse moduli space) and other times no single space can be found and the families themselves are studied.\\
To formalize this notion of families the most fruitful approach has been the functorial one. In this chapter the vey basics of this approach are introduced.
\medskip

\noindent Most definitions given in this chapter follow \cite{Alper} and \cite{Bejleri1}.

\section{Moduli problems and moduli spaces}
In this section we introduce the basic concept of the functorial approach and some of the required category theory. In the next section the topics more specific to finding fine moduli spaces are presented.

\subsection{Yoneda lemma and embedding}
All categories are considered small. If $\Cc$ is a category, the notation $X\in \Cc$ is to be read as ``$X$ is an object in $\Cc$".\\
If $A$ and $B$ are objects in a category (resp. categories / functors) then $\Hom(A,B)$ denotes the set of morphisms (resp. functors / natural transformations) from $A$ to $B$.

\begin{definition}[Presheaf]
A contravariant functor $F:\Cc\op\to \Set$ is called a \textbf{presheaf} on $\Cc$. For any fixed category $\Cc$, the presheaves on $\Cc$ form a category $\Fun(\Cc\op,\Set)$ with morphisms given by natural transformations.\\
If $T\in \Cc$ then we call the elements of $F(T)$ \textbf{families} over $T$.
\end{definition}

\begin{definition}[Hom-functor]
Let $\Cc$ be a category and $X\in \Cc$. We define the \textbf{Hom-functor} of $X$ to be
\[h_X:\functorDef{\Cc\op}{\Set}{T}{\Hom(T,X)}{f:T\to S}{\circ f:\Hom(S,X)\to \Hom(T,X)}\]
\end{definition}
\begin{remark}
The Hom-functor is a presheaf.
\end{remark}

\begin{definition}[Moduli problem]
Let $S$ be a scheme. A presheaf on $\Sch S$ is called a \textbf{moduli problem}.
\end{definition}

\begin{remark}
Usually we study moduli problems of the following form
\[\functorDef{\Sch S}{\Set}{T}{\quot{\cpa{\text{families over $T$}}}\sim}{f:T\to R}{\text{pullback of families along }f}\]
where $\sim$ refers to the equivalence relation we are studying the families of classes of.
\end{remark}

\noindent To proceed we will need the following

\begin{lemma}[Yoneda Lemma]\label{YonedaLemma}
Let $\Cc$ be a category and $X\in \Cc$. For all contravariant functors $F:\Cc\op\to\Set$ the following sets are in a natural bijection\footnote{when we write ``morphism" when referring to functors we mean natural transformations.}
\[\Hom(h_X,F)\longleftrightarrow F(X).\]
\end{lemma}
\begin{proof}
Given a natural trasformation $\zeta$, we can take its image in $F(X)$ to be $\zeta_X(id_X)$.\\
On the other hand, for any given element $u\in F(X)$ we can define an arrow $h_X(T)\to F(T)$ for any $T\in \Cc$ by taking $f\mapsto F(f)(u)$. This collection of maps defines a natural transformation from $h_X$ to $F$ by commutativity of the following diagram for all $g:S\to T$
\[\begin{tikzcd}
	{\Hom(X,X)} \\
	& {\Hom(T,X)} & {\Hom(S,X)} \\
	& {F(T)} & {F(S)} \\
	{F(X)}
	\arrow["{\circ f}"', from=1-1, to=2-2]
	\arrow["{\circ (f\circ g)}", from=1-1, to=2-3]
	\arrow["{F(\cdot)(u)}"', from=1-1, to=4-1]
	\arrow["{\circ g}"', from=2-2, to=2-3]
	\arrow["{F(\cdot)(u)}"', from=2-2, to=3-2]
	\arrow["{F(\cdot)(u)}", from=2-3, to=3-3]
	\arrow["{F(g)}", from=3-2, to=3-3]
	\arrow["{F(f)}", from=4-1, to=3-2]
	\arrow["{F(f\circ g)}"', from=4-1, to=3-3]
\end{tikzcd}\]
\end{proof}

\begin{definition}[Yoneda embedding]
We define the \textbf{Yoneda embedding} of a category $\Cc$ to be the following functor
\[\yo:\functorDef{\Cc}{\Fun(\Cc\op,\Set)}{X}{h_X}{f:X\to Y}{f\circ:h_X\to h_Y}\]
\end{definition}

\begin{proposition}
The functor $\yo$ is an embedding, i.e. it is fully faithful.
\end{proposition}
\begin{proof}
Recall that a functor $F:\Cc\to \Dc$ is fully faithful if for any two objects $A,B\in\Cc$ we have $\Hom_\Cc(A,B)\cong \Hom_\Dc(F(A),F(B))$. In our case we want to check if
\[\Hom(X,Y)\cong\Hom(h_X,h_Y),\]
which is exactly the statement of the Yoneda lemma.
\end{proof}
\begin{remark}\label{YonedaEmbeddingInjectiveOnIsoClasses}
$\yo$ is injective up to isomorphism on isomorphism classes of objects in $\Cc$.
\end{remark}
\begin{proof}
Suppose $\zeta:h_X\to h_Y$ is an isomorphism of functors, then $id_X\in h_X(X)$ corresponds to some $f=\zeta(id_X)\in h_Y(X)$. Similarly $id_Y\in h_Y(Y)$ corresponds to some $g=\zeta\ii(id_Y):X\to Y$. By naturality of $\yo$ we see can pull back $id_X$ and $f$ by $g$ to see that 
\[id_Y=\zeta(g)=\zeta(id_X\circ g)=f\circ g\]
and similarly we can show that $g\circ f=id_X$.
\end{proof}

\begin{lemma}\label{YonedaEmbeddingPreservesLimits}
The Yoneda embedding preserves limits.
\end{lemma}
\begin{proof}
Suppose $X$ is the limit of the diagram $\cpa{f_{ij}:X_j\to X_i}$. If we apply the Yoneda embedding to the diagram we obtain
\[\cpa{\circ f_{ij}:h_{X_j}\to h_{X_i}}\]
Let $F$ be any presheaf on $\Cc$ and suppose that we have morphisms $F\to h_{X_i}$ which make the diagram commute, then for all $T\in \Cc$ we have compatible and natural $F(T)\to \Hom(T,X_i)$. If $f\in T(T)$ then these arrows define several $f_i\in \Hom(T,X_i)$ which compose respecting the diagram. By the universal property of limits this defines uniquely a morphism $f_\ell\in \Hom(T,X)$ and we see that the assignment $f\mapsto f_\ell$ is the unique map from $F(T)$ to $\Hom(T,X)$ which makes the diagram in $\Set$ commute. Since all that we have done is natural in $T$ we have effectively constructed a morphism $F\to h_X$ as we desired.
\end{proof}

\subsection{Representable functors}
\begin{definition}[Representable functor]
A presheaf $F$ on $\Cc$ is \textbf{representable} if there exists a natural isomorphism $\zeta:F\to h_X$ for some $X\in \Cc$. In this case we say that the pair\footnote{usually we just say that $X$ represents $F$} $(X,\zeta)$ \textbf{represents} $F$.\\
If $a\in F(T)$ we call the map $\zeta_T(a):T\to X$ the \textbf{classifying map} of the family $a$.
\end{definition}
\begin{definition}[Universal family]
Instead of the pair $(X,\zeta)$ for $\zeta:F\to h_X$, we may consider the pair $(X,\xi)$ where $\xi\in F(X)$ is the image of $\zeta\ii$ under the bijection predicated by the Yoneda lemma.\\
The object $\xi$ is called the \textbf{universal family} of $X$.
\end{definition}
\begin{remark}
The universal family is given by
\[\zeta\ii_X(id_X).\]
\end{remark}


\begin{definition}[Fine moduli space]
Let $F$ be a moduli functor. A scheme $X\in \Sch S$ is a \textbf{fine moduli space} for $F$ if $X$ represents $F$.
\end{definition}

\begin{notation}
If $U$ is a subscheme of $T$ and $i:U\to T$ is the inclusion morphism, then if $\xi\in F(T)$ we will refine its restriction to $U$ to be
\[\xi\res U=F(i)(\xi).\]
\end{notation}

\begin{remark}
Because the Yoneda embedding is injective on isomorphism classes (\ref{YonedaEmbeddingInjectiveOnIsoClasses}), moduli spaces are unique up to isomorphism.
\end{remark}

\begin{example}[Projective space]
Consider the functor
\[\Pj_n:\functorDef{\mathrm{Sch}\op}{\Set}{S}{\quot{\cpa{(\Lc,s_0,\cdots,s_n)\mid \emat{\Lc\text{ line bundle on $S$, }s_0,\cdots,s_n\in\Lc(S),\\\forall x\in S,\ \ps{(s_0)_x,\cdots,(s_n)_x}_{\Oc_{S,x}}=\Lc_x}}}\sim}{f}{\text{pullback of sheaves and sections via $f$}}\]
where $(\Lc,(s_i))\sim (\Lc',(s_i'))$ is there exists a sheaf isomorphism $\al:\Lc\to\Lc'$ such that $s_i=\al^\ast s_i'$ for all $i\in\cpa{0,\cdots, n}$.\medskip

\noindent It is a well know fact (Proposition 5.1.31 in \cite{QingLiu}) that $\Pj_n(S)\cong \Hom(S,\Pj^n_\Z)$ and that pullbacks behave as expected, thus $\Pj^n_\Z$ is a fine moduli space for $\Pj_n$.\\
From the statement of Proposition 5.1.31 in \cite{QingLiu} it is also clear that $\Oc_{\Pj^n_\Z}(1)$ is a universal family.
\end{example}

\noindent Fine moduli spaces do not always exist. The simplest obstructions to having a fine moduli spaces are
\begin{itemize}
\item the functor is not a Zariski sheaf (see proposition (\ref{RepresentableModuliFunctorsAreZariskiSheaves}))
\item existence of non trivial automorphisms.
\end{itemize}

\noindent To get an idea of why the second condition is an obstruction consider the following
\begin{proposition}
Let $F\in \Psh{\C}{\Set}$ be a moduli functor. If there exists an \textbf{isotrivial family} $\Ec\in F(S)$ for $S\in \Sch\C$ variety, i.e.
\begin{itemize}
\item for all $s,t\in S(\C)$, the fiber $F(s)(\Ec)=\Ec_s=\Ec_t$ 
\item the family $\Ec$ is not the pullback of an object $E\in F(\Spec \C)$ along the structural morphism $S\to \Spec \C$
\end{itemize}
then there exists no fine moduli space for $F$.
\end{proposition}
\begin{proof}
This is Proposition 0.3.21 in \cite{Alper}. [ISERT LATER MAYBE]
\end{proof}

\noindent A weaker notion of moduli space is that of coarse moduli space:
\begin{definition}[Coarse moduli space]
Let $F$ be a moduli problem. A pair $(X,\zeta)$ for $X\in \Sch S$ and $\zeta:F\to h_X$ natural transformation is a \textbf{coarse moduli space} for $F$ if
\begin{itemize}
\item $\zeta_{\Spec \K}:F(\Spec\K)\to \Hom(\Spec\K,X)$ is a bijection for all algebraically closed fields $\K$
\item for any scheme $Y$ and $\eta:F\to h_{Y}$ natural transformation there exists a unique morphism $\al:X\to Y$ such that $\eta=h_\al\circ \zeta$.
\end{itemize}
\end{definition}
\begin{remark}
A fine moduli space is also a coarse moduli space.
\end{remark}
\begin{proof}
The first condition is trivial. For the second condition, if $(Y,\eta)$ is as defined above and $(X,u)$ is the fine moduli space with universal family $u$ then we can take $\al=\eta_X(u)$.
\end{proof}


\section{Zariski sheaves and gluing of fine moduli spaces}
In order to show that a moduli problem is representable, one approach we can follow is trying to emulate the gluing properties of sheaves. Indeed it is possible to show that representable functors are sheaves of some kind and this realization will eventually lead us to the representability criterion we will use later in this article. 

\subsection{Zariski sheaves}
First, let us formalize a way in which a functor can be a sheaf. First we recall the definition of equalizer:

\begin{definition}[Equalizer]
Let $\Cc$ be a category, $A,B,C\in \Cc$ and $f,g:B\to C$. We say that the pair $(A,h)$ is an \textbf{equalizer} of the diagram \[B\underset{g}{\overset{f}{\rightrightarrows}} C\]
if $h:A\to B$ is such that $f\circ h=g\circ h$ and if $(Q,q)$ is another such pair then there exists a unique morphism $Q\to A$ which makes the diagram commute
\[\begin{tikzcd}
	A & B & C \\
	Q
	\arrow["h", from=1-1, to=1-2]
	\arrow["f", shift left, from=1-2, to=1-3]
	\arrow["g"', shift right, from=1-2, to=1-3]
	\arrow["q"', from=2-1, to=1-2]
	\arrow[dashed, from=2-1, to=1-1]
\end{tikzcd}\]
\end{definition}

\begin{definition}[Zariski sheaf]
A moduli problem $F\in \Psh S\Set$ is a \textbf{Zariski sheaf} if for any $S$-scheme $X$ and any Zariski open cover $\cpa{U_i\to X}$ the following diagram is an equalizer
% https://q.uiver.app/#q=WzAsMyxbMCwwLCJGKFgpIl0sWzIsMCwiXFxkaXNwbGF5c3R5bGUgXFxwcm9kX2sgRihVX2spIl0sWzQsMCwiXFxkaXNwbGF5c3R5bGUgXFxwcm9kX3tpLGp9RihVX2lcXGNhcCBVX2opIl0sWzAsMV0sWzEsMiwiIiwwLHsib2Zmc2V0IjotMX1dLFsxLDIsIiIsMix7Im9mZnNldCI6MX1dXQ==
\[\begin{tikzcd}
	{F(X)} && {\displaystyle \prod_k F(U_k)} && {\displaystyle \prod_{i,j}F(U_i\cap U_j)}
	\arrow[from=1-1, to=1-3]
	\arrow[shift left, from=1-3, to=1-5]
	\arrow[shift right, from=1-3, to=1-5]
\end{tikzcd}\]
where the arrows are induced by the inclusions.
\end{definition}


\begin{proposition}[Representable moduli functors are Zariski sheaves]\label{RepresentableModuliFunctorsAreZariskiSheaves}
Let $F:\Psh S\Set$ be a moduli problem, then if there exists a fine moduli space $M$ for $F$ it must be the case that $F$ is a Zariski sheaf.
\end{proposition}
\begin{proof}
By composing with the natural isomorphism we may assume $F=h_M$. Let $X$ be an $S$-scheme and $\cpa{U_i\to X}$ a Zariski open cover for it. We want to show that the following diagram is an equalizer
% https://q.uiver.app/#q=WzAsMyxbMCwwLCJIb20oVSxNKSJdLFsyLDAsIlxcZGlzcGxheXN0eWxlIFxccHJvZF9pIEhvbShVX2ksTSkiXSxbNCwwLCJcXGRpc3BsYXlzdHlsZSBcXHByb2Rfe2ksan1Ib20oVV9pXFx0aW1lc19VVV9qLE0pIl0sWzAsMSwiUmVzIl0sWzEsMiwicHJfMV5cXGFzdCIsMCx7Im9mZnNldCI6LTF9XSxbMSwyLCJwcl8yXlxcYXN0IiwyLHsib2Zmc2V0IjoxfV1d
\[\begin{tikzcd}
	{\Hom(X,M)} && {\displaystyle \prod_k \Hom(U_k,M)} && {\displaystyle \prod_{i,j}\Hom(U_i\cap U_j,M)}
	\arrow[from=1-1, to=1-3]
	\arrow[shift left, from=1-3, to=1-5]
	\arrow[shift right, from=1-3, to=1-5]
\end{tikzcd}\]
The arrows in this case correspond to restriction of morphisms so what we need to verify is
\begin{itemize}
\item $\mathrm{res}^{U_i}_{U_i\cap U_j}\circ \mathrm{res}^X_{U_i}=\mathrm{res}^{U_j}_{U_i\cap U_j}\circ \mathrm{res}^X_{U_j}$
\item a collection of maps $\cpa{f_i:U_i\to M}$ such that $f_i\res{U_i\cap U_j}=f_j\res{U_i\cap U_j}$ glues uniquely to a map $f:X\to M$
\end{itemize}
both of which are true.
\end{proof}

\subsection{Open cover of a moduli problem}
\noindent Let us now define the analogue of an open cover for functors
\begin{definition}[Subfunctor]
Let $G:\Cc\to \Dc$ be a functor. A \textbf{subfunctor} of $G$ is a pair $(F,i)$ consisting of a functor $F:\Cc\to \Dc$ and a natural transformation $i:F\to G$ such that $i_X:F(X)\to G(X)$ is a monomorphism for all $X\in \Cc$.
\end{definition}
\begin{remark}
If $\Dc=\Set$ then we can express the same data equivalently as follows:\\
A functor $F:\Cc\to \Set$ is a subfunctor of $G:\Cc\to \Set$ if for all $X\in \Cc$ and for all $f\in \Hom_\Cc(A,B)$
\[F(X)\subseteq G(X),\qquad\text{and}\qquad F(f)=G(f)\res{F(A)}.\]
In this case we write $F\subseteq G$.
\end{remark}

\begin{definition}[Fibered product of presheaves]
Let $F,G,H:\Cc\op\to\Set$ be presheaves together with two natural transformations $\eta:F\to H$ and $\zeta:G\to H$. We define their fibered product as the following functor
\[F\times_H G:\functorDef{\Cc\op}{\Set}{X}{F(X)\times_{H(X)}G(X)}{f:A\to B}{(b_1,b_2)\mapsto(F(f)(b_1),G(f)(b_2))}\]
where the fibered product in defined through the maps $\eta_X$ and $\zeta_X$. The map $(F\times_H G)(f)$ is well defined because if $(b_1,b_2)\in F(B)\times_{H(B)}G(B)$ then $\eta_B(b_1)=\zeta_B(b_2)$, thus
\[\eta_A(F(f)(b_1))=H(f)(\eta_B(b_1))=H(f)(\zeta_B(b_2))=\zeta_A(G(f)(b_2)).\]
\end{definition}

\begin{definition}[Open subfunctor]
Let $F:\Psh S\Set$ be a moduli problem. We say that a subfunctor $G\subseteq F$ is \textbf{open} if for any $S$-scheme $T$ and any natural transformation $h_T\to F$, there exists an open subscheme $U$ of $T$ such that $U$ represents $h_T\times_FG$, i.e.
% https://q.uiver.app/#q=WzAsNCxbMCwwLCJoX1UiXSxbMCwxLCJoX1QiXSxbMSwxLCJGIl0sWzEsMCwiRyJdLFszLDJdLFsxLDJdLFswLDMsIiIsMCx7InN0eWxlIjp7ImJvZHkiOnsibmFtZSI6ImRhc2hlZCJ9fX1dLFswLDEsIiIsMix7InN0eWxlIjp7ImJvZHkiOnsibmFtZSI6ImRhc2hlZCJ9fX1dLFswLDIsIiIsMSx7InN0eWxlIjp7Im5hbWUiOiJjb3JuZXItaW52ZXJzZSJ9fV1d
\[\begin{tikzcd}
	{h_U} & G \\
	{h_T} & F
	\arrow[dashed, from=1-1, to=1-2]
	\arrow[dashed, from=1-1, to=2-1]
	\arrow["\ulcorner"{anchor=center, pos=0.125}, draw=none, from=1-1, to=2-2]
	\arrow[from=1-2, to=2-2]
	\arrow[from=2-1, to=2-2]
\end{tikzcd}\]
\end{definition}

\begin{remark}
By the Yoneda lemma, giving a natural transformation like in the above definition is equivalent to choosing a family $\xi\in F(T)$. We can thus rephrase the definition as follows:\\
A subfunctor $G\subseteq F$ is open if for any $S$-scheme $T$ and any family $\xi\in F(T)$ there exists an open subscheme $U\subseteq T$ such that the following diagram is natural in $R$, commutes and for any map $f:R\to U$ there exists a $g:R\to U$ such that $f=\subseteq \circ g$ if and only if $F(f)(\xi)\in G(R)$\footnote{the ``only if" is trivially true by commutativity but for the ``if" we are using the fact that $h_U\cong h_T\times_FG$.}
% https://q.uiver.app/#q=WzAsNCxbMCwwLCJcXE1vcihSLFUpIl0sWzAsMSwiXFxNb3IoUixUKSJdLFsyLDAsIkcoUikiXSxbMiwxLCJGKFIpIl0sWzAsMSwiXFxzdWJzZXRlcVxcY2lyYyIsMix7InN0eWxlIjp7InRhaWwiOnsibmFtZSI6Imhvb2siLCJzaWRlIjoidG9wIn19fV0sWzAsMiwiRyhcXHN1YnNldGVxIFxcY2lyYyB+flxcY2RvdH4pKFxceGkpIl0sWzEsMywiRihcXGNkb3QpKFxceGkpIiwyXSxbMiwzLCJcXHN1YnNldGVxIiwwLHsic3R5bGUiOnsidGFpbCI6eyJuYW1lIjoiaG9vayIsInNpZGUiOiJ0b3AifX19XV0=
\[\begin{tikzcd}
	{\Hom(R,U)} && {G(R)} \\
	{\Hom(R,T)} && {F(R)}
	\arrow["\subseteq\circ"', hook, from=1-1, to=2-1]
	\arrow["{G(\subseteq \circ ~~\cdot~)(\xi)}", from=1-1, to=1-3]
	\arrow["{F(\cdot)(\xi)}"', from=2-1, to=2-3]
	\arrow["\subseteq", hook, from=1-3, to=2-3]
\end{tikzcd}\]
\end{remark}

\begin{definition}[Open cover of a functor]
Let $F:\Psh S\Set$ be a moduli problem. A collection of open subfunctors $\cpa{F_i}$ is an \textbf{open cover} of $F$ if for any $S$-scheme $T$ and any natural transformation $h_T\to F$, there exists an open cover $\cpa{U_i}$ of $T$ such that $U_i$ represents $h_T\times_F F_i$ for all $i$.
\end{definition}
\begin{remark}
Like above, we can rephrase the definition as follows:\\
A collections of open subfunctors $F_i\subseteq F$ form an open cover of $F$ if for any $S$-scheme $T$ and any family $\xi\in F(T)$, there exists an open cover $\cpa{U_i}$ of $T$ such that $\xi\res{U_i}\in F_i(U_i)$ for all $i$.
\end{remark}

\subsection{Representability criterion}
Finally, we come to the main result for this chapter

\begin{theorem}[Representability by open cover]\label{RepresentabilityByOpenSubfunctorCover}
Let $F:\Psh{S}{\Set}$ be a moduli problem which is a Zariski sheaf and let $\cpa{F_i}$ be an open cover of it by representable subfunctors, then $F$ is representable.
\end{theorem}
\begin{proof}
For this proof we will mainly follow the version presented in \cite{GortzAlgGeoISchemes} (Theorem 8.9 at page 212).\\
Let $X_i$ be the fine moduli space for $F_i$ and let $\xi_i\in F_i(X_i)$ be their universal families.\\
Note that for all $S$-schemes $T$ we have
\[(F_i\times_F F_j)(T)=F_i(T)\times_{F(T)}F_j(T)=F_i(T)\cap F_j(T)\subseteq F(T),\]
it follows that $F_i\times_F F_j=F_j\times_F F_i\doteqdot F_{i,j}$. We can define analogously $F_{i,j,k}$.
\medskip

\noindent Since $F_j$ is an open subfunctor of $F$, there exists an open subscheme $U_{ij}\subseteq X_i$ which represents $h_{X_i}\times_F F_j\cong F_i\times_F F_j=F_{i,j}$. We can define $U_{ji}\subseteq X_j$ similarly and since they are both moduli spaces for $F_{i,j}$ they are isomorphic. Let $\vp_{ji}:U_{ij}\to U_{ji}$ be the isomorphism given by $\vp_{ji}=\al_{U_{ij}}(id_{U_{ij}})$ for $\al$ natural isomorphism which makes the following diagram commute
% https://q.uiver.app/#q=WzAsNixbMCwwLCJoX3tVX3tpan19Il0sWzAsMSwiaF97VV97aml9fSJdLFsxLDAsImhfe1hfaX1cXHRpbWVzX0YgRl9qIl0sWzIsMCwiRl97aSxqfSJdLFsyLDEsIkZfe2ksan0iXSxbMSwxLCJGX2lcXHRpbWVzX0YgaF97WF9qfSJdLFsyLDMsIlxcY29uZyIsMyx7InN0eWxlIjp7ImJvZHkiOnsibmFtZSI6Im5vbmUifSwiaGVhZCI6eyJuYW1lIjoibm9uZSJ9fX1dLFs1LDQsIlxcY29uZyIsMyx7InN0eWxlIjp7ImJvZHkiOnsibmFtZSI6Im5vbmUifSwiaGVhZCI6eyJuYW1lIjoibm9uZSJ9fX1dLFszLDQsIj0iLDMseyJzdHlsZSI6eyJib2R5Ijp7Im5hbWUiOiJub25lIn0sImhlYWQiOnsibmFtZSI6Im5vbmUifX19XSxbMCwxLCJcXGFscGhhIiwyXSxbMCwyLCJcXGNvbmciLDMseyJzdHlsZSI6eyJib2R5Ijp7Im5hbWUiOiJub25lIn0sImhlYWQiOnsibmFtZSI6Im5vbmUifX19XSxbMSw1LCJcXGNvbmciLDMseyJzdHlsZSI6eyJib2R5Ijp7Im5hbWUiOiJub25lIn0sImhlYWQiOnsibmFtZSI6Im5vbmUifX19XV0=
\[\begin{tikzcd}
	{h_{U_{ij}}} & {h_{X_i}\times_F F_j} & {F_{i,j}} \\
	{h_{U_{ji}}} & {F_i\times_F h_{X_j}} & {F_{i,j}}
	\arrow["\cong"{marking, allow upside down}, draw=none, from=1-2, to=1-3]
	\arrow["\cong"{marking, allow upside down}, draw=none, from=2-2, to=2-3]
	\arrow["{=}"{marking, allow upside down}, draw=none, from=1-3, to=2-3]
	\arrow["\alpha"', from=1-1, to=2-1]
	\arrow["\cong"{marking, allow upside down}, draw=none, from=1-1, to=1-2]
	\arrow["\cong"{marking, allow upside down}, draw=none, from=2-1, to=2-2]
\end{tikzcd}\]
Note that if $T$ is an $S$-scheme and $f\in h_{U_{ij}}(T)$ then
\[h_{\vp_{ji}}(f)=\al_{U_{ij}}(id_{U_{ij}})\circ f =\al_T(id_{U_{ij}}\circ f)=\al_T(f),\]
so $\al$ is the image of $\vp_{ji}$ under the Yoneda embedding.
\medskip

\noindent We now want to show that the $X_i$ can be glued along the $U_{ij}$ using the isomorphisms $\vp_{ji}$. First we need to show that $\vp_{ji}(U_{ij}\cap U_{ik})=U_{ji}\cap U_{jk}$ and then we have to verify the cocycle condition $\vp_{ki}=\vp_{kj}\circ \vp_{ji}$.\\
The first condition follows immediately from the fact that $F_k$ is an open subfunctor and our construction of the $\vp_{ji}$.\\
Since the Yoneda embedding preserves limits (\ref{YonedaEmbeddingPreservesLimits}) it preserves fibered products, so we see that the following diagram commutes
% https://q.uiver.app/#q=WzAsOCxbMSwwLCJoX3tVX3tpan19XFx0aW1lc197aF97WF9pfX1oX3tVX3tpa319Il0sWzEsMSwiaF97VV97aml9fVxcdGltZXNfe2hfe1hfan19aF97VV97amt9fSJdLFsyLDAsIkZfe2ksan1cXHRpbWVzX3tGX2l9IEZfe2ksa30iXSxbMywwLCJGX3tpLGosa30iXSxbMywxLCJGX3tpLGosa30iXSxbMiwxLCJGX3tpLGp9XFx0aW1lc197Rl9qfSBGX3tqLGt9Il0sWzAsMCwiaF97VV97aWp9XFxjYXAgVV97aWt9fSJdLFswLDEsImhfe1Vfe2ppfVxcY2FwIFVfe2prfX0iXSxbMiwzLCJcXGNvbmciLDMseyJzdHlsZSI6eyJib2R5Ijp7Im5hbWUiOiJub25lIn0sImhlYWQiOnsibmFtZSI6Im5vbmUifX19XSxbNSw0LCJcXGNvbmciLDMseyJzdHlsZSI6eyJib2R5Ijp7Im5hbWUiOiJub25lIn0sImhlYWQiOnsibmFtZSI6Im5vbmUifX19XSxbMyw0LCI9IiwzLHsic3R5bGUiOnsiYm9keSI6eyJuYW1lIjoibm9uZSJ9LCJoZWFkIjp7Im5hbWUiOiJub25lIn19fV0sWzAsMiwiXFxjb25nIiwzLHsic3R5bGUiOnsiYm9keSI6eyJuYW1lIjoibm9uZSJ9LCJoZWFkIjp7Im5hbWUiOiJub25lIn19fV0sWzEsNSwiXFxjb25nIiwzLHsic3R5bGUiOnsiYm9keSI6eyJuYW1lIjoibm9uZSJ9LCJoZWFkIjp7Im5hbWUiOiJub25lIn19fV0sWzYsMCwi44KIIl0sWzcsMSwi44KIIl0sWzYsNywiXFx2cF97aml9IiwyXV0=
\[\begin{tikzcd}
	{h_{U_{ij}\cap U_{ik}}} & {h_{U_{ij}}\times_{h_{X_i}}h_{U_{ik}}} & {F_{i,j}\times_{F_i} F_{i,k}} & {F_{i,j,k}} \\
	{h_{U_{ji}\cap U_{jk}}} & {h_{U_{ji}}\times_{h_{X_j}}h_{U_{jk}}} & {F_{i,j}\times_{F_j} F_{j,k}} & {F_{i,j,k}}
	\arrow["\cong"{marking, allow upside down}, draw=none, from=1-3, to=1-4]
	\arrow["\cong"{marking, allow upside down}, draw=none, from=2-3, to=2-4]
	\arrow["{=}"{marking, allow upside down}, draw=none, from=1-4, to=2-4]
	\arrow["\cong"{marking, allow upside down}, draw=none, from=1-2, to=1-3]
	\arrow["\cong"{marking, allow upside down}, draw=none, from=2-2, to=2-3]
	\arrow["{\yo}", from=1-1, to=1-2]
	\arrow["{\yo}", from=2-1, to=2-2]
	\arrow["{\vp_{ji}}"', from=1-1, to=2-1]
\end{tikzcd}\]
thus to prove that $\vp_{kj}\circ \vp_{ji}=\vp_{ki}$ it is enough to see that $id_{F_{i,j,k}}\circ id_{F_{i,j,k}}=id_{F_{i,j,k}}$.
\bigskip

\noindent
We can thus define $X$ to be the scheme obtained by gluing the $X_i$ along the $U_{ij}$.\\
Observe that $\xi_i=\vp_{ji}^\ast\xi_j$, so if we look at these families as elements of $F(X)$ we see that $\xi_i\res{U_ij}=\xi_j\res{U_{ij}}$. Since $F$ is a Zariski sheaf, the $\xi_i$ can be glued to a family $\xi\in F(X)$.\medskip

\noindent We now only need to verify that $(X,\xi)$ is a fine moduli space for $F$:\\
Let $T$ be an $S$-scheme and and let us consider a family $\zeta\in F(T)$. Since $\cpa{F_i}$ is an open cover of $F$, there exists an open cover $\cpa{V_i}$ of $T$ such that $\zeta\res{V_i}\in F_i(V_i)\cong \Hom(V_i,X_i)$. Since $F$ is a sheaf and $\zeta_i\res{V_i\cap V_j}=\zeta_j\res{V_i\cap V_j}$, the morphisms $V_i\to X_i$ corresponding to the $\zeta_i$ glue to a morphism $f:T\to X$ such that $f^\ast\xi=\zeta$ (by construction).
\end{proof}