\chapter{Moduli Spaces}
Set theoretic issues: whenever I write that something is an element of a class, what I mean is that that object satisfies the proposition that defines the class.

\section{Introduction to moduli problems}
\begin{definition}[Presheaf]
A contravariant functor $F:\Cc\op\to \Set$ is called a \textbf{presheaf} on $\Cc$.
\end{definition}

\begin{definition}[Moduli problem]
Let $S$ be a scheme. A presheaf on $\Sch S$ is called a \textbf{moduli problem}.
\end{definition}

\section{Fine and Coarse moduli spaces}

\section{Zariski sheaves and gluing of fine moduli spaces}

\comm{\begin{definition}[Grothendieck pretopology]
Given a category $\Cc$, a \textbf{Grothendieck pretopology} on it is given by a collection $\Tc$ of sets of morphisms $\cpa{U_i\to U}$ such that the following properties hold:
\begin{itemize}
\item If $f:V\to U$ is an isomorphism, then $\cpa{f:V\to U}$ is an element of $\Tc$.
\item If $f:V\to U$ is a morphism and $\cpa{U_i\to U}\in \Tc$ then the set\footnote{we are implicitly assuming that the category $\Cc$ contains at least the fibered products which appear in this definition.} $\cpa{U_i\times_U V\to V}$ is an element of $\Tc$ (the morphisms are given by the projection from the fibered product to $V$).
\item If $\cpa{U_i\to U}\in \Tc$ and for all $i$ we have a set $\cpa{V_{ij}\to U_i}\in \Tc$, then the set of compositions $\cpa{V_{ij}\to U}$ is an element of $\Tc$.
\end{itemize}
A category together with a Grothendieck pretopology is called a \textbf{site}. The sets in $\Tc$ are called \textbf{covers}.
\end{definition}
\begin{remark}
The axioms above try to mimic the behavior of open covers for a topological space.
\end{remark}
\begin{definition}[Zariski site]
The \textbf{Zariski site} is the category of $S$-schemes together with the following collection of covers:\\
a set of morphisms $\cpa{f_i:U_i\to U}$ is a cover in the Zariski site if every morphism is an open immersion and $\bigcup_i f_i(U_i)=U$, i.e. the family is \textbf{jointly surjective}.
\end{definition}

\begin{definition}[Sheaf on a site]
Let $\Cc$ be a site and $F:\Cc^{op}\to Set$ be a presheaf, then $F$ is a sheaf on $\Cc$ if for any cover $\cpa{U_i\to U}$ the following diagram is an equalizer\footnote{$pr_1^\ast\circ Res=pr_2^\ast\circ Res$ and if we have $q:Q\to \prod_i F(U_i)$ such that $pr_1^\ast\circ q=pr_2^\ast\circ q$ then there exists a morphism $Q\to F(U)$ that makes the resulting diagram commute.}
% https://q.uiver.app/#q=WzAsMyxbMCwwLCJGKFUpIl0sWzIsMCwiXFxkaXNwbGF5c3R5bGUgXFxwcm9kX2lGKFVfaSkiXSxbNCwwLCJcXGRpc3BsYXlzdHlsZSBcXHByb2Rfe2ksan1GKFVfaVxcdGltZXNfVVVfaikiXSxbMCwxLCJSZXMiXSxbMSwyLCJwcl8xXlxcYXN0IiwwLHsib2Zmc2V0IjotMX1dLFsxLDIsInByXzJeXFxhc3QiLDIseyJvZmZzZXQiOjF9XV0=
\[\begin{tikzcd}
	{F(U)} && {\displaystyle \prod_iF(U_i)} && {\displaystyle \prod_{i,j}F(U_i\times_UU_j)}
	\arrow["Res", from=1-1, to=1-3]
	\arrow["{pr_1^\ast}", shift left, from=1-3, to=1-5]
	\arrow["{pr_2^\ast}"', shift right, from=1-3, to=1-5]
\end{tikzcd}\]
where $Res:F(U)\to \prod_i F(U_i)$ is given by the pullback of $U_i\to U$ in each component. Similarly $pr_k^\ast$ are induced by the two canonical projections from the fibered products in the obvious way.
\end{definition}
\begin{remark}
If $X$ is a fixed topological space and $\Cc$ is the category whose objects are open subsets of $X$ and whose morphisms are inclusions then we can consider $\Cc$ as a site by taking as covers the open covers in $X$. Then a functor $F:\Cc^{op}\to Set$ is a sheaf on $X$ if and only if it's a sheaf on $\Cc$.
\end{remark}}








\begin{definition}[Equalizer]
Let $\Cc$ be a category, $A,B,C\in \Cc$ and $f,g:B\to C$. We say that the pair $(A,h)$ is an \textbf{equalizer} of the diagram \[B\underset{g}{\overset{f}{\rightrightarrows}} C\]
if $h:A\to B$ is such that $f\circ h=g\circ h$ and if $(Q,q)$ is another such pair then there exists a unique morphism $Q\to A$ which makes the diagram commute
\[\begin{tikzcd}
	A & B & C \\
	Q
	\arrow["h", from=1-1, to=1-2]
	\arrow["f", shift left, from=1-2, to=1-3]
	\arrow["g"', shift right, from=1-2, to=1-3]
	\arrow["q"', from=2-1, to=1-2]
	\arrow[dashed, from=2-1, to=1-1]
\end{tikzcd}\]
\end{definition}


\begin{definition}[Zariski sheaf]
A moduli problem $F\in \Psh S\Set$ is a \textbf{Zariski sheaf} if for any $S$-scheme $X$ and any Zariski open cover $\cpa{U_i\to X}$ the following diagram is an equalizer
% https://q.uiver.app/#q=WzAsMyxbMCwwLCJGKFgpIl0sWzIsMCwiXFxkaXNwbGF5c3R5bGUgXFxwcm9kX2sgRihVX2spIl0sWzQsMCwiXFxkaXNwbGF5c3R5bGUgXFxwcm9kX3tpLGp9RihVX2lcXGNhcCBVX2opIl0sWzAsMV0sWzEsMiwiIiwwLHsib2Zmc2V0IjotMX1dLFsxLDIsIiIsMix7Im9mZnNldCI6MX1dXQ==
\[\begin{tikzcd}
	{F(X)} && {\displaystyle \prod_k F(U_k)} && {\displaystyle \prod_{i,j}F(U_i\cap U_j)}
	\arrow[from=1-1, to=1-3]
	\arrow[shift left, from=1-3, to=1-5]
	\arrow[shift right, from=1-3, to=1-5]
\end{tikzcd}\]
where the arrows are induced by the inclusions.
\end{definition}


\begin{proposition}[Representable moduli functors are Zariski sheaves]
Let $F:\Psh S\Set$ be a moduli problem, then if there exists a fine moduli space $M$ for $F$ it must be the case that $F$ is a Zariski sheaf.
\end{proposition}
\begin{proof}
By composing with the natural isomorphism we may assume $F=h_M$. Let $X$ be an $S$-scheme and $\cpa{U_i\to X}$ a Zariski open cover for it. We want to show that the following diagram is an equalizer
% https://q.uiver.app/#q=WzAsMyxbMCwwLCJIb20oVSxNKSJdLFsyLDAsIlxcZGlzcGxheXN0eWxlIFxccHJvZF9pIEhvbShVX2ksTSkiXSxbNCwwLCJcXGRpc3BsYXlzdHlsZSBcXHByb2Rfe2ksan1Ib20oVV9pXFx0aW1lc19VVV9qLE0pIl0sWzAsMSwiUmVzIl0sWzEsMiwicHJfMV5cXGFzdCIsMCx7Im9mZnNldCI6LTF9XSxbMSwyLCJwcl8yXlxcYXN0IiwyLHsib2Zmc2V0IjoxfV1d
\[\begin{tikzcd}
	{\Mor(X,M)} && {\displaystyle \prod_k \Mor(U_k,M)} && {\displaystyle \prod_{i,j}\Mor(U_i\cap U_j,M)}
	\arrow[from=1-1, to=1-3]
	\arrow[shift left, from=1-3, to=1-5]
	\arrow[shift right, from=1-3, to=1-5]
\end{tikzcd}\]
The arrows in this case correspond to restriction of morphisms, so the thesis is equivalent to the fact that restriction to a given set doesn't depend on the intermediate restrictions and that morphisms of schemes that coincide on double intersections glue to the union, both of which are true.
\end{proof}


\begin{definition}[Subfunctor]
Let $G:\Cc\to \Dc$ be a functor. A \textbf{subfunctor} of $G$ is a pair $(F,i)$ consisting of a functor $F:\Cc\to \Dc$ and a natural transformation $i:F\to G$ such that $i_X:F(X)\to G(X)$ is a monomorphism for all $X\in \Cc$.
\end{definition}
\begin{remark}
If $\Dc=\Set$ then we can express the same data equivalently as follows:\\
A functor $F:\Cc\to \Set$ is a subfunctor of $G:\Cc\to \Set$ if for all $X\in \Cc$ and for all $f\in Mor_\Cc(A,B)$
\[F(X)\subseteq G(X),\qquad\text{and}\qquad F(f)=G(f)\res{F(A)}.\]
In this case we write $F\subseteq G$.
\end{remark}

\begin{definition}[Fibered product of presheaves]
Let $F,G,H:\Cc\op\to\Set$ be presheaves together with two natural transformations $\xi^1:F\to H$ and $\xi^2:G\to H$. We define their fibered product as follows:\\
If $X\in\Cc$ then 
\[(F\times_H G)(X)=F(X)\times_{H(X)}G(X),\] 
if $f:A\to B$ then\footnote{the map is well defined because $\xi^1_A(F(f)(b_1))=H(f)(\xi^1_B(b_1))=H(f)(\xi^2_B(b_2))=\xi^2_A(G(f)(b_2))$.}
\[(F\times_HG)(f):\funcDef{F(B)\times_{H(B)}G(B)}{F(A)\times_{H(A)}G(A)}{(b_1,b_2)}{(F(f)(b_1),G(f)(b_2))}.\]
\end{definition}
\begin{definition}[Open subfunctor]
Let $F:\Psh S\Set$ be a moduli problem. We say that a subfunctor $G\subseteq F$ is \textbf{open} if for any $S$-scheme $T$ and any natural trasformation $h_T\to F$, the pullback $h_T\times_FG$ is representable by an open subscheme of $T$.
\end{definition}

\begin{remark}
By the Yoneda lemma, giving a natural transformation like in the above definition is equivalent to choosing a family $\xi\in F(T)$. We can thus rephrase the definition as follows:\\
A subfunctor $G\subseteq F$ is open if for any $S$-scheme $T$ and any family $\xi\in F(T)$ there exists an open subscheme $U\subseteq T$ such that the following diagram is natural in $R$ and commutes
% https://q.uiver.app/#q=WzAsNCxbMCwwLCJcXE1vcihSLFUpIl0sWzAsMSwiXFxNb3IoUixUKSJdLFsyLDAsIkcoUikiXSxbMiwxLCJGKFIpIl0sWzAsMSwiXFxzdWJzZXRlcVxcY2lyYyIsMix7InN0eWxlIjp7InRhaWwiOnsibmFtZSI6Imhvb2siLCJzaWRlIjoidG9wIn19fV0sWzAsMiwiRyhcXHN1YnNldGVxIFxcY2lyYyB+flxcY2RvdH4pKFxceGkpIl0sWzEsMywiRihcXGNkb3QpKFxceGkpIiwyXSxbMiwzLCJcXHN1YnNldGVxIiwwLHsic3R5bGUiOnsidGFpbCI6eyJuYW1lIjoiaG9vayIsInNpZGUiOiJ0b3AifX19XV0=
\[\begin{tikzcd}
	{\Mor(R,U)} && {G(R)} \\
	{\Mor(R,T)} && {F(R)}
	\arrow["\subseteq\circ"', hook, from=1-1, to=2-1]
	\arrow["{G(\subseteq \circ ~~\cdot~)(\xi)}", from=1-1, to=1-3]
	\arrow["{F(\cdot)(\xi)}"', from=2-1, to=2-3]
	\arrow["\subseteq", hook, from=1-3, to=2-3]
\end{tikzcd}\]
and a map $f\in \Mor(R,T)$ factors as $R\overset g\to U\subseteq T$ \underline{if and only if} $F(f)(\xi)\in G(R)$\footnote{the ``only if" is trivially true by commutativity but for the ``if" we are using the fact that $h_U\cong h_T\times_FG$.}.
\end{remark}

\begin{definition}[Open cover of a functor]
Let $F:\Psh S\Set$ be a moduli problem. A collection of open subfunctors $\cpa{F_i}$ is an \textbf{open cover} of $F$ if for any $S$-scheme $T$ and any natural transformation $h_T\to F$, the open subschemes $U_i$ that represent the pullbacks $h_T\times_F F_i$ form an open cover of $T$.
\end{definition}
\begin{remark}
Like above, we can rephrase the definition as follows:\\
A collections of open subfunctors $F_i\subseteq F$ form an open cover of $F$ if for any $S$-scheme $T$ and any family $\xi\in F(T)$, there exists an open cover $\cpa{U_i}$ of $T$ such that $\xi\res{U_i}\in F_i(U_i)$ for all $i$.
\end{remark}

\noindent The definitions above let us state the following criterion for representability:
\begin{proposition}[Representability by open cover]\label{RepresentabilityByOpenSubfunctorCover}
Let $F:\Psh{S}{\Set}$ be a moduli problem which is a Zariski sheaf and let $\cpa{F_i}$ be an open cover of it by representable subfunctors, then $F$ is representable.
\end{proposition}
\begin{proof}
Let $X_i$ be the fine moduli space for $F_i$ and let $\xi_i\in F_i(X_i)$ be their universal families.\\
Since $F_j$ is an open subfunctor of $F$, there exists an open subscheme $U_{ij}\subseteq X_i$ such that the diagram is a cartesian square
% https://q.uiver.app/#q=WzAsNCxbMCwwLCJoX3tVX3tpan19Il0sWzEsMCwiaF97WF9pfSJdLFswLDEsIkZfaiJdLFsxLDEsIkYiXSxbMCwyXSxbMCwxXSxbMiwzXSxbMSwzXSxbMCwzLCIiLDEseyJzdHlsZSI6eyJuYW1lIjoiY29ybmVyLWludmVyc2UifX1dXQ==
\[\begin{tikzcd}
	{h_{U_{ij}}} & {h_{X_i}} \\
	{F_j} & F
	\arrow[from=1-1, to=2-1]
	\arrow[from=1-1, to=1-2]
	\arrow[from=2-1, to=2-2]
	\arrow[from=1-2, to=2-2]
	\arrow["\ulcorner"{anchor=center, pos=0.125}, draw=none, from=1-1, to=2-2]
\end{tikzcd}\]
Evaluating the diagram on $U_{ij}$ yields
% https://q.uiver.app/#q=WzAsNCxbMCwwLCJcXE1vcihVX3tpan0sVV97aWp9KSJdLFsxLDAsIlxcTW9yKFVfe2lqfSxYX2kpIl0sWzAsMSwiRl9qKFVfe2lqfSkiXSxbMSwxLCJGKFVfe2lqfSkiXSxbMCwyXSxbMCwxXSxbMiwzXSxbMSwzXV0=
\[\begin{tikzcd}
	{\Mor(U_{ij},U_{ij})} & {\Mor(U_{ij},X_i)} \\
	{F_j(U_{ij})} & {F(U_{ij})}
	\arrow[from=1-1, to=2-1]
	\arrow[from=1-1, to=1-2]
	\arrow[from=2-1, to=2-2]
	\arrow[from=1-2, to=2-2]
\end{tikzcd}\]
The top left set contains $id_{U_{ij}}$, which corresponds to the universal family in \[F_j(U_{ij})\times_{F(U_{ij})}\Mor(U_{ij},X_i),\]
given by $(\xi\res{U_{ij}},\iota_i)$, where $\iota_i:U_{ij}\to X_i$ is the inclusion and $\xi_j\res{U_{ij}}=F_j(f_i)(\xi_j)$ for % https://q.uiver.app/#q=WzAsNixbMCwwLCJoX3tVX3tpan19KFVfe2lqfSkiXSxbMSwwLCJGX2ooVV97aWp9KVxcdGltZXNfe0YoVV97aWp9KX1oX3tYX2l9KFVfe2lqfSkiXSxbMiwwLCJGX2ooVV97aWp9KSJdLFszLDAsIlxcTW9yKFVfe2lqfSxYX2opIl0sWzAsMSwiaWRfe1Vfe2lqfX0iXSxbMywxLCJmIl0sWzAsMSwiXFxjb25nIiwxLHsic3R5bGUiOnsiYm9keSI6eyJuYW1lIjoibm9uZSJ9LCJoZWFkIjp7Im5hbWUiOiJub25lIn19fV0sWzEsMiwicHJfMSJdLFsyLDMsIlxcY29uZyIsMSx7InN0eWxlIjp7ImJvZHkiOnsibmFtZSI6Im5vbmUifSwiaGVhZCI6eyJuYW1lIjoibm9uZSJ9fX1dLFs0LDUsIiIsMSx7InN0eWxlIjp7InRhaWwiOnsibmFtZSI6Im1hcHMgdG8ifX19XSxbNCwwLCJcXGluIiwzLHsic3R5bGUiOnsiYm9keSI6eyJuYW1lIjoibm9uZSJ9LCJoZWFkIjp7Im5hbWUiOiJub25lIn19fV0sWzUsMywiXFxpbiIsMyx7InN0eWxlIjp7ImJvZHkiOnsibmFtZSI6Im5vbmUifSwiaGVhZCI6eyJuYW1lIjoibm9uZSJ9fX1dXQ==
\[\begin{tikzcd}
	{h_{U_{ij}}(U_{ij})} & {F_j(U_{ij})\times_{F(U_{ij})}h_{X_i}(U_{ij})} & {F_j(U_{ij})} & {\Mor(U_{ij},X_j)} \\
	{id_{U_{ij}}} &&& f_i
	\arrow["\cong"{description}, draw=none, from=1-1, to=1-2]
	\arrow["{pr_1}", from=1-2, to=1-3]
	\arrow["\cong"{description}, draw=none, from=1-3, to=1-4]
	\arrow[maps to, from=2-1, to=2-4]
	\arrow["\in"{marking, allow upside down}, draw=none, from=2-1, to=1-1]
	\arrow["\in"{marking, allow upside down}, draw=none, from=2-4, to=1-4]
\end{tikzcd}\]
The images of these two elements in $F(U_{ij})$ are $\xi_j\res{U_{ij}}$ (since $F_j\subseteq F$) and $\xi_i\res{U_{ij}}=\iota_i^\ast\xi_i$, so by commutativity
\[\xi_j\res{U_{ij}}=\xi_i\res{U_{ij}}.\]
\medskip

\noindent
We now evaluate the cartesian square that defines $U_{ji}$ on $U_{ij}$:
% https://q.uiver.app/#q=WzAsNCxbMCwwLCJcXE1vcihVX3tqaX0sVV97aWp9KSJdLFsxLDAsIlxcTW9yKFVfe2lqfSxYX2opIl0sWzAsMSwiRl9pKFVfe2lqfSkiXSxbMSwxLCJGKFVfe2lqfSkiXSxbMCwyXSxbMCwxXSxbMiwzXSxbMSwzXV0=
\[\begin{tikzcd}
	{\Mor(U_{ij},U_{ji})} & {\Mor(U_{ij},X_j)} \\
	{F_i(U_{ij})} & {F(U_{ij})}
	\arrow[from=1-1, to=2-1]
	\arrow[from=1-1, to=1-2]
	\arrow[from=2-1, to=2-2]
	\arrow[from=1-2, to=2-2]
\end{tikzcd}\]
Given what we have already said, $(\xi_i\res{U_{ij}},f_i)\in F_i(U_{ij})\times_{F(U_{ij})}h_{X_j}(U_{ij})$, so it defines a morphism $\vp_{ij}:U_{ij}\to U_{ji}$. We observe that $\vp_{ij}$ and $\vp_{ji}$ are inverses of each other. Indeed if we consider $\vp_{ji}\circ \vp_{ij}$ as an element of the top left set in the diagram
% https://q.uiver.app/#q=WzAsNCxbMCwwLCJcXE1vcihVX3tpan0sVV97aWp9KSJdLFsxLDAsIlxcTW9yKFVfe2lqfSxYX2kpIl0sWzAsMSwiRl9qKFVfe2lqfSkiXSxbMSwxLCJGKFVfe2lqfSkiXSxbMCwyXSxbMCwxXSxbMiwzXSxbMSwzXV0=
\[\begin{tikzcd}
	{\Mor(U_{ij},U_{ij})} & {\Mor(U_{ij},X_i)} \\
	{F_j(U_{ij})} & {F(U_{ij})}
	\arrow[from=1-1, to=2-1]
	\arrow[from=1-1, to=1-2]
	\arrow[from=2-1, to=2-2]
	\arrow[from=1-2, to=2-2]
\end{tikzcd}\]
we notice that the two projections are given by \[\iota_i\circ \vp_{ji}\circ \vp_{ij}=f_j\circ \vp_{ij}=f_i\]
and
\[F_j(f_i\circ \vp_{ji}\circ \vp_{ij})(\xi_j)=F_j(f_j\circ \vp_{ij})(\xi_j)=F_j(f_i)(\xi_j)=\xi_j\res{U_{ij}},\]
which are the same projections as $id_{U_{ij}}$, so $\vp_{ji}\circ \vp_{ij}=id_{U_{ij}}$. A symmetric argument yields $\vp_{ij}\circ \vp_{ji}=id_{U_{ji}}$. Note that with our notation $\vp_{ii}=id_{U_{ii}}=id_{X_i}$.\\
We also remark that \[F(\vp_{ij})(\xi_j)=F_j(\iota_j\circ \vp_{ij})(\xi_j)=F_j(f_i)(\xi_j)=\xi_j\res{U_{ij}}=\xi_i\res{U_{ij}}.\]\medskip

\noindent We now want to show that the $X_i$ can be glued along the $U_{ij}$ using the isomorphisms $\vp_{ij}$.
\setlength{\leftmargini}{0cm}
\begin{itemize}
\item[$\boxed{\text{Intersection}}$] We want to check that \[\vp_{ij}(U_{ij}\cap U_{ik})= U_{ji}\cap U_{jk}.\]
Given the symmetry of the indicies and having already proven that $\vp_{ji}=\vp_{ij}\ii$, we just need to show inclusion. By the equivalent definition of open subfunctor, we know that the morphism $\vp_{ij}:U_{ij}\cap U_{ik}\to U_{ji}$ factors through $U_{ji}\cap U_{jk}$ if and only if $\vp_{ij}^\ast\xi_j\res{U_{ji}\cap U_{jk}}\in F(U_{ij}\cap U_{ik})$, which is true because \[\vp_{ij}^\ast\xi_j\res{U_{ji}\cap U_{jk}}=\xi_i\res{U_{ij}\cap U_{ik}}.\]
\item[$\boxed{\text{Cocycle cond.}}$] We now want to verify that
\[\vp_{jk}\res{U_{ji}\cap U_{jk}}\circ \vp_{ij}\res{U_{ij}\cap U_{ik}}=\vp_{ik}\res{U_{ik}\cap U_{ij}},\]
but this simply follows from the fact that both maps pullback $\xi_k\res{U_{ki}\cap U_{kj}}$ to $\xi_i\res{U_{ij}\cap U_{ik}}$.
\end{itemize}
\setlength{\leftmargini}{0.5cm}
We can thus define $X$ to be the scheme obtained by gluing the $X_i$ along the $U_{ij}$. Moreover, since $F$ is a Zariski sheaf, the $\xi_i$ must glue to a family $\xi\in F(X)$\footnote{because $\xi_i\res{U_ij}=\xi_j\res{U_{ij}}$.}.\medskip

\noindent We now only need to verify that $(X,\xi)$ is a fine moduli space for $F$:\\
Let $T$ be an $S$-scheme and and let us consider a family $\zeta\in F(T)$. Since $\cpa{F_i}$ is an open cover of $F$, there exists an open cover $\cpa{V_i}$ of $T$ such that $\zeta_{U_i}\in F_i(V_i)\cong \Mor(V_i,X_i)$. Since $F$ is a sheaf $\zeta_i\res{V_i\cap V_j}=\zeta_j\res{V_i\cap V_j}$, so the morphisms $V_i\to X_i$ corresponding to the $\zeta_i$ glue to a morphism $f:T\to X$ such that $f^\ast\xi=\zeta$ (by construction).
\end{proof}