\chapter{Grassmannians as projective varieties}
In this chapter we introduce Grassmannians from the point of view of classical algebraic geometry. We are interested in Grassmannians in the context of classification problems because given their definition, we can expect them to be moduli spaces for families of quotient vector spaces. In the next chapter we will indeed find that they are fine moduli spaces for a functor that formalizes \textit{families of fixed rank vector subspaces of $\K^n$}.

We first define Grassmannians set-theoretically, then we will find a bijection between this set and a Zariski-closed subset of some projective space. This bijection will allow us to endow the Grassmannians with the structure of projective algebraic varieties.

\section{First definitions and conventions}
\begin{notation}
In this chapter we use $V$ and $W$ to denote a fixed $n$-dimensional and a fixed $k$-dimensional $\K$-vector space respectively. 
Unless otherwise stated, we understand $\Bc=\cpa{v_1,\cdots, v_n}$ to be a basis of $V$ and $\Dc=\cpa{w_1,\cdots, w_k}$ to be a basis of $W$. We use $u_i$ and $q_i$ to indicate general elements of $V$ and $W$ respectively.

When a basis $\Fc$ for a vector space $U$ of dimension $\ell$ is fixed, we denote the isomorphism which sends $\Fc$ to the canonical basis of $\K^\ell$ by $[\cdot]_\Fc:U\to \K^\ell$. We denote the canonical basis of $\K^\ell$ by $\Can_\ell=\cpa{e_1,\cdots, e_\ell}$.
\end{notation}

\begin{definition}[Grassmannian]
Let $k\leq n$ be a pair of positive integers. We define the \textbf{$(n,k)$-Grassmannian} to be the following set
\[\Gr(k,V)=\quot{\cpa{\vp\in\Hom_\K(V,W)\mid \vp\text{ surjective}}}\sim\]
where $\vp\sim \psi$ if and only if $\ker \vp=\ker \psi$. To simplify notation we will usually write $\Gr(k,n)$.
\end{definition}

\begin{remark}
We may equivalently define $\Gr(k,n)$ to be the following set:
\[\cpa{\ker \vp\mid \vp\in \Hom_\K(V,W),\ \rnk \vp=k}=\cpa{H\subseteq V\mid \dim H=n-k}.\]
It is common in the literature to give this set the notation $\Gr(n-k,n)$ instead, but fixing a basis for $V$ yields a bijection between $\Gr(k,n)$ and $\Gr(n-k,n)$, namely $H\mapsto H^\perp$.
\end{remark}

\begin{lemma}\label{kerAkerBVSActionOfGLk}
Let $\vp,\psi\in \Hom_\K(V,W)$ be linear maps of full rank. The following conditions are equivalent:
\begin{enumerate}
    \item $\ker\vp=\ker\psi$,
    \item there exists $\theta\in \GL(W)$ such that $\vp=\theta\circ \psi$. 
\end{enumerate}
\end{lemma}
\begin{proof}
The implication $2.\implies 1.$ is a straight forward computation, the other can be derived by completing a basis of the kernels to a basis $\Bc$ of $V$ and defining $\theta$ to be the change of basis between the images of $\Bc$ under $\vp$ and $\psi$.
\end{proof}

\noindent We conclude this introductory section with some notation and conventions.
\begin{definition}[Multiindicies]
We define a \textbf{$(k,n)$-multiindex} as an element of the set $\cpa{1,\cdots, n}^k$. Our notation for a multiindex $I$ will usually be $I=(i_1,\cdots, i_k)$. 
We denote the set of \textbf{ordered $(k,n)$-multiindicies} by
\[\omega(k,n)=\cpa{(i_1,\cdots, i_k)\in \cpa{1,\cdots, n}^k\mid i_1<\cdots<i_k}.\]
If $I\in \omega(k,n)$, we write
\begin{itemize}
\item $\wh I$ for the element of $\omega(n-k,n)$ whose entries are the elements of $\cpa{1,\cdots, n}$ missing from $I$ and 
\item $\sigma_I$ for the permutation that sends the concatenation $\wh I\ast I$ to $\pa{1,\cdots, n}$.
\end{itemize}
\end{definition}

\begin{remark}
If $I=(i_1,\cdots, i_k)$ is a $(k,n)$-multiindex and $u_1,\cdots, u_n$ are $n$ vectors of $V$, we define
\[u_I=u_{i_1}\wedge\cdots\wedge u_{i_k}.\]
Note also that if $\Bc$ is a basis of $V$ then
\[\cpa{v_I\mid I\in \omega(k,n)}\]
yields a basis of $\bw V$, which we call the \textbf{basis induced by $\Bc$} and denote by $\wedge^k\Bc$.
\end{remark}

\begin{notation}
For $\Fc$ and $\Gc$ bases of $U$ and $Z$ respectively, we define
\[\eta_\Fc=[\cdot]_{\wedge^{\dim U}\Fc}:\bw[\dim U]U\to \K,\quad \eta^\Fc_\Gc=\eta_\Gc\ii\circ \eta_\Fc:\bw[\dim Z]Z\to\bw[\dim U]U.\]
\end{notation}



\section{The Pl\"ucker embedding}
In this section we define an injection from the Grassmannian to a projective space.
Our approach differs slightly from the usual one\footnote{briefly illustrated in \cite{matroids}, pages 79 and 80} because we consider equivalence classes of maps rather than equivalence classes of bases of subspaces.

\begin{definition}[Pl\"ucker map]
Let $k\leq n$ be a pair of positive integers. We define the \textbf{Pl\"ucker map} as\footnote{the map $\wedge^k\vp$ is well defined because if we view it as a map $\wedge^k\vp:V^{\times k}\to \bw W$ then it is multilinear and alternating.}
\[\wedge^k:\funcDef{\Hom_\K(V,W)}{\Hom_\K(\bw V,\bw W)}{\vp}{\wedge^k\vp},\]
where $(\wedge^k\vp)(u_1\wedge\cdots\wedge u_k)=\vp(u_1)\wedge\cdots\wedge\vp(u_k).$
\end{definition}

\begin{remark}\label{CodomainOfPluckerMap}
The codomain of the Pl\"ucker map is isomorphic to $\bw V$, indeed
\[\textstyle \Hom_\K\pa{\bw V,\bw W}\cong \pa{\bw V}^{\vee}\cong \bw V.\]
For fixed bases of $V$ and $W$ we can write one such isomorphism concretely as
\[\zeta_{\Bc,\Dc}:\funcDef{\Hom_\K(\bw V,\bw W)}{\bw V}{\psi}{\displaystyle\sum_{I\in\omega(k,n)} \eta_\Dc(\psi(v_I))v_I}.\]
When the bases are clear from context we simply write $\zeta$.
\end{remark}
\begin{notation}
We define $\phi_{\Bc,\Dc}\doteqdot \zeta_{\Bc,\Dc}\circ \wedge^k:\Hom_\K(V,W)\to \bw V$
\end{notation}


\begin{proposition}\label{ImagePluckerMapIsCone}
The image of the Pl\"ucker map is a cone.
\end{proposition}
\begin{proof}
We have $\la\wedge^k\vp=\wedge^k(\al\circ \vp)$ for any $\al\in \GL(W)$ with determinant $\la$.
\end{proof}


\begin{lemma}\label{WhenPluckerMapIsZero}
If $\vp\in\Hom_\K(V,W)$ then $\rnk \vp<k$ if and only if $\wedge^k(\vp)=0$.
\end{lemma}
\begin{proof}
$\wedge^k(\vp)$ is the zero map if an only if the set $\cpa{\vp(u_1),\cdots, \vp(u_k)}$ is linearly dependent for any choice of $u_1,\cdots, u_k$, i.e. $\vp$ is not of full rank.
\end{proof}

\begin{lemma}\label{CharacterizationOfKernels}
Let $\vp:V\to W$ be a full rank linear map, then
\[\ker\vp=\cpa{z\in V\mid \forall u_2,\cdots, u_k\in V,\ \wedge^k(\vp)(z\wedge u_2\wedge\cdots\wedge u_k)=0}.\]
\end{lemma}
\begin{proof}
The inclusion $\subseteq$ is trivial. If $\vp(z)\neq 0$ we can find $k-1$ vectors of the desired form by completing $\vp(z)$ to a basis $\vp(z),q_2,\cdots,q_k$ of $W$ and then taking $u_i$ to be any element of $\vp\ii(q_i)$. This preimage is not empty by surjectivity of $\vp$.
\end{proof}

\begin{proposition}[Injectivity of the Pl\"ucker map up to scalars]\label{PluckerMapInjectiveOnGrassmanniansUpToScalar}
Let $\sim$ be the equivalence relation on $\Hom_\K(V,W)$ which defines $\Gr(k,n)$, then for any two full rank linear maps $\vp,\psi:V\to W$
\[\vp\sim \psi\coimplies \exists \la\in\K^\ast\ s.t.\ \wedge^k(\vp)=\la\wedge^k(\psi).\]
\end{proposition}
\begin{proof}
We prove both implications:
\setlength{\leftmargini}{0cm}
\begin{itemize}
\item[$\boxed{\implies}$] By lemma (\ref{kerAkerBVSActionOfGLk}), if $\vp\sim \psi$ then there exists $\theta\in \GL(W)$ such that $\vp=\theta\circ \psi$, thus
\[\wedge^k(\vp)=\wedge^k(\theta\circ \psi)=(\det\theta) \wedge^k(\psi).\]
\item[$\boxed{\impliedby}$] It is enough to apply lemma (\ref{CharacterizationOfKernels}) as follows:
\begin{align*}
\ker\vp=&\cpa{z\in V\mid \forall u_2,\cdots, u_k\in V,\ \wedge^k(\vp)(z\wedge u_2\wedge \cdots\wedge u_k)=0}=\\
=&\cpa{z\in V\mid \forall u_2,\cdots, u_k\in V,\ \la\wedge^k(\psi)(z\wedge u_2\wedge\cdots\wedge u_k)=0}=\\
=&\cpa{z\in V\mid \forall u_2,\cdots, u_k\in V,\ \wedge^k(\psi)(z\wedge u_2\wedge \cdots\wedge u_k)=0}=\ker \psi.
\end{align*}
\end{itemize}
\setlength{\leftmargini}{0.5cm}
\end{proof}


\begin{definition}[Pl\"ucker embedding]
Let us fix bases $\Bc$ and $\Dc$ of $V$ and $W$. We define the \textbf{Pl\"ucker embedding} as follows
\[\Pl_\Bc:\funcDef{\Gr(k,n)}{\Pj(\bw V)}{[\vp]}{\displaystyle\spa{\sum_{I\in \omega(k,n)} \eta_\Dc(\wedge^k\vp(v_I))v_I}}\]
The entries of the $\binom nk$-tuple $\spa{\cpa{\eta_\Dc(\wedge^k(\vp(v_I)))}_{I\in\omega(k,n)}}$ are called the \textbf{Pl\"ucker coordinates} of $[\vp]$. We will give a cleaner form of the Pl\"ucker coordinates once we express this map in terms of matricies.
\end{definition}


\begin{remark}
If the Pl\"ucker embedding is well defined, it does not depend on the choice of basis for $W$. Indeed changing the basis of $W$ simply multiplies all Pl\"ucker coordinates by the same nonzero scalar\footnote{the determinant of the change of basis}, so the resulting point in $\Pj(\bw V)$ is left unchanged.
\end{remark}

\begin{proposition}\label{PluckerEmbeddingWellDefinedInjective}
The Pl\"ucker embedding is well defined and injective.
\end{proposition}
\begin{proof}
Because of proposition (\ref{PluckerMapInjectiveOnGrassmanniansUpToScalar}) and lemma (\ref{WhenPluckerMapIsZero}), there exists a unique map $p$ such that the diagram commutes
% https://q.uiver.app/#q=WzAsNCxbMCwwLCJcXGNwYXtcXHZwXFxpbiBcXEhvbV9cXEsoVixXKVxcbWlkIFxccm5rIFxcdnA9a30iXSxbMCwxLCJcXEdyKGssbikiXSxbMSwxLCJcXFBqKFxcYmlnd2VkZ2Vea1xcSG9tX1xcSyhWLFcpKSJdLFsxLDAsIlxcYncgXFxIb21fXFxLKFYsVylcXG56Il0sWzAsMSwiXFxwaV9cXHNpbSJdLFsxLDIsInAiLDIseyJzdHlsZSI6eyJib2R5Ijp7Im5hbWUiOiJkYXNoZWQifX19XSxbMCwzLCJcXHdlZGdlXmsiXSxbMywyLCJcXFBqIl1d
\[\begin{tikzcd}
	{\cpa{\vp\in \Hom_\K(V,W)\mid \rnk \vp=k}} & {\bw \Hom_\K(V,W)\nz} \\
	{\Gr(k,n)} & {\Pj(\bw \Hom_\K(V,W))}
	\arrow["{\wedge^k}", from=1-1, to=1-2]
	\arrow["{\pi_\sim}", from=1-1, to=2-1]
	\arrow["\Pj", from=1-2, to=2-2]
	\arrow["p"', dashed, from=2-1, to=2-2]
\end{tikzcd}\]
It follows that $\Pl_\Bc$ is well defined because $\Pl_\Bc=\Pj(\zeta_{\Bc,\Dc})\circ p$.

By proposition (\ref{PluckerMapInjectiveOnGrassmanniansUpToScalar}) we have that $p$ is injective, so $\Pl_\Bc$ must also be injective because $\zeta_{\Bc,\Dc}$ is an isomorphism.
\end{proof}


\begin{remark}
$\Pl_\Bc\circ \pi_\sim=\Pj(\zeta_{\Bc,\Dc}\circ\wedge^k)=\Pj(\phi_{\Bc,\Dc})$.
\end{remark}



\subsection{Matrix form}
\begin{notation}
If $A$ is a $k\times n$ matrix and $I$ is a $(k,n)$-multiindex, we denote the \textbf{$I$-minor of $A$} by $A_I$, i.e.
\[A_I=\mat{a_{1,i_1} &\cdots &a_{1,i_{k}}\\\vdots &\ddots &\vdots\\a_{k,i_1}&\cdots&a_{k,i_k}}.\]
If $B$ is an $\al\times \beta$ matrix, $i\in\cpa{1,\cdots, \al}$ and $j\in \cpa{1,\cdots, \beta}$, we denote the $(\al-1)\times (\beta-1)$ matrix obtained from $B$ by deleting the $i$-th row and the $j$-th column with $B_{\times i,\times j}$.
\end{notation}
If we fix bases $\Bc$ for $V$ and $\Dc$ for $W$ we can identify $V$ with $\K^n$, $W$ with $\K^k$ and $\Hom_\K(V,W)$ with $\Mc(k,n)$. Under these identifications we have 
\[\Gr(k,n)=\quot{\cpa{A\in \Mc(k,n)\mid \rnk A=k}}{\sim},\] 
where $A\sim B\coimplies \exists P\in \GL_k\text{ such that } A=PB$.\bigskip

Because $\wedge^k\vp(u_1\wedge\cdots\wedge u_k)=\det\mat{[\vp(u_1)]_\Dc|\cdots|[\vp(u_k)]_\Dc} w_{(1,\cdots,k)}$ we have
\[\phi:\funcDef{\Mc(k,n)}{\bw \K^n}{A}{\displaystyle \sum_{I\in \omega(k,n)}\det A_I e_I}\]
\[\Pl:\begin{array}{ccccc}
{\Gr(k,n)} & \longrightarrow & {\Pj(\bw \K^n)} & = & \Pj^{\binom nk-1}\\
{[A]_\sim} & \longmapsto & {\displaystyle \spa{\sum_{I\in \omega(k,n)}\det A_I e_I}_{\K^\ast}} & = & \spa{\cpa{\det A_I}_{I\in\omega(k,n)}}_{\K^\ast}
\end{array}\]







\section{The image of the Pl\"ucker embedding is closed}\label{ImagePluckerEmbeddingIsClosed}
Thus far we have identified $\Gr(k,n)$ with a subset of some projective space. We seek to show that this subset is closed in the Zariski topology.

\subsection{Some linear algebra results}
\begin{definition}[Divisibility]
We say that $\omega\in \bw V$ is \textbf{divisible} by $v\in V$ if there exists $\e\in \bw[k-1]V$ such that $\omega=\e\wedge v$.
\end{definition}
\begin{lemma}\label{Divisibility}
$\omega\in \bw V$ is divisible by $v\in V\nz$ if and only if $\omega\wedge v=0$.
\end{lemma}
\begin{proof}
If $\omega=\e\wedge v$ then $\omega\wedge v=\e\wedge v\wedge v=0$. If $\omega\wedge v=0$ then by writing $\omega$ in a basis containing $v$ we can see that the simple multivectors with nonzero coefficients must contain $v$ as a factor, so we can factor out $v$ by multilinearity and get a decomposition of the form $\omega=\e\wedge v$.
\end{proof}

\begin{corollary}[Total decomposability criterion]\label{TotalDecomposabilityCriterion}
Let $\omega\in \bw V$ and define 
\[D_\omega=\cpa{v\in V\mid \omega\wedge v=0}.\]
If $\dim D_\omega\geq k$ then $\omega=\la v_1\wedge \cdots\wedge v_k$ for any set of linearly independent vectors $\cpa{v_1,\cdots, v_k}$ in $D_\omega$ and some scalar $\la$. 
Moreover $\la\neq 0$ if and only if $\dim D_\omega= k$.
\end{corollary}
\begin{proof}
For the first part of the result we may just iterate the above lemma. If $\la=0$ then $D_\omega=V$, so its dimension is not $k$. If the dimension is greater than $k$ then we may subtract two total decompositions differing only by one vector and use linear independence to check that the coefficients must have been zero.
\end{proof}


\begin{proposition}\label{CanonicalIso}
There is a canonical isomorphism between the set of linear maps $\Hom_\K(\bw V,\bw[n]V)$ and $\bw[n-k]V$ given by
\[\Xi:\funcDef{\bw[n-k]V}{\Hom_\K(\bw V, \bw[n]V)}{\omega}{\omega\wedge \cdot}\]
For any basis $\Bc$ of $V$, the inverse of $\Xi$ is given by 
\[\Gamma_\Bc:\funcDef{\Hom_\K(\bw V,\bw[n]V)}{\bw[n-k]V}{\psi}{\displaystyle \sum_{I\in\omega(n-k,n)}\sgn\sigma_I\eta_\Bc(\psi(v_{\wh I}))v_I}\]
\end{proposition}
\begin{proof}
$\Xi$ is clearly base independent and linear. Concluding from here is simply a matter of computing $\Gamma_\Bc(\Xi(\omega))$ by writing $\omega$ in terms of its coordinates in $\wedge^k\Bc$ and verifying that $\Xi(\Gamma_\Bc(\psi))$ and $\psi$ agree on $\wedge^k\Bc$.
\end{proof}
\begin{corollary}\label{UpToScalarCanonicalIso}
Let $\psi\in \Hom_\K(\bw V,\bw W)$. If $\Bc=\cpa{v_1,\cdots, v_n}$ and $\Bc'=\cpa{v_1',\cdots, v_n'}$ are bases for $V$ and $\Dc=\cpa{w_1,\cdots,w_k}$ and $\Dc'=\cpa{w_1',\cdots,w_k'}$ are bases for $W$, there exists $\mu\in\K^\ast$ such that
\[\sum_{I\in\omega(n-k,n)}\sgn\sigma_I\eta_\Dc(\psi(v_{\wh I}))v_I=\mu \sum_{I\in\omega(n-k,n)}\sgn\sigma_I\eta_{\Dc'}(\psi(v'_{\wh I}))v'_I.\]
\end{corollary}
\begin{proof}
Note that
\[\sum_{I\in\omega(n-k,n)}\sgn\sigma_I\eta_\Dc(\psi(v_{\wh I}))v_I=\Xi\ii(\eta^\Dc_\Bc\circ \psi)\]
and similarly the other expression is $\Xi\ii(\eta^{\Dc'}_{\Bc'}\circ \psi)$. 
It is therefore enough to show that $\eta^\Bc_\Dc=\mu\eta^{\Bc'}_{\Dc'}$ for some $\mu\in \K^\ast$, which is true because 
\[\dim_\K\Hom_\K\pa{\bw[n]V,\bw W}=1\] 
and both $\eta^{\Bc}_{\Dc}$ and $\eta^{\Bc'}_{\Dc'}$ are not the zero map.
\end{proof}





\subsection{Rank condition for the image}
\begin{lemma}\label{DecomposabilityOfMultilinearForm}
Fix bases $\Bc$ of $V$ and $\Dc$ of $W$. 
A multilinear alternating form $\psi\in \Hom_\K(\bw V,\bw W)$ is in the image of the Pl\"ucker map $\wedge^k$ if and only if there exists $\la\in\K$ and linearly independent vectors $z_1,\cdots,z_{n-k}$ such that
\[\sum_{I\in \omega(n-k,n)}\sgn\sigma_I\eta_\Dc(\psi(v_{\wh I}))v_I=\la z_{(1,\cdots, n-k)}.\]
\end{lemma}
\begin{proof}
We show both implications
\setlength{\leftmargini}{0cm}
\begin{itemize}
\item[$\boxed{\implies}$] If $\psi=\wedge^k\vp$, the equality follows by choosing $z_1,\cdots, z_{n-k}$ to be a basis of $\ker\vp$. Completing this set to a basis of $V$ and using corollary (\ref{UpToScalarCanonicalIso}) gives the result after a simple calculation.
\item[$\boxed{\impliedby}$] Let $\Zc=\cpa{z_1,\cdots, z_n}$ be a basis of $V$ which extends the given $z_1,\cdots, z_{n-k}$. We can take $\vp$ to be
\[\vp(z_i)=\begin{cases}
0 & \text{if }1\leq i\leq n-k\\
\pa{\mu\la\sgn\sigma_{(1,\cdots,n-k)}} w_1 & \text{if }i=n-k+1\\
w_{i-n+k} & \text{if }i>n-k+1
\end{cases}\]
where $\mu\in\K^\ast$ is such that $\eta^\Bc_\Dc=\mu \eta^{\Zc}_{\Dc}$.
\end{itemize}
\setlength{\leftmargini}{0.5cm}
\end{proof}

\begin{definition}
Let $\Bc$ be a basis of $V$. If $\omega=\sum_{J\in\omega(k,n)}p_J v_J$ we define
\[\Phi_{\Bc}(\omega):\funcDef{V}{\bigwedge^{n-k+1}V}{v}{\displaystyle\sum_{I\in \omega(n-k,n)}\sgn\sigma_I p_{\wh I}v_I\wedge v}.\]
\end{definition}

\begin{remark}
For any basis $\Dc$ of $W$ we have $\Phi_{\Bc}(\omega)(v)=\Xi\ii(\eta^\Dc_\Bc\circ \zeta_{\Bc,\Dc}\ii(\omega))\wedge v$.
\end{remark}

\begin{proposition}\label{RankCriterionForImageOfPlucker}
A $k$-multivector $\omega\in \bw V$ is in the image of $\phi_{\Bc,\Dc}$ if and only if $\Phi_\Bc(\omega)$ has rank at most $k$.
\end{proposition}
\begin{proof}
$\omega\in \imm \phi_{\Bc,\Dc}$ if and only if $\zeta_{\Bc,\Dc}\ii(\omega)\in \imm \wedge^k$ by definition, so what we want to show is that $\psi\in \imm \wedge^k$ if and only if the rank of the map 
\[\Upsilon_{\Bc,\Dc}(\psi):v\mapsto \Xi\ii(\eta^\Dc_\Bc\circ \psi)\wedge v=\sum_{I\in\omega(n-k,n)}\sgn\sigma_I\eta_\Dc(\psi(v_{\wh I}))v_I\wedge v\] 
is at most $k$.

For the $\implies$ arrow, let $\psi=\wedge^k\vp$ and choose a basis $\Zc=\cpa{z_1,\cdots, z_n}$ for $V$ which extends a basis of $\ker\vp$. Because of how we proved lemma (\ref{DecomposabilityOfMultilinearForm}), we see that if $v\in\ker\vp$ then $\Upsilon_{\Bc,\Dc}(\wedge^k\vp)(v)=\la z_{(1,\cdots, n-k)}\wedge v$, which is zero by linear dependence. Thus the nullity of $\Upsilon_{\Bc,\Dc}(\wedge^k\vp)$ is at least $\dim\ker\vp=n-k$.\medskip

Given $z_1,\cdots, z_{n-k}$ linearly independent vectors in $\ker\Upsilon_{\Bc,\Dc}(\psi)$, by the total decomposability criterion (\ref{TotalDecomposabilityCriterion}) there exists $\la\in\K$ such that
\[\sum_{I\in \omega(n-k,n)}\sgn\sigma_I\eta_\Dc(\psi(v_{\wh I}))v_I=\la z_1\wedge\cdots \wedge z_{n-k}.\]
This concludes by lemma (\ref{DecomposabilityOfMultilinearForm}).
\end{proof}


\begin{theorem}\label{ImageOfPhiClosed}
The image of $\phi_{\Bc,\Dc}$ is a Zariski closed subset of $\bw V$.
\end{theorem}
\begin{proof}
We seek to translate the rank condition (\ref{RankCriterionForImageOfPlucker}) into equations on the coordinates of $\bw V$. Let $\K[z_I\mid I\in\omega(k,n)]$ be the coordinate ring of $\bw \K^n$. If $B^I\in \Mc\pa{\binom{n}{n-k+1},n,\K}$ is the matrix which represents $\Phi_{\Bc}(v_I)$ in the bases induced by $\Bc$ and $\Dc$ then we define
\[M_{\Bc,\Dc}=\sum_{I,\omega(k,n)}B^I z_I=\pa{\sum_{I\in\omega(k,n)}(B^I)_{i,j}z_I}_{i,j}.\]
This matrix represents $\Phi_{\Bc}$ in the following way: if $\omega=\sum_{I\in\omega(k,n)}p_I v_I$,
\[\Phi_{\Bc}(\omega)(v)=\sum_{I\in \omega(k,n)}p_I \Phi_{\Bc}(v_I)(v)=\sum_{I\in \omega(k,n)}p_I B^I v=\rbar{M_{\Bc}}_{z_I=p_I}v.\]
It follows that the coordinates of the $k$-multivectors in the image of $\phi_{\Bc,Dc}$ are exactly those that satisfy the determinantal criterion for the rank being at most $k$, i.e.
\begin{align*}
\imm \phi_{\Bc,\Dc}=&\cpa{\sum_{I\in \omega(k,n)}p_I v_I\mid \rnk \rbar{M_{\Bc,\Dc}}_{z_I=p_I}<k+1}=\\
=&V(\cpa{\det m\mid \text{$m$ is a $(k+1)\times (k+1)$ minor of $M$}}).
\end{align*}
\end{proof}
\begin{corollary}\label{ImageOfPluckerEmbeddingIsClosed}
$\Pl_\Bc$ endows $\Gr(k,n)$ with the structure of a projective variety.
\end{corollary}
\begin{proof}
Since $\imm\phi_{\Bc,\Dc}$ is a cone (\ref{ImagePluckerMapIsCone}) and Zariski closed we see that 
\[\Pj(\imm \phi_{\Bc,\Dc})=\imm \Pl_\Bc\]
is Zariski closed.
We conclude by recalling that $\imm\Pl_\Bc$ is in bijection with $\Gr(k,n)$ by injectivity of $\Pl_\Bc$ (\ref{PluckerEmbeddingWellDefinedInjective}).
\end{proof}

\begin{remark}
The determinants we used to show that the image of the Pl\"ucker embedding is closed do not generate the ideal of that variety. The most well known set of generators for that ideal are the \textbf{Pl\"ucker relations} (Theorem 2.4.3 in \cite{matroids}, page 80).
\end{remark}




