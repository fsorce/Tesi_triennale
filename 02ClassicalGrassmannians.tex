\chapter{Grassmannians as projective varieties}
In this chapter we introduce Grassmannians from the point of view of classical algebraic geometry. We are interested in Grassmannians in the context of classification problems because their definition leads us to suspect that they are a moduli space for certain families of vector spaces. In the next chapter we will indeed find that they are fine moduli spaces for a functor that formalizes \textit{families of $k$-vector subspaces of $\K^n$}.\medskip

We first define Grassmannians set-theoretically, then we will find a bijection between this set and Zariski-closed subset of some projective space. This bijection will allow us to endow the Grassmannian with the structure of a projective algebraic variety.

\section{First definitions and conventions}
\begin{definition}[Grassmannian]
Let $k\leq n$ be a pair of positive integers. We define the \textbf{$(n,k)$-Grassmannian}, denoted\footnote{the field will be omitted when clear from context} $\Gr(k,n,\K)$, as the set of $(n-k)$-dimensional\footnote{The most natural choice might be to take the $k$-dimentional subspaces, but for later convenience we adopt this convention. It is also worth noting that if we fix a basis, the map $H\mapsto H^\perp$ gives a  bijection between $\Gr(k,n)$ and $\Gr(n-k,n)$.} $\K$-vector subspaces of $\K^n$.
\end{definition}


\begin{remark}
We may equivalently define $\Gr(k,n)$ to be the following set:
\[\cpa{\ker \vp\mid \vp\in \Hom_\K(\K^n,\K^k),\ \rnk \vp=k}.\]
\end{remark}
%\begin{proof}
%If $H\in \Gr(k,n)$, let $z_1,\cdots, z_n$ be a basis of $\K^n$ such that $z_1,\cdots, z_{n-k}$ is a basis of $H$ and let $e_1,\cdots, e_k$ be any basis of $\K^k$. We can view $H$ as the kernel of the (rank $k$) linear map given by
%\[\vp(z_i)=\begin{cases}
%0 &\text{if }i\leq n-k\\
%e_{i-n+k} &\text{otherwise}
%\end{cases}\]
%On the other hand, if $\vp$ is a rank $k$ linear map then, by the rank-nullity theorem, its kernel is an $n-k$ dimensional subspace of $\K^n$.
%\end{proof}

\begin{lemma}\label{kerAkerBVSActionOfGLk}
Let $\vp,\psi\in \Hom_\K(\K^n,\K^k)$ be linear maps of full rank. The following conditions are equivalent:
\begin{enumerate}
    \item $\ker\vp=\ker\psi$,
    \item there exists $\theta\in \GL(\K^k)$ such that $\vp=\theta\circ \psi$. 
\end{enumerate}
\end{lemma}
\begin{proof}
The implication $2.\implies 1.$ is a straight forward computation, the other can be derived by completing a basis of $H$ to a basis $\Bc$ of $\K^n$ and defining $\theta$ to be the change of basis between the images of $\Bc$ under $\vp$ and $\psi$.
%Let us prove both implications:
%\setlength{\leftmargini}{0cm}
%\begin{itemize}
%\item[$\boxed{2.\implies 1.}$] $\ker \vp=\ker(\theta\circ \psi)=\psi\ii(\ker \theta)=\psi\ii(\cpa{0})=\ker \psi$. 
%\item[$\boxed{1.\implies 2.}$] Let $z_1,\cdots, z_n$ be a basis of $\K^n$ such that $\ker\vp=\ker\psi=\Span(z_1,\cdots, z_{n-k})$. By construction $\vp(z_{n-k+1}),\cdots, \vp(z_n)$ and $\psi(z_{n-k+1}),\cdots, \psi(z_n)$ are bases of $\K^k$. 
%Let $\theta$ be the change of basis on $\K^k$ determined by $\theta(\psi(z_i))=\vp(z_i)$ for all $n-k<i\leq n$. By construction $\theta$ is nonsingular and $\vp$ agrees with $\theta\circ \psi$ on a basis of $\K^n$.
%\end{itemize}
%\setlength{\leftmargini}{0.5cm}
\end{proof}

\begin{corollary}\label{LinearQuotientDefinition}
We may redefine Grassmannians in terms of linear maps as follows:
\[\Gr(k,n)=\quot{\cpa{\vp\in \Hom_\K(\K^n,\K^k)\mid \vp\ \text{surjective.}}}\sim\]
where $\vp\sim \psi$ if and only if $\exists \theta\in \GL(\K^k)$ such that $\vp=\theta\circ \psi$.
\end{corollary}
\bigskip

\noindent We conclude this introductory section with some notation and conventions.
\begin{definition}[Multiindicies]
We define a \textbf{$(k,n)$-multiindex} as an element of $\cpa{1,\cdots, n}^k$. Our notation for a multiindex $I$ will usually be $I=(i_1,\cdots, i_k)$.\\
We denote the set of \textbf{ordered $(k,n)$-multiindicies} by
\[\omega(k,n)=\cpa{(i_1,\cdots, i_k)\in \cpa{1,\cdots, n}^k\mid i_1<\cdots<i_k}.\]
If $I\in \omega(k,n)$, we write\footnote{$\cup$ and $\ast$ denote the union of the underlying sets and concatenation respectively.}
\begin{itemize}
\item $\wh I$ for the element of $\omega(n-k,n)$ such that $I\cup \wh I=\cpa{1,\cdots, n}$ and 
\item $\sigma_I$ for the permutation that sends $\wh I\ast I$ to $\pa{1,\cdots, n}$.
\end{itemize}
If $A$ is a $k\times n$ matrix and $I$ is a $(k,n)$-multiindex, we denote the \textbf{$I$-minor of $A$} by $A_I$, i.e.
\[A_I=\mat{a_{1,i_1} &\cdots &a_{1,i_{k}}\\\vdots &\ddots &\vdots\\a_{k,i_1}&\cdots&a_{k,i_k}}.\]
If $B$ is an $\al\times \beta$ matrix, $i\in\cpa{1,\cdots, \al}$ and $j\in \cpa{1,\cdots, \beta}$ we denote the $(\al-1)\times (\beta-1)$ matrix obtained from $B$ by deleting the $i$-th row and the $j$-th column with $B_{\times i,\times j}$.
\end{definition}

\begin{remark}
If $I=(i_1,\cdots, i_k)$ is a $(k,n)$-multiindex and $\Bc=\cpa{e_1,\cdots,e_n}$ is a basis of $\K^n$ we define
\[e_I=e_{i_1}\wedge\cdots\wedge e_{i_k}.\]
Note that
\[\cpa{e_{i_1}\wedge\cdots\wedge e_{i_k}\mid 1\leq i_1<\cdots<i_k\leq n}=\cpa{e_I\mid I\in \omega(k,n)}\]
forms a basis for $\bigwedge^k\K^n$, which we call the \textbf{basis induced by $\Bc$} and denote with $\wedge^k\Bc$ or simply $\Bc$ with a slight abuse of notation.
\end{remark}

\begin{notation}
Whenever a basis $\Bc$ of $\K^\ell$ is fixed, we will identify $\bigwedge^\ell\K^\ell$ with $\K$ by sending only element of $\wedge^\ell\Bc$ to $1\in\K$. This isomorphism is denoted $\eta_\Bc:\bigwedge^\ell\K^\ell\to \K$. If $\Dc$ is a basis of $\K^m$ then we define $\eta^\Bc_\Dc=\eta_\Dc\ii\circ \eta_\Bc$.
\end{notation}

\begin{remark}[Matrix form for the Grassmannian]
If we fix bases $e_1,\cdots, e_n$ of $\K^n$ and $e_1,\cdots, e_k$ of $\K^k$, then we can identify $\Hom_\K(\K^n,\K^k)$ with the set of $k\times n$ matrices with coefficients in $\K$. As a consequence of this we find yet another form for $\Gr(k,n)$:
\[\Gr(k,n)=\quot{\cpa{A\in \Mc(k,n)\mid \rnk A=k}}{\sim},\] 
where $A\sim B\coimplies \exists P\in \GL(k)\ s.t.\ A=PB$.
\end{remark}

\section{The Pl\"ucker embedding}
In this section we define an injection from the Grassmannian to a projective space.
The idea behind this map is to take appropriate wedge products in such a way as to transform the several vectors defining a vector subspace into a single vector and then to projectivize. 
Our approach differs slightly from the usual one\footnote{briefly illustrated in \cite{matroids}, pages 79 and 80} because we consider equivalence classes of maps rather than equivalence classes of bases.

\begin{definition}[Pl\"ucker map]
Let $k\leq n$ be a pair of positive integers. We define the \textbf{Pl\"ucker map} as\footnote{the map $\wedge^k\vp$ is well defined because if we view it as a map $\wedge^k\vp:(\K^n)^k\to \bigwedge^k\K^k$ then it is multilinear and alternating.}
\[\wedge^k:\funcDef{\Hom_\K(\K^n,\K^k)}{\Hom_\K(\bigwedge^k\K^n,\bigwedge^k\K^k)}{\vp}{\wedge^k\vp},\]
where $(\wedge^k\vp)(v_1\wedge\cdots\wedge v_k)=\vp(v_1)\wedge\cdots\wedge\vp(v_k).$
\end{definition}

\begin{remark}
If $\Bc=\cpa{v_1,\cdots, v_k}$ is a basis of $\K^k$, $\Can=\cpa{e_1,\cdots,e_k}$ is the canonical basis and $[\cdot]_\Bc:\K^k\to\K^k$ is the isomorphism which sends $v_i$ to $e_i$ then
\[\wedge^k(\vp)(v_1\wedge\cdots\wedge v_k)=\det\mat{[\vp(v_1)]_\Bc|\cdots|[\vp(v_k)]_\Bc} e_1\wedge\cdots\wedge e_k.\]
\end{remark}

\begin{remark}\label{CodomainOfPluckerMap}
The codomain of the Pl\"ucker map is isomorphic to $\bigwedge^k\K^n$, indeed
\[\Hom_\K\pa{\bigwedge^k\K^n,\bigwedge^k\K^k}\cong \pa{\bigwedge^k\K^n}^{\vee}\cong \bigwedge^k\K^n.\]
If $\Bc=\cpa{e_1,\cdots, e_n}$ is a basis of $\K^n$ and $\Dc=\cpa{e_1,\cdots, e_k}$ is a basis of $\K^k$ then we can write one such isomorphism concretely as
\[\zeta_{\Bc,\Dc}:\funcDef{\Hom_\K(\bigwedge^k\K^n,\bigwedge^k\K^k)}{\bigwedge^k\K^n}{\psi}{\displaystyle\sum_{I\in\omega(k,n)} \eta_\Dc(\psi(e_I))e_I}.\]
When the bases are fixed we simply write $\zeta$.
\end{remark}

\begin{notation}
If bases are fixed we define $\phi\doteqdot \zeta\circ \wedge^k$.
\end{notation}

\begin{remark}[Matrix form of the Pl\"ucker map]
If we fix bases $e_1,\cdots, e_n$ of $\K^n$ and $e_1,\cdots, e_k$ of $\K^k$ then, up to identifying $\Hom_\K(\K^n,\K^k)$ with $\Mc(k,n)$, we have
\[\phi:\funcDef{\Mc(k,n)}{\bigwedge^k\K^n}{A}{\sum_{I\in \omega(k,n)}\det A_I e_I}\]
\end{remark}


\begin{proposition}\label{ImagePluckerMapIsCone}
The image of the Pl\"ucker map is a cone.
\end{proposition}
\begin{proof}
We have $\la\wedge^k\vp=\wedge^k(\al\circ \vp)$ for any $\al\in \GL(\K^k)$ with determinant $\la$. 
%For any $\la\in \K^\ast$ and any map $\vp\in \Hom_\K(\K^n,\K^k)$ note that
%\[\la\wedge^k(\vp)=\wedge^k(\al\circ \vp)\]
%for any automorphism $\al$ of $\K^k$ with determinant $\la$. We can construct one such $\al$ by fixing a basis of $\K^k$ and defining $\al$ to be the map corresponding to the matrix
%\[\mat{
%\la &   &        &\\
%    & 1 &        &\\
%	&   & \ddots &\\
%	&   &        &  1
%}\]
\end{proof}


\begin{lemma}\label{WhenPluckerMapIsZero}
If $\vp\in\Hom_\K(\K^n,\K^k)$ then $\rnk \vp<k$ if and only if $\wedge^k(\vp)=0$.
\end{lemma}
\begin{proof}
$\wedge^k(\vp)$ is the zero map if an only if the set $\cpa{\vp(v_1),\cdots, \vp(v_k)}$ is linearly dependent for any choice of $v_1,\cdots, v_k$, i.e. $\vp$ is not of full rank.
\end{proof}

\begin{lemma}\label{CharacterizationOfKernels}
Let $\vp:\K^n\to \K^k$ be a full rank linear map, then
\[\ker\vp=\cpa{z\in\K^n\mid \forall w_2,\cdots, w_k\in \K^n,\ \wedge^k(\vp)(z\wedge w_2\wedge\cdots\wedge w_k)=0}.\]
\end{lemma}
\begin{proof}
The inclusion $\subseteq$ is trivial. If $\vp(z)\neq 0$ we can find $k-1$ vectors of the desired form by completing $\vp(z)$ to a basis $\vp(z),v_2,\cdots,v_k$ of $\K^k$ and then taking $w_i$ to be any element of $\vp\ii(v_i)$. The set is not empty by surjectivity of $\vp$.
%If $\vp(z)=0$ then for any $w_2,\cdots, w_k\in \K^k$ we see that 
%\[\wedge^k(\vp)(z\wedge w_2\wedge\cdots\wedge w_k)=0\wedge \vp(w_2)\wedge\cdots\wedge \vp(w_k)=0.\]
%Suppose now that $\vp(z)\neq 0$ and let $v_2,\cdots, v_k$ be such that $\cpa{\vp(z), v_2,\cdots,v_k}$ forms a basis for $\K^k$. Since $\vp$ is surjective, there exist $w_2,\cdots, w_k$ such that $\vp(w_i)=v_i$ for all $2\leq i\leq k$.
%By construction 
%\[\wedge^k(\vp)(z\wedge w_2\wedge\cdots\wedge w_k)=\vp(z)\wedge v_2\wedge\cdots\wedge v_k\neq0.\]
\end{proof}

\begin{proposition}[Injectivity of the Pl\"ucker map up to scalars]\label{PluckerMapInjectiveOnGrassmanniansUpToScalar}
Let $\sim$ be the equivalence relation defined in corollary (\ref{LinearQuotientDefinition}), then for any two full rank linear maps $\vp,\psi:\K^n\to \K^k$
\[\vp\sim \psi\coimplies \exists \la\in\K^\ast\ s.t.\ \wedge^k(\vp)=\la\wedge^k(\psi).\]
\end{proposition}
\begin{proof}
We prove both implications:
\setlength{\leftmargini}{0cm}
\begin{itemize}
\item[$\boxed{\implies}$] If $\vp=\theta\circ \psi$ for $\theta\in \GL(\K^k)$ then
\[\wedge^k(\vp)=\wedge^k(\theta\circ \psi)=(\det\theta) \wedge^k(\psi).\]
\item[$\boxed{\impliedby}$] From lemma (\ref{kerAkerBVSActionOfGLk}) we see that it is enough to prove that $\ker \vp=\ker \psi$. We conclude by applying lemma (\ref{CharacterizationOfKernels}) as follows:
\begin{align*}
\ker\vp=&\cpa{z\in\K^n\mid \forall w_2,\cdots, w_k\in \K^n,\ \wedge^k(\vp)(z\wedge w_2\wedge \cdots\wedge w_k)=0}=\\
=&\cpa{z\in\K^n\mid \forall w_2,\cdots, w_k\in \K^n,\ \la\wedge^k(\psi)(z\wedge w_2\wedge\cdots\wedge w_k)=0}=\\
=&\cpa{z\in\K^n\mid \forall w_2,\cdots, w_k\in \K^n,\ \wedge^k(\psi)(z\wedge w_2\wedge \cdots\wedge w_k)=0}=\ker \psi.
\end{align*}
\end{itemize}
\setlength{\leftmargini}{0.5cm}
\end{proof}

\begin{remark}
Because of proposition (\ref{PluckerMapInjectiveOnGrassmanniansUpToScalar}) and lemma (\ref{WhenPluckerMapIsZero}), there exists a unique $h$ such that the diagram commutes
% https://q.uiver.app/#q=WzAsMyxbMCwwLCJcXGNwYXtcXHZwXFxpbiBcXEhvbV9cXEsoXFxLXm4sXFxLXmspXFxtaWQgXFxybmsgXFx2cD1rfSJdLFswLDEsIlxcR3IoayxuKSJdLFsxLDAsIlxcUGooXFxiaWd3ZWRnZV5rXFxIb21fXFxLKFxcS15uLFxcS15rKSkiXSxbMCwxLCJcXHBpIl0sWzAsMiwiW1xccGhpXSJdLFsxLDIsImgiLDIseyJzdHlsZSI6eyJib2R5Ijp7Im5hbWUiOiJkYXNoZWQifX19XV0=
\[\begin{tikzcd}
	{\cpa{\vp\in \Hom_\K(\K^n,\K^k)\mid \rnk \vp=k}} & {\Pj(\Hom_\K(\bigwedge^k\K^n,\bigwedge^k\K^k))} \\
	{\Gr(k,n)}
	\arrow["\pi_\sim", from=1-1, to=2-1]
	\arrow["{[\wedge^k]}", from=1-1, to=1-2]
	\arrow["h"', dashed, from=2-1, to=1-2]
\end{tikzcd}\]
Moreover, such an $h$ must be injective by proposition (\ref{PluckerMapInjectiveOnGrassmanniansUpToScalar}).
\end{remark}


\begin{definition}[Pl\"ucker embedding]
Let us fix a basis $e_1,\cdots, e_n$ of $\K^n$ and a basis $e_1,\cdots, e_k$ of $\K^k$. We define the \textbf{Pl\"ucker embedding} as follows\footnote{we can omit the basis in which we calulated the determinats because we will soon see that the resulting point in $\Pj(\bigwedge^k\K^n)$ does not depend on this choice.}
\[\Pl:\funcDef{\Gr(k,n)}{\Pj(\bigwedge^k\K^n)}{[\vp]}{\displaystyle\spa{\sum_{1\leq i_1<\cdots<i_k\leq n}\det(\vp(e_{i_1})\mid\cdots\mid\vp(e_{i_k}))e_{i_1}\wedge\cdots\wedge e_{i_k}}}\]
The entries of the homogeneous $\binom nk$-tuple associated to $[\vp]\in \Gr(k,n)$ are called the \textbf{Pl\"ucker coordinates} of $[\vp]$. 
\end{definition}

\begin{remark}[Well defined and injective]
If we fix bases for $\K^n$ and $\K^k$ and $\zeta$ is the isomorphism $\Hom_\K(\bigwedge^k\K^n,\bigwedge^k\K^k)\to \bigwedge^k\K^n$ discussed during remark (\ref{CodomainOfPluckerMap}), we see that the following diagram commutes
% https://q.uiver.app/#q=WzAsMyxbMCwxLCJcXEdyKGssbikiXSxbMCwwLCJcXFBqKFxcYmlnd2VkZ2Vea1xcSG9tX1xcSyhcXEtebixcXEteaykpIl0sWzEsMCwiXFxQaihcXGJpZ3dlZGdlXmtcXEtebikiXSxbMCwxLCJoIiwyXSxbMSwyLCJcXFBqKFxcemV0YSkiXSxbMCwyLCJcXG1hdGhybXtQbH0iLDJdXQ==
\[\begin{tikzcd}
	{\Pj(\Hom_\K(\bigwedge^k\K^n,\bigwedge^k\K^k))} & {\Pj(\bigwedge^k\K^n)} \\
	{\Gr(k,n)}
	\arrow["h"', from=2-1, to=1-1]
	\arrow["{\Pj(\zeta)}", from=1-1, to=1-2]
	\arrow["{\Pl}"', from=2-1, to=1-2]
\end{tikzcd}\]
This proves that the Pl\"ucker embedding is well defined and injective.
\end{remark}

\begin{remark}
$\Pl\circ \pi_\sim=\Pj(\zeta\circ\wedge^k)=\Pj(\phi)$.
\end{remark}


\begin{remark}
The Pl\"ucker embedding depends on the choice of basis for $\K^n$ but not on the one for $\K^k$. 

Changing the basis of $\K^k$ simply multiplies all Pl\"ucker coordinates by the same nonzero scalar (the determinant of the change of basis), which does not change the point they describe in $\Pj(\bigwedge^k\K^k)$.

The dependence on the basis of $\K^n$ is inevitable because $\GL(\K^n)$ acts transitively on $\Gr(k,n)$ viewed as the set of $(n-k)$-dimentional subspaces of $\K^n$.
\end{remark}

\begin{remark}[Matrix form of the Pl\"ucker embedding]
If we fix a basis $e_1,\cdots, e_n$ of $\K^n$ and identify $\Hom_\K(\K^n,\K^k)$ with $\Mc(k,n)$  then
\[\Pl:\funcDef{\Gr(k,n)}{\Pj(\bigwedge^k\K^n)}{[A]_\sim}{\spa{\sum_{I\in \omega(k,n)}\det A_I e_I}_{\K^\ast}}\]
\end{remark}



\section{The image of the Pl\"ucker embedding is closed}\label{ImagePluckerEmbeddingIsClosed}
Thus far we have identified $\Gr(k,n)$ with a subset of some projective space. We seek to show that this subset is closed in the Zariski topology.

\subsection{Some linear algebra results}
\begin{definition}[Divisibility]
We say that $\omega\in \bigwedge^k\K^n$ is \textbf{divisible} by $v\in\K^n$ if there exists $\e\in \bigwedge^{k-1}\K^n$ such that $\omega=\e\wedge v$.
\end{definition}
\begin{lemma}\label{Divisibility}
Let $\omega\in \bigwedge^k\K^n$. For any given nonzero vector $v$, $\omega$ is divisible by $v$ if and only if $\omega\wedge v=0$.
\end{lemma}
\begin{proof}
If $\omega=\e\wedge v$ then $\omega\wedge v=\e\wedge v\wedge v=0$. If $\omega\wedge v=0$ then by writing $\omega$ in a base containing $v$ we can see that the simple multivectors with nonzero coefficients must contain $v$ as a factor, so can factor out $v$ by multilinearity and get a decomposition of the form $\omega=\e\wedge v$.
%If $\omega=\e\wedge v$ then $\omega\wedge v=\e\wedge v\wedge v=0$.\\
%Suppose now that $\omega\wedge v=0$. Let $v_1,\cdots, v_n$ be a basis of $\K^n$ such that $v_1=v$. 
%If we write
%\[\omega=\sum_{I\in\omega(k,n)}p_I v_I\]
%then we see that for any given multiindex $I$, either $p_I=0$ or $v_I\wedge v=0$. 
%Since $v_1,\cdots, v_n$ is a basis, $v_I\wedge v_1=0$ if and only if $1\in I$, i.e. $v_I=v\wedge v_{\pa{i_2,\cdots, i_k}}$, therefore
%\[\omega=v\wedge \under{\doteqdot (-1)^{k-1}\e}{\pa{\sum_{2\leq i_2<\cdots i_k\leq n} p_{\pa{1,i_2,\cdots, i_k}} e_{\pa{i_2,\cdots, i_k}}}}=\e\wedge v.\]
\end{proof}

\begin{corollary}[Total decomposibility criterion]\label{TotalDecomposibilityCriterion}
Let $\omega\in \bigwedge^k\K^n$ and define 
\[D_\omega=\cpa{v\in\K^n\mid \omega\wedge v=0}.\]
If $\dim D_\omega\geq k$ then $\omega=\la v_1\wedge \cdots\wedge v_k$ for any set of linearly independent vectors $\cpa{v_1,\cdots, v_k}$ in $D_\omega$ and some scalar $\la$. 
Moreover $\la\neq 0$ if and only if $\dim D_\omega= k$.
\end{corollary}
\begin{proof}
For the first part of the result we may just iterate the above lemma. If $\la=0$ then $D_\omega=\K^n$, so its dimention is not $k$. If the dimention is greater than $k$ then we may subtract two total decompositions differing only by one vector and use linear independence to check that the coefficients must have been zero.
%The set $D_\omega$ is clearly a subspace of $\K^n$. Let $\cpa{v_1,\cdots, v_k}$ be linearly independent vectors of this space. By iterating the above lemma we see that
%\[\omega=\la\wedge v_1\wedge\cdots\wedge v_k\]
%for some $\la\in \bigwedge^0\K^n=\K$.\\
%If $\la=0$ then clearly $D_\omega=\K^n$. If $v_{k+1}$ is such that $\omega\wedge v_{k+1}=0$ and $\cpa{v_1,\cdots, v_{k+1}}$ is linearly independent then, proceding as above,
%\[\la v_1\wedge\cdots\wedge v_{k}=\omega=\mu v_1\wedge\cdots\wedge v_{k-1}\wedge v_{k+1}.\]
%By multilinearity
%\[0=v_1\wedge\cdots\wedge v_{k-1}\wedge(\la v_k-\mu v_{k+1}),\]
%i.e. $\la v_k-\mu v_{k+1}\in \Span(v_1,\cdots, v_{k-1})$. By linear independence $\la v_k-\mu v_{k+1}=0$ and thus, again by linear independence, $\la=\mu=0$.
\end{proof}


\begin{proposition}\label{CanonicalIso}
There is a canonical isomorphism between $\Hom_\K(\bigwedge^k\K^n,\bigwedge^n\K^n)$ and $\bigwedge^{n-k}\K^n$ given by
\[\Xi:\funcDef{\bigwedge^{n-k}\K^n}{\Hom_\K(\bigwedge^k\K^n, \bigwedge^n\K^n)}{\omega}{\omega\wedge \cdot}.\]
For any basis $\Bc=\cpa{e_1,\cdots, e_n}$ of $\K^n$ the map 
\[\Gamma_\Bc:\funcDef{\Hom_\K(\bigwedge^k\K^n,\bigwedge^n\K^n)}{\bigwedge^{n-k}\K^n}{\psi}{\sum_{I\in\omega(n-k,n)}\sgn\sigma_I\eta_\Bc(\psi(e_{\wh I}))e_I}\]
is the inverse of $\Xi$.
\end{proposition}
\begin{proof}
The map is clearly base independent and linear. Let $\wedge^k\Bc$ be the basis induced on $\bigwedge^k\K^n$ by $\Bc$. Concluding from here is simply a matter of computing $\Gamma_\Bc(\Xi(\omega))$ by writing $\omega$ in terms of its coordinates in $\wedge^k\Bc$ and verifying that $\Xi(\Gamma_\Bc(\psi))$ and $\psi$ agree on $\wedge^k\Bc$. 
%The map is clearly base independent and linear.\\
%For all $e_J$
%\begin{align*}
%\Xi(\Gamma_\Bc(\psi))(e_J)=&\sum_{I\in\omega(n-k,n)}\sgn\sigma_I\eta_\Bc(\psi(e_{\wh I}))e_I\wedge e_J=\\
%=&\sgn\sigma_{\wh J}\eta_\Bc(\psi(e_{J}))e_{\wh J}\wedge e_J=\\
%=&\eta_\Bc(\psi(e_{J}))e_{(1,\cdots,n)}=\\
%=&\psi(e_{J}),
%\end{align*}
%so $\Xi(\Gamma_\Bc(\psi))$ agrees with $\psi$ on a basis.\medskip

%\noindent
%If $\omega=\sum_{J\in\omega(n-k,n)} p_J e_J$ then
%\begin{align*}
%\sgn\sigma_I\eta_\Bc(\omega\wedge e_{\wh I})=&\sum_{J\in\omega(n-k,n)}p_J\sgn\sigma_I\eta_\Bc(e_J\wedge e_{\wh I})=\\
%=&p_I\eta_\Bc(\sgn \sigma_I e_I\wedge e_{\wh I})=\\
%=&p_I\eta_\Bc(e_{(1,\cdots, n)})=\\
%=&p_I,
%\end{align*}
%thus \[\Gamma_\Bc(\Xi(\omega))=\sum_{I\in\omega(n-k,n)}\sgn\sigma_I\eta_\Bc(\omega\wedge e_{\wh I})e_I=\sum_{I\in\omega(n-k,n)}p_Ie_I=\omega.\]
\end{proof}
\begin{corollary}\label{UpToScalarCanonicalIso}
Let $\psi\in \Hom_\K(\bigwedge^k\K^n,\bigwedge^k\K^k)$. If $\Bc=\cpa{e_1,\cdots, e_n}$ and $\Bc'=\cpa{e_1',\cdots, e_n'}$ are two bases for $\K^n$ and $\Dc=\cpa{e_1,\cdots,e_k}$ and $\Dc'=\cpa{e_1',\cdots,e_k'}$ are bases for $\K^k$, there exists $\mu\in\K\nz$ such that
\[\sum_{I\in\omega(n-k,n)}\sgn\sigma_I\eta_\Dc(\psi(e_{\wh I}))e_I=\mu \sum_{I\in\omega(n-k,n)}\sgn\sigma_I\eta_{\Dc'}(\psi(e'_{\wh I}))e'_I.\]
\end{corollary}
\begin{proof}
Note that
\[\sum_{I\in\omega(n-k,n)}\sgn\sigma_I\eta_\Dc(\psi(e_{\wh I}))e_I=\Xi\ii(\eta^\Dc_\Bc\circ \psi)\]
and similarly the other expression is $\Xi\ii(\eta^{\Dc'}_{\Bc'}\circ \psi)$. 
It is therefore enough to show that $\eta^\Bc_\Dc=\mu\eta^{\Bc'}_{\Dc'}$ for some $\mu\in \K\nz$, which is true because $\dim_\K\Hom_\K(\bigwedge^n\K^n,\bigwedge^k\K^k)=1$ and both $\eta^{\Bc}_{\Dc}$ and $\eta^{\Bc'}_{\Dc'}$ are not the zero map.
\end{proof}

\subsection{Rank condition for the image}
\begin{lemma}\label{DecomposabilityOfMultilinearForm}
Fix bases $\Bc=\cpa{e_1,\cdots, e_n}$ of $\K^n$ and $\Dc=\cpa{e_1,\cdots, e_k}$ of $\K^k$. A multilinear alternating form $\psi\in \Hom_\K(\bigwedge^k\K^n,\bigwedge^k\K^k)$ is in the image of the Pl\"ucker map $\wedge^k$ if and only if there exists $\la\in\K$ and linearly independent vectors $z_1,\cdots,z_{n-k}$ such that
\[\sum_{I\in \omega(n-k,n)}\sgn\sigma_I\eta_\Dc(\psi(e_{\wh I}))e_I=\la z_{(1,\cdots, n-k)}.\]
\end{lemma}
\begin{proof}
We show both implications
\setlength{\leftmargini}{0cm}
\begin{itemize}
\item[$\boxed{\implies}$] If $\psi=\wedge^k\vp$, the equality follows by choosing $z_1,\cdots, z_{n-k}$ to be a basis of $\ker\vp$. Completing this set to a basis of $\K^n$ and using corollary (\ref{UpToScalarCanonicalIso}) gives the result after a simple calculation.
\item[$\boxed{\impliedby}$] Let $\Zc=\cpa{z_1,\cdots, z_n}$ be a basis of $\K^n$ which extends the given $z_1,\cdots, z_{n-k}$. We can take $\vp$ to be
\[\vp(z_i)=\begin{cases}
0 & \text{if }1\leq i\leq n-k\\
\pa{\mu\la\sgn\sigma_{(1,\cdots,n-k)}} e_1 & \text{if }i=n-k+1\\
e_{i-n+k} & \text{if }i>n-k+1
\end{cases}\]
where $\mu\in\K\nz$ is such that $\eta^\Bc_\Dc=\mu \eta^{\Zc}_{\Dc}$.
\end{itemize}
\setlength{\leftmargini}{0.5cm}
%For simplicity we omit the $\eta_\Dc$. 
%\setlength{\leftmargini}{0cm}
%\begin{itemize}
%\item[$\boxed{\implies}$] Suppose that $\psi=\wedge^k(\vp)$ and let $\cpa{z_1,\cdots, z_n}$ be a basis of $\K^n$ such that $z_1,\cdots, z_{n-k}$ are linearly independent vectors in $\ker \vp$, then
%\begin{align*}
%&\sum_{I\in \omega(n-k,n)}\sgn\sigma_I\wedge^k\vp(e_{\wh I})e_I\pasgnl={(\ref{UpToScalarCanonicalIso})}\\
%&\qquad\qquad\qquad=\mu\sum_{I\in \omega(n-k,n)}\sgn\sigma_I\wedge^k\vp(z_{\wh I})z_I=\\
%&\qquad\qquad\qquad=\pa{\mu\ \sgn\sigma_{(1,\cdots,n-k)}\wedge^k\vp(z_{(n-k+1,\cdots,n)}) }z_{\pa{1,\cdots, n-k}}.
%\end{align*}
%\item[$\boxed{\impliedby}$] Let $z_1,\cdots, z_n$ be a basis of $\K^n$ which extends $\cpa{z_1,\cdots, z_{n-k}}$ and define $\wt\vp$ by
%\[\wt\vp(z_i)=\begin{cases}
%0 & \text{if }1\leq i\leq n-k\\
%e_1 & \text{if }i=n-k+1\\
%e_{i-n+k} & \text{if }i>n-k+1
%\end{cases}\]
%Let $\al=\wedge^k\wt\vp(z_{(n-k+1,\cdots, n)})$ and consider the following chain of equalities
%\begin{align*}
%\sum_{I\in \omega(n-k,n)}\sgn\sigma_I\psi(e_{\wh I})e_I=&\la z_{(1,\cdots, n-k)}=\\
%=&\frac\la\al\wedge^k\wt\vp(z_{(n-k+1,\cdots, n)}) z_{(1,\cdots, n-k)}=\\
%=&\pa{\frac\la\al\sgn\sigma_{(1,\cdots,n-k)}}\sum_{I\in \omega(n-k,n)}\sgn\sigma_I\wedge^k\wt\vp(z_{\wh I})z_I=\\
%=&\pa{\frac\la\al\sgn\sigma_{(1,\cdots,n-k)}}\mu\sum_{I\in \omega(n-k,n)}\sgn\sigma_I\wedge^k\wt\vp(e_{\wh I})e_I
%\end{align*}
%where the third equality follows from the construction of $\wt\vp$ and the last is (\ref{UpToScalarCanonicalIso}). If we now define $\vp$ by
%\[\vp(z_i)=\begin{cases}
%0 & \text{if }1\leq i\leq n-k\\
%\mu\pa{\frac\la\al\sgn\sigma_{(1,\cdots,n-k)}} e_1 & \text{if }i=n-k+1\\
%e_{i-n+k} & \text{if }i>n-k+1
%\end{cases}\]
%with the same reasoning we see that
%\[\sum_{I\in \omega(n-k,n)}\sgn\sigma_I\psi(e_{\wh I})e_I=\sum_{I\in \omega(n-k,n)}\sgn\sigma_I\wedge^k\vp(e_{\wh I})e_I.\]
%By linear independence, this shows that for all $J\in\omega(k,n)$ we have
%\[\psi(e_J)=\wedge^k \vp(e_J),\]
%so $\psi$ and $\wedge^k(\vp)$ agree on a basis of $\bigwedge^k\K^n$ and are thus the same map. 
%\end{itemize}
%\setlength{\leftmargini}{0.5cm}
\end{proof}

\begin{definition}
Let $\Bc=\cpa{e_1,\cdots, e_n}$ and $\Dc=\cpa{e_1,\cdots, e_k}$ be bases of $\K^n$ and $\K^k$ respectively. For any $\psi\in \Hom_\K(\bigwedge^k\K^n,\bigwedge^k\K^k)$ we define $\Phi_{\Bc,\Dc}(\psi)$ to be
\[\Phi_{\Bc,\Dc}(\psi):\funcDef{\K^n}{\bigwedge^{n-k+1}\K^n}{v}{\displaystyle\sum_{I\in \omega(n-k,n)}\sgn\sigma_I\eta_\Dc(\psi(e_{\wh I}))e_I\wedge v}.\]
\end{definition}

\begin{remark}\label{RankPhiIsBaseIndependent}
The rank of $\Phi_{\Bc,\Dc}(\psi)$ does not depend on the choice of basis. Indeed if we change basis, by corollary (\ref{UpToScalarCanonicalIso}) we see that
\[\sum_{I\in \omega(n-k,n)}\sgn\sigma_I\eta_\Dc(\psi(e_{\wh I}))e_I\wedge v=\mu\sum_{I\in \omega(n-k,n)}\sgn\sigma_I\eta_{\Dc'}(\psi(e'_{\wh I}))e'_I\wedge v,\]
so $\ker \Phi_{\Bc,\Dc}(\psi)$ does not depend on the basis and thus neither do nullity or rank.\bigskip

For this reason we will write propositions which only concern the rank of $\Phi_{\Bc,\Dc}(\psi)$ omitting the bases.
\end{remark}

\begin{remark}
$\Phi_{\Bc,\Dc}(\psi)$ is linear in $\psi$.
\end{remark}

\begin{proposition}\label{RankCriterionForImageOfPlucker}
An alternating multilinear map $\psi\in \Hom_\K(\bigwedge^k\K^n,\bigwedge^k\K^k)$ is in the image of the Pl\"ucker map $\wedge^k$ if and only if $\Phi(\psi)$ has rank at most $k$.
\end{proposition}
\begin{proof}
For the $\implies$ arrow, choose a basis $\Zc=\cpa{z_1,\cdots, z_n}$ for $\K^n$ which extends a basis for $\ker\vp$. Because of how we proved lemma (\ref{DecomposabilityOfMultilinearForm}), we see that if $v\in\ker\vp$ then $\Phi_{\Zc,\Dc}(\wedge^k\vp)(v)=\la z_{(1,\cdots, n-k)}\wedge v$, which is zero by linear dependence. Thus the nullity of $\Phi(\wedge^k\vp)$ is at least $\dim\ker\vp=n-k$.\bigskip

Given $n-k$ linearly independent vectors in $\ker\Phi(\psi)$, by the total decomposibility criterion (\ref{TotalDecomposibilityCriterion}) there exists $\la\in\K$ such that
\[\sum_{I\in \omega(n-k,n)}\sgn\sigma_I\eta_\Dc(\psi(z_{\wh I}))z_I=\la z_1\wedge\cdots \wedge z_{n-k}.\]
This concludes by lemma (\ref{DecomposabilityOfMultilinearForm}).
%Let $\Dc$ be any basis of $\K^k$. We prove both implications
%\setlength{\leftmargini}{0cm}
%\begin{itemize}
%\item[$\boxed{\implies}$] Suppose that $\psi=\wedge^k(\vp)$ and let $\Zc=\cpa{z_1,\cdots, z_{n}}$ be a basis of $\K^n$ such that the first $n-k$ vectors are a basis of $\ker \vp$. Note that
%\[\sum_{I\in \omega(n-k,n)}\sgn\sigma_I\eta_\Dc(\psi(z_{\wh I}))z_I=\sgn\sigma_{(1,\cdots,n-k)}\eta_\Dc(\wedge^k\vp(z_{(n-k+1,\cdots,n)}))z_{(1,\cdots, n-k)}.\]
%If $v\in \ker \vp$ then $z_{(1,\cdots, n-k)}\wedge v=0$ and therefore by the above equality $v\in \ker \Phi_{\Zc,\Dc}(\psi)$. This means that $\Phi(\psi)$ has a nullity of at least $n-k$ (i.e. rank at most $k$).
%\item[$\boxed{\impliedby}$] Suppose that $\cpa{z_1\cdots, z_{n-k}}$ are linearly independent elements of $\ker \Phi(\psi)$. By the total decomposability criterion (\ref{TotalDecomposibilityCriterion}) there exists $\la\in\K$ such that
%\[\sum_{I\in \omega(n-k,n)}\sgn\sigma_I\eta_\Dc(\psi(z_{\wh I}))z_I=\la z_1\wedge\cdots \wedge z_{n-k}.\]
%This concludes by lemma (\ref{DecomposabilityOfMultilinearForm}).
%\end{itemize}
%\setlength{\leftmargini}{0.5cm}
\end{proof}

\begin{definition}
Let $\Bc=\cpa{e_1,\cdots, e_n}$ and $\Dc=\cpa{e_1,\cdots, e_k}$ be bases of $\K^n$ and $\K^k$ respectively. Let $\zeta_{\Bc,\Dc}:\Hom_\K(\bigwedge^k\K^n,\bigwedge^k\K^k)\to\bigwedge^k\K^n$ be the isomorphism discussed during remark (\ref{CodomainOfPluckerMap}). We define
\[\wt \Phi_{\Bc,\Dc}:\funcDef{\bigwedge^k\K^n}{ \Hom_\K\pa{\K^n,\bigwedge^{n-k+1}\K^n}}{\omega}{\Phi_{\Bc,\Dc}(\zeta_{\Bc,\Dc}\ii(\omega))}\]
\end{definition}

\begin{proposition}
The rank of $\wt \Phi_{\Bc,\Dc}(\omega)$ does not depend on the choice of basis.
\end{proposition}
\begin{proof}
If $\omega=\sum_{I\in\omega(k,n)}p_Ie_I$ then an easy calculation shows that $v$ is an element of $\ker \Phi_{\Bc,\Dc}(\zeta_{\Bc,\Dc}\ii(\omega))$ if and only if for all $I\in \omega(k,n)$ either $p_I=0$ or $e_{\wh I}\wedge v=0$, which is the same as saying $\omega\wedge v= 0$. The last condition is base independent so $\ker \Phi_{\Bc,\Dc}(\zeta_{\Bc,\Dc}\ii(\omega))$ must be also.
%Let $\omega=\sum_{I\in\omega(k,n)}p_Ie_I$.
%\begin{align*}
%\Phi_{\Bc,\Dc}(\zeta_{\Bc,\Dc}\ii(\omega))(v)=&\sum_{I\in\omega(n-k,n)}\sgn\sigma_I\eta_\Dc(\zeta_{\Bc,\Dc}\ii(\omega)(e_{\wh I}))e_I\wedge v=\\
%=&\sum_{I\in\omega(n-k,n)}\sgn\sigma_Ip_{\wh I}e_I\wedge v,
%\end{align*}
%so $v\in \ker \Phi_{\Bc,\Dc}(\zeta_{\Bc,\Dc}\ii(\omega))$ if and only if for all $I\in \omega(k,n)$ either $p_I=0$ or $e_{\wh I}\wedge v=0$, which is the same as saying $\omega\wedge v= 0$ if and only if $\omega=0$ or $v=0$. This is a base independent condition, so the rank of $\wt\Phi_{\Bc,\Dc}(\omega)$ does not depend on the choice of basis.
\end{proof}

\begin{remark}
$\wt\Phi_{\Bc,\Dc}$ is linear.
\end{remark}



\noindent
Let us define a matrix with coefficients in\footnote{what we will later call the Bracket ring} $\K[z_I\mid I\in\omega(k,n)]$ which represents $\wt\Phi_{\Bc,\Dc}$:\medskip

Let $B^I\in \Mc\pa{\binom{n}{n-k+1},n,\K}$ be the matrix which represents $\Phi_{\Bc,\Dc}(\zeta_{\Bc,\Dc}\ii(e_I))$ in the bases induced by $\Bc$ and $\Dc$. By linearity
\[\wt \Phi_{\Bc,\Dc}\pa{\sum_{I\in \omega(k,n)}a_I e_I}(v)=\sum_{I\in \omega(k,n)}a_I \Phi_{\Bc,\Dc}(\zeta_{\Bc,\Dc}\ii(e_I))(v)=\sum_{I\in \omega(k,n)}a_I B^I v.\]
We define the matrix which represents $\wt\Phi_{\Bc,\Dc}$ to be
\[M_{\Bc,\Dc}=\sum_{I,\omega(k,n)}B^I z_I=\pa{\sum_{I\in\omega(k,n)}(B^I)_{i,j}z_I}_{i,j}.\]


\begin{remark}
The rank of $\Phi_{\Bc,\Dc}(\sum_{I\in\omega(k,n)}p_Ie_I)$ is exactly the rank of $\rbar{M}_{z_I=p_I}$.
\end{remark}

\noindent The previous remark together with proposition (\ref{RankCriterionForImageOfPlucker}) tells us that
\begin{align*}
\imm (\zeta_{\Bc,\Dc}\circ \wedge^k)=&\cpa{\sum_{I\in \omega(k,n)}p_Ie_I\mid \rnk \rbar{M}_{z_I=p_I}<k+1}=\\
=&V(\cpa{\det m\mid m\text{ is a $(k+1)\times(k+1)$ minor of $M$}}),
\end{align*}
which is evidently a Zariski-closed subset of $\bigwedge^k\K^n$.\bigskip

It follows trivially that the projectivization\footnote{recall (\ref{ImagePluckerMapIsCone}) that $\imm \wedge^k$ is a cone.} of this set (i.e. the image of $\Pl$) is closed in $\Pj(\bigwedge^k\K^n)$, so we found a bijection between $\Gr(k,n)$ and a projective variety, which we can use to endow $\Gr(k,n)$ with the structure of one.


\begin{remark}
The determinants we used to show that the image of the Pl\"ucker embedding is closed do not generate the ideal of that variety. The most well known set of generators for that ideal are the \textbf{Pl\"ucker relations} (Theorem 2.4.3 in \cite{matroids}, page 80).
\end{remark}










