\chapter{Classical Grassmannians}

\section{First definitions and conventions}
\begin{definition}[Grassmannian]
Let $k\leq n$ be a pair of positive integers. We define the \textbf{$(n,k)$-Grassmannian}, denoted\footnote{we shall often omit the field when clear from context} $\Gr(k,n,\K)$, as the set of $(n-k)$-dimensional $\K$-vector subspaces of $\K^n$.
\end{definition}

\begin{remark}
We may equivalently define $\Gr(k,n)$ to be the following set:
\[\cpa{\ker \vp\mid \vp\in \Hom_\K(\K^n,\K^k),\ \rnk \vp=k}.\]
\end{remark}
\begin{proof}
If $H\in \Gr(k,n)$, let $z_1,\cdots, z_n$ be a basis of $\K^n$ such that $H=\Span(z_1,\cdots, z_{n-k})$ and let $e_1,\cdots, e_k$ be any basis of $\K^k$, then we can view $H$ as the kernel of the (rank $k$) linear map defined by
\[\vp(z_i)=\begin{cases}
0 &\text{if }i\leq n-k\\
e_{i-n+k} &\text{otherwise}
\end{cases}\]
Viceversa, if $\vp$ is a rank $k$ linear map then by the rank-nullity theorem its kernel is an $n-k$ dimentional subspace of $\K^n$.
\end{proof}

\begin{lemma}\label{kerAkerBVSActionOfGLk}
Let $\vp,\psi\in \Hom_\K(\K^n,\K^k)$ be linear maps of full rank. The following conditions are equivalent:
\begin{enumerate}
    \item $\ker\vp=\ker\psi$,
    \item there exists $\theta\in \GL(\K^k)$ such that $\vp=\theta\circ \psi$. 
\end{enumerate}
\end{lemma}
\begin{proof}
Let us prove both implications:
\setlength{\leftmargini}{0cm}
\begin{itemize}
\item[$\boxed{2.\implies 1.}$] $\ker(\theta\circ \psi)=\psi\ii(\ker \theta)=\psi\ii(\cpa{0})=\ker \psi$. 
\item[$\boxed{1.\implies 2.}$] Let $z_1,\cdots, z_n$ be a basis of $\K^n$ such that $\ker\vp=\ker\psi=\Span(z_1,\cdots, z_{n-k})$. By construction $\vp(z_{n-k+1}),\cdots, \vp(z_n)$ and $\psi(z_{n-k+1}),\cdots, \psi(z_n)$ are bases of $\K^k$. Let $\theta$ be the linear automorphism of $\K^k$ determined by $\theta(\psi(z_i))=\vp(z_i)$ for all $n-k<i\leq n$. By construction $\theta$ is nonsingular and $\vp$ agrees with $\theta\circ \psi$ on a basis of $\K^n$.
\end{itemize}
\setlength{\leftmargini}{0.5cm}
\end{proof}

\begin{corollary}\label{LinearQuotientDefinition}
We may redefine Grassmannians in terms of linear maps as follows:
\[\Gr(k,n)=\quot{\cpa{\vp\in \Hom_\K(\K^n,\K^k)\mid \vp\ \text{surjective.}}}\sim\]
where $\vp\sim \psi$ if and only if $\exists \theta\in \GL(\K^k)$ such that $\vp=\theta\circ \psi$.
\end{corollary}

\noindent We end the prelude of this chapter by introducing some notation and conventions
\begin{definition}[Multiindicies]
We define a \textbf{$(k,n)$-multiindex} as an element of $\cpa{1,\cdots, n}^k$. Our notation for a multiindex $I$ will usually be $I=(i_1,\cdots, i_k)$.\\
We denote the set of \textbf{ordered $(k,n)$-multiindicies} with
\[\omega(k,n)=\cpa{(i_1,\cdots, i_k)\in \cpa{1,\cdots, n}^k\mid i_1<\cdots<i_k}.\]
If $A$ is a $k\times n$ matrix and $I$ is a $(k,n)$-multiindex, we denote the $I$-minor by $A_I$, i.e.
\[A_I=\mat{a_{1,i_1} &\cdots &a_{1,i_{k}}\\\vdots &\ddots &\vdots\\a_{k,i_1}&\cdots&a_{k,i_k}}.\]
If $B$ is an $\al\times \beta$ matrix, $i\in\cpa{1,\cdots, \al}$ and $j\in \cpa{1,\cdots, \beta}$ we use $B_{\times i,\times j}$ to denote the $(\al-1)\times (\beta-1)$ matrix obtained from $B$ by deleting the $i$-th row and the $j$-th column.
\end{definition}

\begin{remark}
The set
\[\cpa{e_{i_1}\wedge\cdots\wedge e_{i_k}\mid 1\leq i_1<\cdots<i_k\leq n}\]
forms a basis for $\bigwedge^k\K^n$. For brevity, for all multiindicies $I=(i_1,\cdots, i_k)$ we shall define
\[e_I=e_{i_1}\wedge\cdots\wedge e_{i_k}.\]
\end{remark}

\begin{notation}
Whenever a base of $\K^\ell$ is fixed, we will identify $\bigwedge^\ell\K^\ell$ with $\K$ by sending the wedge of the ordered basis to $1\in\K$.
\end{notation}

\section{The Pl\"ucker embedding}
To make the study of Grassmannians easier, we want to identify $\Gr(k,n)$ with a projective variety.\\
Intuitively we seek to transform objects defined by several vectos into objects given by a single vector and then take the projective space construction. This conversion can be made by taking wedge products appropriately.

\begin{definition}[Pl\"ucker map]
Let $k\leq n$ be a pair of positive integers. We define the \textbf{Pl\"ucker map} as\footnote{the map $\wedge^k\vp$ is well defined because if we view it as a map $\wedge^k\vp:(\K^n)^k\to \bigwedge^k\K^k$ then it is multilinear and alternating.}:
\[\wedge^k:\funcDef{\Hom_\K(\K^n,\K^k)}{\Hom_\K(\bigwedge^k\K^n,\bigwedge^k\K^k)}{\vp}{\wedge^k\vp},\]
where $(\wedge^k\vp)(v_1\wedge\cdots\wedge v_k)=\vp(v_1)\wedge\cdots\wedge\vp(v_k).$\\
Note that if $e_1,\cdots, e_k$ is a basis of $\K^k$ then\footnote{the columns of the determinant are the coordinates of the vectors given in the base $e_1,\cdots, e_k$.}
\[\wedge^k(\vp)(v_1\wedge\cdots\wedge v_k)=\det\mat{\vp(v_1)|\cdots|\vp(v_k)} e_1\wedge\cdots\wedge e_k,\]
which we will often identify with $\det\mat{\vp(v_1)|\cdots|\vp(v_k)}$ when the choice of basis is clear.
\end{definition}

\begin{remark}\label{CodomainOfPluckerMap}
The codomain of the Pl\"ucker map is isomorphic to $\bigwedge^k\K^n$, indeed
\[\Hom_\K\pa{\bigwedge^k\K^n,\bigwedge^k\K^k}\cong \pa{\bigwedge^k\K^n}^{\vee}\cong \bigwedge^k\K^n.\]
The isomorphism depends on the choice of basis for $\K^n$ and $\K^k$. If $e_1,\cdots, e_n$ is a basis of $\K^n$ and $e_1,\cdots, e_k$ is a basis of $\K^k$ then the isomorphism takes on the following form
\[\funcDef{\Hom_\K(\K^n,\K^k)}{\bigwedge^k\K^n}{\vp}{\displaystyle\sum_{1\leq i_1<\cdots< i_k\leq n} \det(\vp(e_{i_1})\mid \cdots\mid \vp(e_{i_k}))e_{i_1}\wedge\cdots\wedge e_{i_k}},\]
where the determinant is defined assuming that the $\vp(e_j)$ are viewed as their coordinates in the base $e_1,\cdots, e_k$.
\end{remark}


\begin{remark}
The image of the Pl\"ucker map is a cone.
\end{remark}
\begin{proof}
For any $\la\in \K^\ast$ and any map $\vp\in \Hom_\K(\K^n,\K^k)$ we see that
\[\la\wedge^k(\vp)=\wedge^k(\al\circ \vp),\]
for any automorphism $\al$ of $\K^k$ with determinant\footnote{For example we may fix a base $e_1,\cdots, e_k$ of $\K^k$ and define $\al(e_i)=\begin{cases}
\la e_1 &\text{if }i=1\\
e_i &\text{otherwise}
\end{cases}$} $\la$.
\end{proof}


\begin{remark}\label{WhenPluckerMapIsZero}
$\rnk \vp<k$ if and only if $\wedge^k(\vp)=0$.
\end{remark}
\begin{proof}
$\wedge^k(\vp)$ is the zero map if an only if the set $\cpa{\vp(v_1),\cdots, \vp(v_k)}$ is linearly dependent for any choice of $v_1,\cdots, v_k$, i.e. $\vp$ is not of full rank.
\end{proof}

\begin{lemma}\label{CharacterizationOfKernels}
Let $\vp:\K^n\to \K^k$ be a full rank linear map, then
\[\ker\vp=\cpa{z\in\K^n\mid \forall w_2,\cdots, w_k\in \K^n,\ \wedge^k(\vp)(z\wedge w_2\wedge\cdots\wedge w_k)=0}.\]
\end{lemma}
\begin{proof}
If $\vp(z)=0$ then for any $w_2,\cdots, w_k\in \K^k$ we see that 
\[\wedge^k(\vp)(z\wedge w_2\wedge\cdots\wedge w_k)=0\wedge \vp(w_2)\wedge\cdots\wedge \vp(w_k)=0.\]
Suppose now that $\vp(z)\neq 0$ and let $v_2,\cdots, v_k$ be such that $\cpa{\vp(z), v_2,\cdots,v_k}$ form a basis for $\K^k$. Since $\vp$ is surjective there exist $w_2,\cdots, w_k$ such that $\vp(w_i)=v_i$ for all $2\leq i\leq k$.
By construction 
\[\wedge^k(\vp)(z\wedge w_2\wedge\cdots\wedge w_k)=\vp(z)\wedge v_2\wedge\cdots\wedge v_k\neq0.\]
\end{proof}

\begin{proposition}\label{PluckerMapInjectiveOnGrassmanniansUpToScalar}
Let $\sim$ be the equivalence relation defined in corollary (\ref{LinearQuotientDefinition}), then for any two full rank linear maps $\vp,\psi:\K^n\to \K^k$
\[\vp\sim \psi\coimplies \exists \la\in\K^\ast\ s.t.\ \wedge^k(\vp)=\la\wedge^k(\psi).\]
\end{proposition}
\begin{proof}
Let us prove both implications:
\setlength{\leftmargini}{0cm}
\begin{itemize}
\item[$\boxed{\implies}$] If $\vp=\theta\circ \psi$ for $\theta\in GL(\K^k)$ then it follows easily from known properties of the determinant that \[\wedge^k(\vp)=\wedge^k(\theta\circ \psi)=(\det\theta) \wedge^k(\psi).\]
\item[$\boxed{\impliedby}$] From lemma (\ref{kerAkerBVSActionOfGLk}) we see that it is enough to prove that $\ker \vp=\ker \psi$. We conclude by applying lemma (\ref{CharacterizationOfKernels}) as follows:
\begin{align*}
\ker\vp=&\cpa{z\in\K^n\mid \forall w_2,\cdots, w_k\in \K^n,\ \wedge^k(\vp)(z\wedge w_2\wedge \cdots\wedge w_k)=0}\\
=&\cpa{z\in\K^n\mid \forall w_2,\cdots, w_k\in \K^n,\ \la\wedge^k(\psi)(z\wedge w_2\wedge\cdots\wedge w_k)=0}=\\
=&\cpa{z\in\K^n\mid \forall w_2,\cdots, w_k\in \K^n,\ \wedge^k(\psi)(z\wedge w_2\wedge \cdots\wedge w_k)=0}=\ker \psi.
\end{align*}
\end{itemize}
\setlength{\leftmargini}{0.5cm}
\end{proof}

\begin{remark}
Because of proposition (\ref{PluckerMapInjectiveOnGrassmanniansUpToScalar}) and remark (\ref{WhenPluckerMapIsZero}) there exists a unique $h$ such that the diagram commutes
% https://q.uiver.app/#q=WzAsMyxbMCwwLCJcXGNwYXtcXHZwXFxpbiBcXEhvbV9cXEsoXFxLXm4sXFxLXmspXFxtaWQgXFxybmsgXFx2cD1rfSJdLFswLDEsIlxcR3IoayxuKSJdLFsxLDAsIlxcUGooXFxiaWd3ZWRnZV5rXFxIb21fXFxLKFxcS15uLFxcS15rKSkiXSxbMCwxLCJcXHBpIl0sWzAsMiwiW1xccGhpXSJdLFsxLDIsImgiLDIseyJzdHlsZSI6eyJib2R5Ijp7Im5hbWUiOiJkYXNoZWQifX19XV0=
\[\begin{tikzcd}
	{\cpa{\vp\in \Hom_\K(\K^n,\K^k)\mid \rnk \vp=k}} & {\Pj(\Hom_\K(\bigwedge^k\K^n,\bigwedge^k\K^k))} \\
	{\Gr(k,n)}
	\arrow["\pi", from=1-1, to=2-1]
	\arrow["{[\wedge^k]}", from=1-1, to=1-2]
	\arrow["h"', dashed, from=2-1, to=1-2]
\end{tikzcd}\]
Moreover, such an $h$ must be injective by proposition (\ref{PluckerMapInjectiveOnGrassmanniansUpToScalar}).
\end{remark}


\begin{definition}[Pl\"ucker embedding]
Let us fix a basis $e_1,\cdots, e_n$ of $\K^n$ and a basis $e_1,\cdots, e_k$ of $\K^k$. We define the \textbf{Pl\"ucker embedding} as follows
\[\Pl:\funcDef{\Gr(k,n)}{\Pj(\bigwedge^k\K^n)}{[\vp]_\sim}{[(\det(\vp(e_{i_1})\mid\cdots\mid\vp(e_{i_k})))_{1\leq i_1<\cdots<i_k\leq n}]_{\K^\ast}}.\]
\end{definition}

\begin{remark}
If we fix bases for $\K^n$ and $\K^k$ and $\zeta$ is the isomorphism $\Hom_\K(\bigwedge^k\K^n,\bigwedge^k\K^k)\to \bigwedge^k\K^n$ discussed during remark (\ref{CodomainOfPluckerMap}), we see that the following diagram commutes
% https://q.uiver.app/#q=WzAsMyxbMCwxLCJcXEdyKGssbikiXSxbMCwwLCJcXFBqKFxcYmlnd2VkZ2Vea1xcSG9tX1xcSyhcXEtebixcXEteaykpIl0sWzEsMCwiXFxQaihcXGJpZ3dlZGdlXmtcXEtebikiXSxbMCwxLCJoIiwyXSxbMSwyLCJcXFBqKFxcemV0YSkiXSxbMCwyLCJcXG1hdGhybXtQbH0iLDJdXQ==
\[\begin{tikzcd}
	{\Pj(\bigwedge^k\Hom_\K(\K^n,\K^k))} & {\Pj(\bigwedge^k\K^n)} \\
	{\Gr(k,n)}
	\arrow["h"', from=2-1, to=1-1]
	\arrow["{\Pj(\zeta)}", from=1-1, to=1-2]
	\arrow["{\Pl}"', from=2-1, to=1-2]
\end{tikzcd}\]
This proves that the Pl\"ucker embedding is well defined and injective.
\end{remark}

\begin{remark}
If we define $\phi=\zeta\circ \wedge^k$ then we see that \[\Pl\circ \pi=\Pj(\phi).\]
This form will be instrumental in the next chapter.
\end{remark}

\noindent
The entries of the homogeneous $\binom nk$-tuple associated to $[\vp]\in \Gr(k,n)$ are called the \textbf{Pl\"ucker coordinates} of $[\vp]$. The Pl\"ucker coordinates are unique up to multiplying each by the same nonzero scalar.

\begin{remark}
The Pl\"ucker embedding depends on the choice of basis for $\K^n$ but not on the one for $\K^k$, since the effect of changing the basis of $\K^k$ is that of multiplying all Pl\"ucker coordinates by the same scalar (the determinant of the change of basis), which doesn't change the point they describe in $\Pj(\bigwedge^k\K^k)$.\\
The dependence on the base of $\K^n$ corresponds to the fact that $\GL(\K^n)$ acts transitively on $\Gr(k,n)$ viewed as the set of $(n-k)$-dimentional subspaces of $\K^n$.
\end{remark}



\section{The image of the Pl\"ucker embedding is closed}
Thus far we have identified $\Gr(k,n)$ with a subset of some projective space. We now seek to show that it is a closed subset in the Zariski topology. Our approach mostly readapts parts of \cite{McKernan}.
\medskip

\noindent First we need some linear algebra results

\begin{lemma}\label{Divisibility}
Let $\omega\in \bigwedge^k\K^n$. For any given nonzero vector $v$ there exists $\e\in \bigwedge^{k-1}\K^n$ such that $\omega=\e\wedge v$ if and only if $\omega\wedge v=0$.
\end{lemma}
\begin{proof}
If $\omega=\e\wedge v$ then $\omega\wedge v=\e\wedge v\wedge v=0$.\\
Suppose now that $\omega\wedge v$. Let $v_1,\cdots, v_n$ be a basis of $\K^n$ such that $v_1=v$. If we write
\[\omega=\sum_{I\in\omega(k,n)}p_I v_I\]
then we see that for any given multiindex $I$ either $p_I=0$ or $v_I\wedge v=0$. Since $v_1,\cdots, v_n$ is a basis, $v_I\wedge v_1=0$ if and only if $1\in I$, i.e. $v_I=v\wedge v_{\pa{i_2,\cdots, i_k}}$, therefore
\[\omega=v\wedge \pa{\sum_{2\leq i_2<\cdots i_k\leq n} p_{\pa{1,i_2,\cdots, i_k}} e_{\pa{i_2,\cdots, i_k}}}\]
\end{proof}

\begin{corollary}[Total decomposibility criterion]\label{TotalDecomposibilityCriterion}
Let $\omega\in \bigwedge^k\K^n$. If $\dim\cpa{v\in\K^n\mid \omega\wedge v=0}\geq k$ then $\omega=\la v_1\wedge \cdots\wedge v_k$ for any set of linearly independent vectors $\cpa{v_1,\cdots, v_k}$ in $\cpa{v\in\K^n\mid \omega\wedge v=0}$ and some scalar $\la$.
\end{corollary}
\begin{proof}
The set $\cpa{v\in\K^n\mid \omega\wedge v=0}$ is clearly a subspace of $\K^n$. Let $\cpa{v_1,\cdots, v_k}$ be linearly independent vectors of this space. By iterating the above lemma we see that
\[\omega=\la\wedge v_1\wedge\cdots\wedge v_k\]
for some $\la\in \bigwedge^0\K^n=\K$.
\end{proof}

\begin{lemma}\label{DecomposabilityOfMultilinearForm}
A multilinear alternating form $\psi\in \Hom_\K(\bigwedge^k\K^n,\bigwedge^k\K^k)$ is in the image of the Pl\"ucker map $\wedge^k$ if and only if there exists a basis $e_1,\cdots, e_n$ of $\K^n$ and an element $\la$ of $\K$ such that
\[\sum_{I\in \omega(n-k,n)}\psi(e_{\wh I})e_I=\la e_{(1,\cdots, n-k)}.\]
\end{lemma}
\begin{proof}
Let us fix a basis $e_1,\cdots, e_k$ of $\K^k$. Using this basis we identify $\bigwedge^k\K^k$ with $\K$.\\
Suppose that $\psi=\wedge^k(\vp)$ and let $e_1,\cdots, e_n$ be a basis of $\K^n$ such that $e_1,\cdots, e_{n-k}$ is a basis of $\ker \vp$, then
\[\sum_{I\in \omega(n-k,n)}\wedge^k\vp(e_{\wh I})e_I=\pa{\vp(e_{n-k+1})\wedge \cdots\wedge \vp(e_{n})} e_{\pa{1,\cdots, n-k}}.\]
Suppose now that we have the decomposibility above and let us define $\vp$ by
\[\vp(e_i)=\begin{cases}
0 & \text{if }1\leq i\leq n-k\\
\la e_1 & \text{if }i=n-k+1\\
e_{i-n+k} & \text{if }i>n-k+1
\end{cases}\]
We now compute a second form for $\sum_{I\in \omega(n-k,n)}\psi(e_{\wh I})e_I$
\begin{align*}
\sum_{I\in \omega(n-k,n)}\psi(e_{\wh I})e_I=&\la e_{(1,\cdots, n-k)}=\\
=&\vp(e_{n-k+1})\wedge\cdots\wedge\vp(e_n) e_{(1,\cdots, n-k)}=\\
=&\sum_{I\in \omega(n-k,n)}\wedge^k\vp(e_{\wh I})e_I
\end{align*}
where the second equality follows from the construction of $\vp$.\\
We have shown that for all $J\in\omega(k,n)$ we have
\[\psi(e_J)=\wedge^k \vp(e_J),\]
so $\psi$ and $\wedge^k(\vp)$ agree on a basis of $\bigwedge^k\K^n$ and are thus the same map. 
\end{proof}

\noindent Let us consider the following map: let $\psi:\Hom_\K(\bigwedge^k\K^n,\bigwedge^k\K^k)$ be any alternating multilinear map, we define $\Phi(\psi)$ as
\[\Phi(\psi):\funcDef{\K^n}{\bigwedge^{n-k+1}\K^n}{v}{\displaystyle\sum_{I\in \omega(n-k,n)}\psi(e_{\wh I})e_I\wedge v}\]
where $\wh I$ is the ordered $k$-tuple of the indicies in $\cpa{1,\cdots, n}$ missing from $I$.



\begin{proposition}\label{RankCriterionForImageOfPlucker}
An alternating multilinear map $\psi\in \Hom_\K(\bigwedge^k\K^n,\bigwedge^k\K^k)$ is in the image of the Pl\"ucker map $\wedge^k$ if and only if $\Phi(\psi)$ has rank at most $k$.
\end{proposition}
\begin{proof}
Suppose that $\psi=\wedge^k(\vp)$ and let $\cpa{z_1,\cdots, z_{n-k},z_{n-k+1},\cdots, z_{n}}$ be a basis of $\K^n$ such that the first $n-k$ vectors are a basis of $\ker \vp$. From the proof of lemma (\ref{DecomposabilityOfMultilinearForm}) we see that if $v\in \ker \vp$ then $v\in \ker \Phi(\psi)$, i.e. $\Phi(\psi)$ has a nullity of at least $n-k$ (or rank at most $k$).
\medskip

\noindent
Suppose now that $\cpa{z_1\cdots, z_{n-k}}$ are linearly independent elements of $\ker \Phi(\psi)$. By the Total decomposibility criterion (\ref{TotalDecomposibilityCriterion}) we have that there exists $\la\in\K$ such that
\[\sum_{I\in \omega(n-k,n)}\psi(e_{\wh I})e_I=\la z_1\wedge\cdots \wedge z_{n-k}.\]This concludes by lemma (\ref{DecomposabilityOfMultilinearForm}).
\end{proof}

\noindent Let us now consider the map
\[\Phi:\funcDef{\bigwedge^k\K^n}{ \Hom_\K\pa{\K^n,\bigwedge^{n-k+1}\K^n}}{\psi^\ast}{\Phi(\zeta\ii(\psi^\ast))}\]
Note that $\Phi$ is linear, so we can represent $\Phi$ as a matrix with coefficients in $\bigwedge^k\K^n$:\\
we fix a basis $\cpa{e_I}$ of $\bigwedge^k\K^n$ and write $\Phi(\sum a_I e_I)=\sum a_I \Phi(\zeta\ii e_I)$. Since each $\Phi(\zeta\ii e_I)$ is linear, it can be viewed as a matrix $\pa{b_{i,j}^I}_{i,j}$ and so the matrix associated to $\Phi$ is $\pa{\sum b_{i,j}^I e_I}_{i,j}$.
\bigskip

\noindent Proposition (\ref{RankCriterionForImageOfPlucker}) tells us that the image of $\zeta\circ\wedge^k$ can by identified by imposing that the rank of the matrix representing $\Phi$ defined above is at most $n-k$, which is equivalent to the vanishing of its $(n-k+1)\times (n-k+1)$ minors, which is a closed condition.

It follows trivially that the projectivization\footnote{recall that $\imm \wedge^k$ is a cone.} of this set (i.e. the image of $\Pl$) is also closed in $\Pj(\bigwedge^k\K^n)$, so we found a bijection between $\Gr(k,n)$ and a projective variety, which we can use to endow $\Gr(k,n)$ with the structure of one.













