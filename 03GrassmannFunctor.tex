\chapter{Representability of the Grassmannian functor}
In this chapter we will work with $\K^n$ and $\K^k$ instead of abstract vector spaces. This means that we have canonical bases $\Can_n=\cpa{e_1,\cdots, e_n}$ and $\Can_k=\cpa{e_1,\cdots, e_k}$ and that we identify $\Hom_\K(\K^n,\K^k)$ with $\Mc(k,n)$.
\medskip

To distinguish the scheme morphisms we define in this chapter from the morphisms of varieties defined previously we use a superscript $s$ (for ``set-theoretic") for the latter, i.e.
\[\phi^s:\funcDef{\Mc(k,n)}{\bigwedge^k\K^n}{A}{\displaystyle\sum_{I\in \omega(k,n)}\det A_I e_I},\quad \Pl^s:\funcDef{\Gr(k,n)}{\Pj(\bigwedge^k\K^n)}{[A]_\sim}{\spa{\phi^s(A)}_{\K^\ast}}\]

\begin{notation}
Let $I$ be an ideal of the ring $A$ and $J$ be a homogeneous ideal of the graded ring $B$. We adopt the following notation
\[V(I)=\cpa{\pf\in\Spec A\mid I\subseteq \pf},\qquad V_+(J)=\cpa{\pf\in\Proj B\mid I\subseteq \pf}.\]
\end{notation}


\section{Grassmannians as projective schemes}

\begin{definition}[Bracket ring]
We define the \textbf{bracket ring} (see page 79 of \cite{matroids}) as the ring of polynomial functions on $\bigwedge^k\K^n$, i.e.
\[\Bc_{k,n}\doteqdot\frac{\K[z_I\mid I\in \cpa{1,\cdots, n}^k]}{(\cpa{z_I-\sgn(\sigma)z_{\sigma(I)}}_{\sigma\in \Sf_k})}\cong \K[z_I\mid I\in \omega(k,n)].\]
We define $\Bc_{k,n}^+$ to be the ideal generated by the indeterminates $z_I$.
\end{definition}

\begin{definition}[Ring of generic matrices]
Let $\K[X_{k,n}]\doteqdot\K[x_{1,1},\cdots,x_{k,n}]$ denote the polynomial ring with $k\cdot n$ variables. We define the \textbf{generic matrix} as
\[X=\mat{
    x_{1,1} & \cdots & x_{1,n}\\
    \vdots & \ddots & \vdots\\
    x_{k,1} &\cdots & x_{k,n}}.\]
By the same token we use $X_I$ to denote the generic $k\times k$ minor determined by the multiindex $I$ and $\det X_I$ to write the formal determinant of this minor.
\end{definition}
\begin{remark}
The ring $\K[X_{k,n}]$ is the coordinate ring of $\Mc(k,n)$. 
\end{remark}

\begin{remark}
The familiar $\Mc(k,n)$ and $\bigwedge^k\K^n$ can be identified with the $\K$-points of the affine schemes $\Spec \K[X_{k,n}]$ and $\Spec \Bc_{k,n}$ respectively (Example 2.3.32 of \cite{QingLiu}). We will use this identification for the rest of the chapter.
\end{remark}

\begin{definition}[Pl\"ucker ring homomorphism]
We define the \textbf{Pl\"ucker ring homomorphism} or simply \textbf{Pl\"ucker homomorphism} as
\[\phi^\#:\funcDef{\Bc_{k,n}}{\K[X_{k,n}]}{z_I}{\det X_I}\]
For brevity we will denote $\Spec \phi^\#$ by $\phi$.
\end{definition}

\begin{remark}\label{PluckerRingHomomorphismWorksForKPoints}
It is clear by construction that
\[\phi\res{\Mc(k,n)}(A)=\sum_{I\in\omega(k,n)}\det A_I e_I=\phi^s(A).\]
\end{remark}

\begin{proposition} $\ker\phi^\#$ is a homogeneous prime ideal and $\Bc_{k,n}^+\not\subseteq \ker\phi^\#$.
\end{proposition}
\begin{proof}
$\ker\phi^\#$ is prime because $\K[X_{k,n}]$ is an integral domain and $z_I\notin \ker \phi^\#$ because $\deg \phi^\#(z_I)=\deg(\det X_I)=k>0$. 
To show homogeneity let us note that if $g$ is homogeneous of degree $d$ then $\phi^\#(g)$ is homogeneous of degree $kd$. If follows that if $f_d$ is the homogeneous component of $f$ of degree $d$ and $0=\phi^\#(f)=\sum_{d\in\N}\phi^\#(f_d)$ then $\phi^\#(f_d)=0$ for all $d\in\N$.
\end{proof}

\begin{proposition}
Let $t:\mathrm{Var}/\K\to \Sch \K$ be the fully faithful functor defined as in Proposition 2.6 of \cite{Hartshorne}. Then $V_+(\ker\phi^\#)\cong t(\imm\Pl^s)$.
\end{proposition}
\begin{proof}
Because $t$ is fully faithful, we only need to show that $V_+(\ker(\phi^\#))(\K)\cong \imm \Pl^s$.
Passing to the corresponding cones, this is equivalent to 
\[\imm \phi^s\cong V(\ker\phi^\#)(\K)=\ol{\imm \phi\res{\Mc(k,n)}}=\ol{\imm \phi^s},\]
which is true because $\imm \phi^s\pasgnl={(\ref{ImageOfPhiClosed})}\ol{\imm \phi^s}$.
\end{proof}

From now on $\Gr(k,n)$ will also have the scheme structure of $V_+(\ker \phi^\#)$. What we used to write $\Gr(k,n)$ corresponds to $\Gr(k,n)(\K)$.

\subsection{Standard affine cover of the Grassmannian scheme}
Recall that projective space admits a standard affine cover given by the loci where one indeterminate does not vanish. In our case we see that
\[\Proj \Bc_{k,n}=\bigcup_{I\in\omega(k,n)}\Spec \pa{\pa{\Bc_{k,n}}_{z_I}^0}=\bigcup_{I\in\omega(k,n)}\Spec \pa{\K\spa{\frac{z_J}{z_I}\mid J\in \omega(k,n)}},\]
where the subscript denotes localization with multiplicative part $\cpa{1,z_I,z_I^2,\cdots}$ and the superscript $0$ denotes the fact that we are only considering terms of degree $0$ in this ring (this is the notation used in \cite{QingLiu}).
\medskip

This open affine cover of $\Proj\Bc_{k,n}$ induces an open cover on $\Gr(k,n)$ as follows:
\[\Gr(k,n)=V_+(\ker\phi^\#)=\bigcup_{I\in \omega(k,n)}\Spec\pa{\pa{\frac{\Bc_{k,n}}{\ker\phi^\#}}_{z_I}^0}.\]

\begin{notation}
Let us fix $I\in\omega(k,n)$, then we denote the restriction of $\phi^\#$ as
\[\phi^\#_I:\funcDef{\K\spa{\frac{z_J}{z_I}\mid J\in \omega(k,n)}}{\K[X_{k,n}]_{\det X_I}^0}{\frac{z_J}{z_I}}{\frac{\det X_J}{\det X_I}}\]
\end{notation}

\begin{remark}
By the first isomorphism theorem we have
\[\frac{\pa{\Bc_{k,n}}_{z_I}^0}{\ker \phi^\#_I}\cong \imm \phi^\#_I=\K\spa{\frac{\det X_J}{\det X_I}\mid J\in \omega(k,n)}\doteqdot \K\spa{\frac{\det X_J}{\det X_I}}.\]
\end{remark}

\begin{remark}
Applying a property of localization we have
\[\pa{\frac{\Bc_{k,n}}{\ker\phi^\#}}_{z_I}=\frac{\pa{\Bc_{k,n}}_{z_I}}{(\ker\phi^\#)_{z_I}},\]
thus
\[\pa{\frac{\Bc_{k,n}}{\ker\phi^\#}}_{z_I}^0=\pa{\frac{\pa{\Bc_{k,n}}_{z_I}}{(\ker\phi^\#)_{z_I}}}^0=\frac{\pa{\Bc_{k,n}}_{z_I}^0}{\ker \phi^\#_I}\cong \K\spa{\frac{\det X_J}{\det X_I}}\]
\end{remark}

In summary we have shown that, up to some canonical identifications,
\[\Gr(k,n)=\bigcup_{I\in \omega(k,n)}\Spec\pa{\K\spa{\frac{\det X_J}{\det X_I}}}\doteqdot \bigcup_{I\in \omega(k,n)}\Gr_I(k,n).\]

\begin{notation}
Let $I$ be a $(k,n)$-multiindex, $i\in\cpa{1,\cdots,k}$ and $j\in\cpa{1,\cdots, n}$. We define $I^i_j$ to be the multiindex which is the same as $I$ but with the $i$-th entry replaced with $j$.
\end{notation}

\begin{lemma}\label{FormalBasisChange}
If $I\in\omega(k,n)$ then the following equality holds in $\K[X_{k,n}]_{\det X_I}$
\[X_I\ii X=\mat{
w_{I^1_{1}} & \cdots & w_{I^1_n}\\
\vdots      & \ddots & \vdots\\
w_{I^k_{1}} & \cdots & w_{I^k_n}
},\qquad\text{where }w_J=\frac{\det X_J}{\det X_I}\]
\end{lemma}
\begin{proof}
Recall that if $\Adj(X_I)$ is the adjugate matrix of $X_I$ then
\[\pa{X_I}\ii=\frac1{\det X_I}\Adj(X_I)=\frac1{\det X_I}\mat{(-1)^{i+j}\det (X_I)_{\times j\times i}}_{\smat{1\leq i\leq k\\1\leq j\leq k}}.\]
We can verify the identity for each element:
\[\frac{(\Adj(X_I) X)_{i,j}}{\det X_I}=\frac1{\det X_I}{\sum_{\ell=1}^k\pa{(-1)^{i+\ell} \det \pa{X_I}_{\times \ell,\times i}}x_{\ell,j}}=\frac{\det X_{I^i_j}}{\det X_I}=w_{I^i_j}.\]
\end{proof}

\begin{remark}
$(X_I\ii X)_J=X_I\ii X_J$, in particular
$(X_I\ii X)_I$ is the identity matrix.
\end{remark}

\begin{proposition}\label{GrassmannianIsCoveredByAffineSpaces}
$\Gr_I(k,n)$ is isomorphic to $\A^{k(n-k)}_\K$ as a scheme.
\end{proposition}
\begin{proof}
Since both schemes are affine, it is enough to show that their coordinate rings are isomorphic. Without loss of generality we may assume that $I=\pa{1,\cdots, k}$. 
For brevity we set $w_J=\frac{\det X_J}{\det X_I}$. 

Let $M$ be the formal matrix whose $(i,j)$-entry is $w_{I^i_j}$. Lemma (\ref{FormalBasisChange}) shows that $M=X_I\ii X$, so $\det M_J=\det X_I\ii \det X_J=w_J$. This shows that
\[\K\spa{\frac{\det X_J}{\det X_I}\mid J\in \omega(k,n)}=\K\spa{\frac{\det X_J}{\det X_I}\mid J=I^j_{\ell_j},\ j\in \cpa{1,\cdots, k},\ \ell_j\notin I}.\]
Let $R$ denote this ring. To conclude we want to show that it is isomorphic to $\K[Y_{k,n-k}]=\K[y_{1,1},\cdots, y_{k,n-k}]$.
\medskip

Let us consider the following ring homomorphism
\[\chi:\funcDef{\K[Y_{k,n-k}]}{R}{y_{i,j}}{w_{I^i_{j+k}}}.\]
It is surjective by construction, so we just need to show that it is injective to find the desired isomorphism.

Suppose that there exists a nonzero polynomial $p\in \K[Y_{k,n-k}]$ which maps to $0$. If $\ol \K$ is an algebraic closure\footnote{we can take any field extension $\K\subseteq \F$ where $\F$ is an infinite field.} of $\K$ we can consider the lift 
\[\wt\chi:\funcDef{\ol\K[Y_{k,n-k}]}{\wt R=\ol \K[w_{I^i_j}]}{y_{i,j}}{w_{I^i_{j+k}}}\]
Note that if $\chi(p)=0$ then $\wt\chi(p)=0$ because $R\subseteq \wt R$ and $\wt \chi\res{\K[Y_{k,n-k}]}=\chi$. Consider now any matrix of the form
\[A=\mat{I_k\mid \wt A}=\pa{a_{i,j}}_{i,j}\]
where $I_k$ is the $k\times k$ identity matrix and $\wt A\in \Mc(k,n-k,\ol \K)$. 
From what we have said above it follows that $\det A_{I^i_j}=a_{i,j}$, so 
\[p(\wt A)=p\pa{\pa{\det A_{I^i_j}}_{\smat{i\in\cpa{1,\cdots, k},~~~\\ j\in \cpa{k+1,\cdots, n}}}}=\wt \chi(p)(A)=0.\] 
This shows that $p$ has infinitely many roots in $\ol K$, so if we fix the value of $k(n-k)-1$ indeterminates the resulting polynomial is the $0$ polynomial. 
If we reiterate this reasoning we eventually prove that $p=0$ in $\ol \K[Y_{k,n-k}]$, but $0\in \K[Y_{k,n-k}]\subseteq \ol \K[Y_{k,n-k}]$, so $p$ is the zero polynomial in the original ring, contradicting our hypothesis.
\end{proof}

\begin{remark}\label{IntersectionIsAffine}
Since $\Gr_I(k,n)$ and $\Gr_J(k,n)$ are affine and $\Gr(k,n)$ is projective and thus separated, $\Gr_I(k,n)\cap \Gr_J(k,n)$ is affine for any choice of multiindices.
\end{remark}




\section{Grassmannian moduli problem}

Let us consider the following moduli problem
\[\gr(k,n):\functorDef{(\Sch\K)\op}{\Set}{T}{\quot{\cpa{\al:\Oc_T^n\onto Q}}\sim}{f:S\to T}{(\al:\Oc_T^n\to Q)\mapsto (f^\ast\al:\Oc_S^n\to f^\ast Q)}\]
where $Q$ is a locally free sheaf of rank $k$ on $T$ and two surjections $\al:\Oc_T^n\onto Q$, $\beta:\Oc_T^n\onto V$ are equivalent if and only if there exist an isomorphism of sheaves $\theta:Q\to V$ such that the diagram commutes
\[\begin{tikzcd}
	{\Oc_T^n} & Q \\
	& V
	\arrow["\al", two heads, from=1-1, to=1-2]
	\arrow["\beta"', two heads, from=1-1, to=2-2]
	\arrow["\theta", from=1-2, to=2-2]
\end{tikzcd}\]
We have functoriality because of the composition properties of pullbacks.

\begin{remark}
This functor formalizes the classification problem of $(n-k)$-dimensional subspaces of an $n$-dimensional space. Indeed
\[\gr(k,n)(\Spec\K)=\quot{\cpa{\al:\Oc_{\Spec \K}^n\onto Q}}\sim\cong\quot{\cpa{\vp:\K^n\onto \K^k}}\sim=\Gr(k,n)(\K).\]
For the middle isomorphism we used the fact that sheaves over a point are skyscrapers and that $\Oc_{\Spec\K,\Spec \K}=\K$. The last equality is our first definition for the Grassmannian up to the choice of a basis.
\end{remark}

\begin{remark}
We could have defined the moduli problem equivalently as follows:
\[\gr'(k,n):\functorDef{(\Sch\K)\op}{\Set}{T}{\cpa{\Fc\mid \emat{\Fc\text{ vector subbundle of $\Oc_T^n$ of}\\ \text{ rank $k$ s.t. $\Oc^n_T/\Fc$ is locally free}}}}{f:S\to T}{\Fc\mapsto f^\ast\Fc},\]
indeed the following is the data of a natural isomorphism
\[\correspDef{\gr(k,n)(T)}{\gr'(n-k,n)(T)}{[q:\Oc^n_T\onto Q]}{\ker q}{[\Oc^n_T\to \Oc^n_T/\Fc]}{\Fc}\]
We chose to adopt the first definition because it is easier to verify whether a map is a valid quotient (as in, we do not need to compute a quotient sheaf) and because the first definition generalizes well to objects like the functor of quotients, which we will introduce in the next chapter.
\end{remark}


\noindent In this section we prove that the Grassmann scheme is a fine moduli space for the Grassmannian moduli problem.

\subsection{Open subfunctor cover of the Grassmannian}
\begin{notation}
For any multiindex $I\in\omega(k,n)$ and any scheme $T$ we define the following morphism of sheaves
\[s_I^T:\funcDef{\Oc_T^k}{\Oc_T^n}{e_j}{e_{i_j}}.\]
If there is no ambiguity we write $s_I$.
\end{notation}

\begin{definition}[Principal subfunctors of the Grassmannian]
Fixed a multiindex $I\in \omega(k,n)$ we define the following functor
\[\gr_I(k,n):\functorDef{(\Sch\K)\op}{\Set}{T}{\quot{\cpa{\Oc_T^n\overset{\al}\onto Q\mid \al\circ s_I\text{ surjective}}}\sim}{f}{\al\mapsto f^\ast \al}\]
where the equivalence relation is the same as the one defined for $\gr(k,n)$.
\end{definition}

\begin{proposition}
The functor $\gr_I(k,n)$ is well defined.
\end{proposition}
\begin{proof}
First we observe that $\gr_I(k,n)(T)$ is well defined because if $\psi=\theta\circ \al$ with $\theta$ isomorphism of sheaves then on each stalk we have
\[\psi_x\circ (s_I)_x=\theta_x\circ \vp_x\circ (s_I)_x,\]
which is surjective if and only if $\vp_x\circ (s_I)_x$ is surjective.

Consider now a morphism $f:S\to T$, then
\[f^\ast\al \circ s_I^S=f^\ast\al\circ f^\ast s_I^T=f^\ast(\al\circ s_I^T)\]
is surjective if and only if it is surjective on all stalks, i.e. if and only if for all $s\in S$ we have that the following map is surjective
\[f^\ast(\al\circ s_I^T)_s=(\al\circ s_I^T)_{f(s)}\otimes_{\Oc_{T,f(s)}}id_{\Oc_{S,s}},\]
which is true because the tensor product is right-exact.
\end{proof}


\begin{proposition}\label{GrIAreOpenSubfunctors}
The $\gr_I(k,n)$ are open subfunctors of $\gr(k,n)$.
\end{proposition}
\begin{proof}
The inclusion $\gr_I(k,n)(T)\subseteq \gr(k,n)(T)$ is apparent, so we just need to show that if we fix a quotient $[\al:\Oc_T^n\onto Q]$ in $\gr(k,n)(T)$ then we can find an open subscheme of $T$ which represents $h_T\times_{\gr(k,n)}\gr_I(k,n)$.\medskip

Let us fix a representative $\al$ for the given quotient. The locus where $\al\circ s_I:\Oc_T^k\to Q$ is surjective is the complement of the support of its cokernel sheaf $\Kc$, i.e. 
\[(\al\circ s_I)_x\text{ surjective} \coimplies x\notin \Supp\Kc.\]
Note that by the definition of $\sim$ and properties of isomorphisms of sheaves, the first condition does not depend of the choice of representative for $[\al]$, so $\Supp\Kc$ only depends on $[\al]$. Note that $\Kc$ is of finite type because the codomains are locally free of finite rank, so $\Supp\Kc$ is closed\footnote{For more detail see \href{https://stacks.math.columbia.edu/tag/01B4}{Section 01B4} in \cite{stacks}} and hence $U_I=T\bs \Supp \Kc$ is open.\medskip

We now want to show that $U_I$ represents the functor $h_T\times_{\gr(k,n)}\gr_I(k,n)$, that is we want to show that if $f:S\to T$ is a morphism of $\K$-schemes then $f$ factors through $U_I$ if and only if $[f^\ast\al:\Oc_S^n\to f^\ast Q] \in \Gr_I(S)$.\medskip

Note that $f(s)\in U_I$ if and only if $(\al\circ s_I^T)_{f(s)}$ is surjective which, by Nakayama's lemma applied to the cokernels, is equivalent to the surjectivity of
\[(\al\circ s_I^T)\res{f(s)}:k(f(s))^n\to Q_{f(s)}\otimes_{\Oc_{T,f(s)}}k(f(s)).\]
Observe that, up to standard identifications,
\begin{align*}
f^\ast(\al\circ s^T_I)\res s=&f^\ast(\al\circ s^T_I)_s\otimes_{\Oc_{S,s}} id_{k(s)}=\\
=&((\al\circ s^T_I)_{f(s)}\otimes_{\Oc_{T,f(s)}}id_{\Oc_{S,s}})\otimes_{\Oc_{S,s}}id_{k(s)}=\\
=&(\al\circ s^T_I)_{f(s)}\otimes_{\Oc_{T,f(s)}}id_{k(s)}=\\
=&((\al\circ s^T_I)_{f(s)}\otimes_{\Oc_{T,f(s)}}id_{k(f(s))})\otimes_{k(f(s))}id_{k(s)}=\\
=&(\al\circ s^T_I)\res{f(s)}\otimes_{k(f(s))} id_{k(s)}.
\end{align*}
Note that we used the fact that $\Oc_{T,f(s)}\to \Oc_{S,s}\to k(s)=\Oc_{T,f(s)}\to k(f(s))\to k(s)$. Since field extensions do not change the rank of linear maps, this shows that
\[f^\ast(\al\circ s^T_I)\res s\text{ is surjective }\quad\coimplies\quad (\al\circ s^T_I)\res{f(s)}\text{ is surjective}.\]
By Nakayama's lemma we can again consider equivalently $f^\ast(\al\circ s^T_I)_s=(f^\ast\al)_{s}\circ (s^S_I)_s$.


We have thus shown that $f(s)\in U_I$ if and only if $(f^\ast\al)_{s}\circ (s^S_I)_s$ is surjective, i.e. $f$ factors through $U_I$ if and only if $(f^\ast\al)\circ s^S_I$ is surjective, i.e. $f^\ast\al\in \gr_I(k,n)(S)$.
\end{proof}


\begin{proposition}\label{GrIAreOpenCover}
The collection $\cpa{\gr_I(k,n)}$ is an open cover of $\gr(k,n)$.
\end{proposition}
\begin{proof}
For any $\K$-scheme $S$ and any quotient $[\al]\in \Gr(k,n)(S)$ (without loss of generality we choose a representative $\al$) we need to show that for any $s\in S$ there exists a multiindex $I$ such that $s\in U_I$ defined as in the previous proposition.

We are therefore looking for a multiindex $I$ such that $(\al\circ s_I)_s$ is surjective. By Nakayama's lemma this is equivalent to showing that there exists an $I$ such that
\[k(s)^k\overset{s_I}\to k(s)^n\overset{\al_s}\to Q_s\otimes_{\Oc_{S,s}}k(s)\]
is surjective, which is trivially true since $\rnk \al_s=k$.
\end{proof}


\subsection{Representability of the Grassmannian functor}
\begin{lemma}\label{UniqueIsomorphismInGrassmannFunctor}
Let $T$ be a scheme and $[\al:\Oc_T^n\onto Q],[\beta:\Oc_T^n\onto Q']\in\gr(k,n)$. If $[\al]=[\beta]$ then the isomorphism $\theta:Q\to Q'$ such that $\beta=\theta\circ \al$ is unique.
\end{lemma}
\begin{proof}
First, observe that if $\al=\beta$ then by surjectivity and commutativity $\theta=id_Q$.
Let $\theta,\eta:Q\to Q'$ be isomorphisms such that $\beta=\theta\circ \al$ and $\beta=\eta\circ \al$. Then $\theta\ii \circ \eta:Q\to Q$ is an isomorphism such that $\theta\ii\circ \eta\circ \al=\theta\ii\circ \beta=\al$, so $\theta\ii\circ \eta=id_Q$ and thus $\theta=\eta$.
\end{proof}

\begin{proposition}\label{GrassmannianIsSheaf}
The Grassmannian functor $\gr(k,n)$ is a Zariski sheaf.
\end{proposition}
\begin{proof}
Consider a $\K$-scheme $T$ and an open cover $\cpa{U_i\to T}$. Let $\al_i:\Oc_{U_i}^n\onto Q_i$ be representatives of quotients such that 
\[\al_i\res{U_i\cap U_j}\sim \al_j\res{U_i\cap U_j}.\]
Because of lemma (\ref{UniqueIsomorphismInGrassmannFunctor}), the isomorphism giving the equivalence above is unique. Let $\vp_{ji}:Q_i\res{U_{i}\cap U_j}\to Q_j\res{U_i\cap U_j}$ be this isomorphism. Because of the uniqueness $\vp_{ii}=id_{Q_i}$ and $\vp_{ki}=\vp_{kj}\circ \vp_{ji}$, so we have the data to glue the $Q_i$ to a locally free sheaf of rank $k$ over $T$, which we denote by $Q$. 

By construction $\al_i:\Oc_{U_i}^n\onto Q\res{U_i}$ for all $i$. 
Let $V\subseteq T$ be an open subset. For any section $s\in \Oc_T^n(V)$ we can define $\al_V(s)$ by gluing the $(\al_i)_V(s\res{U_i})$, which we can do by construction\footnote{More precisely, the $\vp_{ji}$ are the gluing functions on $Q$ and \[\al_j(s\res{U_j})\res{U_i\cap U_j}=\al_j(s\res{U_i\cap U_j})=\vp_{ji}\circ \al_i(s\res{U_i\cap U_j})= \vp_{ji}(\al_j(s\res{U_j})\res{U_i\cap U_j}).\]} of $Q$. It is well known that a sheaf morphism is determined by its restrictions to open sets.
\end{proof}

\begin{proposition}\label{GrIRepresentGrIFunctors}
The affine scheme $\Gr_I(k,n)$ represents $\gr_I(k,n)$.
\end{proposition}
\begin{proof}
First we prove that for any $\K$-scheme $T$, $\Hom_{\Sch\K}(T,\Gr_I(k,n))\cong \gr_I(T)$, then we need to check naturality.

By definition $\Gr_I(k,n)=\Spec\pa{\K \spa{\frac{\det X_J}{\det X_I}}}$, so
\[\Hom_{\Sch\K}(T,\Gr_I(k,n))\cong \Hom_{\K\text{-alg}}\pa{\K\spa{\frac{\det X_J}{\det X_I}},\Oc_T(T)}.\]
For a map $\al:\Oc^n_T\to \Oc_T^k$, we define $M(U)$ as the matrix which represents $\al_U:\Oc_T^n(U)\to\Oc_T^k(U)$ in the canonical bases.
We define the following maps
\[\correspDef{\Hom_{\K\text{-alg}}\pa{\K\spa{\frac{\det X_J}{\det X_I}},\Oc_T(T)}}{\cpa{\al:\Oc_T^n\to \Oc_T^k\mid \al\circ s_I=id_{\Oc_T^k}}}{\vp}{\eta(\vp)}{\rho(\al):\frac{\det X_J}{\det X_I}\mapsto \frac{\det (M(T))_J}{\det (M(T))_I}}{\al}\]
where $\eta(\vp)$ is defined on an open subset $V$ of $T$ by
\[\eta(\vp)_V(e_j)=
\sum_{i=1}^k (\mathrm{res}^T_V\circ\vp)\pa{\frac{\det X_{I^{i}_j}}{\det X_I}}e_r\pasgnl={(\ref{FormalBasisChange})}(\mathrm{res}^T_V\circ \vp)\pa{X_I\ii X} e_j.
\]
The maps are well defined because $\al\circ s_I=id_{\Oc^k_T}\coimplies M(T)_I=I_k$ and 
\[\dfrac{\det X_{I^{r}_{i_s}}}{\det X_I}=\delta_{r,s}\implies \eta(\vp)\circ s_I=id_{\Oc^k_T}.\]
We can see that $\eta$ and $\rho$ are inverses via the following computations:
\[\mathrm{res}^T_V\circ \rho(\al)(X_I\ii X)=\mathrm{res}^T_V({M(T)_I}\ii M(T))=\mathrm{res}^T_V(I_k\ii M(T))=M(V),\]
\begin{align*}
\rho(\eta(\vp))\pa{\frac{\det X_J}{\det X_I}}=&\frac{\det((\mathrm{res}^T_T\circ \vp)\pa{X_I\ii X}_J)}1=\\
=&\vp(\det((X_I\ii X)_J))=\vp\pa{\frac{\det X_J}{\det X_I}}.
\end{align*}
Observe now that
\[\correspDef{\cpa{\al:\Oc_T^n\to \Oc_T^k\mid \al\circ s_I=id_{\Oc_T^k}}}{\quot{\cpa{\al:\Oc_T^n\onto Q\mid \al\circ s_I\text{ isomorphism}}}\sim}{\al}{[\al]}{(\beta\circ s_I)\ii\circ \beta}{[\beta]}\]
is a bijection. The second map is well defined because if $\beta=\theta\circ \beta'$ then \[(\beta\circ s_I)\ii\circ \beta=(\beta'\circ s_I)\ii\circ \theta\ii\circ \theta\circ \beta'=(\beta'\circ s_I)\ii\circ \beta'\]
and they are inverses because $\beta\sim (\beta\circ s_I)\ii\circ \beta$ by definition of $\sim$ and if $\al\circ s_I=id_{\Oc_T^k}$ then $(\al\circ s_I)\ii\circ \al=\al$.
We conclude by noticing that
\[\quot{\cpa{\al:\Oc_T^n\onto Q\mid \al\circ s_I\text{ isomorphism}}}\sim=\quot{\cpa{\al:\Oc_T^n\onto Q\mid \al\circ s_I\text{ surjective}}}\sim\]
because on all stalks $\al\circ s_I$ is an endomorphism of finitely generated modules.
\bigskip

To prove naturality we consider a morphism  $f:S\to T$ of $\K$-schemes. Recall that
\[\funcDef{\gr_I(k,n)(T)}{\gr_I(k,n)(S)}{[\al]}{[f^\ast \al]}.\]
Under the bijection above, imposing naturality gives
\[\funcDef{\cpa{\al:\Oc_T^n\to \Oc_T^k\mid \al\circ s_I=id_{\Oc_T^k}}}{\cpa{\beta:\Oc_S^n\to \Oc_S^k\mid \beta\circ s_I=id_{\Oc_S^k}}}{\al}{f^\ast\al}\]
since $f^\ast\al\circ s_I^S=f^\ast(\al\circ s_I^T)=f^\ast(id_{\Oc^k_T})=id_{\Oc^k_S}$.
If we impose naturality again we get
\[\funcDef{\Hom_{\K\text{-alg}}\pa{\K\spa{\frac{\det X_J}{\det X_I}},\Oc_T(T)}}{\Hom_{\K\text{-alg}}\pa{\K\spa{\frac{\det X_J}{\det X_I}},\Oc_S(S)}}{\vp}{\rho(f^\ast\eta(\vp))}\]
We claim that $\rho(f^\ast(\eta(\vp)))=f^\#(T)\circ \vp$. Since $\eta$ is the inverse of $\rho$, it is enough to prove that $f^\ast(\eta(\vp))=\eta(f^\#(T)\circ \vp)$. 
Equality holds because for all $s\in S$ both of the maps induced on stalks are represented by the matrix
\[f^\#_s\pa{\pa{\vp(X_I\ii X)}_{f(s)}}.\]
We conclude by recalling that the following diagram commutes
% https://q.uiver.app/#q=WzAsNCxbMSwwLCJcXEhvbV97XFxLXFx0ZXh0ey1hbGd9fVxccGF7XFxLXFxzcGF7XFxmcmFje1xcZGV0IFhfSn17XFxkZXQgWF9JfX0sXFxPY19UKFQpfSJdLFsxLDEsIlxcSG9tX3tcXEtcXHRleHR7LWFsZ319XFxwYXtcXEtcXHNwYXtcXGZyYWN7XFxkZXQgWF9KfXtcXGRldCBYX0l9fSxcXE9jX1MoUyl9Il0sWzAsMCwiXFxIb21fe1xcU2NoXFxLfShULFxcR3JfSShrLG4pKSJdLFswLDEsIlxcSG9tX3tcXFNjaFxcS30oUyxcXEdyX0koayxuKSkiXSxbMiwwLCJcXFNwZWMiXSxbMywxLCJcXFNwZWMiXSxbMiwzLCJoX3tcXEdyX0koayxuKX0oZikiLDJdLFswLDEsIlxcSG9tXFxwYXtcXEtcXHNwYXtcXGZyYWN7XFxkZXQgWF9KfXtcXGRldCBYX0l9fSxmXlxcIyhUKX0iXV0=
\[\begin{tikzcd}
	{\Hom_{\Sch\K}(T,\Gr_I(k,n))} & {\Hom_{\K\text{-alg}}\pa{\K\spa{\frac{\det X_J}{\det X_I}},\Oc_T(T)}} \\
	{\Hom_{\Sch\K}(S,\Gr_I(k,n))} & {\Hom_{\K\text{-alg}}\pa{\K\spa{\frac{\det X_J}{\det X_I}},\Oc_S(S)}}
	\arrow["\Spec", from=1-1, to=1-2]
	\arrow["{h_{\Gr_I(k,n)}(f)}"', from=1-1, to=2-1]
	\arrow["{\Hom\pa{\K\spa{\frac{\det X_J}{\det X_I}},f^\#(T)}}", from=1-2, to=2-2]
	\arrow["\Spec", from=2-1, to=2-2]
\end{tikzcd}\]
\end{proof}

\begin{theorem}\label{GrassmannianIsModuliSpace}
The Grassmann scheme $\Gr(k,n)$ is a fine moduli space for the Grassmann functor $\gr(k,n)$.
\end{theorem}
\begin{proof}
We know that $\cpa{\gr_I(k,n)\to\gr(k,n)}$ is an open cover (\ref{GrIAreOpenCover}), that the functor $\gr(k,n)$ is a Zariski sheaf (\ref{GrassmannianIsSheaf}) and that $h_{\Gr_I(k,n)}\cong \gr_I(k,n)$ (\ref{GrIRepresentGrIFunctors}). If we can show that these isomorphisms restrict well to double intersections we have the desired result by proposition (\ref{MapGluingForZariskiSheaves}).

Let $T$ be a scheme and let us consider a morphism
\[f\in\Hom_{\Sch\K}\pa{T,\Gr_I(k,n)\cap \Gr_J(k,n)}=\Hom_{\Sch\K}\pa{T,\Gr(k,n)_{z_Iz_J}}.\]
Applying a standard result for morphisms towards an affine scheme\footnote{see remark (\ref{IntersectionIsAffine}).} we get 
\[f^\#(T)\in\Hom_{\K\text{-alg}}\pa{\pa{\K\spa{\det X_L}_{\det X_I\det X_J}}^0,\Oc_T(T)}.\]
By the universal property of localization, we may identify this set with
\[\cpa{\beta\in \Hom_{\K\text{-alg}}\pa{\K\spa{\frac{\det X_L}{\det X_I}},\Oc_T(T)}\mid \beta\pa{\frac{\det X_J}{\det X_I}}\in \Oc_T(T)^\ast}.\]
Applying the functor $\eta$  defined during the proof of proposition (\ref{GrIRepresentGrIFunctors}), which we will denote $\eta^I$ to emphasize which determinant we consider at the denominator, we obtain\footnote{the condition on the image of $\frac{\det X_J}{\det X_I}$ corresponds to $\det(\al\circ s_J)$ being invertible, and thus to $\al\circ s_J$ being an isomorphism.} 
\[\eta^I(f^\#(T))\in\cpa{\al:\Oc_T^n\to \Oc^k_T\mid \al\circ s_I=id_{\Oc^k_T},\ \al\circ s_J\text{ isomorphism}}.\]
Observe that we can identify this set with
\[{\cpa{\al:\Oc^n_T\to Q\mid \al\circ s_I,\ \al\circ s_J\text{ surjective}}}/_{\sim}=(\gr_I(k,n)\times_{\gr(k,n)}\gr_J(k,n))(T),\]
so to conclude the proof we just need to verify that $\eta^I(f^\#(T))\sim \eta^J(f^\#(T))$ in $\gr(k,n)$. By lemma (\ref{FormalBasisChange}), the matrix $X_J\ii X_I$ can be described only using elements in the ring $\K\spa{\det X_L}_{\det X_I\det X_J}^0$. We can thus define $\theta$ by setting $\theta_V(e_j)=f^\#(V)(X_J\ii X_I)e_j$. It is clear by construction that $\theta\circ \eta^I(f^\#(T))=\eta^J(f^\#(T))$. Defining $\delta$ from $X_I\ii X_J$ analogously yields an inverse of $\theta$, realizing the sought out equivalence. 
\end{proof}