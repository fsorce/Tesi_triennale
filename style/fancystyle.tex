%PER CAMBIARE I MARGINI
%\usepackage[margin=4cm]{geometry}
%\usepackage{geometry}
%\usepackage[inner=3.5483cm,outer=5.2817cm,bottom=4.27cm,top=4.4cm]{geometry}
%\usepackage[inner=2cm,outer=4cm, bindingoffset=0.9085cm]{geometry}
\usepackage[inner=2.5cm,outer=2.5cm, bindingoffset=1cm]{geometry}
%\usepackage[inner=3.5cm,outer=5cm,bottom=4cm,top=4cm]{geometry}

\usepackage{emptypage} % Pagine vuote non numerate
\usepackage{fancyhdr} % Sistema headers
\usepackage[Lenny]{fncychap} % Capitoli fighi
\ChTitleVar{\Huge\bfseries}
\usepackage[hang,flushmargin]{footmisc} % Footnote non indentata
%========== Stile header e footer ==============

\pagestyle{fancy}
% Left, Right, Even(pages), Odd(pages), Center
\fancyhead[R]{\thepage}
\fancyhead[L]{\leftmark}
%\fancyhead[RE]{\rightmark}
\fancyfoot[C]{}

\renewcommand{\chaptermark}[1]{\markboth{\textsc{#1}}{}}
\renewcommand{\sectionmark}[1]{\markright{\textsc{#1}}}

%\renewcommand{\headrulewidth}{0.05pt}
%\renewcommand{\footnoterule}{\kern 0pt	\hrule width \textwidth height 0.5pt \kern 5pt}

%Numeri pagina in prima pagina capitoli
\makeatletter
\let\ps@plain\ps@empty
\makeatother

%==== Colore dei footnote, link e citazioni ====
\definecolor{DarkBlue}{HTML}{00518B}
\hypersetup{
    colorlinks=true,
    linkcolor=DarkBlue,
    filecolor=DarkBlue,
    citecolor = DarkBlue,
    urlcolor=DarkBlue,
}
\renewcommand\thefootnote{\textcolor{blue}{\arabic{footnote}}}